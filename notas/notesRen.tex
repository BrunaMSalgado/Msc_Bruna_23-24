\documentclass[10pt,a4paper]{amsart}
\usepackage[utf8]{inputenc}
\usepackage[english]{babel}
\usepackage[mathcal]{euscript}
\usepackage{mathrsfs}  
\usepackage{amsmath}
\usepackage{amsfonts}
\usepackage{mathrsfs}  
\usepackage{amssymb}
\usepackage{stmaryrd}
\usepackage{graphicx}
\usepackage{mathtools}
\usepackage{pgfplots}
\usepackage{xfrac}
\usepackage{listings}
\usepackage{xcolor}
\usepackage{lstlinebgrd}
\usepackage{ebproof}
\usepackage{graphicx,float}
\usepackage[colorlinks=true, linkcolor=cyan, urlcolor=cyan, citecolor=cyan]{hyperref}
\usepackage[shortlabels]{enumitem}
\usepackage[textsize=tiny,textwidth=45]{todonotes}

\newcommand{\klcomp}{\star}
\newcommand{\parI}{\mathop{\bowtie}}
\newcommand{\seqI}{\mathop{\triangleright}}
\DeclareMathOperator{\demon}{\square}
\DeclareMathOperator{\angel}{\Diamond}
\makeatletter
\DeclareRobustCommand{\iscircle}{\mathord{\mathpalette\is@circle\relax}}
\newcommand\is@circle[2]{%
  \begingroup
  \sbox\z@{\raisebox{\depth}{$\m@th#1\bigcirc$}}%
  \sbox\tw@{$#1\square$}%
  \resizebox{!}{\ht\tw@}{\usebox{\z@}}%
  \endgroup
}
\makeatother
\DeclareMathOperator{\statt}{\iscircle_\prog{p}}
\newcommand{\schfont}[1]{\mathcal{#1}}
\newcommand{\sch}{\schfont{S}}
\newcommand{\conv}[1]{\mathrm{conv}\, {#1}}
%%%%%%%%%%%%% Macros
\newcommand{\renato}[1]{\textcolor{teal}{RN Note: #1}}
\newcommand{\codiag}{\triangledown}
%%%% Categories
\newcommand{\catfont}[1]{\mathsf{#1}}
\newcommand{\cop}{\catfont{op}}
\newcommand{\Law}{\catfont{Law}}
\newcommand{\catV}{\catfont{V}}
\newcommand{\catX}{\catfont{X}}
\newcommand{\catC}{\catfont{C}}
\newcommand{\catCat}{\catfont{C}}
\newcommand{\catCop}{\catfont{C^{op}}}
\newcommand{\catD}{\catfont{D}}
\newcommand{\catE}{\catfont{E}}
\newcommand{\catA}{\catfont{A}}
\newcommand{\catB}{\catfont{B}}
\newcommand{\catP}{\catfont{P}}
\newcommand{\catMet}{\catfont{Met}}
\newcommand{\catCPTP}{\catfont{CPTP}}
\newcommand{\catCPS}{\catfont{CPS}}
\newcommand{\catCP}{\catfont{CP}}
\newcommand{\catQ}{\catfont{Q}}
\newcommand{\catSet}{\catfont{Set}}
\newcommand{\catFinSet}{\catfont{FinSet}}
\newcommand{\catPO}{\catfont{PO}}
\newcommand{\catCompFunc}{\catfont{CompFunc}}
\newcommand{\catBan}{\catfont{Ban}}
\newcommand{\catVect}{\catfont{CVect}}
\newcommand{\WstarCPSU}{\catfont{Wstar_{CPSU}}}
\newcommand{\WstarCPSUop}{\left(\catfont{Wstar_{CPSU}}\right)^{\catfont{op}}}
\newcommand{\catI}{\catfont{I}}
\newcommand{\Set}{\catfont{Set}}
\newcommand{\Top}{\catfont{Top}}
\newcommand{\Pos}{\catfont{Pos}}
\newcommand{\Inj}{\catfont{Inj}}
\newcommand{\Det}{\catfont{RMhat}}
\newcommand{\CoAlg}[1]{\catfont{CoAlg}\left (#1 \right )}
\newcommand{\Mon}{\catfont{Mon}}
\newcommand{\Mnd}{\catfont{Mnd}(\catC)}
\newcommand{\SMnd}{\catfont{Mnd}(\Set)}
\newcommand{\CLat}{\catfont{CLat}}
\newcommand{\SLat}{\catfont{SLat}}
\newcommand{\Stone}{\catfont{Stone}}
\newcommand{\Spectral}{\catfont{Spectral}}
\newcommand{\CompHaus}{\catfont{CompHaus}}
\newcommand{\Subs}[2]{\catfont{Sub}_{}}
\newcommand{\Cone}{\catfont{Cone}}
\newcommand{\LCone}{\catfont{LCone}}
\newcommand{\StComp}{\catfont{StablyComp}}
\newcommand{\PosC}{\catfont{PosComp}}
\newcommand{\Haus}{\catfont{Haus}}
\newcommand{\Meas}{\catfont{Meas}}
\newcommand{\Coh}{\catfont{CohDom}}
\newcommand{\Ord}{\catfont{Ord}}
\newcommand{\Dcpo}{\catfont{DCPO}}
\newcommand{\Dom}{\catfont{Dom}}
\newcommand{\EndoC}{[\catC,\catC]}
\newcommand{\Dcpop}{\catfont{DCPO}^\catfont{p}}
%% General functors
\newcommand{\funfont}[1]{#1}
\newcommand{\funF}{\funfont{F}}
\newcommand{\funU}{\funfont{U}}
\newcommand{\funG}{\funfont{G}}
\newcommand{\funT}{\funfont{T}}
\newcommand{\funI}{\funfont{I}}
%% Particular kinds of functors
\newcommand{\sfunfont}[1]{\mathrm{#1}}
\newcommand{\Pow}{\sfunfont{P}}
\newcommand{\PP}{\sfunfont{V}}
\newcommand{\Dist}{\sfunfont{V}_{=1,\omega}}
\newcommand{\PDist}{\sfunfont{V}_{\leq 1,\omega}}
\newcommand{\Maybe}{\sfunfont{M}}
\newcommand{\List}{\sfunfont{L}}
\newcommand{\UForg}{\sfunfont{U}}
\newcommand{\Forg}[1]{\sfunfont{U}_{#1}}
\newcommand{\Id}{\sfunfont{Id}}
\newcommand{\Vie}{\sfunfont{V}}
\newcommand{\Disc}{\funfont{D}}
\newcommand{\Weight}{\sfunfont{W}}
\newcommand{\homf}{\sfunfont{hom}}
\newcommand{\Yoneda}{\sfunfont{Y}}
%% Diagram functors
\newcommand{\Diag}{\mathscr{D}}
\newcommand{\KDiag}{\mathscr{K}}
\newcommand{\LDiag}{\mathscr{L}}
%% Monads
\newcommand{\monadfont}[1]{\mathbb{#1}}
\newcommand{\monadT}{\monadfont{T}}
\newcommand{\monadS}{\monadfont{S}}
\newcommand{\monadU}{\monadfont{U}}
\newcommand{\monadH}{\monadfont{H}}
\newcommand{\str}{\mathrm{str}}
%% Adjunctions
\newcommand\adjunct[2]{\xymatrix@=8ex{\ar@{}[r]|{\top}\ar@<1mm>@/^2mm/[r]^{{#2}}
& \ar@<1mm>@/^2mm/[l]^{{#1}}}}
\newcommand\adjunctop[2]{\xymatrix@=8ex{\ar@{}[r]|{\bot}\ar@<1mm>@/^2mm/[r]^{{#2}}
& \ar@<1mm>@/^2mm/[l]^{{#1}}}}
%% Retractions
\newcommand\retract[2]{\xymatrix@=8ex{\ar@{}[r]|{}\ar@<1mm>@/^2mm/@{^{(}->}[r]^{{#2}}
& \ar@<1mm>@/^2mm/@{->>}[l]^{{#1}}}}
%% Limits
\newcommand{\pv}[2]{\langle #1, #2 \rangle}
\newcommand{\limt}{\mathrm{lim}}
\newcommand{\pullbackcorner}[1][dr]{\save*!/#1+1.2pc/#1:(1,-1)@^{|-}\restore}
\newcommand{\pushoutcorner}[1][dr]{\save*!/#1-1.2pc/#1:(-1,1)@^{|-}\restore}
%% Colimits
\newcommand{\colim}{\mathrm{colim}}
\newcommand{\inl}{\mathrm{inl}}
\newcommand{\inr}{\mathrm{inr}}
%% Distributive categories
\newcommand{\distr}{\mathrm{dist}}
\newcommand{\undistr}{\mathrm{undist}}
%% Closedness
\newcommand{\curry}[1]{\mathrm{curry}{#1}}
\newcommand{\app}{\mathrm{app}}
%% Misc. operations
\newcommand{\const}[1]{\underline{#1}}
\newcommand{\comp}{\cdot}
\newcommand{\id}{\mathrm{id}}
\newcommand{\sw}{\mathrm{sw}}
\newcommand{\spt}{\mathrm{sp}}
\newcommand{\sh}{\mathrm{sh}}
\newcommand{\jn}{\mathrm{jn}}
\newcommand{\dist}{\mathrm{dist}}
\newcommand{\unfold}{\mathrm{unfold}}
\newcommand{\fold}{\mathrm{fold}}
%% Factorisations
\newcommand{\EClass}{E}
\newcommand{\MClass}{M}
\newcommand{\MConeClass}{\mathcal{M}}
%%%%%%%%%%%%%%%% End of Categorical Stuff

%%%% Misc
%% Operations
\newcommand{\blank}{\, - \,}
\newcommand{\sem}[1]{\left \llbracket #1 \right \rrbracket}
\newcommand{\asem}[1]{ \llparenthesis #1 \rrparenthesis}
\newcommand{\closure}[1]{\overline{#1}}
\DeclareMathOperator{\img}{\mathrm{im}}
\DeclareMathOperator{\dom}{\mathrm{dom}}
\DeclareMathOperator{\codom}{\mathrm{codom}}
%% Sets of numbers
\newcommand{\Nats}{\mathbb{N}}
\newcommand{\Reals}{\mathbb{R}}
\newcommand{\Rats}{\mathbb{Q}}
\newcommand{\Rz}{\Reals_{\geq 0}}
\newcommand{\Complex}{\mathbb{C}}
%% Writing
\newcommand{\cf}{\emph{cf.}}
\newcommand{\ie}{\emph{i.e.}}
\newcommand{\eg}{\emph{e.g.}}
\newcommand{\df}[1]{\emph{\textbf{#1}}}
%%%%%%%%%%%%%%%% End of Misc

%%%% Programming Stuff
%% Types
\newcommand{\typefont}[1]{\mathbb{#1}}
\newcommand{\typeOne}{1}
\newcommand{\typeTwo}{2}
\newcommand{\typeA}{\typefont{A}}
\newcommand{\typeB}{\typefont{B}}
\newcommand{\typeC}{\typefont{C}}
\newcommand{\typeV}{\typefont{V}}
\newcommand{\typeD}{\typefont{D}}
\newcommand{\typeE}{\typefont{E}}
\newcommand{\typeF}{\typefont{F}}
\newcommand{\typeI}{\typefont{I}}
%% RuleName
\newcommand{\rulename}[1]{(\mathrm{#1})}
%% Sequents
\newcommand{\jud}{\vdash}
\newcommand{\vljud}{\triangleright}
\newcommand{\cojud}{\vdash_{\co}}
\newcommand{\vl}{\mathtt{v}}
\newcommand{\co}{\mathtt{c}}
% Program font
\newcommand{\prog}[1]{\ensuremath{\tt #1}}
\newcommand{\blue}[1]{\textcolor{blue}{#1}}
\newcommand{\pseq}[3]{#1 \leftarrow #2; #3}
\newcommand{\ppm}[4]{(#1,#2) \leftarrow #3; #4}
\newcommand{\pinl}[1]{\prog{inl}(#1)}
\newcommand{\pinr}[1]{\prog{inr}(#1)}
\newcommand{\pcase}[5]{\prog{ case } #1 \prog{ \hspace{2pt} of \hspace{2pt} } \pinl{#2} \Rightarrow #3 ; \pinr{#4} \Rightarrow #5}
%% Sets of terms
\newcommand{\ValuesBP}[2]{\mathsf{Values}(#1, #2)}
\newcommand{\TermsBP}[2]{\mathsf{Terms}(#1, #2)}
\newcommand{\closValP}[1]{\ValuesBP{\emptyset}{#1}}
\newcommand{\closTermP}[1]{\TermsBP{\emptyset}{#1}}
\newcommand{\closVal}{\closValP{\typeA}}
\newcommand{\closTerm}{\closTermP{\typeA}}
%% Contextual equivalence
\newcommand{\ctxeq}{\equiv_{\prog{ctx}}}

%%%% End of Programming Stuff
\newcommand{\Shuff}{\mathrm{Sf}}

%%%% Domain theory
\newcommand{\upclos}{\mathord{\uparrow}}
\newcommand{\dwclos}{\mathord{\downarrow}}

%%%% Quantum stuff
\newcommand{\Hilb}{\catfont{Hilb}}
\newcommand{\tr}{\text{Tr}}

%%%% Norms
\newcommand{\euclideannorm}[1]{\left\lVert #1  \right\rVert_{2}}
\newcommand{\spectralnorm}[1]{\left\lVert #1  \right\rVert_{\infty}}
\newcommand{\tracenorm}[1]{\left\lVert #1  \right\rVert_{1}}
\newcommand{\diamondnorm}[1]{\left\lVert #1  \right\rVert_{\diamondsuit}}
\newcommand{\lonenorm}[1]{\left\lVert #1  \right\rVert_{ L^{1} }}
\newcommand{\gentracenorm}[1]{\left\lVert #1  \right\rVert_{ L^\infty }}
\newcommand{\gendiamondnorm}[1]{\left\lVert #1  \right\rVert_{ \diamondsuit \text{ gen}}}
\newcommand{\opnorm}[1]{\left\lVert #1  \right\rVert_{\text{op}}}
\newcommand{\norm}[1]{\left\lVert #1  \right\rVert}
\newcommand{\cbnorm}[1]{\left\lVert #1  \right\rVert_{\text{cb}}}
\newcommand{\projnorm}[1]{\left\lVert #1  \right\rVert_{\pi}}

%%%% Tensor
\newcommand{\projtensor}{\widehat{\otimes}_\pi}

\usepackage[left=2cm,right=2cm,top=2cm,bottom=2cm]{geometry}
\usepackage{tcolorbox}
\usepackage{braket}
\usepackage{quantikz}
\usepackage{tikz-cd}
\linespread{1.10}
\author{\dots}
\title{Notes}
\pgfplotsset{compat=1.18}

\lstset{
language=C,
showstringspaces=false,
keywordstyle=\color{blue},
basicstyle=\fontsize{10}{13}\ttfamily,
emph={exit,blue,unif,then,wait},emphstyle=\color{blue},
breaklines=true,
escapeinside={*@}{*@}
}


\usepackage{proof}
\usepackage{amsthm}
\usepackage[all]{xy}
%for definition
\theoremstyle{definition}
\newtheorem{definition}{Definition}[section]

%for examples
\theoremstyle{definition}
\newtheorem{example}[definition]{Example}

%lemmas
\theoremstyle{definition}
\newtheorem{lemma}[definition]{Lemma}

%proposition
\theoremstyle{definition}
\newtheorem{proposition}[definition]{Proposition}
%corollary
\theoremstyle{definition}
\newtheorem{corollary}[definition]{Corollary}

%theorem
\theoremstyle{definition}
\newtheorem{theorem}[definition]{Theorem}

% Renato's macros
\newcommand{\until}{\> \textcolor{blue}{\prog{for}} \>}
\newcommand{\then}{\textcolor{blue}{then}}
\newcommand{\progife}[3]{{ \blue{\prog{if}} \> #1 \> \blue{\prog{then}} \> 
{\prog #2} \> \blue{\prog{else}} \> {\prog #3}}}
\newcommand{\progwhile}[2]{{\blue{\prog{while}} \>  #1 \> \blue{\prog{do}} \> \{ \> {#2}  \> \}}}
\newcommand{\scomp}{\, \blue{;} \,}
\newcommand{\prem}[1]{(\text{if\/ }#1)}
\newcommand{\nline}{\vspace{-5mm}}
\newcommand{\ssto}[1][]{~\to^{#1}~}
\newcommand{\bsto}{~\Downarrow~}
\newcommand{\stp}{\mathit{stop}}
\newcommand{\skp}{\mathit{skip}}
\newcommand{\err}{\mathit{err}}
\providecommand{\comma}{,\operatorname{}\linebreak[1]}		% possibly line-beaking comma
\newcommand{\sep}{\kern1pt\comma\kern-1pt}
\newcommand{\lrule}[3]{\textbf{#1}\quad\frac{#2}{#3}}
\newcommand{\ptt}{{\prog tt}}
\newcommand{\pff}{{\prog ff}}
\newcommand{\unif}{{\prog \blue{unif}}}
\newcommand{\meas}[1]{\mathcal{M}(#1)}
\newcommand{\pmap}{ \xrightharpoonup{\hspace{0.1cm}} }

\begin{document}
\title{Xelatex is bloated}
\maketitle
\section{Probabilistic computation} \label{subsec:syntax_pc}

We now briefly illustrate the case of probabilistic computation, using as basic
examples two main topics in probability theory~\cite{dudley18} -- probabilistic
predicates and random walks on the real line.  Our illustration will be
grounded on a standard probabilistic model, namely the category $\Ban$ of
Banach spaces and linear contractions~\cite{dahlqvist19}. As discussed
in~\cite{dahlqvist2023syntactic} this category has a $\Met$-enriched
monoidal-closed structure and it is well-known that it has binary coproducts
given by the direct sum equipped with the $\ell_1$ norm. Thus in order to fit
$\Ban$ in our framework we only need to show that its binary coproduct
structure is $\Met$-enriched as well. We prove this next. 

Let us start with some preliminaries. The $\Met$-enriched structure of $\Ban$
is induced by the \emph{operator norm}. Specifically given a linear map 
$T : V \to W$ between Banach spaces we have,
\[
        \norm{T} = \sup \{ \norm{T(v)} \mid v \in V, \norm{v} = 1 \}
\]
Linear contractions will be precisely those linear maps $T$ such that $\norm{T}
\leq 1$, and the distance between two contractions $T$ and $S$ is set as
$\norm{ T - S}$. Given $T : V \to W$ and $S : U \to W$ their co-pairing $[T,S]
: V \oplus U \to W$ is defined by $[T,S](v,u) = T(v) + S(u)$. The fact that the
operator $[T,S]$ is contractive follows from the inequation $\norm{[T,S]} \leq
\max \{ \norm{T}, \norm{S} \}$ -- which is straightforward to prove when
one notices that every unitary vector $(v,u) \in V \oplus U$ can be rewritten
as,
\[
        \left (\norm{v} \frac{1}{\norm{v}} v, \norm{u} \frac{1}{\norm{u}} u \right )
        \hspace{1cm}
        \norm{v} + \norm{u} = 1
\]
The fact that the coproduct structure of $\Ban$ is $\Met$-enriched then follows
rather directly,
\begin{align*}
        d([T,S] , [T',S']) 
        & = 
        \norm{ [T,S] - [T',S'] }
        \\
        & = 
        \norm{ [T - T', S - S'] }
        \\
        & \leq
        \max \{ \norm{T - T'}, \norm{S - S'} \}
        \\
        & =
        \max \{ d(T,T'), d(S,S') \}
\end{align*}
Next, we will involve the notion of a measure (see \eg\ \cite[Chapter
10]{aliprantis06} or \cite[Chapter 2]{panangaden09}).

\begin{definition} For a measurable space $(X,\Sigma_X)$ a measure is a
        function $\mu : \Sigma_X \to \Reals$ such that $\mu(\emptyset) = 0$ and
        moreover it is $\sigma$-additive, \ie\ 
        \[
        \mu \left (\bigcup_{i =1}^{\infty} U_i \right ) = \sum_{i = 1}^{\infty}
        \mu(U_i) 
        \] 
where $(U_i)_{i \in \omega}$ is any family of pairwise disjoint measurable
sets.  A measure $\mu$ is called positive if $\mu(U) \geq 0$ for all measurable
sets $U$.  
\end{definition}

For a measurable space $X$ the set of measures $\Measu(X)$ forms a vector space
via pointwise extension. It also forms a Banach space when equipped with the
total variation norm,
\[
        \lVert \mu \rVert = 
        \sup \left \{ \sum_{i = 1}^n \, \lVert \mu(U_i) \rVert \mid
             \{ U_1, \dots, U_n \} \text{ is a measurable partition }
        \right \}
\]
An extremely useful fact is that $\norm{\mu} = \mu^{+}(X) + \mu^{-}(X)$ where
$\mu^{+}$ and $\mu^{-}$ are the positive and negative parts of $\mu$
respectively (see details in~\cite[Section 8.2. and Section
10.10]{aliprantis06}).

We proceed by presenting a simple a metric $\lambda$-theory on which to reason
about predicates and random walks, as previously discussed. Our (only) ground
type will be $\mathtt{real}$ to represent measures over real numbers, \ie\ we
set $\sem{\mathtt{real}} = \Measu(\Reals)$.  Concerning operations we take a
pre-determined set of predicates $p : \mathtt{real} \to \typeI \oplus \typeI$,
whose interpretation takes the form $\sem{p}(\mu) = (\mu(U),
\mu(\overline{U}))$ for some measurable subset of $U \subseteq \Reals$. In
other words $U \subseteq \Reals$ corresponds to the subspace in which the
predicate is supposed to hold. We also take a pre-determined set of actions $a
: \typeI \to (\typeA \multimap \typeA)$ and a pre-determined set of measures $m
: \typeI \to \mathtt{real}$, whose interpretation takes no particular form.
Finally we consider an addition operation $+ : \mathtt{real},\mathtt{real} \to
\mathtt{real}$ whose interpretation is given by $\mu \otimes \nu \mapsto
+_\ast(\mu \otimes \nu)$ where $+_\ast$ is the pushforward measure construction
w.r.t. the measurable map $+ : \Reals \times \Reals \to \Reals$.  Now, given a
measure $m$ and actions $a,b$ consider the following 'abstract' Bernoulli
trial,
\[
        p : \mathtt{real} \multimap \typeI \oplus \typeI
        \vljud \underbrace{ \text{case } p(m) \text{ of } \inl(x) \Rightarrow a(x) ; 
        \inr(y) \Rightarrow b(y)}_{\mathtt{bern}(p)} : \typeA \multimap \typeA
\]
Note that if $p_1(x) =_\epsilon p_2(x)$ then  $\mathtt{bern}(\lambda x. p_1(x))
=_\epsilon \mathtt{bern}(\lambda x. \, p_2(x))$ holds, as per our equational
system. Such is useful to approximate Bernoulli trials that are hard to compute
as the following example illustrates.
\begin{example}[Predicates as limits of Cauchy sequences]
        Consider the predicate,
        \[
                x : \mathtt{real} \vljud
                p_{\frac{1}{2}\sqrt{2}}(x) : \typeI \oplus \typeI
        \]
        that returns true if $x < \frac{1}{2}\sqrt{2}$ and false otherwise.
        Given the irrationality of $\frac{1}{2}\sqrt{2}$ it is natural to
        consider successive approximations $p_{q_n}(x) :  \typeI \oplus \typeI$
        $(n \in \Nats)$ of this predicate in which the condition $x <
        \frac{\sqrt{2}}{2}$ is replaced by $x < q_n$ for $q_n$ a rational
        number. We show next how our framework makes this idea precise. Take
        the sequence of rational numbers $(q_n)_{n \in \Nats}$ defined by,
        \[
                \begin{cases}
                        q_0 & = 1
                        \\
                        q_{n + 1} & = \frac{1}{2} (q_n + \frac{2}{q_n})
                \end{cases}
        \]
        which corresponds to Heron's method for approximating $\sqrt{2}$ (\ie\
        $\lim_{n \to \infty} q_n = \sqrt{2}$). We then postulate as axioms in
        our deductive system that $(p_{\frac{1}{2} q_n}(x))_{n \in \Nats}$ is a
        Cauchy sequence and furthermore that it converges to $p_{\frac{1}{2}
        \sqrt{2}}(x)$.  Such is asserted explicitly by setting,
        \begin{equation}
                \label{eq:sound}
                \begin{cases}
                \forall \epsilon > 0. \, \exists m \in \Nats.
                \, \forall n \geq m. \, p_{\frac{1}{2}q_n}(x) =_\epsilon p_{\frac{1}{2}q_{n+1}} (x)
                & \text{(Cauchy sequence)}
                \\
                \forall \epsilon > 0. \, \exists m \in \Nats.
                \, \forall n \geq m. \, p_{\frac{1}{2}q_n}(x) 
                =_\epsilon p_{\frac{1}{2} \sqrt{2}} (x)
                & \text{(Convergence)}
                \end{cases}
        \end{equation}
        for appropriate choices of $m$ (which in our context is irrelevant to
        detail). Then note that according to our deductive system we have that
        $\mathtt{bern}(\lambda x. \, p_{\frac{1}{2} q_n}(x))_{n \in \Nats}$ is
        a Cauchy sequence that furthermore converges to $\mathtt{bern}(\lambda
        x. \, p_{\frac{1}{2} \sqrt{2}}(x))$. Then next step is to prove
        that our axiomatics is sound.
        \begin{align*}
                \sem{p_{\frac{1}{2} \sqrt{2}}(x)}(\mu)
                & = \left (\mu \left (-\infty, \frac{1}{2} \sqrt{2} \right ), 
                \mu(X) - \mu \left (-\infty, \frac{1}{2}\sqrt{2} \right ) \right )
                &
                \\
                & = \left (\mu \left (\bigcup_{n \in \Nats} 
                        \left (-\infty, \frac{1}{2} q_n \right )\right ), 
                \mu(X) - \mu \left (-\infty, \frac{1}{2}\sqrt{2} \right ) \right )
                & 
                \left \{(\frac{1}{2}q_n)_{n \in \Nats} 
                \nearrow \frac{1}{2}\sqrt{2} \right \}
                \\
                & = \left (\sup_{n \in \Nats} \mu \left ( 
                        \left (-\infty, \frac{1}{2} q_n \right )\right ), 
                \mu(X) - \mu \left (-\infty, \frac{1}{2}\sqrt{2} \right ) \right )
                & 
                \left \{ \text{Measure properties} \right \}
                \\
                & = \left (\lim_{n \to \infty} \mu \left ( 
                        \left (-\infty, \frac{1}{2} q_n \right )\right ), 
                \mu(X) - \mu \left (-\infty, \frac{1}{2}\sqrt{2} \right ) \right )
                & 
                \left \{ \text{Limits coincide with sup. of inc. seq.} \right \}                         \\
                & = \left (\lim_{n \to \infty} \mu \left ( 
                        \left (-\infty, \frac{1}{2} q_n \right )\right ), 
                \lim_{n \to \infty}
                \mu \overline{\left (-\infty, \frac{1}{2}q_n \right )} \right )
                & 
                \left \{ \text{Measure properties} \right \}
                \\
                & = \lim_{n \to \infty} \left (\mu \left ( 
                        \left (-\infty, \frac{1}{2} q_n \right )\right ), 
                \mu \overline{\left (-\infty, \frac{1}{2}q_n \right )} \right )
                & 
                \\
                & = \lim_{n \to \infty}
                \sem{p_{\frac{1}{2}q_n}(x)}(\mu)
                &
        \end{align*}
\end{example}

\begin{example}[From predicates approximation to random walk approximation] 
        Let us now capitalise on the previous example to reason about
        approximations of random walks. First consider the term,
        \[
                (-) \vljud \underbrace{\lambda x_1. \, \dots \, x_k. \, y. \,
                x_1 (\dots (x_k(y)) \dots)}_{\mathtt{sequence_k}}
        \]
        which sequences $k$ terms given as input. Then take the $\lambda$-term
        $\mathtt{sequence_k} \> \mathtt{bern}(\lambda x. \, p(x)) \dots \,
        \mathtt{bern}(\lambda x. \, p(x)) : \typeA \to \typeA$ which
        intuitively represents an abstract random walk of $k$-steps. In order
        to keep our notation simple we will abbreviate this last $\lambda$-term to
        $\mathtt{rwalk}(\lambda x. \, p(x))$.  It follows from our system that
        if $p_1(x) =_\epsilon p_2(x)$ then,
        \[
                \mathtt{rwalk}(\lambda x. p_1(x)) =_{n \cdot \epsilon}
                \mathtt{rwalk}(\lambda x. p_2(x)) 
        \]
        In particular from the previous example we can deduce that
        $\mathtt{rwalk}\left (\lambda x. \, p_{\frac{1}{2} q_n} (x) \right )$
        is a Cauchy sequence that converges to $\mathtt{rwalk}\left (\lambda x.
        \, p_{\frac{1}{2} \sqrt{2}} (x) \right )$.

        As a final illustration of the synergy between syntax and semantics
        that our framework provides, suppose now that the actions $a,b : \typeI
        \to \typeA \multimap \typeA$ involved in $\mathtt{bern}(\lambda x. \,
        p(x))$ are concrete jumps on the real line. To this effect we set the
        interpretations of $a$ and $b$ to be,
        \[
                        \sem{a}(1)(\mu) = +_\ast(\mu \otimes \mathtt{unif}(0,1))
                        \hspace{2cm}
                        \sem{b}(1)(\mu) = +_\ast(\mu \otimes \mathtt{unif}(-1,0))
        \]
        where $\mathtt{unif}(0,1)$ is the uniform distribution on the interval
        $[0,1]$ and analogously for $\mathtt{unif}(-1,0)$.  Operationally $a$
        corresponds to a jump to the right with magnitude between $0$ and $1$,
        and analogously for $b$. Suppose we have another action $c : \typeI \to
        (\typeA \multimap \typeA)$ whose interpretation is that of $a$ except
        for the fact that $\mathtt{unif}(0,1)$ is replaced by
        $\mathtt{unif}(0,1+\delta)$. What will be the effect on the random walk
        when replacing $a$ by $c$? 

        Our approach for answering the previous question starts by
        `decomposing' the actions $a$ and $c$, via the axioms,
        \[
                a(\ast) =_0 \lambda z. \, +(z, \mathtt{unif}(0,1)(\ast))
                \hspace{1cm}
                c(\ast) =_0 \lambda z. \, +(z, \mathtt{unif}(0,1 + \delta)(\ast))
        \]
        whose soundness is straightforward to prove. Note that we are slightly
        abusing notation by using $\mathtt{unif}(0,1)$ (resp.
        $\mathtt{unif}(0,1+\delta)$) both as syntactic and semantic objects.
        The next step is to observe that  one can put an upper bound between
        $a(\ast)$ and $c(\ast)$ via the previous axioms and an upper bound
        between the $\lambda$-terms $\mathtt{unif}(0,1)(\ast)$ and
        $\mathtt{unif}(0,1+\delta)(\ast)$. The latter upper bound can be
        obtained semantically, by computing $\norm{ \mathtt{unif}(0,1) -
        \mathtt{unif}(0,1+\delta) }$ as follows:
        \begin{align*}
        & \norm{ \mathtt{unif}(0,1) - \mathtt{unif}(0,1+\delta) }
        \\
        & =
        \norm { \mathtt{unif}(0,1) - \mathtt{unif}(0,1+\delta) }^+ (\Reals)
        +
        \norm { \mathtt{unif}(0,1) - \mathtt{unif}(0,1+\delta) }^- (\Reals)
        \end{align*}
        and we proceed by computing the left-hand side of the addition,
        \begin{align*} 
        & \norm { \mathtt{unif}(0,1) - \mathtt{unif}(0,1+\delta) }^+ (\Reals)
        \\
        &  = 
        \sup \{ \mathtt{unif}(0,1)(U) - \mathtt{unif}(0,1 + \delta)(U)
        \mid U \subseteq \Reals \}
        \\
        & 
        =
        \sup \{ \mathtt{unif}(0,1)(U \cap [0,1]) 
        - \mathtt{unif}(0,1 + \delta)(U \cap [0,1]) 
        - \mathtt{unif}(0,1 + \delta)(U \cap (1,1 + \delta]) 
        \mid U \subseteq \Reals \}
        \\
        &
        = \sup \left \{ \left (1 - \frac{1}{1+\delta} \right ) 
                \mathtt{unif}(0,1)(U \cap [0,1]) 
        - \mathtt{unif}(0,1 + \delta)(U \cap (1,1 + \delta]) 
        \mid U \subseteq \Reals \right \}
        \\
        & = 1 - \frac{1}{1+\delta} 
        \end{align*}
        Via an analogous reasoning one shows that the right-hand side of the
        addition is equal to $\frac{\delta}{1 + \delta}$ and then proceeds
        syntactically in the following manner.
        \begin{align*}
               \text{case } p(m) \text{ of } \inl(x) \Rightarrow a(x) ; 
               \inr(y) \Rightarrow b(y)
               & =_0
               \text{case } p(m) \text{ of } \inl(x) \Rightarrow x \text{ to} 
               \ast. \, a(\ast) 
               ; 
               \inr(y) \Rightarrow b(y)
               \\
               & =_{2 \cdot {\frac{\delta}{1 + \delta}}}
               \text{case } p(m) \text{ of } \inl(x) \Rightarrow x \text{ to} 
               \ast. \, c(\ast) 
               ; 
               \inr(y) \Rightarrow b(y)
               \\
               & =_0
               \text{case } p(m) \text{ of } \inl(x) \Rightarrow c(x) 
               ; 
               \inr(y) \Rightarrow b(y)
        \end{align*}
        Thus if $\mathtt{rwalk}(\lambda x. p(x))$ is the random walk that
        involves action $a$ and $\mathtt{rwalk'}(\lambda x. p(x))$ the random
        walk that involves action $c$ we deduce from the framework the metric
        equation,
        \[
                \mathtt{rwalk}(\lambda x. p(x)) =_{2n \cdot \frac{\delta}{1 + \delta}}
                \mathtt{rwalk'}(\lambda x. p(x)) 
        \]
        which will tend converge to $0$ as $\delta$ tends to $0$.
\end{example}

\section{More content}

Let $\mathcal{V}$ denote a commutative and unital quantale, $\otimes :
\mathcal{V} \times \mathcal{V} \to \mathcal{V}$ the corresponding binary
operation, and $k$ the corresponding unit~\cite{paseka00}.  We start by
recalling two definitions concerning ordered
structures~\cite{GHK+03,JGL-topology}. 

\begin{definition}
	Consider a complete lattice $L$.  For every $x, y \in L$ we say that
	$y$ is \emph{way-below} $x$ (in symbols, $y \ll x$) if for every
	subset $X \subseteq L$ whenever $x \leq \bigvee X$ there exists a
	\emph{finite} subset $A \subseteq X$ such that $y \leq \bigvee A$.
	The lattice $L$ is called \emph{continuous} iff for every $x \in L$,
	\begin{flalign*}
		x = \bigvee \{ y  \mid y \in L\ \text{and} \ y \ll x \}
	\end{flalign*}
\end{definition}

\begin{definition}
	Let $L$ be a complete lattice. A \emph{basis} $B$ of $L$ is a subset
	$B \subseteq L$ such that for every $x \in L$ the set
	$B \cap \{ y \mid y \in L\ \text{and} \ y \ll x \}$ is directed and
	has $x$ as the least upper bound.
\end{definition}
We assume that the underlying lattice of $\mathcal{V}$ is continuous and has a
basis $B \ni k$ closed under finite joins and the multiplication of the
quantale $\otimes$. As explained in~\cite{dahlqvist22}, these assumptions allow
us to work \emph{only} with a specified subset of $\mathcal{V}$-equations
chosen \eg for computational reasons. We also assume that $\mathcal{V}$ is
\emph{integral}, \ie that the unit $k$ is the top element of $\mathcal{V}$, a
common assumption in quantale theory~\cite{velebil19}. Next we present two
examples of quantales obeying these conditions. Other examples can be found
in~\cite{dahlqvist22}.

\begin{example}
        The Boolean quantale $((\{0 \leq 1\}, \vee), \otimes := \wedge)$ is
        \emph{finite} and thus continuous~\cite{GHK+03}. Since it is
        continuous, $\{0,1\}$ itself is a basis for the quantale that satisfies
        the conditions above. For the metric quantale (also known as Lawvere
        quantale) $(([0,\infty], \wedge), \otimes := +)$, the way-below
        relation corresponds to the \emph{strictly greater} relation with
        $\infty > \infty$, and a basis for the underlying lattice that
        satisfies the conditions above is the set of extended non-negative
        rational numbers. 
\end{example}


\begin{definition}[$\mathcal{V} \lambda$-theories]\label{defn:theory}
  Consider a tuple $(G,\Sigma)$ consisting of a class $G$ of ground types and a
  class $\Sigma$ of sorted operation symbols.  A \emph{$\mathcal{V}
  \lambda$-theory} $((G,\Sigma),Ax)$ is a triple such that $Ax$ is a class of
  $\mathcal{V}$-equations-in-context over graded $\lambda$-terms built from
  $(G,\Sigma)$.
\end{definition}
The elements of $Ax$ are called the \emph{axioms} of the theory. 

\begin{definition}
  \label{defn:vcat}
  A (small) $\mathcal{V}$-category is a pair $(X,a)$ where $X$ is a
  class (set) and $a : X \times X \to \mathcal{V}$ is a function that
  satisfies:
  \begin{flalign*}
    k \leq a(x,x) \qquad \text{ and }  \qquad
    a(x,y) \otimes a(y,z) \leq a(x,z) \hspace{2cm}
    (x,y,z \in X)
  \end{flalign*}
  For two $\mathcal{V}$-categories $(X,a)$ and $(Y,b)$, a
  $\mathcal{V}$-functor $f : (X,a) \to (Y,b)$ is a function
  $f : X \to Y$ that satisfies the inequality
  $a(x,y) \leq b(f(x),f(y))$ for all $x,y \in X$.
\end{definition}
Small $\mathcal{V}$-categories and $\mathcal{V}$-functors form a
category which we denote by $\VCat$.  A $\mathcal{V}$-category $(X,a)$
is called \emph{symmetric} if $a(x,y) = a(y,x)$ for all $x,y \in
X$. We denote by $\VCatSy$ the full subcategory of $\VCat$ whose
objects are symmetric. Every $\mathcal{V}$-category carries a natural
order defined by $x \leq y$ whenever $k \leq a(x,y)$. A
$\mathcal{V}$-category is called \emph{separated} if its natural order
is anti-symmetric. We denote by $\VCatSe$ the full subcategory of
$\VCat$ whose objects are separated.

\begin{example}
  For $\mathcal{V}$ the Boolean quantale, $\VCatSe$ is the category $\Pos$ of
  partially ordered sets and monotone maps;  $\VCatSS$ is simply the category
  $\catfont{Set}$ of sets and functions.  For $\mathcal{V}$ the metric quantale,
  $\VCatSS$ is the category $\Met$ of metric spaces and non-expansive maps.
  Other examples can be found for example in~\cite{dahlqvist22}.
\end{example}

The inclusion functor $\VCatSe \hookrightarrow \VCat$ has a left adjoint
\cite{hofmann20}. It is constructed first by defining the equivalence relation
$x \sim y$ whenever $x \leq y$ and $y \leq x$ (for $\leq$ the natural order
introduced earlier). Then this relation induces the separated
$\mathcal{V}$-category $(X/_\sim, \tilde a)$ where $\tilde a$ is defined as
$\tilde a([x],[y]) = a(x,y)$ for every $[x],[y] \in X/_\sim$. The left adjoint
of the inclusion functor $\VCatSe \hookrightarrow \VCat$ sends every
$\mathcal{V}$-category $(X,a)$ to $(X/_\sim, \tilde a)$.  Next, note that every
category $\VCat$ is autonomous with the tensor $(X,a) \otimes (Y,b) := (X
\times Y, a \otimes b)$ where $a\otimes b$ is defined as $(a \otimes b)((x,y),
(x',y')) = a(x,x') \otimes b(y,y')$ and the set of $\mathcal{V}$-functors
$\VCat((X,a),(Y,b))$ is equipped with the map,
\[
(f,g) \mapsto \bigwedge_{x \in X} b(f(x),g(x))
\]
It is also known that the categories $\VCatSy$, $\VCatSe$, and $\VCatSS$
inherit the autonomous structure of $\VCat$ whenever $\mathcal{V}$ is
integral~\cite{dahlqvist22}.

\begin{definition}\label{defn:enr_aut}
  A $\VCat$-enriched autonomous category $\catC$ is an autonomous and
  $\VCat$-enriched category $\catC$ such that the bifunctor $\otimes : \catC
  \times \catC \to \catC$ is a $\VCat$-functor and the adjunction $(- \otimes
  X) \dashv (X \multimap -)$ is a $\VCat$-adjunction.  We obtain analogous
  notions of enriched autonomous category by replacing $\VCat$ (as basis of
  enrichment) with $\VCatSe$, $\VCatSy$, or $\VCatSS$.
\end{definition}

\begin{theorem}
        The categories $\VCatSe$ and $\VCatSS$ have $(\VCat)$-enriched binary
        coproducts.
\end{theorem}

\begin{proof}
        Given two $\mathcal{V}$-categories $(X,a_X)$ and $(Y,a_Y)$ the carrier of
        the coproduct is given by $X + Y$ with $a_{X + Y}$ given by,
        \[
                \begin{cases}
                        a_{X + Y}(\inl(x_1),\inr(x_2)) = a_X(x_1,x_2) \\
                        a_{X + Y}(\inl(y_1),\inr(y_2)) = a_Y(y_1,y_2) \\
                        a_{X + Y}(\inl(x), \inr(y)) = 
                        a_{X + Y}(\inr(y), \inl(x))  = \bot
                \end{cases}
        \]
        and the co-pairing is defined as in $\catfont{Set}$.
        Concerning the enrichment we calculate,
        \begin{align*}
                a([f,g], [f',g']) & = \bigwedge_{z \in X + Y}
                b([f,g](z), [f',g'](z)) \\
                                  & = 
                                  \bigwedge_{x \in X} 
                                  b([f,g](\inl(x)),[f',g'](\inl(x)))
                                  \wedge
                                  \bigwedge_{y \in Y}
                                  b([f,g](\inr(y)),[f',g'](\inr(y)))
                                  \\
                                  & =
                                  \bigwedge_{x \in X} 
                                  b(f(x)),f'(x))
                                  \wedge
                                  \bigwedge_{y \in Y}
                                  b(g(y),g'(y))
                                  \\
                                  & = a(f,f') \wedge b(g,g')
        \end{align*}
\end{proof}


\bibliographystyle{alpha} 
\bibliography{biblio}


\end{document}
