\documentclass[a4paper, 11pt]{article}
\usepackage{comment} % enables the use of multi-line comments (\ifx \fi) 
\usepackage{lipsum} %This package just generates Lorem Ipsum filler text. 
\usepackage{fullpage} % changes the margin
\usepackage{libertine}
\linespread{1.1}
\begin{document}
%Header-Make sure you update this information!!!!
\noindent
\large\textbf{Thesis Proposal} \hfill \textbf{Renato Neves} \\
Date: \today

\section*{A Metric Equational System for Quantum Computation}

\subsection*{Motivation}

The notion of program equivalence and underlying theories typically hinges on
the idea that equivalence is binary, \emph{i.e.} programs are either equivalent
or they are not~\cite{winskel93}. Whilst in the case of classical programming
such is usually fine, in other computational paradigms it is a too
coarse-grained perspective~\cite{dahlqvist23b}. This is manifested in quantum
computation at multiple fronts \cite{han19,dahlqvist23b,watrous18}. For
example, the fact that one works only with a finite set of basic quantum gates,
and that these are used to approximate the remaining ones, already strongly
suggests the development of appropriate notions of \emph{approximate} program
equivalence~\cite{dahlqvist23b}. Also the central issue of
quantum decoherence calls for such notions, for it is unreasonable to expect
that our idealised quantum algorithm will run perfectly on a quantum device.
What we observe is instead a mere approximation~\cite{han19,preskill18}.  Yet
another example is found in the day-to-day substitution of `costly' quantum
operations by less costly ones, which although not the same are regarded as
being approximately the same.

A notion of approximate equivalence for quantum programming was introduced and
briefly studied in~\cite{dahlqvist23b}. Technically it uses linear
$\lambda$-calculus as basis which has well-known deep connections to both logic
and category theory~\cite{girard95,benton94}. It then integrates the notion of
approximate equivalence in the calculus via the so-called diamond norm, which
induces a metric (roughly, a distance function) on the space of quantum
programs (seen as completely positive trace-preserving
super-operators)~\cite{watrous18}. Some positive results were achieved in this
setting but much remains to be done, as detailed next.

\subsection*{Goals}

\dots 
\subsection*{Research Plan}

The concurrent quantum language adopted for this project is an \emph{extension}
of the imperative concurrent language proposed in~\cite{brookes96} with basic
quantum features, more specifically a set of unitary gates and measurements.
The semantics of the concurrent language without quantum features is also
present in~\cite{brookes96}. We will start by first implementing this simpler
semantics in \textsc{Haskell}.  The next step is to study the quantum
simulation library for \textsc{Haskell} presented in~\cite{altenkirch10}, which
we will use for simulating programs written in our concurrent quantum
language.

We will then move on to the main goal of the project: to implement a semantics
for the aforementioned concurrent quantum language. Next we will explore the
use of concurrency, w.r.t. the reduction of superposition times, in different
famous quantum algorithms/protocols.  The final step of the project is to write
a report detailing the conclusions obtained throughout the work. The research
plan is summarised in the table below. 

As an additional assurance that the project will proceed according to plan, a Ph.D.
student supervised by professors Renato Neves and Luís Barbosa will act as
co-supervisor as well. This student is one of the main contributors of the
quantum concurrent language that is object of implementation in this project,
and is therefore a valuable element of the supervising team. 

\begin{center}
\begin{tabular}{ | c | c | c | c | c | c | c | c | c | c | c | }
  \hline
  Task & Oct & Nov & Dez & Jan & Fev & Mar & Apr & May & Jun & Jul \\
  \hline
  Implementation of~\cite{brookes96} & X & X &  & & & & & & & \\
  \hline
  Study of Quantum Concurr.  & & & X & X  & & & & & & \\
  \hline
  Implementation & & & & X  & X & X & & & & \\
  \hline
  Case-studies & & & &  & & & X & X & & \\
  \hline
  Dissertation & & & &  & & & & & X & X \\
  \hline
\end{tabular}
\end{center}


%% Bibliography
\bibliographystyle{alpha}
\bibliography{biblio}

\end{document}
