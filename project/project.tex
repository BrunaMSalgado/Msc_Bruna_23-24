\documentclass[a4paper, 11pt]{article}
\usepackage{comment} % enables the use of multi-line comments (\ifx \fi) 
\usepackage{lipsum} %This package just generates Lorem Ipsum filler text. 
\usepackage{fullpage} % changes the margin
\usepackage{libertine}
\linespread{1.1}
\begin{document}
%Header-Make sure you update this information!!!!
\noindent
\large\textbf{Thesis Proposal} \hfill \textbf{Renato Neves} \\
Date: \today

\section*{A Metric Equational System for Quantum Computation}

\subsection*{Motivation and Context}

Program equivalence and underlying theories typically hinge on the idea that
equivalence is binary, \emph{i.e.} two programs are either equivalent or they
are not~\cite{winskel93}. Whilst in the case of classical programming this is
usually fine, in other computational paradigms it is a too coarse-grained
perspective~\cite{dahlqvist23b}.  Specifically in quantum computation it
manifests itself at multiple fronts \cite{han19,dahlqvist23b,watrous18}. For example
the fact that one works only with a \emph{finite} set of basic quantum gates, and that
these are used to approximate the remaining ones, already strongly suggests the
development of appropriate notions of \emph{approximate} program
equivalence~\cite{dahlqvist23b}. Also the central issue of quantum decoherence
calls for such notions, for it is unreasonable to expect that our idealised
quantum algorithm will run perfectly on a quantum device --  what we will
observe is instead a mere approximation~\cite{han19,preskill18}.  Yet another
example is found in the day-to-day substitution of `costly' quantum operations
by less costly ones: the latter might not be equivalent to the original ones
but are informally regarded as being `almost' equivalent.

A notion of approximate equivalence for quantum programming was (briefly)
studied in~\cite{dahlqvist23}. Technically it uses linear $\lambda$-calculus as
basis -- \emph{i.e.} programs are written as linear $\lambda$-terms -- which has
 deep connections to both logic and category
theory~\cite{girard95,benton94}. A notion of approximate equivalence is then
integrated in the calculus via the so-called \emph{diamond norm}, which induces a
metric (roughly, a distance function) on the space of quantum programs (seen
semantically as completely positive trace-preserving
super-operators)~\cite{watrous18}. Some positive results were achieved in this
setting but much remains to be done.

\subsection*{Goals}

The notion of approximate equivalence for quantum programming explored
in~\cite{dahlqvist23} does not take important operations into account.
Specifically the corresponding mathematical model does not include
measurements, nor classical control flow, nor discard operations. Also the
corresponding typing system is often times too strict, and cannot properly
handle multiple uses of the same resource, such as sampling exactly $n$-times
from a distribution.  The overarching goal of this M.Sc. project is to tackle
the aforementioned limitations.

A successful completion of this goal will provide a fully-fledged quantum
programming language on which to study metric program equivalence in various
scenarios. This includes not only quantum algorithmics -- where for example the
number of iterations in Grover's algorithm involves approximations -- but also
in quantum information theory, where for example quantum teleportation and the
problem of the discrimination of quantum states have important
rôles~\cite{nielsen_chuang_2010}.

\subsection*{Research Plan}

The first three months of this project are devoted to a background study on the
topics of programming theory, $\lambda$-calculus, and (graded) typing systems
that are suited to the use of a resource multiple
times~\cite{winskel93,dahlqvist23,dahlqvist23b}. The next three months are
allocated to extending the quantum model in~\cite{dahlqvist23} with
measurement, classical control flow, and also discard operations. The subsequent two
months will be dedicated to enriching the typing system in~\cite{dahlqvist23}
so it can properly support multiple uses of the same resource. Finally the
following two months will be devoted to writing the dissertation. The table below
summarises the work plan just described. Throughout the whole project we will
use a number of simple case-studies to illustrate and benchmark the prospective
results.

\begin{center}
\begin{tabular}{ | c | c | c | c | c | c | c | c | c | c | c | }
  \hline
  Task & Oct & Nov & Dez & Jan & Fev & Mar & Apr & May & Jun & Jul \\
  \hline
  Background Study & X & X & X & & & & & & & \\
  \hline
  New Quantum Operations  & & & & X  & X & X & & & & \\
  \hline
  Enriched Typing System & & & &  & & & X & X & & \\
  \hline
  Dissertation & & & &  & & & & & X & X \\
  \hline
\end{tabular}
\end{center}


%% Bibliography
\bibliographystyle{alpha}
\bibliography{biblio}

\end{document}
