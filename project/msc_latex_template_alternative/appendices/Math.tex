\chapter{Mathematical backgound} \label{ch:math_back}

\section{Equivalence Relations and Quotients in Sets}

\begin{definition}
  A relation $\sim$ on a set \( S \) is an \emph{equivalence relation} if it is
\begin{itemize}
  \item reflexive: for all \( x \in S \), \( x \sim x \),
  \item symmetric: for all \( x, y \in S \), if \( x \sim y \) then \( y \sim x \), and
  \item transitive: for all \( x, y, z \in S \), if \( x \sim y \) and \( y \sim z \), then \( x \sim z \).
\end{itemize} 
\end{definition}

\begin{definition}
  Given an equivalence relation on a set \( S \), we can describe the so-called \emph{equivalence classes}. If \( s \in S \), then the \emph{equivalence class} of \( s \) is the set of all elements related to it:
\[
[s] = \{ r \in S \mid r \sim s \}.
\]
That is, \( [s] \) is the set of all elements that are considered “the same” as \( s \) under the relation \( \sim \). For a given set \( S \) and an equivalence relation \( \sim \) on \( S \), we define the \emph{quotient set}, denoted \( S /\sim \), whose elements are all the equivalence classes of elements in \( S \). 
Observe that the quotient mapping
\[
q : S \longrightarrow S/\!\sim,
\]
which takes an element \( s \in S \) to its equivalence class \( [s] \), has the property that a map \( f : X \to Y \) extends along \( q \),
\[
\begin{tikzpicture}
  \matrix (m) [matrix of math nodes,row sep=4em,column sep=7em,minimum width=2em]
  {
   S &  S/\!\sim   \\
      & P \\
  };
  \path[-stealth]
    (m-1-1) edge  node [above] {$q$} (m-1-2)
    (m-1-1) edge  node [above] {$f$} (m-2-2)
    (m-1-2) edge [dotted]  node [right] {} (m-2-2);
    ;
\end{tikzpicture}
\]
\end{definition}
 just in case $f$ respects the equivalence relation, in the sense that $s \sim p$ implies
 $f(s)=f(p)$.


For instance, consider a set of cars \( S \). We can define an equivalence relation on \( S \) by grouping cars according to their colour. This results in subsets such as the set of blue cars, the set of red cars, the set of green cars, and so on — these subsets are the \emph{equivalence classes}. Moreover, the collection of all such equivalence classes forms a new set, called the \emph{quotient set}.

\section{Category theory} \label{sec:catgories}


Category theory originated as an effort to connect and unify two distinct areas of mathematics.  The goal was to study and classify specific geometric structures—such as topological spaces, manifolds, and bundles—by associating them with corresponding algebraic structures like groups, rings, and abelian groups. It became clear that a language was needed to connect geometric and algebraic objects—one not explicitly tailored to geometry or algebra. Only a language of such generality could allow meaningful discussion across both fields.  This is the birth of category theory.
Described as ``a language about nothing, and therefore about everything'' category theory provides a highly general way of discussing mathematical concepts.  It was invented by Samuel Eilenberg and Saunders MacLane \cite{eilenbergGeneralTheoryNatural1945}. They organized various mathematical structures into categories called geometric and algebraic. To connect these categories, they defined functors, which map objects and morphisms from one category to another, much like functions do. They further introduced natural transformations, which provide a way to compare functors, translating the results of one functor into those of anothers. \cite{yanofskyMonoidalCategoryTheory2024}

 As previously mentioned, in light os its deep connection with lambda calculus and its capacity to encompass multiple ``perspectives'', thereby broadening the applicability of results, we adopt a categorical interpretation.



\subsection{Categories}

\begin{definition}
   A \emph{category} $\catC$ consists of
   \begin{itemize}
    \item a collection of objects $A, B, C, \ldots$, denoted $|\catC|$ or $\text{Obj}(\catC)$;
    \item for every two pairs of objects $A$ and $B$, a collection of morphisms $f, g, \ldots $, usually denoted $\catC(A,B)$, $\mathrm{Hom}_{\catC}(A,B)$, or $\mathrm{Hom}(A,B)$ if there is no ambiguity. 
   \end{itemize}
    The collection for morphisms has the following structure:
    \begin{itemize}
      \item Each morphism has a specified domain($A$) and codomain($B$) and the notation $ f : A \to B $ indicates that $ f $ is a morphism from object $ A $ to object $ B $.  
       \item Every object $ A $ has an identity morphism $ \id_A : A \to A $.  
      \item For any pair of morphisms $ f : A \to B $ and $ g : B \to C $, there exists a composite morphism $ g \circ f : X \to Z $. We will also write $g \circ f$ as $f \cdot g$ or simply $f g$.
    \end{itemize}

     The composition is required to satisfy the two following laws: if $f : A \to B, g : B \to C,$ and $h:C \to D$ are morphisms, then
     \begin{itemize}
      \item  $f \circ \id_A = f = \id_B \circ f$;
      \item  $  (f \circ g) \circ h = f \circ (g \circ h) $.
     \end{itemize}
\end{definition}




%Sets

\begin{example}
 $\catSet$ is the category whose objects are sets and whose  morphisms are functions between them. Given a function \(f: A \to B\), it assigns to each element \(a \in A\) a unique element \(f(a) \in B\). 
 For any two functions \(f: A \to B\) and \(g: B \to C\), their composition is defined by
$$
(g \circ f)(a) = g(f(a)) \quad \text{for all } a \in A.
$$
This composition is \emph{associative}. That is, for any further function \(h: C \to D\), we have
\[
(h \circ g) \circ f = h \circ (g \circ f),
\]
since for every \(a \in A\),
\[
((h \circ g) \circ f)(a) = h(g(f(a))) = (h \circ (g \circ f))(a).
\]

Moreover, for every set \(A\), there exists an \emph{identity function}
\[
\id_A : A \to A, \quad \text{defined by } \id_A(a) = a,
\]
which satisfies the unit laws for composition:
\[
f \circ \id_A = f \quad \text{and} \quad \id_B \circ f = f
\]
for any function \(f: A \to B\).

Therefore, \(\catSet\), with sets as objects and functions as morphisms, satisfies the axioms of a category.
\end{example}

Another common type of example consists of categories of sets equipped with additional structure, along with functions that preserve that structure.


\begin{definition}
A \emph{partially ordered set} or \emph{partial order} is a set $A$ equipped with a binary relation $\leq_A$ satisfying the following properties for all $a, b, c \in A$:
\begin{itemize}
    \item Reflexivity: $a \leq_A a$;
    \item Transitivity: If $a \leq_A b$ and $b \leq_A c$, then $a \leq_A c$;
    \item Antisymmetry: If $a \leq_A b$ and $b \leq_A a$, then $a = b$.
\end{itemize}
\end{definition}

\begin{example}
The set of real numbers \(\mathbb{R}\), equipped with the usual ordering \(\leq\), forms a poset. Moreover, it is \emph{linearly ordered} (or \emph{totally ordered}), since for any \(x, y \in \mathbb{R}\), either \(x \leq y\) or \(y \leq x\) holds.
\end{example}

\begin{example}
  Each partially ordered set naturally defines a category. Let \( (P, \leq) \) be a poset. We define a category \( \catfont{B(P, \leq)} \), often denoted simply by \( \catfont{B(P)} \) or even \( \catfont{P} \), where the objects are the elements of \( P \), and there is a unique morphism \( p \to q \) if and only if \( p \leq q \). The reflexivity of the order \( \leq \) ensures the existence of identity morphisms, while transitivity guarantees that morphisms compose appropriately. Moreover, since there is at most one morphism between any two objects, composition is trivially associative. 
\end{example}

\begin{definition}
Given two partial orders \((A, \leq_A)\) and \((B,\leq_B)\), a function \(m: A \to B\) is called a \emph{monotone map} (or \emph{order-preserving map}) if for all \(a, a' \in A\),
\[
a \leq_A a' \quad \Rightarrow \quad m(a) \leq_B m(a').
\]
\end{definition}



\begin{example}
  $\catPO$ is the category of all partial orders and all monotone maps. First, for any poset \(A\), the identity function \(\id_A : A \to A\) is monotone. Indeed, for all \(a \in A\),
\[
a \leq_A a \quad \Rightarrow \quad \id_A(a) \leq_A \id_A(a).
\]

Next, given monotone maps \(f : A \to B\) and \(g : B \to C\), their composition \(g \circ f : A \to C\) is also monotone. For all \(a, a' \in A\), if \(a \leq_A a'\), then
\[
f(a) \leq_B f(a') \quad \text{and} \quad g(f(a)) \leq_C g(f(a')),
\]
so it follows that
\[
(g \circ f)(a) \leq_C (g \circ f)(a').
\]

\end{example}



\begin{example}
 $\catVect$ is the category of finite complex vector spaces and linear mappings.
\end{example}

\begin{comment}
\begin{example}
Given a functional programming language $L$, we can define a category $\catCompFunc$.  In this category, the objects represent the data types of the language \( L \), and the morphisms correspond to computable functions or programs. A function is considered \emph{computable} if a computer program is capable of executing that function.

The composition of morphisms is defined as follows: given two morphisms \( f \colon X \to Y \) and \( g \colon Y \to Z \), the composition \( g \circ f \colon X \to Z \) is defined by applying \( g \) to the output of \( f \). This composition is often written as \( f;g \).

Additionally, the identity morphism \( \id_X \colon X \to X \) represents the ``identity program," which returns its input without making any changes (i.e., it "does nothing").
\end{example}
\end{comment}


\begin{definition} 
 A morphism $f : A \to B$  in a category $\catC$ be a category is called an \emph{isomorphism} if there exists a morphism $f^{-1} : B \to A$ such that
\[
f^{-1} \circ f = \id_A \quad \text{and} \quad f \circ g = \id_B.
\]
In this case, $f^{-1}$ is called the \emph{inverse} of $f$, and it is unique. If such an isomorphism exists, we say that $A$ and $B$ are \emph{isomorphic}, written
$A \cong B.$
\end{definition}

One of the central ideas in category theory is \emph{duality}. Simply put, for a given definition of a structure, there is often a corresponding dual concept obtained by reversing the directions of all the morphisms. 

\begin{definition} 
Let \(\catC\) be a category. The \emph{opposite category}, denoted \(\catC^{\mathrm{op}}\), is defined as follows:
\begin{itemize}
  \item The objects of \(\catC^{\mathrm{op}}\) are the same as those of \(\catC\).
  \item For any pair of objects \(A, B\), the hom-set in \(\catC^{\mathrm{op}}\) is defined by
  \[
  \textit{Hom}_{\catC^{\mathrm{op}}}(A, B) = \textit{Hom}_{\catC}(B, A),
  \]
  that is, each morphism \(f: A \to B\) in \(\catC^{\mathrm{op}}\) corresponds to a morphism \(f: B \to A\) in \(\catC\).
  \item Composition in \(\catC^{\mathrm{op}}\) is defined using the composition in \(\catC\), but in reverse order. That is, if
  \[
  A \xrightarrow{ \quad f \quad } B \xrightarrow{\quad g \quad } C
  \]
  are morphisms in \(\catC^{\mathrm{op}}\), corresponding to morphisms
  \[
  C \xrightarrow{ \quad g \quad} B \xrightarrow{ \quad f \quad} A
  \]
  in \(\catC\), then the composition in \(\catC^{\mathrm{op}}\) is defined by
  \[
  g \circ f := f \circ_{\catC} g.
  \]
\end{itemize}

Thus, \(\catC^{\mathrm{op}}\) reverses the direction of morphisms and composition while retaining the same collection of objects.
\end{definition}

\begin{definition}
  A subcategory \( \catD \) of a category \( \catC \) is a category such that 
  \begin{itemize}
    \item All the objects of \( \catD \) are objects of \( \catC \);
    \item  For any objects \( A \) and \( B \) in \( \catD \), we have \( \mathrm{Hom}_{\catD}(A, B) \subseteq \mathrm{Hom}_{\catC}(A, B) \).
    \item The indentities in $\catD$ are those of $\catC$ and the composition in $\catD$ is the respective restriction relative to $\catC$.
  \end{itemize}
  
\end{definition}

\begin{example}
  The category $\catFinSet$, whose objects are finite sets and whose morphisms are functions between them, forms a subcategory of the category $\catSet$.
\end{example}


\begin{definition}
  A category is called \emph{small} if both its collection of objects and its collection of morphisms form sets.
A category is called \emph{locally small} if, for every pair of objects, the corresponding hom-set is a set.
\end{definition}


% Partial order
%Vect
%Banach spaces and short maps
%CPTP
%Linguagens de programação


%We say that a diagram commutes when for every two vertices X,Y in the diagram, all the paths from X to Y (following arrows) yield equal morphisms. 

%c_op
%iso

\subsection{Products and coproducts}

 A category frequently possesses a more intricate structure than a mere collection of objects and their morphisms. The existence of particular relationships among certain objects and morphisms can give some objects important properties.

A diagram is said to commute if, for every pair of objects $A$ and $B$ in the diagram, all directed paths from 
$A$ to $B$ yield equal morphisms.

\begin{definition}
  An object \( 0 \) in a category \( \catC \) is called an \emph{initial object} if for every object \( A \in \catC  \), there exists a unique morphism  $f: 0 \to A $.

\end{definition}

\begin{definition}
  An object \( 1 \) in a category \( \catC  \) is called a \emph{terminal object} if for every object \( A \in \catC  \), there exists a unique morphism $ f: A \to 1 $ .
\end{definition}

\begin{example}
In the category \( \catSet \), the empty set \( \emptyset \) is an initial object, since for any set \( S \), there exists a unique function$f : \emptyset \to S.$
This function is unique because there are no elements in \( \emptyset \) to map to.

Any singleton set, such as \( \{*\} \) or \( \{a\} \), is a terminal object in this category. For any set \( S \), there exists a unique function $f : S \to \{*\}$,
which maps every element of \( S \) to the sole element of the singleton set
\end{example}

\begin{example}
  Let $(P, \leq)$ be a partial order and $\catfont{P}$ be its associated
 category.
  Here, the initial object is the \emph{bottom element}—an element that is less than or equal to every other element in $P$. The terminal object in $\catP$ is the \emph{top element}—an element that is greater than or equal to every other element in $P$.
\end{example}



\begin{definition}
  Consider a category $\catC$.  We say that it has (binary) products if for any
objects $A$ and $B$ in $\catC$ there also exists an object $A \times B$ in
$\catC$ with morphisms $\pi_A : A \times B \to A$ and $\pi_B :  A \times B \to  B$
that satisfy a certain universal property: specifically for every two morphisms
$f  : C \to A$ and $g : C \to B$ there exists a \emph{unique} morphism $\langle f,g \rangle :
C \to A \times B $ called \emph{pairing} that makes the diagram below commute.
\[
\begin{tikzpicture}
  \matrix (m) [matrix of math nodes,row sep=4em,column sep=7em,minimum width=2em]
  {
   & C &  \\
    A  & A \times B & B\\
  };
  \path[-stealth]
    (m-1-2) edge  node [above] {$f$} (m-2-1)
    (m-1-2) edge  node [above] {$g$} (m-2-3)
    (m-2-2) edge  node [below] {$\pi_A$} (m-2-1)
    (m-2-2) edge  node [below] {$\pi_B$} (m-2-3)
    (m-1-2) edge [dotted]  node [right] {$\langle f,g \rangle$} (m-2-2);
    ;
\end{tikzpicture}
\]
\end{definition}

  \begin{definition}
Let \( A \times B \) be a product of objects \( A \) and \( B \), and let \( A' \times B' \) be a product of objects \( A' \) and \( B' \) in a category $\catC$. Suppose we are given morphisms \( f : A \to A' \) and \( g : B \to B' \). 
Then there exists a unique morphism
\[
f \times g : a \times b \to a' \times b'
\]
such that the following diagram commutes.
\[
\begin{tikzpicture}
  \matrix (m) [matrix of math nodes,row sep=4em,column sep=7em,minimum width=2em]
  {
   A & A \times B & B \\
    A'  & A' \times B' & B'\\
  };
  \path[-stealth]
    (m-1-2) edge  node [below] {$\pi_A$} (m-1-1)
    (m-1-2) edge  node [below] {$\pi_B$} (m-1-3)
    (m-1-1) edge  node [left] {$f$} (m-2-1)
    (m-1-3) edge  node [right] {$g$} (m-2-3)
    (m-2-2) edge  node [below] {$\pi_A'$} (m-2-1)
    (m-2-2) edge  node [below] {$\pi_B'$} (m-2-3)
    (m-1-2) edge [dotted]  node [right] {$ f\times g$} (m-2-2);
    ;
\end{tikzpicture}
\]
This induced morphism \( f \times g \) is called the \emph{product of the morphisms} \( f \) and \( g \), and it is given explicitly by
\[
f \times g = \langle f \circ \pi_A,\, g \circ \pi_B \rangle.
\]
\end{definition}


\begin{theorem} 
  Let \( A \times B \) be the product of objects \( A \) and \( B \) in a category $\catC$. For any object $C$ and morphisms \( f : C \to A \), \( g : C \to B \), \( h : D \to C \) are morphisms, it holds that:
\[
\langle f \circ h,\, g \circ h \rangle = \langle f, g \rangle \circ h.
\]
\end{theorem}

\begin{proof}

  The universal property of the product induces a unique morphism \( \langle f, g \rangle : C \to A \times B \) such that
$\pi_A \circ \langle f, g \rangle = f \quad \text{and} \quad \pi_B \circ \langle f, g \rangle = g.$
Now, let \( h : D \to C \) be another morphism. Then the compositions \( f \circ h : D \to A \) and \( g \circ h : D \to B \) also induce a unique morphism \( \langle f \circ h,\, g \circ h \rangle : D \to A \times B \) by the universal property of the product. As a result, the following diagram commutes by the universal property of the product.

\[
\begin{tikzpicture}
  \matrix (m) [matrix of math nodes,row sep=3em,column sep=4em,minimum width=8em]
  {
    & D &  \\
    & C &  \\
     & A \times B &\\
    A &  & B\\
  };
  \path[-stealth]
    (m-1-2) edge  node [right] {$h$} (m-2-2)
    (m-1-2) edge  node [left] {$f \circ h$} (m-4-1)
    (m-1-2) edge  node [right] {$g \circ h$} (m-4-3)
    (m-1-2) edge [dotted] [bend left=-20] node [left] {$\langle f \circ h,g \circ h \rangle$} (m-3-2)
    (m-2-2) edge [dotted]  node [right] {$\langle f,g \rangle$} (m-3-2)
    (m-2-2) edge  node [right] {$f$} (m-4-1)
    (m-2-2) edge  node [right] {$g$} (m-4-3)
    (m-3-2) edge  node [below=0.1cm] {$\pi_A$} (m-4-1)
    (m-3-2) edge  node [below=0.1cm] {$\pi_B$} (m-4-3)
    %(m-1-2) edge  node [above] {$f$} (m-2-1)
    %(m-1-2) edge  node [above] {$g$} (m-2-3)   
    %(m-1-2) edge [dotted]  node [right] {$\langle f,g \rangle$} (m-2-2);
    ;
\end{tikzpicture}
\]

\end{proof}


% exemplos

\begin{example}
In the category $\catSet$, the product of two sets $A$ and $B$ is given by their Cartesian product, denoted as
\[
A \times B = \{(a, b) \mid a \in A,\ b \in B\}.
\]
The projection maps are defined by
\[
\pi_A(a, b) = a \quad \text{and} \quad \pi_B(a, b) = b.
\]
Given a set $C$ and morphisms $f: C \to A$ and $g: C \to B$, their pairing is the map
\[
\langle f, g \rangle(c) = (f(c), g(c)).
\]
\end{example}

\begin{example}
  Let $(P, \leq)$ be a partial order and $\catfont{P}$ be its associated category.  Consider a product of elements  \( p \times q \in P\). Then, by definition, there must exist projections satisfying
\[
p \times q \leq p \quad \text{and} \quad p \times q \leq q.
\]
Furthermore, for any element \( x \in P \), if
\[
x \leq p \quad \text{and} \quad x \leq q,
\]
then it follows that
\[
x \leq p \times q.
\]
This operation \( p \times q \) corresponds to what is commonly known as the \emph{greatest lower bound} or \emph{meet}, and is typically denoted by \( p \wedge q \).
\end{example}


\begin{example}
  In the category $\catVect$, the product of two vector spaces $V$ and $W$ corresponds to their direct sum, denoted by $V \oplus W$.
The projection maps are the linear maps
\[
\pi_V : V \oplus W \to V, \quad \pi_V(v, w) = v,
\]
\[
\pi_W : V \oplus W \to W, \quad \pi_W(v, w) = w.
\]
Given any vector space $U$ and linear maps $f: U \to V$ and $g: U \to W$, the unique map $\langle f, g\rangle : U \to V \oplus W$
is defined by
\[
\langle f, g\rangle (u) = (f(u), g(u)).
\]
\end{example}


The \emph{coproduct} is the dual of the \emph{product}—it is obtained by reversing all the morphisms in the definition of a product. Consequently, a product in a category $\catC$ corresponds to a coproduct in the opposite category $\catCop$. More explicitly,

\begin{definition}
Consider a category $\catC$.  We say that it has (binary) coproducts if for any
objects $A$ and $B$ in $\catC$ there also exists an object $A \oplus B$ in
$\catC$ with morphisms $\inl : A \to A \oplus B$ and $\inr : B \to A \oplus B$
that satisfy a certain universal property: specifically for every two morphisms
$f  : A \to C$ and $g : B \to C$ there exists a \emph{unique} morphism $[f,g] :
A \oplus B \to C$ known as \emph{co-pairing} that makes the diagram below commute.
\[
\begin{tikzpicture}
  \matrix (m) [matrix of math nodes,row sep=4em,column sep=7em,minimum width=2em]
  {
   & C &  \\
    A  & A \oplus B & B\\
  };
  \path[-stealth]
    (m-2-1) edge  node [above] {$f$} (m-1-2)
    (m-2-3) edge  node [above] {$g$} (m-1-2)
    (m-2-1) edge  node [below] {$\inl$} (m-2-2)
    (m-2-3) edge  node [below] {$\inr$} (m-2-2)
    (m-2-2) edge [dotted]  node [right] {$[f,g]$} (m-1-2);
    ;
\end{tikzpicture}
\]
\end{definition}

  \begin{definition}
Let \( A \oplus B \) be a coproduct of objects \( A \) and \( B \), and let \( A' \oplus B' \) be a coproduct of objects \( A' \) and \( A' \) in a category $\catC$. Suppose we are given morphisms \( f : A \to A' \) and \( g : B \to B' \). 
Then there exists a unique morphism
\[
f \oplus g : A \oplus B \to A' \oplus B'
\]
such that the following diagram commutes.
\[
\begin{tikzpicture}
  \matrix (m) [matrix of math nodes,row sep=4em,column sep=7em,minimum width=2em]
  {
   A  & A \oplus B & B \\
    A'  & A' \oplus B' & B' \\
  };
  \path[-stealth]
    (m-1-1) edge  node [below] {$\inl$} (m-1-2)
    (m-1-3) edge  node [below] {$\inr$} (m-1-2)
    (m-1-1) edge  node [left] {$f$} (m-2-1)
    (m-1-3) edge  node [right] {$g$} (m-2-3)
    (m-2-1) edge  node [below] {$\inl$} (m-2-2)
    (m-2-3) edge  node [below] {$\inr$} (m-2-2)
    (m-1-2) edge [dotted]  node [right] {$ f\oplus g$} (m-2-2);
    ;
\end{tikzpicture}
\]
This induced morphism \( f \oplus g \) is called the \emph{coproduct of the morphisms} \( f \) and \( g \), and it is given explicitly by
\[
f \oplus g = [\inl \circ f,\, \inr \circ g].
\]
\end{definition}

\begin{theorem} 
  Let \( A \oplus B \) be the product of objects \( A \) and \( B \) in a category $\catC$. For any object $C$ and morphisms \( f : A \to C \) and \( g : B \to C \) are morphisms, it holds that:
\[
[h \circ f,\, h \circ g]  =  h \circ [f,g].
\]
\end{theorem}

\begin{proof}
This result is a direct consequence of the duality with products.
\end{proof}

\begin{proposition} \cite[Proposition 3.12]{awodeyCategoryTheory2010} \label{prop:cop_unique_iso}
  Coproducts are unique up to isomorphism. Explicitly, this can be formulated as follows: let \((C, \inl: A \to C, \inr: B \to C)\) and \((C', \inl': A \to C', \inr': B \to C')\) be two coproducts of objects \(A\) and \(B\) in a category. Then there exists a unique isomorphism \(\varphi: C \to C'\) such that
\[
\varphi \cdot \inl = \inl' \quad \text{and} \quad \varphi \cdot \inr = \inr'.
\]
\end{proposition}

\begin{example}
  In the category $\catSet$, the coproduct \( A \oplus B \) of two sets is their disjoint union, which can be constructed as
\[
A \oplus B = \{(a, 1) \mid a \in A\} \cup \{(b, 2) \mid b \in B\}.
\]
The canonical coproduct injections are defined by
\[
\inl(a) = (a, 1), \quad \inr(b) = (b, 2).
\]
Given any set \(C\) and functions \(f: A \to C\) and \(g: B \to C\), the copairing \([f, g]: A \oplus B \to C\) is defined by
\[
[f, g](x, \delta) = 
\begin{cases}
f(x) & \text{if } \delta = 1, \\
g(x) & \text{if } \delta = 2.
\end{cases}
\]
\end{example}

\begin{example}
  Let $(P, \leq)$ be a partial order and $\catfont{P}$ be its associated category.  
Consider a coproduct of elements \( p \oplus q \in P \). Then, by definition, there must exist injections satisfying
\[
p \leq p \oplus q \quad \text{and} \quad q \leq p \oplus q.
\]
Furthermore, for any element \( z \in P \), if
\[
p \leq z \quad \text{and} \quad q \leq z,
\]
then it follows that
\[
p \oplus q \leq z.
\]
This operation \( p + q \) corresponds to what is commonly known as the \emph{least upper bound} or \emph{join}, and is typically denoted by \( p \vee q \).
\end{example}


\begin{example}
  In $\catVect$ the coproduct coincides with the product. In such cases, this structure is called a \emph{biproduct}. 
  In  $\catVect$ the injection maps are the linear maps
\[
\inl: V \to  V \oplus W, \quad  \inl(v)= (v,0),
\]
\[
\inr : W \to V \oplus W, \quad \inr(w) = (0,w).
\]
Given any vector space $U$ and linear maps $f: V \to U$ and $g: W \to U$, the unique map $[ f, g] : V \oplus W \to U$
is defined by
\[
[f, g] (v,w) = f(v)+ g(w).
\]
\end{example}


Up until how we have only discussed binary products/coproducts. However, we can also define \emph{ternary products}\(A_1 \times A_2 \times A_3\) with an analogous universal property. That is, there exist three projection morphisms
$\pi_i : A_1 \times A_2 \times A_3 \to A_i \quad \text{for } i = 1, 2, 3,$
and for any object \(B\) and morphisms \(f_i : B \to A_i\), there exists a unique morphism 
$\langle f_1, f_2, f_3 \rangle  : B \to A_1 \times A_2 \times A_3$ 
such that \(\pi_i \cdot \langle f_1, f_2, f_3 \rangle = f_i\) for each \(i = 1, 2, 3\). Such a condition can be formulated for any number of factors and if a category has binary products, then it has all finite products, \ie any finite number $n \geq 1$ of factors. 
Any object $A$ is the unary product of $A$ with itself one time. Observe also that a terminal object is a ``nullary'' product, that is, a product of no objects: given no objects, there exists an object $\mathbf{1}$ with no projectors, and for any other object $A$, there exists a unique arrow $!: A \to \mathbf{1}$ making no additional diagrams commute. One can also define the product of a family of objects \((C_i)_{i \in I}\) indexed by a set \(I\), as an object $\prod_{i \in I} C_i$
together with a family of projection morphisms
\[
\pi_i : \prod_{j \in I} C_j \to C_i \quad \text{for each } i \in I,
\]
such that for every object \(A\) and every family of morphisms \((f_i : A \to C_i)_{i \in I}\), there exists a unique morphism $u : A \to \prod_{i \in I} C_i$ such that $\pi_i \cdot u = f_i \quad \text{for all } i \in I$.
 
Reversing all arrows in the definitions above yields the notion of \emph{finite coproducts} and the \emph{coproduct} of a family of objects $(C_i)_{i \in I}$.

\begin{definition}
A category \(\catC\) is said to have \emph{all small products} if every  set of objects in \(\catC\) has a product.
\end{definition}



\subsection{Functors}
  

Although categories are already interesting on their own, the real strength of category theory lies in understanding how categories relate to one another. Just as functions express relationships between sets, functors play a similar role for categories. A functor maps each object in one category to an object in another category, and it does the same for morphisms, preserving the structure of composition.

\begin{definition}
  Let $\catC$ and $\catD$ be two categories. A \emph{functor} $F: \catC \to \catD$ consists of a mapping that assigns to each object $A$ in $\catC$ an object $FA$ in $\catD$, and to each morphism $f \in \mathrm{Hom}_{\catC}(A, B)$ a morphism $Ff \in \mathrm{Hom}_{\catD}(FA, FB)$, in such a way that the following two conditions are satisfied for all objects $A, B, C$ in $\catC$ and all morphisms $f \in \mathrm{Hom}_{\catC}(A,B)$ and $g \in \mathrm{Hom}_{\catC}(B,C)$:
\[
F(\id_A) = \id_{FA}, \qquad F(g \circ f) = F(g) \circ F(f).
\]

A functor $F: \catC \to \catD$ is said to be \emph{full} if, for all objects $A$ and $B$ in $\catC$, the induced map
\[
F_{A,B}: \mathrm{Hom}_{\catC}(A, B) \longrightarrow \mathrm{Hom}_{\catD}(FA, FB), \quad f \mapsto Ff,
\]
is surjective. The functor is called \emph{faithful} if each $F_{A,B}$ is injective, and \emph{fully faithful} if each $F_{A,B}$ is bijective. A \emph{full embedding} is a functor that is fully faithful and, in addition, injective on objects.
\end{definition}

\begin {example}
Let $\catC$ be a category. Then there exists an \emph{identity functor} $\id_{\catC} : \catC \to \catC,$
which is defined on objects by $\id_{\catC}(A) = A$ for every object $A$ in $\catC$, and analogously on morphisms, that is, $\id_{\catC}(f) = f$ for every morphism $f$ in $\catC$.
\end{example}

\begin{example}
  Consider the natural numbers $\mathbb{N}$ as a partial order category. There is a functor $(-) + 5 : \mathbb{N} \to \mathbb{N}$
that maps each object $m \in \mathbb{N}$ to $m + 5$. This defines a functor because it preserves morphisms: if $m \leq m'$, then $m + 5 \leq m' + 5$. Moreover, the identity morphisms are trivially preserved.
\end{example}

\begin{example}
  Consider the set of real numbers $\mathbb{R}$ and the set of integers $\mathbb{Z}$, each regarded as a partial order category. In this context, there exists a functor $\mathrm{Floor} : \mathbb{R} \to \mathbb{Z} $ that assigns to each real number $r \in \mathbb{R}$ the greatest integer less than or equal to $r$, denoted $\lfloor r \rfloor$. For instance, $\lfloor 6.2 \rfloor = 6$ and $\lfloor -1.66 \rfloor = -2$. 

Similarly, there exists a \emph{ceiling functor} $\mathrm{Ceil} : \mathbb{R} \to \mathbb{Z}$ that maps each real number $r$ to the least integer greater than or equal to $r$, denoted $\lceil r \rceil$.
\end{example}


\begin{example}
   Let $\catP$ and $\catP'$ be partial order categories. Any functor $F \colon \catP \to \catP'$ corresponds precisely to a monotone function between the underlying posets.
\end{example}


\begin{definition}
  Given categories $\catC$, $\catD$, and $\catE$, a \emph{bifunctor}
$F : \catC \times \catD \to \catE$
is simply a functor from the product category $\catC \times \catD$ to $\catE$. In particular, $F$ is a rule that assigns:
\begin{itemize}
    \item to every objects $A \in \catC$ and $B \in \catD$, an object $F(A, B) \in \catE$;
    \item to every morphisms $f : A \to A'$ in $\catC$ and $g : B \to B'$ in $\catD$, a morphism
    $F(f, g) : F(A, B) \to F(A', B') \in \catE.$
\end{itemize}

These assignments must satisfy the following two requirements:

\begin{itemize}
    \item \emph{Respect for composition}:  
    For morphisms $f : A \to A'$, $f' : A' \to A''$ in $\catC$ and $g : B \to B'$, $g' : B' \to B''$ in $\catD$, it should hold that
    \[
    F(f' \circ f,\, g' \circ g) = F(f', g') \circ F(f, g),
    \]
    where the $\circ$ on the right-hand side is composition in $\catE$.

    \item \emph{Respect for identities}:  
    For all objects $A \in \catC$ and $B \in \catD$, it should hold that
    \[
    F(\id_A, \id_B) = \id_{F(A, B)},
    \]
    where $\id_A$ and $\id_B$ are the identity morphisms in $\catC$ and $\catD$, respectively, and $\id_{F(A, B)}$ is the identity morphism in $\catE$.
\end{itemize}

Many times, rather than writing the name of the bifunctor before the input, like $F(A, B)$, we write the bifunctor in infix notation, for example, $a \mathbin{\Box} b$. When we use this notation, the condition
\[
F(f' \circ f,\, g' \circ g) = F(f', g') \circ F(f, g)
\]
becomes
\[
(f' \circ f) \mathbin{\Box} (g' \circ g) = (f' \mathbin{\Box} g') \circ (f \mathbin{\Box} g).
\]
\end{definition}

\subsection{Natural Tranformations}
If category theory is about morphisms, then morphisms between functors should also be a natural concept. These are called \emph{natural transformations},  and provide a way of relating two functors that have the same domain and codomain. Intuitively, if we consider two functors $F, G : \catC \to \catD$ as different ways of assigning images of the category $\catC$ into the category $\catD$, then a natural transformation $\eta : F \Rightarrow G$ is a coherent way of transforming the image of $F$ into the image of $G$.


\begin{definition}
  Let $\catC$ and $\catD$ be categories, and let $F, G : \catC \to \catD$ be functors. A \emph{natural transformation} $\eta : F \Rightarrow G$ is a family of morphisms in $\catD$,
\[
\left( \eta_A : FA \to GA \right)_{A \in \mathrm{Ob}(\catC)},
\]
indexed by the objects of $\catC$, such that for every morphism $f : A \to A'$ in $\catC$, the following diagram commutes. 

\[
\begin{tikzpicture}
  \matrix (m) [matrix of math nodes,row sep=4em,column sep=7em,minimum width=2em]
  {
   FA  & GA  \\
    FA'  & GA'  \\
  };
  \path[-stealth]
    (m-1-1) edge  node [above] {$\eta_A$} (m-1-2)
    (m-2-1) edge  node [below] {$\eta_{A'}$} (m-2-2)
    (m-1-1) edge  node [left] {$Ff$} (m-2-1)
    (m-1-2) edge  node [right] {$Gf$} (m-2-2)
    ;
\end{tikzpicture}
\]
Given a natural transformation \(\eta : F \Rightarrow G\), the morphism \(\eta_A : F(A) \to G(A)\) in \(\catD\) is called the \emph{component} of \(\eta\) at \(A\).
A natural transformation $\eta : F \Rightarrow G$ is represented diagrammatically as 
\[
\begin{tikzpicture}
  \matrix (m) [matrix of math nodes,row sep=0.4em,column sep=2em,minimum width=1em]
  {
   \catC   & \big\Downarrow_{\eta} & \catD \\
  };
  \path[-stealth]
    (m-1-1) edge [bend left=40] node [above] {$F$} (m-1-3)
    (m-1-1) edge [bend left=-40] node [below] {$G$} (m-1-3)
    ;
\end{tikzpicture}
\]

\end{definition}


\begin{example}
  For every functor $F : \catC \to \catD$, there exists a natural transformation
  \[
    \iota_F : F \Rightarrow F
  \]
  known as  \emph{identity natural transformation},  such that for each object $A \in \catC$, each component of $\iota_F$ is the identity morphism:
  \[
    (\iota_F)_A = \id_{F(A)} : F(A) \to F(A).
    \] 
\end{example}

\begin{example}
The \emph{list functor}
\[
\mathrm{List} : \catSet \to \catSet
\]
assigns to each set \( S \) the set of all finite sequences (or lists) of its elements. 

For instance, if \( S = \{a, b, c\} \), then
\[
\mathrm{List}(S) = \{\varepsilon, a, b, c, aa, ab, ac, ba, \ldots, abc, cba, \ldots\},
\]
where \( \varepsilon \) denotes the empty list.

Given a function \( f : S \to T \), where $T=\{1,2\} $, the functor maps it to \(\mathrm{List}(f) : \mathrm{List}(S) \to \mathrm{List}(T)\), which applies \( f \) to each element of a list. For example, if
\[
f(a) = 2, \quad f(b) = 1, \quad f(c) = 2,
\]
then \(\mathrm{List}(f)(aabccba) = 2212212\).

There exists a natural transformation
\[
\mathrm{Reverse} : \mathrm{List} \Rightarrow \mathrm{List},
\]
whose component at a set \(S\), \(\mathrm{Reverse}_S\), maps each list to its reversal. For example:
\[
\mathrm{Reverse}_S(accbab) = babcca.
\]

\end{example}


\begin{definition}
  A natural transformation $\eta : F \Rightarrow G$ between functors $F, G : \catC \to \catD$ is called a \emph{natural isomorphism} if, for every object $A \in \catC$,  $\eta_A : F(A) \to G(A)$ is an isomorphism in $\catD$.
\end{definition}


\subsection{Equivalence of Categories}

In category theory, the concept of isomorphism between categories can be quite strict.
A more forgiving notion is an \emph{equivalence of categories}.

If \( F: \catC \to \catD \) is an isomorphism of categories, then for every object \( B \in \catD \), there exists a \emph{unique} object \( A \in \catC \) such that \( F(A) = B \). This expresses the idea that \( \catC \) and \( \catD \) are structurally identical.
An \emph{equivalence of categories} relaxes this requirement. For every object \( B \in \catD \), there exists an object \( A \in \catC \) such that \( F(A) \) is not necessarily equal to \( B \), but is \emph{isomorphic} to \( B \). 

\begin{definition}
  Categories \( \catC \) and \( \catD \) are  said to be \emph{equivalent} if there exist functors 
\( F: \catC \to \catD \) and \( G: \catD\to \catC \) such that 
\( G \circ F \cong \id_{\catC} \) and \( F \circ G \cong \id_{\catD} \). 
The functors \( F \) and \( G \) are called \emph{quasi-inverses}, and we write \( \catC \simeq \catD \).
This entails that for every \( A \in \catC \), there is a \( B \in \catD \) with \( G(B) \cong A \), 
and for every \( B \in \catD \), there is an \( A \in \catC \) with \( F(A) \cong B \).
\end{definition}

\begin{example}
  One of the simplest examples of an equivalence of categories is the relationship between the one-object category \( \mathbf{1} \) and the category \( \mathbf{2}_I \), which has two objects and a single isomorphism between them. We can visualize this as:
$$
\ast \;\; \quad \simeq \quad a \xlongrightarrow{\;\cong\;} b \;\; 
$$

More precisely, there is a unique functor \( ! : \mathbf{2}_I \to \mathbf{1} \), and a functor \( L : \mathbf{1} \to \mathbf{2}_I \) defined by \( L(\ast) = a \). Clearly, the composition \( ! \circ L \) is equal to \( \id_{\mathbf{1}} \), and \( L \circ ! \cong \id_{\mathbf{2}_I} \), since both objects \( a \) and \( b \) in \( \mathbf{2}_I \) are isomorphic. Thus, \( \mathbf{1} \simeq \mathbf{2}_I \).
\end{example}

\subsection{Adjoints}
If we further weaken the notion of an equivalence of categories, we arrive at the concept of an \emph{adjunction}. 

\begin{definition}
  Given categories \(\catC\) and \(\catD\), a pair of functors \(L: \catC \to \catD\) and \(R: \catD \to \catC \) form an \emph{adjunction} \(L \dashv R\) if there exists a natural isomorphism:
\[
\mathrm{Hom}_{\catD}(L(A), B) 
\xlongrightarrow{\quad \Phi_{A,B} \quad}
\mathrm{Hom}_{\catC}(A, R(B)).
\]
One says that $R$ is right adjoint to $L$, or that $L$ is left adjoint to $R$. Such an adjunction is denoted by $L \dashv R$, where the turn of the symbol $\dashv$ always points to the left adjoint.

\end{definition}

\begin{example}
  Consider the set of real numbers $\mathbb{R}$ and the set of integers $\mathbb{Z}$, each viewed as partial order categories. There is an inclusion functor $\mathrm{inc} : \mathbb{Z} \hookrightarrow \mathbb{R}$ which simply maps each integer to itself.  This inclusion has a left adjoint $L : \mathbb{R} \to \mathbb{Z}$. 

To determine this left adjoint $L$, we use the definition of an adjunction:  for all $N \in \mathbb{Z}$ and $R \in \mathbb{R}$, we have a natural isomorphism:
\[
\mathrm{Hom}_{\mathbb{Z}}(L(R), N) \cong \mathrm{Hom}_{\mathbb{R}}(R, \mathrm{inc}(N)).
\]

Since both $\mathbb{Z}$ and $\mathbb{R}$ are partial orders, the hom-sets contain at most one morphism. Hence, this isomorphism reduces to the logical equivalence:
\[
 L(R) \leq N \text{ if and only if } R \leq \mathrm{inc}(N) = N.
\]

Take $R = 7.27$ as an example. Then the inequality $R \leq N$ holds precisely when $N$ is an integer greater than or equal to $7.27$. That is:
\[
7.27 \nleq 5,\quad 7.27 \nleq 6,\quad 7.27 \nleq 7,\quad 7.27 \leq 8,\quad 7.27 \leq 9, \quad \ldots
\]
By the condition above, we must then have:
\[
L(7.27) \nleq 5,\quad L(7.27) \nleq 6,\quad L(7.27) \nleq 7,\quad L(7.27) \leq 8,\quad L(7.27) \leq 9, \quad \ldots
\]
From this, we conclude that $L(7.27) = 8$. In general, $L(R)$ is the least integer greater than or equal to $R$, \ie, the ceiling function:
\[
L(r) = \lceil r \rceil.
\]

Thus, the inclusion functor $\mathrm{inc}$ has $\lceil \; \rceil$ as a left adjoint, \ie, $\lceil  \; \rceil \dashv \mathrm{inc}$.
The unit of this adjunction is the natural transformation
$\eta : \id_{\mathbb{R}} \Rightarrow \mathrm{inc} \circ \lceil  \;  \rceil,$
which expresses the inequality $r \leq \lceil r \rceil$ for all $r \in \mathbb{R}$. The counit of the adjunction is the identity, since for any integer $n$, it holds that $\lceil N \rceil = N$.
\end{example}



\begin{definition}
  Let \( F: \catC \to \catD \) and \( G:  \catD \to  \catE \) be functors. It is said that \( G \)  \emph{preserves coproducts} if  whenever $ L $ is a coproduct of \( F \), then \( G(L) \) is a coproduct of \( G \circ F \).
\end{definition}

\begin{theorem} \cite[Section 4.6]{yanofskyMonoidalCategoryTheory2024}
 Left adjoints preserve coproducts.
\end{theorem}

% usar a def principal e a ds homsets

%adjoints
% example com o floor
%Adjoints preserve coproducts

\subsection{Monoidal categories}
\begin{definition}
A \textbf{monoid} is a triple $(M, \cdot, u)$, where $M$ is a set equipped with 
a binary operation $\cdot \colon M \times M \to M$ and a distinguished element 
$u \in M$ called the \emph{unit}, satisfying the following axioms for all 
$x, y, z \in M$:
\begin{align*}
    \text{(Associativity)} & & x \cdot (y \cdot z) &= (x \cdot y) \cdot z, \\
    \text{(Unit laws)}     & & u \cdot x &= x = x \cdot u.
\end{align*}
\end{definition}

Monoidal categories are named so because they are categories equipped with an additional structure that resembles the structure of monoids.

\begin{comment}
  
\begin{definition}
 A \emph{strict monoidal category} is a category $\catC$ equipped with a bifunctor $\otimes : \catC \times \catC \to \catC$
called the \emph{tensor product}, which is associative, meaning that for all objects $A, B, C \in \catC$,
\[
A \otimes (B \otimes C) = (A \otimes B) \otimes C. 
\]
In addition, there is a distinguished object $I \in \catC$, called the \emph{unit}, that acts as a unit for $\otimes$, satisfying
\[
I \otimes C = C = C \otimes I \quad \text{for all } C \in \catC. 
\]
\end{definition}



\begin{example}
The partially ordered set of natural numbers $\mathbb{N}$ forms a strict monoidal category when equipped with addition, denoted $(\mathbb{N}, +, 0)$.  The tensor product is given by ordinary addition, and the unit object is $0$, since for any $n \in \mathbb{N}$ we have $0 + n = n = n + 0$, and addition is strictly associative.
Moreover, the tensor product is monotonic with respect to the order: if $m \leq m'$ and $n \leq n'$, then $m + n \leq m' + n'$. This order preservation ensures functoriality of the monoidal structure.
A similar argument applies to multiplication: if $m \leq m'$ and $n \leq n'$, then $m \cdot n \leq m' \cdot n'$. Therefore, $(\mathbb{N}, \cdot, 1)$ also forms a strict monoidal category under the same ordered structure.
\end{example}

\end{comment}

\begin{definition}
 A \emph{monoidal category} consists of a category $\catC$ equipped with a bifunctor $\otimes : \catC \times \catC \to \catC$
called \emph{tensor product} and a distinguished object $I \in \catC$, called \emph{unit} together with natural isomorphisms
\[
\alpha_{A,B,C} : A \otimes (B \otimes C) \rightarrow (A \otimes B) \otimes C,
\]
\[
\lambda_A :\mathbb{I}\otimes A \rightarrow A, \quad \rho_A : A \otimes\mathbb{I}\rightarrow A,
\]
known as \emph{associator}, \emph{left unitor}, and \emph{right unitor}, respectively. We will omit the subscripts when no ambiguity arises.
Moreover, these natural isomorphisms are required to make the following coherence diagrams commute.
\[
\begin{tikzpicture}
  \matrix (m) [matrix of math nodes,row sep=3em,column sep=2em,minimum width=1em]
  {
    & (A \otimes B) \otimes (C \otimes D) & \\
   A \otimes (B \otimes (C \otimes D)) &  & ((A \otimes B) \otimes C) \otimes D  \\
    A \otimes ((B \otimes C) \otimes D)  & &  (A \otimes (B \otimes C)) \otimes D \\
  };
  \path[-stealth]
    (m-1-2) edge  node [above] {$\alpha$} (m-2-3)
    (m-3-3) edge  node [right] {$\alpha \otimes \id$} (m-2-3)
    (m-3-1) edge  node [below] {$\alpha$} (m-3-3)
    (m-2-1) edge  node [left] {$\id \otimes \alpha$} (m-3-1)
    (m-2-1) edge  node [above] {$\alpha$} (m-1-2)
    ;
\end{tikzpicture}
\]


\[
\begin{minipage}{0.45\textwidth}
\centering
\begin{tikzpicture}
  \matrix (m) [matrix of math nodes,row sep=3em,column sep=2em,minimum width=1em]
  {
    A \otimes (\mathbb{I} \otimes A) &   & (A \otimes \mathbb{I}) \otimes A \\
     & A \otimes A & \\
  };
  \path[-stealth]
    (m-1-1) edge  node [above] {$\alpha$} (m-1-3)
    (m-1-1) edge  node [left] {$\id \otimes \lambda_A$} (m-2-2)
    (m-1-3) edge  node [right] {$\rho_A \otimes \id$} (m-2-2);
\end{tikzpicture}
\end{minipage}
\quad \quad
\begin{minipage}{0.45\textwidth}
\centering
\begin{tikzpicture}
  \matrix (m) [matrix of math nodes,row sep=3em,column sep=3em,minimum width=1em]
  {
   \mathbb{I}\otimes\mathbb{I}&   &\mathbb{I}\otimes\mathbb{I} \\
     &\mathbb{I}& \\
  };
  \path[-stealth]
    (m-1-1) edge  node [left] {$\lambda_I$} (m-2-2)
    (m-1-3) edge  node [right] {$\rho_I$} (m-2-2);
  \draw[double equal sign distance] (m-1-1) -- (m-1-3);
\end{tikzpicture}
\end{minipage}
\]

\end{definition}






\begin{definition}
  A monoidal category is said to be \emph{symmetric} when it is
equipped with a natural isomorphism $\sw : A \otimes B \rightarrow B \otimes A$ known as \emph{braiding} such that the following diagrams commute.


\[
\hspace{110pt}
\begin{tikzpicture}
  \matrix (m) [matrix of math nodes,row sep=3em,column sep=3em,minimum width=1em]
  {
    A \otimes\mathbb{I}&   & \mathbb{I}\otimes A \\
     & A & \\
  };
  \path[-stealth]
    (m-1-1) edge  node [above] {$\sw_{A,I}$} (m-1-3)
    (m-1-1) edge  node [left] {$\rho_A$} (m-2-2)
    (m-1-3) edge  node [right=0.15cm] {$\lambda_A$} (m-2-2);
\end{tikzpicture}
\]

\hspace{60pt}
\begin{tikzpicture}
  \matrix (m) [matrix of math nodes,row sep=3em,column sep=4em,minimum width=1em]
  {
  A \otimes (B \otimes C)  & (A \otimes B) \otimes C & C \otimes (A \otimes B) \\
  A \otimes (C \otimes B) & (A \otimes C) \otimes B & (C \otimes A) \otimes B\\
  };
  \path[-stealth]
    (m-1-1) edge  node [above] {$\alpha$} (m-1-2)
    (m-1-2) edge  node [above] {$\sw$} (m-1-3)
    (m-1-3) edge  node [right] {$\alpha$} (m-2-3)
    (m-2-2) edge  node [below] {$\sw \otimes \id$} (m-2-3)
    (m-2-1) edge  node [below] {$\alpha$} (m-2-2)
    (m-1-1) edge  node [right] {$\id \otimes \sw$} (m-2-1)
    ;
\end{tikzpicture}


\end{definition}

\begin{definition}
  A monoidal category $\catC$ is  said to be \emph{closed} if for each object
  $A$ in $\catC$ the functor $- \otimes A$ has a right adjoint,
  denoted by $A \multimap -$. 
\end{definition}

\begin{definition}
        A monoidal category $\catC$ with coproducts is called
        \emph{distributive} if for every object $A$ in $\catC$ the
        functor $- \otimes A$ preserves coproducts. Explicitly
        this means that the morphism,
        \[
                [\inl \otimes \id, \inr \otimes \id] : B \otimes A \oplus C \otimes                     A \to (B \oplus C) \otimes A
        \]
        is actually an isomorphism. We will denote the respective inverse
        by $\dist$. Note that if $\catC$ is monoidal closed then it is automatically
        distributive as left adjoints preserve all colimits.
\end{definition}



\begin{example}
Examples of monoidal closed categories with coproducts include $\catSet$ and $\catVect$. In $\catSet$, the tensor product is the cartesian product, the monoidal unit is the singleton set, the coproduct is the disjoint union, and the internal hom consists of all functions between sets. For $\catVect$, the tensor product is the standard tensor product of complex vector spaces, the unit is the field of complex numbers $\mathbb{C}$, the coproduct is the direct sum, and the internal hom is the space of complex linear maps. %Similarly, in $\catBan$ the tensor product is the projective tensor product, the monoidal unit is $\mathbb{R}$, the coproduct is the direct sum equipped with the $L_1$-norm, and the internal hom consists of all functions between sets corresponds to space of short (linear) maps equiped with the operator norm.
\end{example}


\begin{theorem}[\emph{Coherence Theorem for Symmetric Monoidal Categories}] \cite[Section 6.2]{yanofskyMonoidalCategoryTheory2024}
Any diagram in a symmetric monoidal category constructed only from associators $\alpha$, unitors $\lambda$, $\rho$, the symmetry $\sw$, and inverses and their composition and tensor product necessarily commutes if the two underlying permutations are the same.
\end{theorem}


\section{Banach spaces}

%Before introducing the category of Banach spaces and proving it is a model, we first introduce some preliminary concepts in this setting. 

%The letters \gls{vectorspaces} will often be used to refer to vector spaces and $\gls{field}$ denotes the field of scalars of a vector space.


%\subsection{Normed spaces}

\begin{definition} \label{def:norm}
  A \emph{norm} $\|\cdot\|$ is a function that associates an element of a vector space $V$ with a non-negative real number, such that the following properties hold:
  \begin{enumerate}
    \item Positive definiteness: $\|v\| \geq 0$ for all $v \in V$, with $\|v\| = 0$ if and only if $v = 0$;
    \item Positive scalability: $\|av\| = |\alpha|\|v\|$ for all $v \in V$  $\alpha \in \mathcal{F}$;
    \item The triangle inequality: $\|v + w\| \leq \|v\| + \|w\|$ for all $v, w \in V$.
  \end{enumerate}
\end{definition}

\begin{definition} \label{def:normed_space}
A vector space together with a norm is called a \emph{normed vector space}.
\end{definition}


Every normed space may be regarded as a metric space (\autoref{def:metric_space}), in which the
distance, $d(v,w)$, between vectors $v$ and $w$ is $\|v-w\|$ .

\begin{definition} \label{def:op_norm}
Let \( V \) and \( W \) be normed vector spaces, and let \( T: V \rightarrow W \) be a linear operator. The \emph{operator norm}, denoted by $\|\cdot\|_{\text{op}}$, is defined as
\[
\opnorm{T}= \sup \{ \norm{T(v)} : v \in V, \, \norm{v} = 1 \}.
\]
If \( \opnorm{T} < \infty \) we say that \( T \) is a \emph{bounded operator}; otherwise, if \( \opnorm{T} = \infty \) we say that \( T \) is \emph{unbounded}. 
\end{definition}

\begin{lemma} \cite[Lemma 6.4]{guide2006infinite} \label{lemma:op_norm_submult} %inf_dim_anal
Let \( T: V \rightarrow W \) be a bounded linear operator between normed spaces. Then the following statements hold:
\begin{enumerate}
  \item For every \( v \in V \), we have \( 
  \norm{T(v)} \leq \opnorm{T} \cdot \norm{v}.\)
  \item The operator \( T \) is continuous (w.r.t the metric) if and only if it is bounded.
\end{enumerate}
\end{lemma}




\begin{definition}
  Let \( V \) and \( W \) be normed vector spaces, and let \( T: V \rightarrow W \) be a linear operator. $T$ is called a \emph{short map} if $\opnorm{T} \leq 1$.
\end{definition}



%\subsection{Banach spaces}


\begin{definition} (\emph{Cauchy sequence})
  Suppose $ d $ is a metric on a set $ X $. A sequence $ \{x_n\} \subset X $ is called a \emph{Cauchy sequence} if, for every $ \varepsilon > 0 $, there exists an integer $N \in \mathbb{N} $ such that $ d(x_m, x_n) < \varepsilon $ for all $ m, n > N $.  The metric \( d \) is said to be \emph{complete} if every Cauchy sequence in \( X \) converges to a point in \( X \)
\end{definition}

\begin{definition}
  A \emph{Banach space} is a normed vector space that is complete with respect to the metric induced by its norm. In other words, every Cauchy sequence in the space must converge to a limit within the space.
\end{definition}


\section{Topology}

%Due to this, we will introduce topological concepts and results necessary for the next two sections.


Topology is the abstract mathematical study of concepts like convergence and approximation, among other things, generalizing familiar notions from calculus and analysis. 
Note that, for instance, in a metric space, a sequence $\{x_n\}$ with $n \in \mathbb{N}$ converges to a point $x$ if the distance $d(x_n, x)$ tends to zero; that is, for every $\varepsilon > 0$, there exists $n_0$ such that $d(x_n, x) < \varepsilon$ for all $n \geq n_0$. However, metric spaces are not sufficient to describe all types of convergence.  An example is the pointwise convergence of all real-valued functions on the interval $[0, 1]$. In fact, there is no metric on the space of all real functions on the interval $[0,1]$ for which one can define a distance function $d(f_n, f)$ such that $d(f_n, f) \to 0$ if and only if $f_n(x) \to f(x)$ for every $x \in [0, 1]$ \cite{guide2006infinite}. 
A foundational idea in topology is that of a \emph{neighborhood}—a collection of points considered ``sufficiently close'' to a given point. From this arises the concept of \emph{open sets}, which are sets that serve as neighborhoods for all their points. The collection of all such open sets defines a \emph{topology}, and a set equipped with a topology becomes a \emph{topological space}. This framework introduces some subtleties: for example, traditional sequences are often inadequate for capturing convergence, requiring the more general notion of \emph{nets}, which are indexed over broader structures than the natural numbers.

\begin{definition}
  A \emph{topology} $\tau$ on a set $S$ is a collection of subsets of $S$ satisfying the following properties:
\begin{enumerate}
    \item $\varnothing \in \tau$ and $S \in \tau$.
    \item $\tau$ is closed under finite intersections: if $U_1, U_2, \dots, U_n \in \tau$, then $\bigcap_{i=1}^n U_i \in \tau$.
    \item $\tau$ is closed under arbitrary unions: if $\{U_\alpha\}_{\alpha \in A} \subseteq \tau$, then $\bigcup_{\alpha \in A} U_\alpha \in \tau$.
\end{enumerate}

A nonempty set $S$ equipped with a topology $\tau$ is called a \emph{topological space}, and is denoted by $(S, \tau)$ (or simply $S$ when no ambiguity arises). A member of $\tau$ is called an \emph{open set} in $S$. The complement of an open set is a \emph{closed set}. 

A set \( S \) can have many different topologies.  
The family of all topologies on \( S \) is partially ordered by set inclusion.  
If \( \tau_1 \subset \tau_2 \), that is, if every \( \tau_1 \)-open set is also \( \tau_2 \)-open,  
then we say that \( \tau_1 \) is \emph{weaker} or \emph{coarser} than \( \tau_2 \),  
and that \( \tau_2 \) is \emph{stronger} or \emph{finer} than \( \tau_1 \).
\end{definition}

\begin{example}
Standard examples of topologies are presented below:
\begin{enumerate}
    \item \emph{Trivial (or indiscrete) topology:} On a set $S$, the trivial topology consists only of the sets $\varnothing$ and $S$. These are also the only closed sets.
    
    \item \emph{Discrete topology:} The discrete topology on a set $S$ consists of all possible subsets of $S$. In this topology, every set is both open and closed.
    
    \item \emph{Standard topology on $\mathbb{R}$:} The metric $d(v, w) = |v - w|$ on $\mathbb{R}$ induces a topology where open sets are unions of open intervals. This is known as the standard topology on $\mathbb{R}$.
\end{enumerate}
\end{example}

\begin{definition}
A \emph{neighborhood} of a point \( s \in S \) in a topological space $(S, \tau)$ is any subset \( N \subseteq S  \) that contains \( s \) in its interior. In this case, \( s \) is called an \emph{interior point} of \( N \). 
\end{definition}

\begin{definition}
  The \emph{norm topology} induced by a norm $\|\cdot\|$ is the topology generated by the metric $d(v, w) = \|v - w\|$.
\end{definition}

Topology is about convergence and also about continuity. Consider a  map $f \colon V \to W$, the idea behind continuity is that if we move $v \in V$ only slightly, then $f(v)$ should change by a small amount as well. The less we move $v$, the less $f(v)$ should change. We begin, with a more intuitive definition restricted to the setting of metric spaces.

\begin{definition}
Let $(V,d_V)$ and $(W,d_W)$ be metric spaces. A mapping $f \colon V \to W$ is \emph{sequentially continuous} if for every convergent sequence $(x_n)_{n\in\mathbb{N}}$ in $V$ with $v_n \to v$, the image sequence $(f(v_n))_{n\in\mathbb{N}}$ converges to $f(v)$ in $W$. That is,
\[
v_n \to n \ \text{in } V \ \Rightarrow \ f(v_n) \to f(v) \ \text{in } W.
\]
\end{definition}
More generally, continuity may be defined as follows:
\begin{definition}
  Let $(S_1, \tau_1)$ and $(S_2, \tau_2)$ be topological spaces. A map $f \colon S_1 \to S_2$ is \emph{continuous} if and only if for every open subset $N \subseteq S_2$, the preimage $f^{-1}(N)$ is open in $S_1$.
\end{definition}

\begin{definition}
A net in a set $S$ is a function $s \colon D \to S$, where $D$ is a directed set. The directed set $D$ is called the \emph{index set} of the net and the members of $D$ are \emph{indexes}.
\end{definition}

\begin{definition}
  Let \( S_1 \) and \( S_2 \) be two topological spaces, and let \( s_1 \) be a point in \( S_1 \).  
A map \( f : S_1 \to S_2 \) is said to be \emph{continuous at} \( s_1 \) if and only if, for every open neighborhood \( S_1 \) of \( f(s_1) \), there exists an open neighborhood \( N \) of \( s_1 \) such that  
$
\{f(n) \, |\, n \in N \} \subseteq N.
$
\end{definition}




The proposition and theorem below present continuity in a more intuitive way:
\begin{proposition} \cite[Theorem 2.27]{guide2006infinite}
  Let \( S_1 \) and \( S_2 \) be two topological spaces. A map \( f : S_1 \to S_2 \) is continuous if and only if it is continuous at every point \( s_1 \in S_2 \).
\end{proposition}

\begin{theorem} \cite[Theorem 2.28]{guide2006infinite} \label{def:continuidade_top}
  Let $f \colon S_1 \to S_2$ be a function between topological spaces, and let $s_1 \in S_2$. The following statements are equivalent:
\begin{enumerate}
    \item The function $f$ is continuous at $s$.
    \item For every net $(s_\alpha)$ in $S_1$ converging to $s$, the net $(f(s_\alpha))$ converges to $f(s)$ in $S_2$.
\end{enumerate}
\end{theorem}



%\begin{definition}
%A \emph{neighborhood} of a point $x$ is any set $S$ such that $x$ is contained in $V$. In this case, we say that $x$ is an \emph{interior point} of $V$.
%\end{definition}

\begin{definition}
A \emph{topological vector space} is a vector space \( V \) equipped with a linear topology \( \tau \) such that:
\begin{enumerate}
    \item every singleton \( \{v\} \subset V \) is a closed set, and
    \item the vector space operations (addition and scalar multiplication) are continuous with respect to \( \tau \).  That is, the addition map \( (x, y) \mapsto x + y \), from the Cartesian product \( V \times V \) into \( V, \) is continuous, and the scalar multiplication map \( (r, x) \mapsto r x \), from \( \mathcal{F} \times V \) into \( V, \) is also continuous.
\end{enumerate}

\end{definition}

\begin{definition}
  Let $V$ be a vector space. 
  Linear maps from $V$ to its scalar field are called \emph{linear functionals}. The set of all continuous linear functionals on $V$ forms a vector space, called the \emph{(topological) dual space} of $ V $, and is denoted by  $V^*$. It is common to designate elements of the dual space  $V^*$  by \( v^* \).  %and to write $\langle v, v^* \rangle $in place of \( v^*(v) \).
\end{definition}

%norma espaço dual


\begin{theorem} \cite[Theorem 4.3]{rudin91functional} %funcanl-Rudin
Let \( V \) be a normed vector space. For each \( v^* \in V^* \), define its norm by
\[
\|v^*\| := \sup \left\{ | v^*(v) | : \|v\| = 1 \right\}.
\]
This defines a norm on \( V^* \) under which \( V^* \) is a Banach space. Moreover, for every \( v \in V \), we have
\[
\|v\| = \sup \left\{ | v^*(v) | : \|v^*\| = 1 \right\}.
\]
As a consequence, the map \( v^* \mapsto v^*(v)  \) defines a bounded linear functional on \( V^* \), and its norm equals \( \|v\| \).
\end{theorem}

\begin{definition} \label{def:weak*_top}
Let \( V \) be a vector space. The \emph{weak\(^*\)-topology} on the dual space \( V^* \) is the coarsest topology that makes all evaluation maps
\[
v^* \mapsto v^*(v) 
\]
continuous for every \( v \in V \).
\end{definition}




The following concepts and results are used in \Cref{subsec:w*_cats} and \Cref{ch:future_work}.

\begin{proposition} \cite[Proposition 2.3.10]{pedersenAnalysisNow1989}
  Let $V$ and $W$ be bounded vector spaces and let \( f : V \to W \) be a bounded linear map. Then its dual \( f^* : V^* \to W^* \), defined by
\[
f^*(\varphi) = \varphi \circ f,
\]
is also a bounded linear map.
\end{proposition}


\begin{theorem} \cite[Theorem 9.15]{romanAdvancedLinearAlgebra1992}
Let $\mathcal{H}$ be a finite-dimensional Hilbert space, and let $\varphi$ be a linear functional on $\HilbH^*$. Then there exists a unique vector $v_\varphi \in \HilbH$ such that
\[
\varphi(w) = \langle w, v_\varphi \rangle \quad \text{for all } w \in  \HilbH.
\]
We call $v_\varphi$ the \emph{Riesz vector} for $\varphi$ and denote it by $v_\varphi$.\\
\end{theorem}


Using the Riesz representation theorem, we can define a map 
$\phi^* : V^* \to V$
by setting \(\phi^*(\varphi) = v_\varphi\), where \(v_\varphi\) is the Riesz vector corresponding to \(\varphi \in V^*\). Since the Riesz representation is unique, \(\phi^*\) is well-defined \cite{romanAdvancedLinearAlgebra1992}. 

\begin{comment}

\begin{proposition}
  A $\phi^*: \BoundOp{\HilbH}^* \to \BoundOp{\HilbK}$,  is defined as $\phi^*(\varphi) = \sum_{i,j} \varphi(\ket{j}\bra{i}) \cdot \ket{i}\bra{j}.$
\end{proposition}

\todo[inline,size=\normalsize]{Alternativa 1}


\begin{proof}
  
\end{proof}



 Consider a linear map $\phi: \BoundOp{\HilbH} \to \BoundOp{\HilbK}$. Since it is well known that we can identify $\BoundOp{\HilbH}$ with the matrix algebra $\mathcal{M}_{n}$, where $n = \dim(\HilbH)$, the map $\phi$ can be equivalently represented as $\phi:\mathcal{M}_{n} \to \mathcal{M}_{m}$, where $m = \dim(\HilbK)$.
 
 Consequently, by the Riesz representation theorem, and the definitions of trace and  inner product for matrices we have:
\[ \varphi(\ket{j} \bra{i})= \langle \ket{j} \bra{i},  v_\varphi  \rangle =  \tr(\ket{i}\bra{j} \cdot v_{\varphi}) =  (v_{\varphi})_{i, j}, \]

where $(v_{\varphi})_{i, j}$ denotes the entry in the $i$-th row and $j$-th column of $v_{\varphi}$. 

As a result, we obtain
\[
\phi^*(\varphi) = \sum_{i,j} \varphi(\ket{j}\bra{i}) \cdot \ket{i}\bra{j}
= \sum_{i,j} (v_{\varphi})_{i,j} \cdot \ket{i}\bra{j}
= v_\varphi.
\]

\todo[inline,size=\normalsize]{Alternativa 2: Mudar a assinatura de $\phi^*$ para as matrizes e assim temos}

By the Riesz representation theorem, and the definitions of trace and  inner product for matrices we have:
\[ \varphi(\ket{j} \bra{i})= \langle \ket{j} \bra{i},  v_\varphi  \rangle =  \tr(\ket{i}\bra{j} \cdot v_{\varphi}) =  (v_{\varphi})_{i, j}, \]

where $(v_{\varphi})_{i, j}$ denotes the entry in the $i$-th row and $j$-th column of $v_{\varphi}$. 

As a result, we obtain
\[
\phi^*(\varphi) = \sum_{i,j} \varphi(\ket{j}\bra{i}) \cdot \ket{i}\bra{j}
= \sum_{i,j} (v_{\varphi})_{i,j} \cdot \ket{i}\bra{j}
= v_\varphi.
\]
\end{comment}



\begin{comment}
Let $\sigma= n_1, \ldots, n_s$. For $\phi: \mathcal{M}_\sigma^* \to \mathcal{M}_\sigma $, by the Riesz representation theorem, we have
\[ \varphi(\ket{j} \bra{i})= \langle \ket{j} \bra{i},  v_\varphi  \rangle. \]

Since $\mathcal{M}_\sigma$ is a direct sum, the Riesz vector $v_\varphi$ also decomposes:
\[
v_\varphi = (v_{\varphi,1}, \ldots, v_{\varphi,s}),
\]
with $v_{\varphi,k} \in \mathbb{C}^{n_k \times n_k}$.

Using the definition of inner product for matrices, we have
\[
\langle \ket{j_k}\bra{i_k}, v_{\varphi,k} \rangle = \tr(\ket{i_k}\bra{j_k} \cdot v_{\varphi,k}) =  (v_{\varphi,k})_{i_k, j_k} ,
\]
where $(v_{\varphi,k})_{p_k, q_k}$ denotes the entry in the $i_k$-th row and $j_k$-th column of $v_{\varphi,k}$. Hence,
\[
\varphi(\ket{j}\bra{i}) = \sum_{k=1}^s \langle \ket{j_k} \bra{i_k},  v_{\varphi,k}  \rangle = \sum_{k=1}^s  \tr \left( \ket{i_k} \bra{j_k} \cdot   v_{\varphi,k} \right) =  (v_{\varphi,k})_{i_k, j_k} \equiv (v_{\varphi})_{i,j}.
\]
When evaluated on basis elements $\ket{j}\bra{i}$, only the $k$-th component of $v_\varphi$ contributes (since the basis is orthonormal).


As a result, we obtain
\[
\phi^*(\varphi) = \sum_{i,j} \varphi(\ket{j}\bra{i}) \cdot \ket{i}\bra{j}
= \sum_{i,j} (v_{\varphi})_{i,j} \cdot \ket{i}\bra{j}
= v_\varphi.
\]
\end{comment}







\begin{definition}\label{def:uni_cont}
  A function \( f : (V, d) \to (W, d') \) between two metric spaces is said to be \emph{uniformly continuous} if for every \( \varepsilon > 0 \) there exists some \( \delta > 0 \) (depending only on \( \varepsilon \)) such that  
\[
d(x, y) < \delta \quad \Rightarrow \quad d'(f(x), f(y)) < \varepsilon
\]
for all \( x, y \in X \).  
Any uniformly continuous function is continuous.  
An important property of uniformly continuous functions is that they map Cauchy sequences into Cauchy sequences \cite{guide2006infinite}. 
\end{definition}

%\todo[inline,size=\normalsize]{Em principio estas 2 coisas vão sair}

\begin{definition} \label{def:lip_uni_cont}
  A function \( f : (V, d_V) \to (W, d_W) \) between metric spaces is said to be \emph{Lipschitz continuous} if there exists a real number \( c \geq 0 \) such that for every \( v_1, v_2 \in V \),
\[
d_W(f(v_1), f(v_2)) \leq c \cdot d_V(v_1, v_2).
\]
The number \( c \) is called a \emph{Lipschitz constant} for \( f \).  
Clearly, every Lipschitz continuous function is uniformly continuous  \cite{guide2006infinite}.
\end{definition}

\begin{lemma} \cite[Lemma 7.3.19]{goubault-larrecqNonHausdorffTopologyDomain2013}  \label{lem:completion_unique}
Let \( A \) be a dense subset of a topological space \( V \), and let  
\( f: A \to W \) be a continuous map into a metric space \( W \).  
Then \( f \) has at most one continuous extension \( g: V \to W \), \ie, one such that $f$ and $g$ coincide on $A$.  
This continuous extension \( g \) exists if \( (V,d_V) \) is a metric space, \( (W,d_W) \) is a complete metric space, and \( f \) is uniformly continuous from \( (A,d_V) \) to \( (W,d_W) \). 
\end{lemma}


