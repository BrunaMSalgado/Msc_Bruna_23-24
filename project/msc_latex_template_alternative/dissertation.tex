\documentclass[12pt,a4paper,fleqn,twoside,openright]{report}
\usepackage{dissertation}
\usepackage{accents}
\usepackage{tikz-cd} % Useless, but might be handy to draw fancier lines.
\usepackage{amssymb} % To provide the \varnothing symbol
\usepackage{amsmath}
\usepackage{quantikz}
\usepackage{mathtools}
\usepackage{hyperref}
\usepackage{aliascnt}
\usepackage{amsthm}
\usepackage{hyperref}
\usepackage[utf8]{inputenc} % Specify input encoding
\usepackage{textcomp} % For additional text symbols
\usepackage{xargs}                      % Use more than one optional parameter in a new commands
\usepackage[pdftex,dvipsnames]{xcolor}  % Coloured text etc.
\usepackage[colorinlistoftodos,prependcaption,textsize=tiny]{todonotes}
\newcommandx{\unsure}[2][1=]{\todo[linecolor=red,backgroundcolor=red!25,bordercolor=red,#1]{#2}}
\newcommandx{\change}[2][1=]{\todo[linecolor=blue,backgroundcolor=blue!25,bordercolor=blue,#1]{#2}}
\newcommandx{\info}[2][1=]{\todo[linecolor=OliveGreen,backgroundcolor=OliveGreen!25,bordercolor=OliveGreen,#1]{#2}}
\newcommandx{\improvement}[2][1=]{\todo[linecolor=Plum,backgroundcolor=Plum!25,bordercolor=Plum,#1]{#2}}
\newcommandx{\thiswillnotshow}[2][1=]{\todo[disable,#1]{#2}}

\newtheorem{theorem}{Theorem}[section]
\newtheorem{corollary}{Corollary}[theorem]
\newtheorem{lemma}{Lemma}[section]
\providecommand*{\lemmaautorefname}{Lemma}
\makeglossaries
\makeindex

%\logo{EAAD}{School of Architecture, Art and Design}{}
%\logoB{EAAD}{School of Architecture, Art and Design}{}

%\logo{EC}{School of Sciences}{}
%\logoB{EC}{School of Sciences}{}

%\logo{ED}{Law School}{}
%\logoB{ED}{Law School}{}

\logo{EE}{School of Engineering}{}
\logoB{EE}{School of Engineering}{}

%\logo{EEG}{School of Economics and Management}{}
%\logoB{EEG}{School of Economics and Management}{}

%\logo{ELACH}{School of Letters, Arts and Human Sciences}{}
%\logoB{ELACH}{School of Letters, Arts and Human Sciences}{}

%\logo{EM}{Medical school}{}
%\logoB{EM}{Medical school}{}

%\logo{EP}{School of Psychology}{}
%\logoB{EP}{School of Psychology}{}

%\logo{ESE}{Higher School of Nursing}{}
%\logoB{ESE}{Higher School of Nursing}{}

%\logo{I3Bs}{Research Institute on Biomaterials,}{Biodegradables and Biomimetics}
%\logoB{I3Bs}{Research Institute on Biomaterials,}{Biodegradables and Biomimetics}

%\logo{ICS}{Institute of Social Sciences}{}
%\logoB{ICS}{Institute of Social Sciences}{}

%\logo{IE}{Institute of Education}{}
%\logoB{IE}{Institute of Education}{}

\author{Bruna Filipa Martins Salgado}

\titleA{A Metric Equational System}
\titleB{for Quantum Computation}


\masters{Master in Physics Engineering}
%\area{Area of specialization}
\supervisor{Renato Jorge Araújo Neves}


\bibpunct[,]{(}{)}{;}{a}{,}{,}

\begin{document}
\setlength{\parindent}{0em}

%-- Covers
\input{covers/Covers}

%-- Document setup
\newgeometry{right=25mm, left=25mm, top=25mm, bottom=25mm}
\pagenumbering{roman}

\setlength{\parskip}{0pt}
\setlength{\parindent}{0pt}

%-- Preamble
\input{preamble/Copyright}
\chapter*{Acknowledgements}

First and foremost, I would like to express my deepest gratitude to my supervisor, Professor Renato Neves, for his invaluable guidance, unwavering patience, and for guiding me into this journey into the fascinating world of formal methods. This journey would not have been possible without his expertise and encouragement.

I am also deeply grateful to my parents, who have always supported and encouraged me to give my best, providing everything I needed along the way, and to my brother, who cheers me on just as much as he annoys me with his antics.

My heartfelt thanks also go to my dear friends Vitória, Margarida, Gabriela, Gonçalo, Rui, Fábio, and Miguel—especially for the cheerful lunches that left me smiling like a madwoman in the afternoons, never failing to lift my spirits during moments of endless frustration.

Juliana, Nico, and Vitor will always have my sincere gratitude for helping me whenever they could (or at least trying to), and for the occasional much-needed laugh. This gratitude also extends to another (semi-)frequent lab visitor, Alexandra.

Finally, I would like to thank INESC TEC and FCT for the research  grant attributed for developing this dissertation project, with reference  \texttt{PTDC/CCI-COM/4280/2021}.

\input{preamble/StatementofIntegrity}
\chapter*{Abstract}

Noisy intermediate-scale quantum (NISQ)  computers are expected to operate with severely limited hardware resources. Precisely controlling qubits in these systems comes at a high cost, is susceptible to errors, and faces scarcity challenges. Therefore, error analysis is indispensable for the design, optimization, and assessment of NISQ computing. Nevertheless, the analysis of errors in quantum programs poses a significant challenge. The overarching goal of the M.Sc. project is to provide a fully-fledged
quantum programming language on which to study metric program equivalence
in various scenarios, such as in quantum algorithmics and quantum information theory.


\paragraph{Keywords} approximate equivalence, $\lambda$-calculus, metric equations

\cleardoublepage

\chapter*{Resumo}

Escrever aqui o resumo (pt)

\paragraph{Palavras-chave} palavras, chave, aqui, separadas, por, vírgulas

\cleardoublepage


\phantomsection
\tableofcontents

\cleardoublepage
\listoffigures

% List of tables
\renewcommand*{\listtablename}{List of Tables}
\listoftables
\clearpage

% Acronyms
\printglossary[type=\acronymtype,nonumberlist, title={Acronyms}]

% Glossary
\printglossary[title={Glossary}, nonumberlist]

\cleardoublepage
\pagenumbering{arabic}

%-- Dissertation 
\part{Introductory material}

\chapter{Introduction}

\newacronym{nisq}{NISQ}{Noisy Intermediate-Scale Quantum}

\section{Motivation and Context}


Quantum computing dates back to 1982 when Nobel laureate Richard Feynman proposed the idea of constructing computers based on quantum mechanics principles to efficiently simulate quantum phenomena \cite{feynman2018simulating}. 

The field has since evolved into a multidisciplinary research area that combines quantum mechanics, computer science, and information theory. Quantum information theory, in particular, is based on the idea that if there are new physics laws, there should be new ways to process and transmit information.  In classical information theory, all systems (computers, communication channels, etc.) are fundamentally equivalent, meaning they adhere to consistent scaling laws. These laws, therefore, govern the ultimate limits of such systems. For instance, if the time required to solve a particular problem, such as the factorization of a large number, increases exponentially with the size of the problem, this scaling behavior remains true irrespective of the computational power available.  Such a problem, growing exponentially with the size of the object, is known as a "difficult problem". However, as demonstrated by Peter Shor, the use of a quantum computer with a sufficient number of quantum bits (qubits) could significantly accelerate the factorization of large numbers \cite{shor1994algorithms}.  This advancement poses a significant threat to the security of confidential data transmitted over the Internet, as the RSA algorithm is based on the computational difficulty of factorizing large numbers.

%In classical information theory, all systems (computers, communication channels, etc.) are equivalent. The systems might be quick or slow, but the scaling rules are always the same. Consequently, these laws rule the ultimate limits of the systems. If the calculation time to process a given  problem, for example, the factorization of a number, increases exponentially with the size of the object under consideration, this law of scale will be true regardless of the power of the computer. Such a problem, growing exponentially with the size of the object, is known as a "difficult problem". As shown by Peter Shor, factorizing large numbers would be tremendously accelerated if one has a quantum computer with many quantum bits. This would ruin the actual encoding of confidential data on the internet, as the RSA algorithm is based on the difficulty of factorizing large numbers.
%In 1994, Peter Shor introduced an algorithm that could factorize large numbers in polynomial time, a task that is believed to be intractable for classical computers [\cite{shor1994algorithms}]. In 1996, Lov Grover presented a quantum search algorithm that could search an unsorted database quadratically faster than classical algorithms [\cite{grover1996fast}].

%as classical computers seemed ill-suited for this task. 

%The quantum computing paradigm holds immense promise, as evidenced by several compelling results in computational complexity theory 

The quantum computing paradigm holds immense promise, as evidenced by this compelling result in computational complexity theory.  While hardware advancements have brought the scientific community closer to realizing this potential, the ultimate goal is yet to be accomplished. A \acrfull{nisq} computer equipped with 50-100 qubits may surpass the capabilities of current classical computers, yet the impact of quantum noise, such as decoherence in entangled states, imposes limitations on the size of quantum circuits that can be executed reliably \cite{preskill2018quantum}. Unfortunately, general-purpose error correction techniques \cite{calderbank1996good, gottesman1997stabilizer, steane1996error} consume a substantial number of qubits, making it difficult for \acrshort{nisq} devices to make use of them in the near term. For instance, the implementation of a single logical qubit may require between $10^3$ and $10^4$ physical qubits \cite{fowler2012surface}. As a result, it is unreasonable to expect that the idealized quantum algorithm will run perfectly on a quantum device, instead
only a mere approximation will be observed.

To reconcile quantum computation with \acrshort{nisq} computers, quantum compilers perform transformations for error mitigation \cite{wallman2016noise} and noise-adaptive optimization \cite{murali2019noise}. Additionally, current quantum computers only support a restricted, albeit universal, set of quantum operations. As a result, nonnative operations must be decomposed into sequences of native operations before execution \cite{harrow2002efficient}, \cite{burgholzer2020advanced}. In general, perfect computational universality is not sought, but only the ability to approximate any quantum algorithm, with a preference for minimizing the use of additional gates beyond the original requirements. The assessment of these compiler transformations necessitates a comparison of the error bounds between the source and compiled quantum programs. Additionally, in quantum information theory, it is essential to account for errors arising from malicious attacks or noisy channels \cite{watrous2018theory}. 

This suggests the development of appropriate notions of approximate program equivalence, \textit {in lieu} of the classical program equivalence and underlying theories that typically hinge on the idea that equivalence is binary, \textit{i.e.} two programs are either equivalent or they are not \cite{winskel1993formal}.

%Furthermore, in quantum information theory, the concept of an $\epsilon-\text{approximation}$ channel is fundamental when studying quantum teleportation via noisy channels \cite{watrous2018theory}. 

As previously noted, Shor's algorithm has played a pivotal role in sparking heightened interest within the scientific community toward quantum computing research. Several quantum programming languages have surfaced over the past 25 years \cite{zhao2020quantum,serrano2022quantum}. These include imperative languages such as Qiskit \cite{Qiskit} and Silq \cite{bichsel2020silq}, as well as functional languages such as Quipper \cite{green2013quipper} and Q\# \cite{svore2018q}. On one hand, the design of quantum programming languages is strongly oriented towards implementing quantum algorithms. On the other hand, the  definition of functional paradigmatic languages or functional calculi serves as a valuable tool for delving into theoretical aspects of quantum computing, particularly exploring the foundational basis of quantum computation \cite{zorzi2016quantum}. 

%Given the nature of this work, the focus will be on quantum languages designed with this latter aspect in mind. 
%QPL, a quantum language within the functional programming paradigm, marks a significant milestone in this context \cite{selinger2004towards}. It is a first-order functional language with a static type system that integrates classical control and quantum data, and its denotational semantics is based on superoperators. 

Most of the current research on algorithms and programming languages assumes that addressing the challenge of noise during program execution will be resolved either by the hardware or through the implementation of fault-tolerant protocols designed independently of any specific application \cite{chong2017programming}. As previously stated, this assumption is not realistic in the \acrshort{nisq} era. Nonetheless, there have been efforts to address the challenge of approximate program equivalence in the quantum setting. 

\cite{hung2019quantitative} and \cite{tao2021gleipnir} reason about the issue of noise in a quantum while-language by developing a deductive system to determine how similar a quantum program is from its idealised, noise-free version. The former introduces the ($Q$,$\lambda$)-diamond norm which analyzes the output error given that the input quantum state satisfies some quantum predicate $Q$ to degree $\lambda$. However, it does not specify any practical method for obtaining non-trivial quantum predicates. In fact, the methods used in [\cite{hung2019quantitative}] cannot produce any post conditions other than $(I,0)$ (\textit{i.e.}, the identity matrix $I$ to degree 0, analogous to a ``true” predicate) for large quantum programs. The latter specifically addresses and delves into this aspect.  

An alternative approach was explored in \cite{dahlqvist2022syntactic}, using linear $\lambda$-calculus as basis – \textit{i.e} programs are written as linear $\lambda$-terms – which has deep connections to both logic and category theory \cite{girard1995advances}, \cite{benton1994mixed}. A notion of approximate equivalence is then
integrated in the calculus via the so-called diamond norm , which induces a metric (roughly, a distance function) on the space of quantum programs (seen semantically as completely positive trace-preserving super-operators) [\cite{watrous2018theory}]. The authors argue that their deductive system allows to compute an approximate distance between two quantum programs easily as opposed to computing an exact distance ``semantically" which tends to involve quite complex operators.  Some positive results were achieved in this setting, but much remains to be done.

\section{Goals}
The notion of approximate equivalence for quantum programming explored in \cite{dahlqvist2022syntactic} does not take important operations into account. Specifically, the corresponding mathematical model does not include measurements, classical control flow, or discard operations. Also, the corresponding typing system is often times too strict and cannot properly handle multiple uses of the same resource, such as sampling exactly $n$-times from a distribution. The overarching goal of this M.Sc. project is to tackle the aforementioned limitations. A successful completion of this goal will provide a fully-fledged quantum programming language on which to study metric program equivalence in various scenarios. This includes not only quantum algorithmics – where, for example, the number of iterations in Grover’s algorithm involves approximations – but also quantum information theory, where, for instance, quantum teleportation and the problem
of the discrimination of quantum states have important roles \cite{nielsen2010quantum}.


\section{Document Structure}
%Notação
%dizer que vamos acrescentando novas regras ao longo dos capitulos ao nosso lambda calculua -> construção incremental do sistema
\chapter{State of the Art}

\section{Linear Lambda Calculus}


The Lambda-Calculus, developed by Church and Curry in the 1930s, serves as a formal language capturing the key attribute of higher-order functional languages, treating functions as first-class citizens, allowing them to be passed as arguments [\cite{barendregt1984lambda}].  Beyond its foundational aspects, this calculus incorporates extensions for modeling side effects, including probabilistic or non-deterministic behaviors and shared memory. Centered around the expression of higher-order functions, where functions can serve as inputs or outputs, it emerges as a potent computational tool.  Higher-order functions form a pivotal abstraction in practical programming languages such as LISP, Scheme, ML, and Haskell.

% Exemplo de função que só se possa usar uma vez, porque não estou a perceber? 

In quantum information theory, the role of higher-order functions encompasses two fundamental aspects. The first involves the concept of entangled functions and how well-known quantum phenomena find natural descriptions through such functions. The second concerns the interplay between classical objects and quantum objects in a higher-order context. Quantum computation conventionally handles classical and quantum data, while the higher-order context introduces a third data type: functions. These functions fall into two categories - those "quantum-like" (entangled, single-use) and those "classical-like" (duplicable, reusable). Remarkably, this classification transcends input/output types, highlighting the coexistence of quantum-like functions operating on classical data and classical-like functions operating on quantum data. [\cite{selinger2009quantum}].

\subsection{Syntax}
The grammar and term formation rules of the linear lambda calculus, discussed in [\cite{dahlqvist2022syntactic}], are presented in this subsection.

The definition of the grammar for linear lambda calculus is as follows, where $G$ represents a set of ground types.
\vspace{-10pt}
\begin{equation} \label{eq:grammar}
\centering
\hspace{95pt} \mathbb{A} ::= X \in G \hspace{3 pt} \vert \hspace{3 pt} \mathbb{I}  \hspace{3 pt}  \vert \hspace{3 pt} \mathbb{A}  \otimes  \mathbb{A} \hspace{3 pt} \vert \hspace{3 pt} \mathbb{A} \oplus \mathbb{A} \hspace{3 pt}  \vert \hspace{3 pt}   \mathbb{A} \multimap  \mathbb{A}
\end{equation}
\vspace{-25pt}

 Regarding the term formation rules, $\Sigma$ corresponds to a class of sorted operation symbols $f: \mathbb{A}_{1}, \ldots, \mathbb{A}_{n} \xrightarrow{} \mathbb{A}$, where $n \geq 1$.   Typing contexts are represented as lists $x_{1}: \mathbb{A}_{1}, \ldots, x_{n}: \mathbb{A}_{n}$ of typed variables, with each variable $x_i$ (where $1 \leq i \leq n$) occurring at most once in $x_1, \ldots, x_n$. The typing contexts are denoted by greek letters $\Gamma$, $\Delta$, and $E$.
The concept of shuffling is employed to construct a linear typing system that ensures the admissibility of the exchange rule and enables unambiguous reference to judgment's denotations $[\![ \Gamma \triangleright v: \mathbb{A} ]\!]$. Shuffling is defined as a permutation of typed variables in a sequence of contexts, $\Gamma_1, \ldots, \Gamma_n$, preserving the relative order of variables within each $\Gamma_i$. For instance, if $\Gamma_1=x:\mathbb{A}, y:\mathbb{B}$ and $\Gamma_2=z:\mathbb{C}$, then $z:\mathbb{C}, x:\mathbb{A}, y:\mathbb{B}$ is a valid shuffle of $\Gamma_1, \Gamma_2$. On the other hand, $y:\mathbb{B}, x:\mathbb{A}, z:\mathbb{C}$ is not a shuffle because it alters the occurrence order of $x$ and $y$ in $\Gamma_1$. The set of shuffles in $\Gamma_1, \ldots, \Gamma_n$ is denoted as $\text{Sf} (\Gamma_1, \ldots, \Gamma_n)$.
The term formation rules of the linear lambda calculus are shown in
\autoref{fig:typing_rules_linear}.
\vspace{-10pt}
\begin{figure} [H]
\begin{equation*}
\begin{split}
\begin{aligned}
&
\begin{minipage}[t]{0.3\textwidth}
$\begin{array}{c}
     \Gamma_{i} \triangleright v_{i}: \mathbb{A}_{i} \quad f: \mathbb{A}_{1}, \ldots, \mathbb{A}_{n} \xrightarrow{} \mathbb{A} \in \Sigma \quad E \in \text{Sf}(\Gamma_{1}; \ldots; \Gamma_{n})\\
    \hline
   E \triangleright f( v_{1},\ldots,v_{n}): \mathbb{A}
\end{array}
$
\end{minipage}
\hspace{168pt}
\text{(ax)} 
 \hspace{10pt}
\begin{minipage}[t]{0.3\textwidth}
$\begin{array}{c}
      \\
    \hline
   x:\mathbb{A} \triangleright x:\mathbb{A}
\end{array}
$ \end{minipage}
\hspace{-60pt} \text{(hyp)} \\
&
\begin{minipage}[t]{0.3\textwidth}
$\begin{array}{c}
     \\
    \hline
   - \triangleright *: \mathbb{I}
\end{array}
$
\end{minipage}
\hspace{-80pt}
\text{($\mathbb{I}_{i}$)} 
 \hspace{20pt}
\begin{minipage}[t]{0.3\textwidth}
$\begin{array}{c}
     \Gamma \triangleright v: \mathbb{A} \otimes \mathbb{B} \quad  \Delta,x: \mathbb{A}, y: \mathbb{B}  \triangleright w: \mathbb{C}  \quad E \in \text{Sf}(\Gamma;\Delta)\\
    \hline
   E\triangleright \text{pm } v \text{ to } x \otimes y. w :\mathbb{C}
\end{array}
$ \end{minipage}
\hspace{144pt} (\otimes_{e}) \\
&
\begin{minipage}[t]{0.3\textwidth}
$\begin{array}{c}
     \Gamma \triangleright v: \mathbb{A} \quad  \Delta \triangleright w: \mathbb{B}  \quad E \in \text{Sf}(\Gamma;\Delta) \\
    \hline
   E \triangleright v \otimes w: \mathbb{A} \otimes \mathbb{B} 
\end{array}
$
\end{minipage}
\hspace{60pt} (\otimes_{i}) 
 \hspace{20pt}
 \begin{minipage}[t]{0.3\textwidth}
$\begin{array}{c}
     \Gamma \triangleright v: \mathbb{I} \quad  \Delta \triangleright w: \mathbb{A}  \quad E \in \text{Sf}(\Gamma;\Delta)  \\
    \hline
   E \triangleright v \text { to } *.w: \mathbb{A}  
\end{array}
$ \end{minipage}
\hspace{55pt} (\mathbb{I}_{e}) \\
&
\begin{minipage}[t]{0.3\textwidth}
$\begin{array}{c}
     \Gamma,x:\mathbb{A} \triangleright v: \mathbb{B} \\
    \hline
   \Gamma \triangleright \lambda x:\mathbb{A} . v: \mathbb{A} \multimap \mathbb{B} 
\end{array}
$
\end{minipage}
\hspace{-20pt} (\multimap_{i}) 
 \hspace{5pt}
 \begin{minipage}[t]{0.3\textwidth}
$\begin{array}{c}
     \Gamma \triangleright v: \mathbb{A} \multimap \mathbb{B} \quad  \Delta \triangleright w: \mathbb{A}  \quad E \in \text{Sf}(\Gamma;\Delta)  \\
    \hline
   E \triangleright v w: \mathbb{B}  
\end{array}
$ \end{minipage}
\hspace{83pt} (\multimap_{e}) 
\hspace{5pt}
\begin{minipage}[t]{0.3\textwidth}
$\begin{array}{c}
     \Gamma \triangleright v: \mathbb{A}  \\
    \hline
   \Gamma \triangleright \text{dis}(v):  \mathbb{I} 
\end{array}
$
\end{minipage}
\hspace{-63pt} (\text{dis})
\end{aligned}
\end{split}
\end{equation*}
\caption{Term formation rules of affine lambda calculus.}
\label{fig:typing_rules_linear}
\end{figure}
\vspace{-8pt}
The no-cloning theorem states that it is impossible to duplicate a quantum bit [\cite{wootters1982single}]. This principle is upheld by the type system outlined in \autoref{fig:typing_rules_linear}, which does not allow the repeated use of a variable (seen as a quantum resource).
Nevertheless, the linearity
constraint is often deemed too restrictive, prompting research into relaxing it in various computational paradigms. In [\cite{dahlqvist2023complete}], the controlled use of a resource multiple times is explored within approximate program equivalence paradigms. Moreover, the grammar introduced allows the specification of how many times a resource can be used—a notion particularly relevant in quantum computation, especially within the NISQ era where resources are scarce.
\vspace{-5pt}
\subsection{Metric equational system}
Metric equations [\cite{mardare2016quantitative}, \cite{mardare2017axiomatizability}] are a strong candidate for reasoning about approximate program equivalence. These equations take the form of $t=_{\epsilon} s$, where  $\epsilon$ is a non-negative rational representing the ``maximum distance" between the two terms $t$ and $s$. The metric equational system for linear lambda calculus is depicted in \autoref{fig:metric deductive system} [\cite{dahlqvist2022syntactic}].
\vspace{-10pt}
\begin{figure} [H]
\begin{equation*}
\begin{split}
\begin{aligned}
&
\begin{minipage}[t]{0.3\textwidth}
$\begin{array}{c}
     \\
    \hline
   v=_{0}v
\end{array}
$
\end{minipage}
\hspace{-90pt}
\text{(refl)} 
 \hspace{55pt}
\begin{minipage}[t]{0.3\textwidth}
$\begin{array}{c}
    v=_{q}w \quad w=_{r}u  \\
    \hline
   v=_{q + r} u
\end{array}
$ \end{minipage}
\hspace{-40pt} \text{(trans)} 
\hspace{55pt}
\begin{minipage}[t]{0.3\textwidth}
$\begin{array}{c}
    v=_{q}w \quad r\geq q  \\
    \hline
   v=_{r} w
\end{array}
$ \end{minipage}
\hspace{-45pt} \text{(weak)} \\
&
\begin{minipage}[t]{0.3\textwidth}
$\begin{array}{c}
    \forall r > q . \hspace{4pt} v=_{r} w \\
    \hline
   v=_{q}w
\end{array}
$
\end{minipage}
\hspace{-45pt}
\text{(arch)} 
 \hspace{40pt}
\begin{minipage}[t]{0.3\textwidth}
$\begin{array}{c}
    \forall i \leq n. \hspace{4pt} v=_{q_i} w\\
    \hline
   v=_{\vee q_i} w
\end{array}
$ \end{minipage}
\hspace{-40pt} \text{(join)} 
\hspace{40pt}
\begin{minipage}[t]{0.3\textwidth}
$\begin{array}{c}
    v=_{q} w \quad v'=_{r} w' \\
    \hline
   v \otimes v' =_{q + r} w \otimes w'
\end{array}
$ \end{minipage}
\hspace{-45pt}  \\
&
\begin{minipage}[t]{0.3\textwidth}
$\begin{array}{c}
   \forall i \leq n. \hspace{4pt} v_{i}=_{q_i} w_{i}\\
    \hline
   f(v_{1},...,v_{n})=_{\Sigma q_i} f(w_{1},...,,w_{n}) 
\end{array}
$
\end{minipage}
 \hspace{34pt}
\begin{minipage}[t]{0.3\textwidth}
$\begin{array}{c}
   v=_{q}w  \quad v'=_{r}w'\\
    \hline
    v \text { to } *.v' =_{q+r} w \text { to } *.w'
\end{array}
$ \end{minipage}
\hspace{0pt}
\begin{minipage}[t]{0.3\textwidth}
$\begin{array}{c}
    v=_{q} w  \\
    \hline
  \lambda x : \mathbb{A}. \hspace{4pt} v=_{q} \lambda x:\mathbb{A}. \hspace{4pt} w
\end{array}
$ \end{minipage}
\hspace{20pt}  \\
&
\begin{minipage}[t]{0.3\textwidth}
$\begin{array}{c}
    v=_{q} w \quad  v'=_{r} w'  \\
    \hline
   \text{pm} \hspace{4pt} v \hspace{4pt} \text{to} \hspace{4pt} x \otimes y. \hspace{4pt} v'=_{q + r}\text{pm} \hspace{4pt} w \hspace{4pt} \text{to} \hspace{4pt} x \otimes y .  \hspace{4pt} w'
\end{array}
$
\end{minipage}
\hspace{200pt}
\begin{minipage}[t]{0.3\textwidth}
$\begin{array}{c}
    v =_{q} w \quad v'=_{r} w'   \\
    \hline
  v v' =_{q + r} w w'
\end{array}
$ \end{minipage}
 \\
 &
\begin{minipage}[t]{0.3\textwidth}
$\begin{array}{c}
  \Gamma \triangleright v =_{q} w: \mathbb{A} \quad \Delta \in \text{perm}(\Gamma)\\
    \hline
   \Delta \triangleright v =_{q} w: \mathbb{A}
\end{array}
$
\end{minipage}
\hspace{180pt}
\begin{minipage}[t]{0.3\textwidth}
$\begin{array}{c}
    v =_{q} w \quad v'=_{r} w'    \\
    \hline
  v[v'/x]=_{q + r} w[w'/x]
\end{array}
$ \end{minipage}
\hspace{10pt}
\end{aligned}
\end{split}
\end{equation*}
\caption{Metric equational system}
\label{fig:metric deductive system}
\end{figure}
\vspace{-5pt}
In the quantum paradigm, a potential notion of approximate equivalence arises from the so-called diamond norm [\cite{watrous2018theory}], which induces a metric (roughly, a distance function) on the space of quantum programs (seen semantically as completely positive trace-preserving super-operators). This norm relies on another norm known as the trace norm. The $\lVert  \rVert_{1}$ latter is defined by  $\lVert A \rVert_{1} = \text{Tr}\sqrt{A^{\dag}A}$  for matrices $A \in \mathbb{C}^{n\times n}$.
The trace distance between two super-operators $E, E': \mathbb{C}^{n\times n} \xrightarrow{} \mathbb{C}^{m\times m }$,  denoted as $T(E,E')$, is defined as follows:
\begin{equation} \label{eq:trace_distance}
\begin{centering}
\hspace{80pt}
 T(E,E')= \max\{\lVert (E-E') A \rVert_{1} \hspace{2pt}  \vert \hspace{2pt}  \lVert A \rVert_{1}=1\} 
 \end{centering}
\end{equation}
Unfortunately, this norm is not stable under tensoring [\cite{watrous2018theory}], and consequently, the diamond norm, which is based on the trace norm, is used instead. The diamond norm between two super-operators $E, E': \mathbb{C}^{n\times n} \xrightarrow{} \mathbb{C}^{m\times m }$ is defined as:
\begin{equation}  \label{eq:diamond_distance}
\begin{centering}
\hspace{110pt}
 \lVert E-E'\rVert_{\diamondsuit} =  T(E \otimes I_{n},E' \otimes I_{n}) 
 \end{centering}
\end{equation}
where $I_{n} $ is the identity super-operator over the space $\mathbb{C}^{n\times n}$.
The notion of a diamond norm is used in [\cite{dahlqvist2022syntactic}] which introduces a simple metric theory based on the idea of approximating a quantum operation. The authors argue that their deductive system allows to compute an approximate distance between two quantum programs easily as opposed to computing an exact distance ``semantically" which tends to involve quite complex operators. 
Other works in this spirit include [\cite{hung2019quantitative}] and [\cite{tao2021gleipnir}]. They reason about the issue of noise in a quantum while-language by developing a deductive system to determine how similar a quantum program is from its idealised, noise-free version. The former introduces the ($Q$,$\lambda$)-diamond norm which analyzes the output error given that the input quantum state satisfies some quantum predicate $Q$ to degree $\lambda$. However, it does not specify any practical method for obtaining non-trivial quantum predicates. In fact, the methods used in [\cite{hung2019quantitative}] cannot produce any post conditions other than $(I,0)$ (\textit{i.e.}, the identity matrix $I$ to degree 0, analogous to a ``true” predicate) for large quantum programs. The latter specifically addresses and delves into this aspect. 

\vspace{-7pt}

\subsection{Interpretation}
In order to define the interpretation of judgments $\Gamma \triangleright v: \mathbb{A}$, it is necessary to establish some notation first. Considering $v \in V$, $w \in W$, and $u \in U$  where $V, W, U$ represent vector spaces,  $\textsc{sw}_{V,W} : V\otimes W \xrightarrow{} W \otimes V$, denotes the swap operator, defined as $\textsc{sw}_{V,W}= v\otimes w \mapsto w \otimes v$;    $\rho_{V} : \mathbb{C} \otimes V \xrightarrow{} V $ is the left unitor defined as $\rho_{V}= 1 \otimes v \mapsto v $; $\lambda_{V} : V  \otimes \mathbb{C} \xrightarrow{} V $ is the right unitor defined as $\lambda_{V}= v \otimes 1 \mapsto v$; $\alpha_{V,W,U} : V  \otimes (W \otimes U) \xrightarrow{} (V  \otimes W) \otimes U$ is the left associator, defined as $\alpha_{V,W,U}= v \otimes (w \otimes u) \mapsto (v \otimes w) \otimes u $; and $!_{V}: V \xrightarrow{} \mathbb{C}$ is the trace operation applied to a vector, defined as  $!_{V}= v \xrightarrow{} \text{Tr} \hspace{1pt}v$. Moreover, for all operators $f: V \otimes W \xrightarrow{} U$, the operator $\overline{f} : V \xrightarrow{} (W \multimap U)$ denotes the corresponding curried version, defined as $\overline{f}(v) = w \mapsto  f(v,w)$. The subscripts in these operators will be omitted unless ambiguity arises.

For all ground types $X \in G$  the interpretation of $[\![X]\!]$  is postulated as a vector space $V$. Types are interpreted inductively using the unit $\mathbb{I}$, the tensor $\otimes$, and the linear map $\multimap$. Given a non-empty context $\Gamma=\Gamma',x: \mathbb{A}$, its interpretation is defined by $[\![\Gamma',x: \mathbb{A}]\!] = [\![\Gamma']\!] \otimes [\![\mathbb{A}]\!]$ if $\Gamma'$ is non-empty and $[\![\Gamma',x: \mathbb{A}]\!] = [\![\mathbb{A}]\!]$ otherwise. The empty context $-$ is interpreted as $[\![-]\!] = \mathbb{I}$. Given $X_{1}, . . . ,X_{n} \in V$, the $n$-tensor $(\ldots (X_1 \otimes X_2) \otimes \ldots ) \otimes X_{n}$ is denoted as $X_1 \otimes \ldots \otimes X_{n}$, and similarly for operators. 


``Housekeeping" operators are employed to handle interactions between context interpretation and the vectorial model. Given $\Gamma_{1}, \ldots, \Gamma_{n}$, the operator that splits $[\![\Gamma_{1}, \ldots, \Gamma_{n}]\!]$ into $[\![\Gamma_{1}]\!] \otimes \ldots \otimes [\![\Gamma_{n}]\!]  $ is denoted by $\text{sp}_{\Gamma_1;\ldots;\Gamma_n}: [\![\Gamma_{1}, \ldots, \Gamma_{n}]\!] \xrightarrow{} [\![\Gamma_{1}]\!] \otimes \ldots \otimes [\![\Gamma_{n}]\!] $.
On the other hand, $\text{jn}_{\Gamma_1;\ldots;\Gamma_n}$ denotes the inverse of $\text{sp}_{\Gamma_1;\ldots;\Gamma_n}$. Next, given $\Gamma, x : \mathbb{A}, y : \mathbb{B},\Delta$, the operator permuting $x$ and $y$ is denoted by $\text{exch}_{\Gamma, x : \mathbb{A}, y : \mathbb{B},\Delta}: [\![\Gamma, x : \mathbb{A}, y : \mathbb{B},\Delta]\!] \xrightarrow{} [\![\Gamma, y : \mathbb{B}, x : \mathbb{A}, \Delta]\!] $. The shuffling operator $\text{sh}_{E}: [\![E]\!] \xrightarrow{} [\![\Gamma_1, \ldots, \Gamma_n ]\!]$ is defined as a suitable composition
of exchange operators.

For every operation symbol $f: \mathbb{A}_{1}, \ldots, \mathbb{A}_{n} \xrightarrow{} \mathbb{A}$ we assume the existence of an operator $[\![f]\!]: [\![\mathbb{A}_{1}]\!] \otimes \ldots \otimes [\![\mathbb{A}_{n}]\!] \xrightarrow{}  [\![\mathbb{A}]\!] $.
The interpretation of judgments is defined by induction over derivations according to the rules in \autoref{fig:denotational_sem} [\cite{dahlqvist2022syntactic}].
\vspace{-7pt}
\begin{figure} [H]
\begin{equation*}
\begin{split}
\begin{aligned}
&
\begin{minipage}[t]{0.3\textwidth}
$\begin{array}{c}
      [\![\Gamma_{i} \triangleright v_{i}: \mathbb{A}_{i} ]\!]=m_{i}  \quad f: \mathbb{A}_{1}, \ldots, \mathbb{A}_{n} \in \Sigma \quad E \in \text{Sf}(\Gamma_{1}; \ldots; \Gamma_{n})\\
    \hline
  [\![E \triangleright f( v_{1},\ldots,v_{n}): \mathbb{A} ]\!] = [\![ f]\!] \cdot (m_{1}\otimes \ldots \otimes m_{n}) \cdot \text{sp}_{\Gamma_1;\ldots;\Gamma_n}\cdot \text{sh}_{E}
\end{array}
$
\end{minipage}
\hspace{204pt}
\begin{minipage}[t]{0.3\textwidth}
$\begin{array}{c}
      \\
    \hline
  [\![ x:\mathbb{A} \triangleright x:\mathbb{A}]\!] = \text{id}_{[\![\mathbb{A} ]\!]}
\end{array}
$ \end{minipage} \\
&
\begin{minipage}[t]{0.3\textwidth}
$\begin{array}{c}
     \\  
    \hline
   [\![ - \triangleright *: \mathbb{I}]\!] = \text{id}_{[\![\mathbb{I} ]\!]}
\end{array}
$
\end{minipage}
\hspace{-31pt}
\begin{minipage}[t]{0.3\textwidth}
$\begin{array}{c}
      [\![\Gamma \triangleright v: \mathbb{A} \otimes \mathbb{B} ]\!]=m  \quad [\![\Delta,x: \mathbb{A}, y: \mathbb{B}  \triangleright w: \mathbb{C} ]\!] =n  \quad E \in \text{Sf}(\Gamma;\Delta)\\
    \hline
  [\![ E\triangleright \text{pm } v \text{ to } x \otimes y. w :\mathbb{C}]\!] = n \cdot \text{jn}_{\Delta;\mathbb{A};\mathbb{B}} \cdot \alpha \cdot \text{sw}\cdot (m \otimes \text{id}) \cdot \text{sp}_{\Gamma;\Delta} \cdot \text{sh}_{E}
\end{array}
$ \end{minipage} \\
&
\begin{minipage}[t]{0.3\textwidth}
$\begin{array}{c}  
     [\![ \Gamma \triangleright v: \mathbb{A} ]\!] =m \quad [\![\Delta \triangleright w: \mathbb{B} ]\!]=n \quad E \in \text{Sf}(\Gamma;\Delta) \\
    \hline
  [\![ E \triangleright v \otimes w: \mathbb{A} \otimes \mathbb{B} ]\!] = (m \otimes n) \cdot \text{sp}_{\Gamma;\Delta} \cdot \text{sh}_{E}
\end{array} 
$
\end{minipage}\\
&
 \begin{minipage}[t]{0.3\textwidth}
$\begin{array}{c} 
    [\![\Gamma \triangleright v: \mathbb{I} ]\!]=m  \quad [\![\Delta \triangleright w: \mathbb{A}]\!]=n \quad E \in \text{Sf}(\Gamma;\Delta)  \\
    \hline
  [\![E \triangleright v \text { to } *.w: \mathbb{A} ]\!]=n \cdot \lambda \cdot (m \otimes \text{id}) \cdot \text{sp}_{\Gamma;\Delta} \cdot \text{sh}_{E}
\end{array}
$ \end{minipage} 
\hspace{130 pt}
\begin{minipage}[t]{0.3\textwidth}
$\begin{array}{c} 
     [\![\Gamma,x:\mathbb{A} \triangleright v: \mathbb{B} ]\!] = m \\
    \hline
   [\![ \Gamma \triangleright \lambda x:\mathbb{A} . \hspace{2pt } v: \mathbb{A} \multimap \mathbb{B}]\!] = \overline{m \cdot \text{jn}_{\Gamma;\mathbb{A}}}
\end{array}
$
\end{minipage} \\
&
 \begin{minipage}[t]{0.3\textwidth}
$\begin{array}{c} 
     [\![\Gamma \triangleright v: \mathbb{A} \multimap \mathbb{B} ]\!] = m \quad [\![  \Delta \triangleright w: \mathbb{A} ]\!] =n \quad E \in S\hspace{-3pt}f(\Gamma;\Delta)  \\
    \hline
  [\![ E \triangleright v w: \mathbb{A} ]\!] = \text{app} \cdot (m \otimes n) \cdot \text{sp}_{\Gamma;\Delta} \cdot \text{sh}_{E}
\end{array}
$ \end{minipage}
\hspace{190 pt} %[\![ ]\!]
\begin{minipage}[t]{0.3\textwidth}
$\begin{array}{c}
     [\![\Gamma \triangleright v: \mathbb{A}]\!]  = f \\
    \hline
   [\![ \Gamma \triangleright \text{dis}(v):  \mathbb{I} ]\!] = !_{[\![ \mathbb{A} ]\!]} \cdot f
\end{array}
$
\end{minipage}
\end{aligned}
\end{split}
\end{equation*}
\caption{Judgment interpretation}
\label{fig:denotational_sem}
\end{figure}

% Perguntar ao professor Renato: "Equations corresponding to the axiomatics of autonomous categories" e onde por o background de quantum computing. Also é para fazer glossário com simbolos matemáticos?

\section{Quantum Lambda Calculus}


In the case of quantum lambda calculus, which combines classical and quantum features, it is natural to consider two distinct basic data types: a type $\textit{bit}$ of classical bits and a type $\textit{qbit}$ of quantum bits.  The interpretation of these types is defined as  $[\![\textit{bit}]\!]=\mathbb{C}\oplus\mathbb{C}$ and $[\![\textit{qbit}]\!]=\mathbb{C}^{2\cdot 2}$. The type $\mathbb{I}$ is interpreted as $[\![\mathbb{I}]\!]=\mathbb{C}$.

The following operations are considered: $\textit{new} \hspace{2pt} 0  :\mathbb{I}  \multimap \textit{bit} $, $\textit{new} \hspace{2pt} 1  :\mathbb{I}  \multimap \textit{bit} $, $q : \textit{bit}  \multimap \textit{qbit}$, $\textit{meas}:\textit{qbit} \xrightarrow{} \textit{bit}$, and $\textit{U}:\textit{qbit},\ldots,\textit{qbit} \xrightarrow{} \textit{qbit}^{\otimes n}$. Their correspondent judgment interpretation is shown in \autoref{fig:interpret_ops}. 



\begin{figure}[H]
\begin{equation*}
\begin{split}
\begin{aligned}
&
\hspace{0pt}
\begin{minipage}[t]{0.45\textwidth}
$\begin{aligned}
  [\![\textit{new} \, 0 ]\!] : \hspace{2pt}& \mathbb{C} \multimap \llbracket \textit{bit} \rrbracket  \\
& 1 \mapsto (1,0)
\end{aligned}$
\end{minipage}
\hspace{-30pt}
\begin{minipage}[t]{0.45\textwidth}
$\begin{aligned}
  [\![\textit{new} \, 1 ]\!] :\hspace{2pt}& \mathbb{C} \multimap \llbracket \textit{bit} \rrbracket  \\
  & 1 \mapsto (0,1)
\end{aligned}$
\end{minipage} 
\hspace{-30pt}
\begin{minipage}[t]{0.45\textwidth}
$\begin{aligned}
  [\![q ]\!] : \hspace{2pt}&\llbracket \textit{bit} \rrbracket \multimap \llbracket \textit{qbit} \rrbracket\\
   &(a,b) \mapsto \big(\begin{smallmatrix}
  a & 0\\
  0 & b
\end{smallmatrix}\big) 
\end{aligned}$
\end{minipage} \\
&
\hspace{0pt}
\begin{minipage}[t]{0.45\textwidth}
$\begin{aligned}
  [\![\textit{meas}]\!]:\hspace{2pt} & \llbracket \textit{qbit} \rrbracket \xrightarrow{} \llbracket \textit{bit} \rrbracket  \\
  &\rho \mapsto ( \text{Tr} (M_{0} \rho M_{0}^{\dag}), \text{Tr} (M_{1} \rho M_{1}^{\dag})) 
\end{aligned}$
\end{minipage} 
\hspace{118pt}
\begin{minipage}[t]{0.45\textwidth}
$\begin{aligned}
  [\![\textit{U} ]\!] : \hspace{2pt} & \llbracket \textit{qbit} \rrbracket^{\otimes n} \xrightarrow{} \llbracket 
  \textit{qbit} \rrbracket^{\otimes n} \\
  & \rho \mapsto U \rho \hspace{2pt}  U^{\dag}
% & \rho,...,\rho_{n} \mapsto U \hspace{-2pt}\left(\bigotimes_{i=1}^{n}\rho_{i}\right ) U^{\dag}
\end{aligned}$
\end{minipage} \\
\end{aligned}
\end{split}
\end{equation*}
\caption{Judgment interpretation of the operations in quantum lambda calculus.}
\label{fig:interpret_ops}
\end{figure}
\chapter{The problem and its challenges}


\paragraph{Proof} In order to validate the metric equational system for conditionals, it is necessary to demonstrate its correctness.

The diamond norm is a particular instance of the operator norm. 
\vspace{15pt}

  The norm of a tuple is defined as the sum of the norms of its components, \textit{i.e.}, for any operators $v$ and $w$:
\begin{equation} \label{eq:norm_tuple}
  \lVert (v,w) \rVert = \lVert v \rVert + \lVert w \rVert
\end{equation}

For the \textbf{injections}:

Firstly, it is necessary to prove that the identity operator $I$ has a norm equal to 1.
\begin{lemma} \label{lemid}
  $ \lVert I \rVert_{\sigma} = 1   $
\end{lemma}

\textit{Proof.} \quad Using the definition of operator norm in \autoref{eq:op_norm}, it follows that:
\begin{equation} 
\begin{split}
  \lVert I \rVert_{\sigma} = \text{sup} \{\lVert I (v) \rVert \hspace{2pt} \vert \hspace{2pt}  \lVert v\rVert =1 \} = \text{sup} \{\lVert v \rVert \hspace{2pt} \vert \hspace{2pt}  \lVert v\rVert =1 \} = 1
\end{split}
\end{equation}

\vspace{10pt}

Thereafter, it is imperative to show that the injection operators $\textsc{Il}$ and $\textsc{Ir}$ are have a norm equal to 1.

\begin{lemma} \label{lemil}
  $ \lVert \textsc{Il} \rVert_{\sigma} = 1   $
\end{lemma}

\begin{lemma} \label{lemir}
  $ \lVert \textsc{Ir} \rVert_{\sigma} = 1   $
\end{lemma} 

\textit{Proof.} \quad Employing the definition of operator norm as defined in \autoref{eq:op_norm}, it ensues that:
\begin{equation} 
\begin{split}
  \lVert \textsc{Il} \rVert_{\sigma} &= \text{sup} \{\lVert \textsc{Il} (v) \rVert \hspace{2pt} \vert \hspace{2pt}  \lVert v\rVert =1 \} = \text{sup} \{\lVert (v,0) \rVert \hspace{2pt} \vert \hspace{2pt}  \lVert v\rVert =1 \} = \text{sup} \{\lVert v \rVert + \lVert 0 \rVert  \hspace{2pt} \vert \hspace{2pt}  \lVert v\rVert =1 \} \\
  & = \text{sup} \{\lVert v \rVert \hspace{2pt} + 0    \hspace{2pt}  \vert \lVert v\rVert =1 \} \hspace{160 pt} \text{ \{Positive definiteness\}} \\
  & = \text{sup} \{\lVert v \rVert \hspace{2pt} \vert \hspace{2pt}  \lVert v\rVert =1 \} = 1
\end{split}
\end{equation}

The proof for \autoref{lemir} is analogous to the proof for \autoref{lemil}.
\begin{equation} 
  \begin{split}
    \lVert \textsc{Ir} \rVert_{\sigma} &= \text{sup} \{\lVert \textsc{Ir} (v) \rVert \hspace{2pt} \vert \hspace{2pt}  \lVert v\rVert =1 \} = \text{sup} \{\lVert (0,v) \rVert \hspace{2pt} \vert \hspace{2pt}  \lVert v\rVert =1 \} = \text{sup} \{ \lVert 0 \rVert +\lVert v \rVert   \hspace{2pt} \vert \hspace{2pt}  \lVert v\rVert =1 \} \\
    & = \text{sup} \{0+\lVert v \rVert \hspace{2pt}     \hspace{2pt}  \vert \lVert v\rVert =1 \} \hspace{160 pt} \text{ \{Positive definiteness\}} \\
    & = \text{sup} \{\lVert v \rVert \hspace{2pt} \vert \hspace{2pt}  \lVert v\rVert =1 \} = 1
  \end{split}
  \end{equation}

Futhermore, given the submultiplicative property of the operator norm, for any super-operators $P$ and $Q$,where $\lVert P \rVert_{\sigma} =1  $ the following holds:
\begin{lemma}\label{lemleq}
  $\lVert PQ \rVert_{\sigma} \leq  \lVert Q \rVert_{\sigma}, \quad \lVert P \rVert_{\sigma}  =1 $ 
\end{lemma}

Using these properties it is possible to prove the validity of the metric equations for the injections. Demonstrating the correctness of the metric equations for the injections is equivalent to proving that for any  non‑negative rational $q$ and super-operators $v$ and $w$ such that $d(v,w) \leq q$, where  $d(v,w)$ represents the distance between $v$ and $w$ the following holds:

\begin{theorem} \label{theoremil}
  $d(\textsc{Il}(v),\textsc{Il} (w)) \leq q$
\end{theorem}
\begin{theorem} \label{theoremir}
  $d(\textsc{Ir}(v),\textsc{Ir} (w)) \leq q$
\end{theorem}
\vspace{10pt}
\textit{Proof.} \quad In the quantum paradigm, the distance between two super-operators $E$ and $E'$ corresponds to the diamond norm between $E$ and $E'$. Therefore,
\begin{equation}
\begin{split}
  d(v,w) \leq q \Leftrightarrow \lVert v \otimes I - w \otimes I \rVert_{\sigma} \leq q
\end{split}
\end{equation}

As a result, to prove that $d(\textsc{Il}(v),\textsc{Il} (w)) \leq q$, it suffices to show that:
\begin{align}
  \lVert \textsc{Il}\otimes I (v \otimes I)-\textsc{Il} \otimes I (w \otimes I)\rVert_{\sigma} \leq \lVert v \otimes I - w \otimes I \rVert_{\sigma} \\
  \lVert \textsc{Ir}\otimes I (v \otimes I)-\textsc{Ir} \otimes I (w \otimes I)\rVert_{\sigma} \leq \lVert v \otimes I - w \otimes I \rVert_{\sigma} 
\end{align}
Given that $\textsc{Il}$ and $\textsc{Ir}$ possess a norm equal to 1, as established by Lemmas \ref{lemil} and \ref{lemir} respectively, and considering the multiplicative property of the operator norm with respect to tensor products alongside the fact that the identity operator also exhibits a norm equal to 1, as demonstrated in  \autoref{lemid}, it follows that both $\lVert \textsc{Il} \otimes I \rVert_{\sigma}$ and $\lVert \textsc{Ir} \otimes I \rVert_{\sigma}$ are equal to one 1. Hence, by \autoref{lemleq},
\begin{align}
   \lVert \textsc{Il}\otimes I (v \otimes I)-\textsc{Il} \otimes I (w \otimes I)\rVert_{\sigma}=\lVert \textsc{Il}\otimes I (v \otimes I-w \otimes I)\rVert_{\sigma} \leq \lVert v \otimes I - w \otimes I \rVert_{\sigma} \\
   \lVert \textsc{Ir}\otimes I (v \otimes I)-\textsc{Ir} \otimes I (w \otimes I)\rVert_{\sigma}=\lVert \textsc{Ir}\otimes I (v \otimes I-w \otimes I)\rVert_{\sigma} \leq \lVert v \otimes I - w \otimes I \rVert_{\sigma}
\end{align}

\vspace{10pt}

Now, regarding the metric equation for the \textbf{conditional statement}, before validating its correctness, it is necessary to prove a few intermediate results. 

The first step is to demonstrate that for any super-operators $P$ and $Q$ the following holds:
\begin{lemma}\label{lem1}
  $\lVert [P,Q] \rVert_{\sigma} \leq \max \{ \lVert P \rVert_{\sigma}, \lVert Q \rVert_{\sigma} \}$
\end{lemma}



$\textit{Proof.}$ \quad Employing the definition of the operator norm in \autoref{eq:op_norm}, it follows that:
\begin{equation} \label{eq:cond_opnorm2}
  \begin{split}
  &\text{sup}{\{ \lVert [P,Q] (v) \rVert  \hspace{2pt} |  \hspace{2pt}  \lVert v \rVert=1  \}}  \leq \text{max} \{  \text{sup} \{ \lVert P (w) \rVert  \hspace{2pt} |  \hspace{2pt}  \lVert w \rVert =1 \}, \text{sup} \{\lVert Q (u) \rVert  \hspace{2pt} |  \hspace{2pt}  \lVert u \rVert=1  \} \} \\
  & = \text{sup}{\{ \lVert [P,Q] (w,u) \rVert  \hspace{2pt} |  \hspace{2pt}  \lVert w \rVert+ \lVert u \rVert=1  \}} \leq \text{max} \{  \text{sup} \{ \lVert P (w) \rVert  \hspace{2pt} |  \hspace{2pt}  \lVert w \rVert = 1, \lVert Q (u) \rVert  \hspace{2pt} |  \hspace{2pt}  \lVert u \rVert=1  \} \} \\
  & =  \text{sup}{\{ \lVert P (w) + Q (u) \rVert  \hspace{2pt} |  \hspace{2pt}  \lVert w \rVert+ \lVert u \rVert=1 \rVert=1  \}} \leq \text{max} \{  \text{sup} \{ \lVert P (w) \rVert  \hspace{2pt} |  \hspace{2pt}  \lVert w \rVert =1, \lVert Q (u) \rVert  \hspace{2pt} |  \hspace{2pt}  \lVert u \rVert=1  \} \} \\
  &  =  \text{sup}{\{ \lVert P (w) + Q (u) \rVert  \hspace{2pt} |  \hspace{2pt}  \lVert w \rVert+ \lVert u \rVert=1 \}} \leq \text{sup} \{  \text{max} \{ \lVert P (w) \rVert  \hspace{2pt} |  \hspace{2pt}  \lVert w \rVert =1, \lVert Q (u) \rVert  \hspace{2pt} |  \hspace{2pt}  \lVert u \rVert=1  \} \} \\
\end{split}
\end{equation}

Therefore, by the triangle inequality, proving the inequality in \autoref{eq:cond_opnorm3} suffices to establish  \autoref{lem1}.
\begin{equation} \label{eq:cond_opnorm3}
  \begin{split}
  \text{sup}{\{ \lVert P (w)  \rVert + \lVert Q (u)  \rVert  \hspace{2pt} |  \hspace{2pt}  \lVert w \rVert+ \lVert u \rVert=1  \}} \leq \text{sup} \{  \text{max} \{ \lVert P (w) \rVert  \hspace{2pt} |  \hspace{2pt}  \lVert w  \rVert =1, \lVert Q (u) \rVert  \hspace{2pt} |  \hspace{2pt}  \lVert u \rVert=1  \} \} \\
  \end{split}
\end{equation}


This can be rewritten as:

\begin{equation} 
  \begin{split}
    \lVert w \rVert+ \lVert u \rVert=1 \wedge \text{sup} \{ \lVert P (w)  \rVert + \lVert Q (u)  \rVert  \hspace{2pt}   \}  \leq \text{max}   \left\{ \dfrac{1}{\lVert w \rVert} \lVert P (w) \rVert  \hspace{2pt},  \dfrac{1}{\lVert u \rVert} \lVert Q (u) \rVert   \right\}
\end{split}
\end{equation}

As a result,
\begin{equation} 
  \begin{split}
    \lVert w \rVert+ \lVert u \rVert=1 \wedge \text{sup}{\{ \lVert P (w)  \rVert + \lVert Q (u)  \rVert    \}}  \leq \text{max}   \left\{  \left\lVert P \left( \dfrac{1}{\lVert w \rVert} w \right) \right\rVert  \hspace{2pt},  \left\lVert Q \left( \dfrac{1}{\lVert u \rVert} u \right) \right\rVert   \right\}
\end{split}
\end{equation}

This is equivalent to demonstrating that for $a+b=1$,
\begin{equation} 
\begin{split}
\hspace{110 pt}
    x + y  \leq  \max \left\{   \dfrac{1}{a}x  ,   \dfrac{1}{b} y   \right\} \\
\end{split}
\end{equation}

This is done by arguing by \textit{reductio ad absurdum}, \textit{i.e.}, supposing otherwise leads to a contradiction:
\begin{equation} 
\begin{split} 
    \hspace{90pt}&
     x + y  >  \max \left\{   \dfrac{1}{a}x  ,   \dfrac{1}{b} y   \right\} \\
    & \Rightarrow  x + y > \dfrac{1}{a}x  \wedge x + y > \dfrac{1}{b}y \\
    & \Rightarrow  a (x + y) > x  \wedge b (x + y)> y \\
    & \Rightarrow  a x + a y > x  \wedge b x + by > y \\
    & \Rightarrow  a x + a y > x  \wedge (1-a) x + (1-a)y > y\\
    & \Rightarrow  a x + a y > x  \wedge x-ax + y -ay > y\\
    & \Rightarrow  x < a x + a y   \wedge x > a x + a y  \\
\end{split}
\end{equation}

\vspace{10pt}

Subsequently, it is imperative to prove that:
\begin{lemma}\label{lemiso}
  $ i= [\textsc{Il} \otimes I, \textsc{Ir} \otimes I ]$ \text{is an isomorphism}.
\end{lemma}

\textit{Proof.} \quad The proof is as follows:

For any vector spaces $V$, $W$, and $U$, $i: (V \otimes U) \oplus (W \otimes U) \xrightarrow{} (V  \oplus W) \otimes U $. If $V$ has dimension $m$, $W$ has dimension $n$, and $U$ has dimension $o$, then the space $(V \otimes U) \oplus (W \otimes U) $ has dimension $mo+no=(m+n)\cdot o$. Similarly, the space $(V\oplus W) \otimes U$ has dimension $(m+n)\cdot o$. Hence, the spaces have the same dimension. Given that spaces with the same dimension are isomorphic \cite{hefferon2006linear}, it follows that $i$ is an isomorphism.

\vspace{10pt}

Next, it is necessary to demonstrate that for any operators $P$ and $Q$, the identity operator $I$, and an isomorphism $i=[\textsc{Il} \otimes I, \textsc{Ir} \otimes I ]$ the following holds:

\begin{lemma}\label{lem2}
  $( [P,Q] \otimes I) \cdot  i  = [P \otimes I, Q \otimes I]$
\end{lemma}

Which is equivalent to showing that for any vector spaces $V$, $W$, $U$, and $Z$  and super-operators $P: V \xrightarrow{} Z$, $Q: W \xrightarrow{} Z$, and $I: U \xrightarrow{} U$, the following diagram holds:

\vspace{10pt}


\begin{tikzpicture}
  \matrix (m) [matrix of math nodes,row sep=4em,column sep=7em,minimum width=2em]
  {
    V \otimes U \oplus W \otimes U & (V  \oplus W) \otimes U \\
     Z \otimes U \\
  };
  \path[-stealth]
    (m-1-1) edge node [left] {$[P \otimes I, Q \otimes I]$} (m-2-1)
    (m-1-1) edge node [above] {$i$} (m-1-2)
    (m-1-2) edge node [right=0.2cm] {$[P,Q] \otimes I$} (m-2-1);
\end{tikzpicture}


\vspace{10pt}

\textit{Proof.} \quad The proof is straightforward:
\begin{equation}
\begin{split}
    & ( [P,Q] \otimes I) \cdot  [\textsc{Il} \otimes I, \textsc{Ir} \otimes I ]  \\
    &=  [([P,Q] \otimes I) \cdot (\textsc{Il} \otimes I),([P,Q] \otimes I) \cdot (\textsc{Ir} \otimes I) ]\\
    &=  [P \otimes I, Q \otimes I]
\end{split}
\end{equation}

\vspace{15pt}

Furhtermore, it is imperative to show that the following relation holds:

\begin{lemma}\label{lemi-1}
  $ [P \otimes I, Q \otimes I] \cdot  i^{-1}  = [P,Q] \otimes I$
\end{lemma}

Demonstrating this is equivalent to establishing that for any vector spaces $V$, $W$, $U$, and $Z$, and super-operators $P: V \xrightarrow{} Z$, $Q: W \xrightarrow{} Z$, and $I: U \xrightarrow{} U$, the following diagram commutes:

\vspace{10pt}

\begin{tikzpicture}
  \matrix (m) [matrix of math nodes,row sep=4em,column sep=7em,minimum width=2em]
  {
    V \otimes U \oplus W \otimes U & (V  \oplus W) \otimes U \\
     Z \otimes U \\
  };
  \path[-stealth]
    (m-1-1) edge node [left] {$[P \otimes I, Q \otimes I]$} (m-2-1)
    (m-1-2) edge node [above] {$i^{-1}$} (m-1-1)
    (m-1-2) edge node [right=0.2cm] {$[P,Q] \otimes I$} (m-2-1);
\end{tikzpicture}


\textit{Proof.} \quad The proof is as follows:
\begin{equation}
\begin{split}
    & ( [P,Q] \otimes I) \cdot  i  = [P \otimes I, Q \otimes I]  \hspace{100pt} & \text{\{\autoref{lem2}\}} \\
    \Leftrightarrow &  \hspace{2pt} ( [P,Q] \otimes I) \cdot  i \cdot i^{-1} = [P \otimes I, Q \otimes I] \cdot  i^{-1}\\
    \Leftrightarrow &  \hspace{2pt} ( [P,Q] \otimes I)  = [P \otimes I, Q \otimes I] \cdot  i^{-1}  &\text{\{\autoref{lemiso}\}} \\
\end{split}
\end{equation}

\vspace{10pt}
With \autoref{lem2} and \autoref{lemi-1}, it has been proved that the diagram below is valid:
\vspace{5pt}

\begin{tikzpicture}
  \matrix (m) [matrix of math nodes,row sep=4em,column sep=7em,minimum width=2em]
  {
    V \otimes U \oplus W \otimes U & (V  \oplus W) \otimes U \\
     Z \otimes U \\
  };
  \path[-stealth]
    (m-1-1) edge node [left] {$[P \otimes I, Q \otimes I]$} (m-2-1)
    edge[bend left=5] node [above] {$i$}  (m-1-2) % Adjusted minimum width
    (m-1-2) edge node [right=0.5cm] {$[P,Q] \otimes I$} (m-2-1)
    (m-1-2) edge[bend right=-5] node [below] {$i^{-1}$} (m-1-1); % Added the label to the arrow
\end{tikzpicture}

\vspace{10pt}




%Next, it is necessary to demonstrate that the coproduct of two super-operators $P$ and $Q$ has a norm equal to 1.
%\begin{lemma} \label{lemeither}
  %$  \lVert [P, Q]  \rVert_{\sigma} = 1   $
%\end{lemma}

%\textit{Proof.} \quad Utilizing the definition of the operator norm as defined in Equation \ref{eq:op_norm}, it follows that:
%\begin{equation} 
  %\begin{split}
    %\lVert [P, Q]  \rVert_{\sigma}  \\
  %\end{split}
  %\end{equation}
%\vspace{10pt}

Now, it is possivel to prove that $i$ has a norm equal to 1.

\begin{lemma} \label{lem3}
  $  \lVert i\rVert_{\sigma} \geq 1 $
\end{lemma}

\vspace{10pt}

\textit{Proof.} \quad Considering the vector $(v \otimes u, 0)$ with $\lVert(v \otimes u, 0)\rVert = 1$, and  attending the multiplicative property of the operator norm with respect to tensor products, along with the definition of the norm of a tuple as in \autoref{eq:norm_tuple}, it holds that $\lVert v \rVert = 1$ and $\lVert u \rVert =1$. Therefore, using this same property and definition, it is possible to demonstrate that the following holds:
  \begin{equation}
    \begin{split}
      \lVert [\textsc{Il} \otimes I, \textsc{Ir} \otimes I ] (v \otimes u, 0) \rVert = \lVert(v, 0) \otimes u \rVert = (\lVert v \rVert + \lVert 0 \rVert ) \lVert u \rVert = \lVert v \rVert \lVert u \rVert =1
    \end{split}
  \end{equation}
 
Given the definition of the operator norm as presented in \autoref{eq:op_norm}, it follows that:
\begin{equation}
  \begin{split}
      & \hspace{3pt} \lVert [\textsc{Il} \otimes I, \textsc{Ir} \otimes I ]  \rVert_{\sigma}  = \text{sup} \{ \lVert [\textsc{Il} \otimes I , \textsc{Ir} \otimes I ] (a) \rVert \hspace{2pt} | \hspace{2pt} \lVert a \rVert = 1 \} \\
  \end{split}
  \end{equation}
  From this, it can be deduced that $\lVert i \rVert_{\sigma} \geq 1$.

Subsequently, it is possible to demontrate that $i^{-1}$ has a norm greater than or equal to 1,

\begin{lemma} \label{lem4}
  $  \lVert i^{-1}  \rVert_{\sigma} \leq 1 $
\end{lemma}

\textit{Proof.} \quad Given that $i$ is an isomophism, it follows that 
\begin{equation} 
  \begin{split}
    &\lVert i \cdot i^{-1}  \rVert_{\sigma} = 1  \\
    \leq \hspace{2pt}& \lVert i  \rVert_{\sigma} \cdot \lVert i^{-1}  \rVert_{\sigma} = 1 \hspace{50pt} & \text{\{Norm submultiplicative with respect to compositions\}} \\
    \leq & 1 \cdot \lVert i^{-1}  \rVert_{\sigma} = 1 & \text{\{\autoref{lem4}\}}  \\
    \Leftrightarrow &  \lVert i^{-1}  \rVert_{\sigma} = 1  \\
  \end{split}   
  \end{equation}




Next, one has to prove that for any super-operators $P$ and $Q$ and their respective erroneous versions $P'$ and $Q'$, the following holds:
  \begin{lemma} \label {lemmasum}
    $  \lVert P\cdot Q \otimes I - P'\cdot Q'  \otimes I \rVert_{\sigma} \leq  \lVert (P - P') \otimes I  \rVert_{\sigma} + \lVert (Q - Q') \otimes I \rVert_{\sigma}   $
  \end{lemma} 
  
  \textit{Proof.} \quad Applying the triangle inequality, he submultiplicative property of the operator norm with respect to compositions, and given that a positive and trace-preserving operator map, $E$, has norm $\lVert E \otimes I  \rVert_{\sigma} =1$ (\cite{watrous2018theory}), it follows that:
  
  \begin{equation}
    \begin{split}
      & \lVert P\cdot Q \otimes I - P'\cdot Q' \otimes I  \rVert_{\sigma}  \\
      &= \lVert  P\cdot Q \otimes I- P\cdot Q' \otimes I + P\cdot Q' \otimes I - P'\cdot Q' \otimes I  \rVert_{\sigma}  \\
      &\leq \lVert P\cdot Q \otimes I - P\cdot Q' \otimes I  \rVert_{\sigma} + \lVert P\cdot Q' \otimes I - P'\cdot Q' \otimes I  \rVert_{\sigma}  \\
      &\leq \lVert P \rVert_{\sigma} \lVert Q \otimes I - Q' \otimes I  \rVert_{\sigma} + \lVert P \otimes I - P' \otimes I  \rVert_{\sigma} \lVert Q'  \rVert_{\sigma}  \\
      &= \lVert P \rVert_{\sigma} \lVert (Q  - Q') \otimes I  \rVert_{\sigma} + \lVert (P  - P') \otimes I  \rVert_{\sigma} \lVert Q'  \rVert_{\sigma}  \\
      &= \lVert (P - P') \otimes I  \rVert_{\sigma} + \lVert (Q - Q') \otimes I  \rVert_{\sigma}  \\
    \end{split}
    \end{equation}

\vspace{5pt}

Finally, considering the the semantics  the conditional statement  in \autoref{fig:denotational_sem cond}, demonstrating the conditional statement rule in \autoref{fig:metric conditionals} includes proving that for any super-operators $P$, $Q$, $P'$ and $Q'$,  denoting the distance between super-operators $A$ and $B$ as $d(A,B)$,  the following holds:
\begin{lemma} \label {lemma_max_otimes}
  $\text{d} ([P,Q],[P',Q']) \leq \text{max} \{\text{d} (P,P'),\text{d} (Q,Q')\}$
\end{lemma}
\vspace{10pt}
\textit {Proof.} 
In the quantum paradigm, the distance between two super-operators  corresponds to the diamond norm between the two super-operators. Hence, denoting $ [\textsc{Il} \otimes I, \textsc{Ir} \otimes I ]$ by $i$ it follows that:

%\begin{equation}
%\begin{split}
  %& \text{d} ([P,Q],[P',Q'])  \\
  %&=   \lVert  [P,Q] \otimes I - [P',Q'] \otimes I   \rVert_{1}  \\
  %&=   \lVert [P \otimes I, Q \otimes I]  - [P' \otimes I, Q' \otimes I]  \rVert_{1}  \\
  %&=  \lVert [P - P' \otimes I, Q-Q' \otimes I]  \rVert_{1}   \\
  %&= \lVert [P -P', Q-Q' ] \otimes I \cdot i \rVert_{1}  \\
%\end{split}
%\end{equation}

\begin{equation} \label{eq:proof_theorem1.1_esq}
  \begin{split}
    & \text{d} ([P,Q],[P',Q'])  \\
    &=  \lVert  [P,Q] \otimes I - [P',Q'] \otimes I   \rVert_{\sigma}  \\
    &=   \lVert [P \otimes I, Q \otimes I] \cdot i^{-1}  - [P' \otimes I, Q' \otimes I]  \cdot i^{-1}  \rVert_{\sigma}   \hspace{165pt}  \text{\{\autoref{lemi-1}\}} \\
    &=  \lVert [P - P' \otimes I, Q-Q' \otimes I] \cdot i^{-1}  \rVert_{\sigma}   \\
    & \leq \lVert [P - P' \otimes I, Q-Q' \otimes I]  \rVert \lVert i^{-1}  \rVert \rVert_{\sigma} \hspace{20pt} \text{\{Norm submultiplicative with respect to compositions\}}  \\  
    & \leq \lVert [(P - P') \otimes I, (Q-Q') \otimes I]  \rVert_{\sigma} \hspace{235pt} \text{ \{\autoref{lem4}\}} \\
  \end{split}
  \end{equation}
and
\begin{equation} \label {eq:proof_theorem1.1_dir}
\begin{split}
   &  \text{max} \{\text{d} (P,P'),\text{d} (Q,Q')\} \\
   = &  \text{max}\{ \lVert P \otimes I - P' \otimes I \rVert_{\sigma}, \lVert Q \otimes I - Q'\otimes I \rVert_{\sigma} \}\\
   = &  \text{max}\{ \lVert (P - P') \otimes I \rVert_{\sigma}, \lVert (Q - Q') \otimes I \rVert_{\sigma} \}\\
\end{split}
\end{equation}

Finally, by  \autoref{lem1}, it can be deduced that $\text{d} ([P,Q],[P',Q']) \leq \text{max} \{\text{d} (P,P'),\text{d} (Q,Q')\}$, which concludes the proof of theorem \autoref{lemma_max_otimes}.
\vspace{10pt}


An alternative method to establish \autoref{theorem:1.1} is now presented.
\vspace{5pt}


\textit {Proof.} The proof is as follows:
\begin{equation}
  \begin{split}
    & \text{d} ([P,Q],[P',Q'])  \\
    &=   \lVert  [P,Q] \otimes I - [P',Q'] \otimes I    \rVert_{\sigma} \hspace{2pt} \\
    &=   \lVert  ([P,Q]  - [P',Q']) \otimes I    \rVert_{\sigma} \hspace{2pt} \\
    &=   \lVert  [P-P',Q-Q'] \otimes I  \rVert_{\sigma}   \\
    &=    \lVert  [P-P',Q-Q'] \rVert_{\sigma} \lVert I \rVert_{\sigma}\hspace{2pt} & \hspace {20pt} \text{\{Norm multiplicative with respect to tensor products\}} \\ 
    &=    \lVert  [P-P',Q-Q'] \rVert_{\sigma} & \text{\{\autoref{lemid}\}}  \\
  \end{split}
  \end{equation}
Moreover,
\begin{equation}
  \begin{split}
     &  \text{max} \{\text{d} (P,P'),\text{d} (Q,Q')\} \\
     = &  \text{max}\{ \lVert P \otimes I - P' \otimes I \rVert_{\sigma}, \lVert Q \otimes I - Q'\otimes I \rVert_{\sigma} \}\\
     = &  \text{max}\{ \lVert (P - P') \otimes I \rVert_{\sigma}, \lVert (Q - Q') \otimes I \rVert_{\sigma} \}\\
     = &\text{max}\{ \lVert (P - P') \rVert_{\sigma} \lVert  I \rVert_{\sigma}, \lVert (Q - Q') \rVert_{\sigma} \lVert I \rVert_{\sigma} \} & \hspace{60pt} \text{\{Norm multiplicative with}\\
     && \text{respect to tensor products\}} \\
     = & \text{max}\{ \lVert (P - P') \rVert_{\sigma}, \lVert (Q - Q') \rVert_{\sigma}  \}  & \text{\{\autoref{lemid}\}}  \\
    \end{split}
  \end{equation}

Therefore, by \autoref{lem1}, it can be deduced that $\text{d} ([P,Q],[P',Q']) \leq \text{max} \{\text{d} (P,P'),\text{d} (Q,Q')\}$, which concludes the proof of theorem \autoref{lemma_max_otimes}.

\vspace{10pt}
  


Now, it is finally possible to adress the proof of the metric equation for the conditional statement as a whole. Considering the the semantics of the conditional statement in \autoref{fig:denotational_sem cond}, the rule for the conditional statement in \autoref{fig:metric conditionals} is valid is equivalent to demonstrating that the distance between the evalution of a boolen $B$ followed by the execution of a program $P$ or a program $Q$ and the evalution of a boolean $B'$ followed by the execution of a program $P'$ or a program $Q'$ is less or equal to the  distance between the evaluation of the boolean $B$ and the evaluation of the boolean $B'$ plus the maximum distance between the execution of the programs $P$ and $P'$ and the execution of the programs $Q$ and $Q'$, \textit{ergo}, that for any booleand $B$ and $B'$ super-operators $P$, $Q$, $P'$ and $Q'$, the following holds:

\begin{theorem} \label {theorem:1.1}
  $ \text{d} (B \cdot [P,Q], B' \cdot [P',Q']) \leq \text{d} (B,B') + \text{max} \{\text{d} (P,P'),\text{d} (Q,Q')\}$
\end{theorem}
\vspace{10pt}

\textit {Proof.} Considering that in the quantum paradigm, the distance between two super-operators  corresponds to the diamond norm between the two super-operators, it follows that:
\begin{equation}
\begin{split}
  & \text{d} (B \cdot [P,Q], B' \cdot [P',Q'])  \\
  &=   \lVert  B \cdot [P,Q] \otimes I - B' \cdot [P',Q'] \otimes I   \rVert_{\sigma}  \\
  & \leq \lVert  (B - B')  \otimes I   \rVert_{\sigma} + \lVert  ([P,Q] - [P',Q']) \otimes I   \rVert_{\sigma} & \hspace{100 pt}  \text{\{\autoref{lemmasum}\}} \\
  &= d(B,B') + \lVert  [P,Q]\otimes I - [P',Q'] \otimes I   \rVert_{\sigma} & \hspace{100 pt} \\
  &=  \text{d} (B,B') + \text{d} ([P,Q],[P',Q'])    \\
  &=d(B,B') + \text{max} \{\text{d} (P,P'),\text{d} (Q,Q')\} & \text{\{\autoref{lemma_max_otimes}\}} \\ 
\end{split}
\end{equation}

%
%
%
%
%
%
%
%
%
% Cenas necessárias v1

%
%
%
%
%
%
%
%
%
% lemmas v1

\begin{theorem} \label{theorem:tensor_stability} 
  For all super-operators $Q: \mathbb{C}^{o_1 \times o_1} \oplus \ldots \oplus \mathbb{C}^{o_n \times o_n}  \rightarrow \mathbb{C}^{p_1 \times p_1} \oplus \ldots \oplus  \mathbb{C}^{p_m \times p_m}$ and complex spaces $\mathbb{C}^{q_1 \times q_1} \oplus \ldots \oplus \mathbb{C}^{q_t \times q_t}$  it holds that:
      \begin{equation}
        \lVert Q \otimes I_{\mathbb{C}^{q_1 \times q_1} \oplus \ldots \oplus \mathbb{C}^{q_t \times q_t}} \rVert_{1 \text{ gen}} \leq  \lVert Q \rVert_{\diamondsuit \text{ gen}}
      \end{equation}
      with equality holding under the assumption that $\text{dim}(\mathbb{C}^{q_1 \times q_1} \oplus \ldots \oplus \mathbb{C}^{q_t \times q_t}) \geq  \text{dim}(\mathbb{C}^{o_1 \times o_1} \oplus \ldots \oplus \mathbb{C}^{o_n \times o_n})$.
      \end{theorem}
  
  \begin{proof}
    \begin{align*}
      & \hspace{-30pt} \lVert Q \otimes I_{\mathbb{C}^{q_1 \times q_1} \oplus \ldots \oplus \mathbb{C}^{q_t \times q_t}} \rVert_{1 \text{ gen}} \\
      & \hspace{-30pt}  = \max \{ \lVert Q_{11}  \otimes I_{\mathbb{C}^{q_1 \times q_1}}  \rVert_{1}  + \ldots +  \lVert Q_{1m}  \otimes I_{\mathbb{C}^{q_1 \times q_1}} \rVert_{1},  \ldots ,  \lVert Q_{11} \otimes I_{\mathbb{C}^{q_t \times q_t}} \rVert_{1}   \quad \{\text{\autoref{eq:gen_tensor_identity}, } \\
      & \hspace{-18pt}  + \ldots + \lVert Q_{1m} \otimes I_{\mathbb{C}^{q_t \times q_t}} \rVert_{1}, \ldots,  \lVert Q_{n1}  \otimes I_{\mathbb{C}^{q_1 \times q_1}}  \rVert_{1} + \ldots +  \hspace{80 pt} \text{\autoref{def:gen_1norm}} \} \\
      & \hspace{-16pt}\lVert Q_{nm}  \otimes I_{\mathbb{C}^{q_1 \times q_1}} \rVert_{1}, \ldots,  \lVert Q_{n1} \otimes I_{\mathbb{C}^{q_t \times q_t}} \rVert_{1} + \ldots + \lVert Q_{nm} \otimes I_{\mathbb{C}^{q_t \times q_t}} \rVert_{1} \} \\
      &  \hspace{-30pt}  \leq   \max \{ \|Q_{11}\|_{\diamondsuit} + \ldots + \|Q_{1m}\|_{\diamondsuit}, \hspace{2pt} \ldots \hspace{2pt},  \|Q_{11}\|_{\diamondsuit} + \ldots + \|Q_{1m}\|_{\diamondsuit}, , \hspace{2pt} \ldots \hspace{2pt},   \hspace{12 pt}  \{ \text{\autoref{thm:tensor_stability}}\}\\
      & \|Q_{n1}\|_{\diamondsuit} + \ldots + \|Q_{nm}\|_{\diamondsuit}, \hspace{2pt} \ldots \hspace{2pt}, \|Q_{n1}\|_{\diamondsuit} + \ldots + \|Q_{nm}\|_{\diamondsuit} \} \\
      &  \hspace{-30pt} = \max \{\|Q_{11}\|_{\diamondsuit} + \ldots + \|Q_{1m}\|_{\diamondsuit}, \hspace{2pt} \ldots \hspace{2pt}, \|Q_{n1}\|_{\diamondsuit} + \ldots + \|Q_{nm}\|_{\diamondsuit} \} \\
      & \hspace{-30pt}  = \lVert Q \rVert_{\diamondsuit \text{ gen}}  \hspace{308 pt}  \{ \text{\autoref{def:gen_diamond_norm}}\}
    \end{align*} 
  
  Note that if  $\text{dim}(\mathbb{C}^{q_1 \times q_1} \oplus \ldots \oplus \mathbb{C}^{q_t \times q_t}) \geq  \text{dim}(\mathbb{C}^{o_1 \times o_1} \oplus \ldots \oplus \mathbb{C}^{o_n \times o_n})$, at least one of the vector spaces ${\mathbb{C}^{q_k \times q_k}}$, in the direct sum $\mathbb{C}^{q_1 \times q_1} \oplus \ldots \oplus \mathbb{C}^{q_t \times q_t}$ has higher dimension that any of the vector spaces ${\mathbb{C}^{q_l \times q_l}}$, in the direct sum $\mathbb{C}^{o_1 \times o_1} \oplus \ldots \oplus \mathbb{C}^{o_n \times o_n}$. Consequentlty, in such a case, there is a vector space $\mathbb{C}^{q_k \times q_k}$ such that, by \autoref{thm:tensor_stability},  $ \lVert Q_{ij} \otimes I_{\mathbb{C}^{q_k \times q_k}} \rVert_1 = {\|Q_{ij}\|_{\diamondsuit}}$ for all $1\leq i \leq n$, $1 \leq j \leq m$. As a result, when $\text{dim}(\mathbb{C}^{q_1 \times q_1} \oplus \ldots \oplus \mathbb{C}^{q_t \times q_t}) \geq  \text{dim}(\mathbb{C}^{o_1 \times o_1} \oplus \ldots \oplus \mathbb{C}^{o_n \times o_n})$,
  
    \begin{align*}
      & \hspace{-30pt} \lVert Q \otimes I_{\mathbb{C}^{q_1 \times q_1} \oplus \ldots \oplus \mathbb{C}^{q_t \times q_t}} \rVert_{1 \text{ gen}} \\
      & \hspace{-30pt}  = \max \{ \lVert Q_{11}  \otimes I_{\mathbb{C}^{q_1 \times q_1}}  \rVert_{1}  + \ldots +  \lVert Q_{1m}  \otimes I_{\mathbb{C}^{q_1 \times q_1}} \rVert_{1},  \ldots ,  \lVert Q_{11} \otimes I_{\mathbb{C}^{q_t \times q_t}} \rVert_{1}   \quad \{\text{\autoref{eq:gen_tensor_identity}, } \\
      & \hspace{-18pt}  + \ldots + \lVert Q_{1m} \otimes I_{\mathbb{C}^{q_t \times q_t}} \rVert_{1}, \ldots,  \lVert Q_{n1}  \otimes I_{\mathbb{C}^{q_1 \times q_1}}  \rVert_{1} + \ldots +  \hspace{80 pt} \text{\autoref{def:gen_1norm}} \} \\
      & \hspace{-16pt}\lVert Q_{nm}  \otimes I_{\mathbb{C}^{q_1 \times q_1}} \rVert_{1}, \ldots,  \lVert Q_{n1} \otimes I_{\mathbb{C}^{q_t \times q_t}} \rVert_{1} + \ldots + \lVert Q_{nm} \otimes I_{\mathbb{C}^{q_t \times q_t}} \rVert_{1} \} \\
      &  \hspace{-30pt}  = \max \{ \|Q_{11} \otimes \mathbb{C}^{q_k \times q_k}\|_{1} + \ldots + \|Q_{1m} \otimes \mathbb{C}^{q_k \times q_k}\|_{1}, \hspace{2pt} \ldots \hspace{2pt} , \hspace {80 pt}  \{ \text{\autoref{thm:tensor_stability}}\}\\
      & \hspace{-16pt} \|Q_{n1} \otimes \mathbb{C}^{q_k \times q_k}\|_{1} + \ldots + \|Q_{nm} \otimes \mathbb{C}^{q_k \times q_k}\|_{1} \} \\
      &  \hspace{-30pt} = \max \{\|Q_{11}\|_{\diamondsuit} + \ldots + \|Q_{1m}\|_{\diamondsuit}, \hspace{2pt} \ldots \hspace{2pt}, \|Q_{n1}\|_{\diamondsuit} + \ldots + \|Q_{nm}\|_{\diamondsuit} \} \hspace{40pt}\{ \text{\autoref{thm:tensor_stability}}\} \\
      & \hspace{-30pt}  = \lVert Q \rVert_{\diamondsuit \text{ gen}}  \hspace{308 pt}  \{ \text{\autoref{def:gen_diamond_norm}}\}
    \end{align*} 
  \end{proof}
  
  \begin{corollary} \label{cor:tensor_stability}
    For all super-operators  $Q: \mathbb{C}^{o_1 \times o_1} \oplus \ldots \oplus \mathbb{C}^{o_n \times o_n}  \rightarrow \mathbb{C}^{p_1 \times p_1} \oplus \ldots \oplus  \mathbb{C}^{p_m \times p_m}$ and complex spaces $\mathbb{C}^{q_1 \times q_1} \oplus \ldots \oplus \mathbb{C}^{q_t \times q_t}$   it holds that:
  \begin{equation}
     \lVert Q \otimes I_{\mathbb{C}^{q_1 \times q_1} \oplus \ldots \oplus \mathbb{C}^{q_t \times q_t}} \rVert_{\diamondsuit \text{ gen}} = \lVert Q \rVert_{\diamondsuit \text{ gen}} 
  \end{equation}
  \end{corollary}
  
  \begin{lemma}\label{lem:q(o)}
    Let  $Q: \mathbb{C}^{o_1 \times o_1} \oplus \ldots \oplus \mathbb{C}^{o_n \times o_n}  \rightarrow \mathbb{C}^{p_1 \times p_1} \oplus \ldots \oplus  \mathbb{C}^{p_m \times p_m}$ be a superoperator, then for $O \in \mathbb{C}^{o_1 \times o_1} \oplus \ldots \oplus  \mathbb{C}^{o_m \times o_m}$ it holds that:
    \begin{equation} \label{eq:qo<q}
      \lVert Q(O) \rVert_{1 \text{ gen}} \leq \lVert Q  \rVert_{1 \text{ gen}} \cdot \lVert O  \rVert_{1 \text{ gen}}.
    \end{equation}
  \end{lemma}
  
  \begin{proof}
    $O$ can be written as  $O = \underbrace{\textsc{Il} \cdot \ldots \cdot \textsc{Il}}_{n-1 \times} \cdot \hspace{2pt} O_{1} + \ldots +  \underbrace{\textsc{Ir} \cdot \ldots \cdot \textsc{Ir}}_{n-1 \times} \cdot \hspace{2pt} O_{n} 
  $, where for each $1 \leq i \leq n$, $O_{i} \in \mathbb{C}^{p_i \times p_i}$ and $O_i =  \underbrace{\textsc{Pl} \cdot \ldots \cdot \textsc{Pl}}_{n-i \times} \cdot \underbrace{\textsc{Pr} \cdot \ldots \cdot \textsc{Pr}}_{i-1 \times} \cdot O $. Considering \autoref{def:gen_norm}, it follows that: 
  \begin{equation}
    \lVert O  \rVert_{1 \text{ gen}} = \lVert O_1 \rVert_{1} + \ldots + \lVert O_n \rVert_{1}.
  \end{equation}
  
  Applying $O$ to $Q$ results in:
  \begin{equation}
  \begin{split}
  Q(O) & = \underbrace{\textsc{Il} \cdot \ldots \cdot \textsc{Il}}_{m-1 \times} \cdot \hspace{1pt} Q_{11}  (O_{1}) + \ldots +   \underbrace{\textsc{Ir} \cdot \ldots \cdot \textsc{Ir}}_{m-1 \times}\cdot \hspace{1pt} Q_{1m} (O_{1}) + \ldots +  \underbrace{\textsc{Il} \cdot \ldots \cdot \textsc{Il}}_{m-1 \times} \cdot\hspace{1pt} Q_{n1} (O_{n}) +  \ldots \\
  & \hspace{10pt}  + \underbrace{\textsc{Ir} \cdot \ldots \cdot \textsc{Ir}}_{m-1 \times}\cdot \hspace{1pt} Q_{nm}  (O_{n})
  \end{split}
  \end{equation}
  
  As a result, the generalized trace norm of $Q(O)$ corresponds to:
  \begin{equation} \label{eq:qo}
    \begin{split}
    \lVert Q(O)  \rVert_{1 \text{ gen}} & = \lVert Q_{11} (O_{1}) \rVert_{1} + \ldots + \lVert Q_{1m} (O_{1}) \rVert_{1} + \ldots +  \lVert Q_{n1} (O_{m})  \rVert_{1} +  \ldots +  \lVert Q_{nm} (O_{n}) \rVert_{1}. 
    \end{split}
  \end {equation}
  The generalized trace norm of $Q$ is given by:
  \begin{equation}
    \begin{split} \label{eq:q}
    &\lVert Q  \rVert_{1 \text{ gen}} = \max \{ \lVert Q_{11} \rVert_{1} + \ldots + \lVert Q_{1m} \rVert_{1}, \hspace{2pt} \ldots \hspace{2pt}, \lVert Q_{n1} \rVert_{1} + \ldots + \lVert Q_{nm} \rVert_{1} \} \\
   & = \max \{ \max \{ \lVert Q_{11} (A_{1}) \rVert_{1} \hspace{1pt}  \vert \hspace{1pt}  \lVert A_{1} \rVert_{1} = 1 \} + \ldots + &  \{\text{\autoref{def:trace_norm_superoperator}}\} \\
   & \hspace{15pt}  \max \{  \lVert Q_{1m} (A_{1}) \rVert_{1} \hspace{1pt}  \vert   \lVert A_{1} \rVert_{1} = 1 \} , \hspace{1pt} \ldots \hspace{1pt}, \max \{ \lVert Q_{n1} (A_{n}) \rVert_{1} \hspace{1pt}  \vert \hspace{1pt}  \lVert A_{n} \rVert_{1} = 1 \}   \\
   &\hspace{15pt} + \ldots +  \max \{ \lVert Q_{nn} (A_{n}) \rVert_{1} \hspace{1pt}  \vert \lVert A_{n} \rVert_{1} = 1 \}\}\\
   & = \max \{ \lVert Q_{11} (A_{1}) \rVert_{1} \hspace{1pt}  + \ldots +  \lVert Q_{1m} (A_{1}) \rVert_{1} \hspace{1pt} , \hspace{2pt} \ldots \hspace{2pt}, \lVert Q_{n1} (A_{n}) \rVert_{1}  + \ldots +  \\
   & \hspace{15pt} \lVert Q_{nm} (A_{n}) \rVert_{1} \hspace{1pt}  \vert \hspace{1pt}   \lVert A_{1} \rVert_{1} = 1, \ldots, \lVert A_{n} \rVert_{1} = 1 \}
    \end{split}
  \end{equation}
  
  
  Thus, if  $\lVert O \rVert_{1 \text{ gen}} = 1$,
  \begin{align*}
    \hspace{-30pt}&  \lVert O_1  \rVert_1  + \ldots + \lVert O_n  \rVert_1 = 1  \wedge \lVert Q(O) \rVert_{1 \text{ gen}} \leq  \lVert Q \rVert_{1 \text{ gen}} \\
    \hspace{-30pt} \Leftrightarrow  & \lVert O_1  \rVert_1  + \ldots + \lVert O_n  \rVert_1 = 1  \wedge  \lVert Q_{11} (O_{1}) \rVert_{1} + \ldots +  \lVert Q_{1m} (O_{1}) \rVert_{1} + \ldots +  \hspace{10pt}&   \{\text{\autoref{eq:qo}}, \\
    \hspace{-30pt} & \lVert Q_{n1} (O_{m}) \rVert_{1} +  \ldots  + \lVert Q_{nm} (O_{n}) \rVert_{1}   & \text{\autoref{eq:q}} \} \\
    \hspace{-30pt}& \leq \max \{ \lVert Q_{11} (A_{1}) \rVert_{1} \hspace{1pt} + \ldots +  \lVert Q_{1m} (A_{1}) \rVert_{1}, \hspace{2pt} \ldots \hspace{2pt},  \lVert Q_{n1} (A_{n}) \rVert_{1} + \ldots +   \\
    \hspace{-30pt}&\lVert Q_{nm} (A_{n}) \rVert_{1} \hspace{1pt}  \vert  \lVert A_{1} \rVert_{1} = 1, \ldots, \lVert A_{n} \rVert_{1} = 1 \} \\
    %\hspace{-30pt} \Leftrightarrow  & \lVert O_1  \rVert_1  + \ldots + \lVert O_n  \rVert_1 = 1  \wedge  \lVert Q_{11} \cdot O_{1} \rVert_{1} + \ldots +  \lVert Q_{1m} \cdot O_{1} \rVert_{1} + \ldots +  \hspace{10pt}  \\
    %\hspace{-30pt}& \leq \max \Bigg\{ \left\lVert Q_{11} \left(\frac{O_{1}} {\lVert O_{1} \rVert_1}\right) \right\rVert_{1} \hspace{1pt} + \ldots +  \left\lVert Q_{1m} \left(\frac{O_{1}} {\lVert O_{1} \rVert_1}\right)  \right\rVert_{1} \hspace{1pt}, \hspace{2pt} \ldots \hspace{2pt},   \\
    %\hspace{-30pt}& \left\lVert Q_{n1}  \left(\frac{O_{n}} {\lVert O_{1} \rVert_1}\right) \right\rVert_{1} + \ldots + \left\lVert Q_{nm}  \left(\frac{O_{n}} {\lVert O_{1} \rVert_1}\right) \right\rVert_{1} \hspace{1pt}  \Bigg\}  \\
    \hspace{-30pt} \Leftrightarrow  & \lVert O_1  \rVert_1  + \ldots + \lVert O_n  \rVert_1 = 1  \wedge  \lVert Q_{11} (O_{1}) \rVert_{1} + \ldots +  \lVert Q_{1m} (O_{1}) \rVert_{1} + \ldots +  \hspace{10pt}\\
    \hspace{-30pt} & \lVert Q_{n1} (O_{m}) \rVert_{1} +  \ldots  + \lVert Q_{nm} (O_{n}) \rVert_{1}   \\
    \hspace{-30pt}& \leq \max \{ \left\lVert Q_{11} \left(O_{1} / \lVert O_{1} \rVert_1\right) \right\rVert_{1} \hspace{1pt} + \ldots +  \left\lVert Q_{1m}  \left(O_{1} / \lVert O_{1} \rVert_1\right) \right\rVert_{1}, \hspace{2pt} \ldots \hspace{2pt},   \\
    \hspace{-30pt}& \left\lVert Q_{n1}  \left(O_{n} / \lVert O_{n} \rVert_1\right) \right\rVert_{1} + \ldots + \left\lVert Q_{nm}   \left(O_{n} / \lVert O_{n} \rVert_1\right) \right\rVert_{1} \hspace{1pt} \}  \\
    \hspace{-30pt} \Leftrightarrow  & \lVert O_1  \rVert_1  + \ldots + \lVert O_n  \rVert_1 = 1  \wedge  \lVert Q_{11} (O_{1}) \rVert_{1} + \ldots +  \lVert Q_{1m} (O_{1}) \rVert_{1} + \ldots +  \hspace{10pt}\\
    \hspace{-30pt} & \lVert Q_{n1} (O_{m}) \rVert_{1} +  \ldots  + \lVert Q_{nm} (O_{n}) \rVert_{1}   \\
    \hspace{-30pt}& \leq \max \{ (1 / \lVert O_{1} \rVert_1)   \left(\lVert Q_{11} (O_{1}) \rVert_{1} \hspace{1pt} + \ldots +  \lVert Q_{1m} (O_{1})\rVert_{1} \right), \hspace{2pt} \ldots \hspace{2pt},   \\
    \hspace{-30pt}& (1 / \lVert O_{n} \rVert_1)   \left(\lVert Q_{n1} (O_{n}) \rVert_{1} \hspace{1pt} + \ldots +  \lVert Q_{nm} (O_{n})\rVert_{1} \right) \}  
    \end{align*}
  
    This is equivalent to demonstrating that for $a_1, \ldots, a_n, x_1, \ldots, x_n \in \mathbb{R}^{+}_{0}$ with $a_1+ \ldots + a_n=1$,
    \begin{equation} 
    \begin{split}
        x_1 + \ldots + x_n  \leq  \max \left\{   \dfrac{1}{a_1} x_1  , \ldots , \dfrac{1}{a_n} x_n   \right\} \\
    \end{split}
    \end{equation}
  
    Designating $M = \max \left\{   \dfrac{1}{a_1} x_1  , \ldots , \dfrac{1}{a_n} x_n   \right\}$, from the definition of maximum it follows that, for all $1 \leq i \leq n$, $x_i \leq M \cdot a_i$, and consequently, $x_1 + \ldots + x_n \leq M \cdot (a_1 + \ldots + a_n) = M$. Therefore, it holds that:
    \begin{equation}
      \lVert Q(O) \rVert_{1 \text{ gen}} \leq  \lVert Q \rVert_{1 \text{ gen}}.
    \end{equation} 
  
    As a result, it follows that for an operator $O \in \mathbb{C}^{o_1 \times o_1} \oplus \ldots \oplus  \mathbb{C}^{o_m \times o_m}$,  $ \left\lVert Q\left(\frac{O}{\lVert O \rVert_{1 \text{ gen}}}\right)  \right\rVert_{1 \text{ gen}}$ is upper bounded by $\lVert Q  \rVert_{1 \text{ gen}}$. Thus, 
  \begin{equation}
    \lVert Q(O) \rVert_{1 \text{ gen}} \leq \lVert Q  \rVert_{1 \text{ gen}} \cdot \lVert O  \rVert_{1 \text{ gen}}.
  \end{equation}
  \end{proof}
  
  
  \begin{lemma}\label{lem:gen_trace_submultiplicative}
    The generalized trace norm is submultiplicative with respect to composition of super‑operators, \textit{i.e.}, for all super-operators $Q: \mathbb{C}^{o_1 \times o_1} \oplus \ldots \oplus \mathbb{C}^{o_n \times o_n}  \rightarrow \mathbb{C}^{p_1 \times p_1} \oplus \ldots \oplus  \mathbb{C}^{p_m \times p_m}$ and $S: \mathbb{C}^{p_1 \times p_1} \oplus \ldots \oplus \mathbb{C}^{p_m \times p_m}  \rightarrow \mathbb{C}^{q_1 \times q_1} \oplus \ldots \oplus \mathbb{C}^{q_t \times q_t}$, it holds that:
    \begin{equation} \label{eq:gen_trace_submultiplicative}
      \lVert S  Q \rVert_{1 \text{ gen}} \leq \lVert S \rVert_{1 \text{ gen}} \lVert Q \rVert_{1 \text{ gen}}
    \end{equation}
  \end{lemma}
  
  \begin{proof}
  The composition of two superoperators $Q: \mathbb{C}^{o_1 \times o_1} \oplus \ldots \oplus \mathbb{C}^{o_n \times o_n}  \rightarrow \mathbb{C}^{p_1 \times p_1} \oplus \ldots \oplus  \mathbb{C}^{p_m \times p_m}$ and $S: \mathbb{C}^{p_1 \times p_1} \oplus \ldots \oplus \mathbb{C}^{p_m \times p_m}  \rightarrow \mathbb{C}^{q_1 \times q_1} \oplus \ldots \oplus \mathbb{C}^{q_t \times q_t}$ corresponds to 
  \begin{equation}
    \begin{split}
      & S \cdot  Q =[\underbrace{\textsc{Il} \cdot \ldots \cdot \textsc{Il}}_{t-1 \times} \cdot \hspace{1pt} S_{11} + \ldots +   \underbrace{\textsc{Ir} \cdot \ldots \cdot \textsc{Ir}}_{t-1 \times}\cdot \hspace{1pt} S_{1t},  \hspace{2pt} \ldots  \hspace{2pt},  \underbrace{\textsc{Il} \cdot \ldots \cdot \textsc{Il}}_{t-1 \times} \cdot\hspace{1pt} S_{m1} + \ldots + \underbrace{\textsc{Ir} \cdot \ldots \cdot \textsc{Ir}}_{t-1 \times}\cdot \hspace{1pt} S_{mt}] \\
       & \hspace{21pt}\cdot  [\underbrace{\textsc{Il} \cdot \ldots \cdot \textsc{Il}}_{m-1 \times} \cdot \hspace{1pt} Q_{11} + \ldots +   \underbrace{\textsc{Ir} \cdot \ldots \cdot \textsc{Ir}}_{m-1 \times}\cdot \hspace{1pt} Q_{1m},  \hspace{2pt} \ldots  \hspace{2pt},  \underbrace{\textsc{Il} \cdot \ldots \cdot \textsc{Il}}_{m-1 \times} \cdot\hspace{1pt} Q_{n1} + \ldots + \underbrace{\textsc{Ir} \cdot \ldots \cdot \textsc{Ir}}_{m-1 \times}\cdot \hspace{1pt} Q_{nm}]  \\
       & \hspace{21pt} = [ \underbrace{\textsc{Il} \cdot \ldots \cdot \textsc{Il}}_{t-1 \times} \cdot \hspace{1pt} S_{11} \cdot Q_{11} + \ldots +   \underbrace{\textsc{Ir} \cdot \ldots \cdot \textsc{Ir}}_{t-1 \times}\cdot \hspace{1pt} S_{1t} \cdot Q_{11} + \ldots +  \underbrace{\textsc{Il} \cdot \ldots \cdot \textsc{Il}}_{t-1 \times} \cdot\hspace{1pt} S_{m1} \cdot Q_{1m} +  \ldots \\
       & \hspace{22pt}  + \underbrace{\textsc{Ir} \cdot \ldots \cdot \textsc{Ir}}_{t-1 \times}\cdot \hspace{1pt} S_{mt} \cdot Q_{1m}, \hspace{2pt} \ldots  \hspace{2pt},  \underbrace{\textsc{Il} \cdot \ldots \cdot \textsc{Il}}_{t-1 \times} \cdot \hspace{1pt} S_{11} \cdot Q_{n1} + \ldots +   \underbrace{\textsc{Ir} \cdot \ldots \cdot \textsc{Ir}}_{t-1 \times}\cdot \hspace{1pt} S_{1t} \cdot Q_{n1} + \ldots   \\
       & \hspace{22pt} +  \underbrace{\textsc{Il} \cdot \ldots \cdot \textsc{Il}}_{t-1 \times} \cdot\hspace{1pt} S_{m1} \cdot Q_{nm} + \ldots + \underbrace{\textsc{Ir} \cdot \ldots \cdot \textsc{Ir}}_{t-1 \times}\cdot \hspace{1pt} S_{mt} \cdot Q_{nm}]
    \end{split}
  \end{equation}
  
  
  
  %Attending to the definition of the generalized trace norm (\autoref{def:gen_1norm}), it follows that:
  %\begin{equation}
    %\begin{split}
   %& \lVert S \cdot  Q \rVert_{1 \text{ gen}} \\
   %& =  \max \{ \lVert S_{11} \cdot Q_{11} \rVert_{1} + \ldots + \lVert S_{1t} \cdot Q_{11} \rVert_{1} + \ldots +  \lVert S_{m1} \cdot Q_{1m} \rVert_{1} +  \ldots +  \lVert S_{mt} \cdot Q_{1m} \rVert_{1}, \\
    %&  \hspace{15pt} \hspace{2pt} \ldots \hspace{2pt}, \lVert S_{11} \cdot Q_{n1}\rVert_{1}  + \ldots + \lVert S_{1t} \cdot Q_{n1} \rVert_{1} + \ldots +  \lVert S_{m1} \cdot Q_{nm} \rVert_{1} +  \ldots + \lVert  S_{mt} \cdot Q_{nm} \rVert_{1} \}
    %\end{split}
  %\end{equation}
  
  
  Note that if $Q$ is decomposed as  $Q=[Q_1, \ldots, Q_n]$, where $Q_1, \ldots, Q_n$ are defined as in \autoref{def:gen_norm_either}, then for $ 1 \leq i \leq n$  $Q_i = \underbrace{\textsc{Il} \cdot \ldots \cdot \textsc{Il}}_{m-1 \times} \cdot \hspace{2pt} Q_{i1} + \ldots +  \underbrace{\textsc{Ir} \cdot \ldots \cdot \textsc{Ir}}_{m-1 \times} \cdot \hspace{2pt} Q_{im}$.
  Consequently, $S \cdot Q$ can also be defined as follows:
  \begin{equation}
    S \cdot Q = [S \cdot Q_1, \ldots, S \cdot Q_n].
  \end {equation}
  where 
  \begin{equation}
    \begin{split}
    S \cdot Q_i & = \underbrace{\textsc{Il} \cdot \ldots \cdot \textsc{Il}}_{t-1 \times} \cdot \hspace{1pt} S_{11} \cdot Q_{i1} + \ldots +   \underbrace{\textsc{Ir} \cdot \ldots \cdot \textsc{Ir}}_{t-1 \times}\cdot \hspace{1pt} S_{1t} \cdot Q_{i1} + \ldots +  \underbrace{\textsc{Il} \cdot \ldots \cdot \textsc{Il}}_{t-1 \times} \cdot\hspace{1pt} S_{m1} \cdot Q_{im} +  \ldots \\
    & \hspace{10pt}  + \underbrace{\textsc{Ir} \cdot \ldots \cdot \textsc{Ir}}_{t-1 \times}\cdot \hspace{1pt} S_{mt} \cdot Q_{im},
    \end{split}
  \end{equation}
  
  
  Attending to \autoref{def:gen_norm_either}, it follows that:
  \begin{equation} \label{eq:sq_decomposed_norm}
  \lVert S \cdot  Q \rVert_{1 \text{ gen}} = \max \{ \lVert S \cdot Q_1 \rVert_{1 \text{ gen}}, \ldots, \lVert S \cdot Q_n \rVert_{1 \text{ gen}} \},
  \end{equation}
  where 
  \begin{equation} \label{eq:sqi_norm}
  \lVert S \cdot Q_i \rVert_{1 \text{ gen}} =  \lVert S_{11} \cdot Q_{i1} \rVert_{1} + \ldots + \lVert S_{1t} \cdot Q_{i1} \rVert_{1} + \ldots +  \lVert S_{m1} \cdot Q_{im} \rVert_{1} +  \ldots +  \lVert S_{mt} \cdot Q_{im} \rVert_{1}
  \end{equation} 
  for all $1 \leq i \leq n$.
  
  %Let $P \in \mathbb{C}^{p_1 \times p_1} \oplus \ldots \oplus  \mathbb{C}^{p_m \times p_m}$, with the decomposition $P = \underbrace{\textsc{Il} \cdot \ldots \cdot \textsc{Il}}_{m-1 \times} \cdot \hspace{2pt} P_{1} + \ldots +  \underbrace{\textsc{Ir} \cdot \ldots \cdot \textsc{Ir}}_{m-1 \times} \cdot \hspace{2pt} P_{m} 
  %$, where for each $1 \leq i \leq m$, $P_{i} \in \mathbb{C}^{p_i \times p_i}$ and $P_i =  \underbrace{\textsc{Pl} \cdot \ldots \cdot \textsc{Pl}}_{m-i \times} \cdot \underbrace{\textsc{Pr} \cdot \ldots \cdot \textsc{Pr}}_{i-1 \times} (P) $, considering \autoref{def:gen_norm}, it follows that: 
  %\begin{equation}
    %\lVert P  \rVert_{1 \text{ gen}} = \lVert P_1 \rVert_{1} + \ldots + \lVert P_m \rVert_{1}.
  %\end{equation}
  %The application of $S$ to $P$ corresponds to 
  %\begin{equation}
  %\begin{split}
  %S(P) & = \underbrace{\textsc{Il} \cdot \ldots \cdot \textsc{Il}}_{t-1 \times} \cdot \hspace{1pt} S_{11} \cdot P_{1} + \ldots +   \underbrace{\textsc{Ir} \cdot \ldots \cdot \textsc{Ir}}_{t-1 \times}\cdot \hspace{1pt} S_{1t} \cdot P_{1} + \ldots +  \underbrace{\textsc{Il} \cdot \ldots \cdot \textsc{Il}}_{t-1 \times} \cdot\hspace{1pt} S_{m1} \cdot P_{m} +  \ldots \\
  %& \hspace{15pt}  + \underbrace{\textsc{Ir} \cdot \ldots \cdot \textsc{Ir}}_{t-1 \times}\cdot \hspace{1pt} S_{mt} \cdot P_{m}
  %\end{split}
  %\end{equation}
  
  %As a result, the generalized trace norm of $S(P)$ corresponds to:
  %\begin{equation}
    %\begin{split}
    %\lVert S(P)  \rVert_{1 \text{ gen}} & = \lVert S_{11} \cdot P_{1} \rVert_{1} + \ldots + \lVert S_{1t} \cdot P_{1} \rVert_{1} + \ldots +  \lVert S_{m1} \cdot P_{m} \rVert_{1} +  \ldots +  \lVert S_{mt} \cdot P_{m} \rVert_{1}. 
    %\end{split}
  %\end {equation}
  %The generalized trace norm of $S$ is given by:
  %\begin{equation}
    %\begin{split}
    %&\lVert S  \rVert_{1 \text{ gen}} = \max \{ \lVert S_{11} \rVert_{1} + \ldots + \lVert S_{1t} \rVert_{1}, \hspace{2pt} \ldots \hspace{2pt}, \lVert S_{m1} \rVert_{1} + \ldots + \lVert S_{mt} \rVert_{1} \} \\
   %& = \max \{ \max \{ \lVert S_{11} (A_{11}) \rVert_{1} \hspace{1pt}  \vert \hspace{1pt}  \lVert A_{11} \rVert_{1} = 1 \} + \ldots + &  \{\text{\autoref{def:trace_norm_superoperator}}\} \\
   %& \hspace{15pt}  \max \{  \lVert S_{1t} (A_{1t}) \rVert_{1} \hspace{1pt}  \vert   \lVert A_{1t} \rVert_{1} = 1 \} , \hspace{1pt} \ldots \hspace{1pt}, \max \{ \lVert S_{11} (A_{m1}) \rVert_{1} \hspace{1pt}  \vert \hspace{1pt}  \lVert A_{m1} \rVert_{1} = 1 \}   \\
   %&\hspace{15pt} + \ldots +  \max \{ \lVert S_{11} (A_{m1}) \rVert_{1} \hspace{1pt}  \vert \lVert A_{m1} \rVert_{1} = 1 \}\}
    %\end{split}
  %\end{equation}
  Let $P \in \mathbb{C}^{p_1 \times p_1} \oplus \ldots \oplus  \mathbb{C}^{p_m \times p_m}$, by \autoref{lem:q(o)} it follows that:
  \begin{equation}
    \lVert S(P) \rVert_{1 \text{ gen}} \leq \lVert S  \rVert_{1 \text{ gen}} \cdot \lVert P  \rVert_{1 \text{ gen}}.
  \end{equation}
  As a result, 
  \begin{equation} \label{ineq:gen_trace_submultiplicative_O}
    \lVert S (Q_i (O_i)) \rVert_{1 \text{ gen}} \leq \lVert S  \rVert_{1 \text{ gen}} \cdot \lVert Q_i (O_i)  \rVert_{1 \text{ gen}},
  \end{equation}
  for all $O_i \in \mathbb{C}^{o_i \times o_i}$ and $1 \leq i \leq n$. 
  
  Given that
  \begin{equation}
  \begin{split} 
    S (Q_i (O_i)) & = \underbrace{\textsc{Il} \cdot \ldots \cdot \textsc{Il}}_{t-1 \times} \cdot \hspace{1pt} S_{11} \cdot Q_{i1} \cdot O_{i} + \ldots + \underbrace{\textsc{Ir} \cdot \ldots \cdot \textsc{Ir}}_{t-1 \times} \cdot \hspace{1pt} S_{1t} \cdot Q_{i1} \cdot O_{i} + \ldots +      \\
    & \hspace{15pt}  \underbrace{\textsc{Il} \cdot \ldots \cdot \textsc{Il}}_{t-1 \times}\cdot \hspace{1pt} S_{m1} \cdot Q_{im} \cdot O_{i} + \ldots + \underbrace{\textsc{Ir} \cdot \ldots \cdot \textsc{Ir}}_{t-1 \times} \cdot \hspace{1pt} S_{mt} \cdot Q_{im} \cdot O_{i},
  \end{split}
  \end{equation}
  and that
  \begin{equation}
    \begin{split}
       Q_i (O_i) = \underbrace{\textsc{Il} \cdot \ldots \cdot \textsc{Il}}_{m-1 \times} \cdot \hspace{2pt} Q_{i1} \cdot O_{i} + \ldots +  \underbrace{\textsc{Ir } \cdot \ldots \cdot \textsc{Ir}}_{m-1 \times} \cdot \hspace{2pt} Q_{im} \cdot O_{i},
    \end{split}
  \end{equation}
  considering \autoref{def:gen_norm}, \autoref{ineq:gen_trace_submultiplicative_O} can be rewritten as:
  \begin{equation}
    \begin{split}
    &\lVert S_{11} \cdot Q_{i1} \cdot O_{i} \rVert_{1} + \ldots + \lVert S_{1t} \cdot Q_{i1} \cdot O_{i} \rVert_{1} + \ldots + \lVert S_{m1} \cdot Q_{im} \cdot O_{i} \rVert_{1} + \ldots +  \\
    &  \lVert S_{mt} \cdot Q_{im} \cdot O_{i} \rVert_{1} \leq  \lVert S  \rVert_{1 \text{ gen}} \cdot \lVert Q_{i1} \cdot O_{i} \rVert_{1} + \ldots + \lVert Q_{im} \cdot O_{i} \rVert_{1}
    \end{split}
  \end{equation}
  
  Taking the maximum over all $O \in \mathbb{C}^{o_1 \times o_1} \oplus \ldots \oplus \mathbb{C}^{o_n \times o_n}$ such that $\lVert O_i \rVert_{1} = 1$ yields the following inequality 
  \begin{equation}
    \begin{split}
    & \max \{ \lVert S_{11} \cdot Q_{i1} \cdot O_{i} \rVert_{1} + \ldots + \lVert S_{1t} \cdot Q_{i1} \cdot O_{i} \rVert_{1} + \ldots + \lVert S_{m1} \cdot Q_{im} \cdot O_{i} \rVert_{1} + \ldots +  \\
    &  \lVert S_{mt} \cdot Q_{im}  \cdot O_{i} \rVert_{1} \hspace{1pt} |\lVert O_i \rVert_{1} = 1 \} \leq  \lVert S  \rVert_{1 \text{ gen}} \cdot \max \{ \lVert Q_{i1} \cdot O_{i} \rVert_{1} + \ldots + \lVert Q_{im} \cdot O_{i} \rVert_{1} \hspace{1pt} |\lVert O_i \rVert_{1} = 1 \}.
    \end{split}
  \end{equation}
  This is equivalent to
  \begin{equation}
    \begin{split}
    & \max \{ \lVert S_{11} \cdot Q_{i1} \cdot O_{i} \rVert_{1} \hspace{1pt}|\lVert O_i \rVert_{1} = 1 \} + \ldots + \max \{\lVert S_{1t} \cdot Q_{i1} \cdot O_{i} \rVert_{1}  \hspace{1pt}|\lVert O_i \rVert_{1} = 1 \}  + \ldots +   \\
    & \max \{\lVert S_{m1} \cdot Q_{im} \cdot O_{i} \rVert_{1} \hspace{1pt}|\lVert O_i \rVert_{1} = 1 \} + \ldots + \max \{ \lVert S_{mt} \cdot Q_{im}  \cdot O_{i} \rVert_{1} \hspace{1pt} |\lVert O_i \rVert_{1} = 1 \}  \\
    & \leq  \lVert S  \rVert_{1 \text{ gen}} \cdot \max \{ \lVert Q_{i1} \cdot O_{i} \rVert_{1}  \hspace{1pt} |\lVert O_i \rVert_{1} = 1 \}  + \ldots + \max \{\lVert Q_{im} \cdot O_{i} \rVert_{1} \hspace{1pt} |\lVert O_i \rVert_{1} = 1 \} 
    \end{split}
  \end{equation}
  As a result, attending to \autoref{def:trace_norm_superoperator}, it follows that:
  \begin{equation}
    \begin{split}
    & \lVert S_{11} \cdot Q_{i1} \rVert_{1} + \ldots + \lVert S_{1t} \cdot Q_{i1} \rVert_{1} + \ldots + \lVert S_{m1} \cdot Q_{im} \rVert_{1} + \ldots +  \lVert S_{mt} \cdot Q_{im} \rVert_{1} \\
    &  \leq  \lVert S  \rVert_{1 \text{ gen}} \cdot (\lVert Q_{i1} \rVert_{1} + \ldots + \lVert Q_{im} \rVert_{1})
    \end{split}
  \end{equation}
  Considering \autoref{eq:sqi_norm}, and \autoref{def:gen_norm_gen_inj}, the inequality above can be rewritten as: 
  \begin{equation}
     \lVert S \cdot  Q_i \rVert_{1 \text{ gen}} \leq \lVert S  \rVert_{1 \text{ gen}} \cdot \lVert Q_i  \rVert_{1 \text{ gen}}
  \end{equation}
  for all $1 \leq i \leq n$. Given \autoref{def:gen_norm_either}, and considering the fact that for $a,b,c \in \mathbb{R}$ if $a \leq b$ and $b \leq c$, then $a \leq c$, it follows that:
  \begin{equation}
    \lVert S  Q_i \rVert_{1 \text{ gen}} \leq \lVert S \rVert_{1 \text{ gen}} \lVert Q \rVert_{1 \text{ gen}}
  \end{equation}
  Attending to \autoref{eq:sq_decomposed_norm}, and the fact that if all elements of a set verify a certain property, then the maximum of the set also verifies the property, given it is an element of the set, it follows that:
  \begin{equation}
    \lVert S  Q \rVert_{1 \text{ gen}} \leq \lVert S \rVert_{1 \text{ gen}} \lVert Q \rVert_{1 \text{ gen}}
  \end{equation}
  
  Therefore, the inequality in \autoref{eq:gen_trace_submultiplicative} holds.
  
  %no take the maximum a seguir por as eqs das normas (com trace norm explita com cena da norma 1) e depois é que digo que corresponde a ineq in \autoref{eq:gen_trace_submultiplicative}
  
  % \underbrace{\textsc{Il} \cdot \ldots \cdot \textsc{Il}}_{m-1 \times} \cdot \hspace{2pt} P_{1} + \ldots +  \underbrace{\textsc{Ir} \cdot \ldots \cdot \textsc{Ir}}_{m-1 \times} \cdot \hspace{2pt} P_{m} 
  
  \end{proof}
  
  
  \begin{lemma}\label{lem:gen_diamond_submultiplicative}
    The generalized diamond norm is submultiplicative with respect to composition of super‑operators, \textit{i.e.}, for all super-operators $Q: \mathbb{C}^{o_1 \times o_1} \oplus \ldots \oplus \mathbb{C}^{o_n \times o_n}  \rightarrow \mathbb{C}^{p_1 \times p_1} \oplus \ldots \oplus  \mathbb{C}^{p_m \times p_m}$ and $S: \mathbb{C}^{p_1 \times p_1} \oplus \ldots \oplus \mathbb{C}^{p_m \times p_m}  \rightarrow \mathbb{C}^{q_1 \times q_1} \oplus \ldots \oplus \mathbb{C}^{q_t \times q_t}$, it holds that:
    \begin{equation} \label{eq:gen_trace_submultiplicative}
      \lVert S  Q \rVert_{\diamondsuit \text{ gen}} \leq \lVert S \rVert_{\diamondsuit  \text{ gen}} \lVert Q \rVert_{\diamondsuit  \text{ gen}}
    \end{equation}
  \end{lemma}
  
  \begin{proof}
   
    By \autoref{lem:gen_trace_submultiplicative}, it is possible to state that:
  \begin{equation}
    \begin{split}
     \lVert S Q \otimes I_{\mathbb{C}^{o_1 \times o_1} \oplus \ldots \oplus \mathbb{C}^{o_n \times o_n}} \rVert_{1\text{ gen}} \leq \lVert S \otimes I_{\mathbb{C}^{o_1 \times o_1} \oplus \ldots \oplus \mathbb{C}^{o_n \times o_n}} \rVert_{1\text{ gen}} \cdot \lVert Q \otimes I_{\mathbb{C}^{o_1 \times o_1} \oplus \ldots \oplus \mathbb{C}^{o_n \times o_n}} \rVert_{1\text{ gen}}  
      \end{split}
  \end{equation}
  Attending to \autoref{def:gen_diamond_norm}, it follows that:
  \begin{equation}
     \lVert S Q \rVert_{\diamondsuit \text{ gen}} \leq  \lVert S \otimes I_{\mathbb{C}^{o_1 \times o_1} \oplus \ldots \oplus \mathbb{C}^{o_n \times o_n}} \rVert_{1\text{ gen}} \cdot \lVert Q \rVert_{\diamondsuit \text{ gen}}
  \end{equation}
  Given \autoref{theorem:tensor_stability}, it holds that
  \begin{equation}
    \lVert S \otimes I_{\mathbb{C}^{o_1 \times o_1} \oplus \ldots \oplus \mathbb{C}^{o_n \times o_n}} \rVert_{1\text{ gen}} \leq \lVert S \rVert_{\diamondsuit \text{ gen}}
  \end{equation}
  As a result, the inequality in \autoref{eq:gen_trace_submultiplicative} holds.
  \end{proof}
   
  \begin{lemma} \label{lem:gen_trace_ptp_norm1}
    Let  $Q: \mathbb{C}^{o_1 \times o_1} \oplus \ldots \oplus \mathbb{C}^{o_n \times o_n}  \rightarrow \mathbb{C}^{p_1 \times p_1} \oplus \ldots \oplus  \mathbb{C}^{p_m \times p_m}$ be a positive trace-preserving super-operator. Then, it holds that $\lVert Q \rVert_{1 \text{ gen}} = 1$.
  \end{lemma}
  
  \begin{proof}
    Consider the decomposition $Q = [Q_1, \ldots, Q_n]$, where $Q_1, \ldots, Q_n$ are defined as in \autoref{def:gen_norm_either}. Then for $ 1 \leq i \leq n$, one has that $Q_i: \mathbb{C}^{o_i \otimes o_i} \rightarrow  \mathbb{C}^{p_1 \times p_1} \oplus \ldots \oplus  \mathbb{C}^{p_m \times p_m}$ is defined as  $Q_i = \underbrace{\textsc{Il} \cdot \ldots \cdot \textsc{Il}}_{m-1 \times} \cdot \hspace{2pt} Q_{i1} + \ldots +  \underbrace{\textsc{Ir} \cdot \ldots \cdot \textsc{Ir}}_{m-1 \times} \cdot \hspace{2pt} Q_{im}$.
  
    In this case attending to \autoref{def:gen_norm_either} and \autoref{def:gen_norm_either}, 
    \begin{equation}
        \lVert Q \rVert_{1 \text{ gen}} = \max \{ \lVert Q_1 \rVert_{1  \text{ gen}} + \ldots + \lVert Q_n \rVert_{1  \text{ gen}} \},
    \end{equation}
      where
      \begin{equation} \label{eq:qi_norm}
          \lVert Q_i \rVert_{1  \text{ gen}} = \lVert Q_{i1} \rVert_{1} + \ldots + \lVert Q_{im} \rVert_{1}
      \end{equation}
  for $ 1 \leq i \leq n$.
  
  Note that for all $1 \leq i \leq m$, if $Q$ is positive trace-preserving, then $Q_i$ is also positive trace-preserving given that the composition of positive trace-preserving super-operators is also positive trace-preserving.
  
  Considering the definition of $Q_{ij}$ in \autoref{def:gen_norm_ops} for a fixed $1 \leq i \leq n$, $Q_{i1}, \ldots, Q_{im}$ are always given the same argument.
  Consequentlty, attending to \autoref{thm:Russo–Dye}, \autoref{eq:qi_norm} can be rewritten as:
  \begin{align} 
    \hspace{-30pt} \lVert Q_i \rVert_{1} & = \max \{\text{Tr}\left(Q_{i1}(uu^{*}) \right) \vert \hspace{2pt}  \lVert u \rVert_{1}=1 \} + \ldots +  \max \{\text{Tr}\left(Q_{in}(u u^{*}) \right) \vert \hspace{2pt}  \lVert u \rVert_{1}=1 \}\\ \label{eq:qi_ptp_norm_1}
    \hspace{-30pt} & =  \max \{ \text{Tr}\left(Q_{i1}(uu^{*}) \right) + \ldots + \text{Tr}\left(Q_{in}(u u^{*}) \right) \vert \hspace{2pt}  \lVert u \rVert_{1}=1 \} \\
    \hspace{-30pt} & =  \max \{ \text{Tr}\left(Q_{i1}(uu^{*}) + \ldots + Q_{in}(u u^{*}) \right) \vert \hspace{2pt}  \lVert u \rVert_{1}=1 \}  \label{eq:qi_ptp_norm_3}
  \end{align}
  %\begin{align} 
    %\hspace{-30pt} \lVert Q_i \rVert_{1} & = \max \{\text{Tr}\left(Q_{i1}(u_1u_1^{*}) \right) \vert \hspace{2pt}  \lVert u_1 \rVert_{1}=1 \} + \ldots +  \max \{\text{Tr}\left(Q_{in}(u_m u_m^{*}) \right) \vert \hspace{2pt}  \lVert u_m \rVert_{1}=1 \}\\ \label{eq:qi_ptp_norm_1}
    %\hspace{-30pt} & =  \max \{ \text{Tr}\left(Q_{i1}(u_1u_1^{*}) \right) + \ldots + \text{Tr}\left(Q_{in}(u_m u_m^{*}) \right) \vert \hspace{2pt}  \lVert u_1 \rVert_{1}=1, \ldots, \lVert u_m \rVert_{1}=1 \} \\
    %\hspace{-30pt} & =  \max \{ \text{Tr}\left(Q_{i1}(u_1u_1^{*}) + \ldots + Q_{in}(u_m u_m^{*}) \right) \vert \hspace{2pt}  \lVert u_1 \rVert_{1}=1, \ldots, \lVert u_m \rVert_{1}=1 \}  \label{eq:qi_ptp_norm_3}
  %\end{align}
  \todo[inline,size=\normalsize]{ no "and ..." fazer referencia à definição de trace preserving para espaços com somas diretas qd eu a definir} 
  where $u \in \mathbb{C}^{o_i}$.
  If $Q_i$ is trace-preserving, then considering \autoref{def:gen_norm_gen_inj} and ...
  \begin{equation}
    \begin{split}
    &\text{Tr} (Q_i(u u^{*})) =  \text{Tr} \left( \underbrace{\textsc{Il} \cdot \ldots \cdot \textsc{Il}}_{m-1 \times } \cdot \hspace{2pt} Q_{i1} (u u^{*}) + \ldots + \underbrace{\textsc{Ir} \cdot \ldots \cdot \textsc{Ir}}_{m-1 \times } \cdot  \hspace{2pt} Q_{im}(u u^{*}) \right) \\
    =&   \text{Tr} \left( Q_{i1} (u u^{*}) + \ldots + Q_{im}(u u^{*}) \right) = \text{Tr} (u u^{*}) = \text{Tr} (u u^{*})
  \end{split}
  \end{equation}
  As a result, considering the definition of trace-norm for square matrices, \autoref{def:trace-norm-matriz},  it follows that $\lVert Q_i \rVert_{1} = \text{Tr} (u u^{*}) = 1$ for all $1 \leq i \leq n$. Given that if all elements of a set verify a certain property, then the maximum of the set also verifies the property it follows that $\lVert Q \rVert_{1 \text{ gen}} = 1$.
  
  \end{proof}
  
  \begin{lemma} \label{lem:gen_diamond_cptp_norm}
    Let  $Q: \mathbb{C}^{o_1 \times o_1} \oplus \ldots \oplus \mathbb{C}^{o_n \times o_n}  \rightarrow \mathbb{C}^{p_1 \times p_1} \oplus \ldots \oplus  \mathbb{C}^{p_m \times p_m}$ be a \acrshort{cptp} superoperator. Then, it holds that $\lVert Q \rVert_{\diamondsuit \text{ gen}} = 1$.
  \end{lemma}
  
  \begin{proof}
    Given that $Q$ is a \acrshort{cptp} superoperator, if follows that $ Q \otimes I_{\mathbb{C}^{o_1 \times o_1} \oplus \ldots \oplus \mathbb{C}^{o_n \times o_n}}$ is a positive trace-preserving superoperator. As a result, attending to \autoref{lem:gen_trace_ptp_norm1}, it holds that $\lVert Q \otimes I_{\mathbb{C}^{o_1 \times o_1} \oplus \ldots \oplus \mathbb{C}^{o_n \times o_n}} \rVert_{1 \text{ gen}} = 1$. As a result, considering  \autoref{def:gen_diamond_norm}, it follows that $\lVert Q \rVert_{\diamondsuit \text{ gen}} = 1$.
  \end{proof}
  
  \begin{theorem}
    Let  $Q: \mathbb{C}^{o_1 \times o_1} \oplus \ldots \oplus \mathbb{C}^{o_n \times o_n}  \rightarrow \mathbb{C}^{p_1 \times p_1} \oplus \ldots \oplus  \mathbb{C}^{p_m \times p_m}$ and $S: \mathbb{C}^{p_1 \times p_1} \oplus \ldots \oplus \mathbb{C}^{p_m \times p_m}  \rightarrow \mathbb{C}^{q_1 \times q_1} \oplus \ldots \oplus \mathbb{C}^{q_t \times q_t}$ be super-operators. If $Q$ is a \acrshort{cptp} superoperator, then $\lVert S  Q \rVert_{\diamondsuit \text{ gen}} \leq \lVert S \rVert_{\diamondsuit \text{ gen}}$, and if $S$ is a quantum channel, then $\lVert S  Q \rVert_{\diamondsuit \text{ gen}} \leq \lVert Q \rVert_{\diamondsuit \text{ gen}}$
  \end{theorem}
  
  \begin{proof}
    This result is an immediate consequence of \autoref{lem:gen_diamond_submultiplicative} and \autoref{lem:gen_diamond_cptp_norm}. 
  \end{proof}
  
















  %
  %
  %
  %
  %
  %
  %
  %
  %
  %
  %
%
%
%
%
%
%
%
%
%
%
%
%
%
% ||Q(O)||<= ||Q|| ||O||


 %
  %
  %
  %
  %
  %
  %
  %
  %
  %
  %
%
%
%
%
%
%
%
%
%
%
%
%
%




$O$ can be written as  $O = \sum_{i=1}^{m} \underbrace{\textsc{Il} \cdot \ldots \cdot \textsc{Il}}_{n-i \times} \cdot   \underbrace{\textsc{Ir} \cdot \ldots \cdot \textsc{Ir}}_{i-1 \times} \cdot \hspace{2pt} O_{i} 
$, where  $O_{i} \in \mathbb{C}^{p_i \times p_i}$ and $O_i =  \underbrace{\textsc{Pl} \cdot \ldots \cdot \textsc{Pl}}_{n-i \times} \cdot \underbrace{\textsc{Pr} \cdot \ldots \cdot \textsc{Pr}}_{i-1 \times} \cdot O $. Considering \autoref{def:gen_norm}, it follows that: 
\begin{equation}
  \lVert O  \rVert_{1 \text{ gen}} =   \lVert O_1 \rVert_{1} + \ldots + \lVert O_n \rVert_{1}.
\end{equation}

Applying $O$ to $Q$ results in:
\begin{equation}
\begin{split}
Q(O) & = \sum_{i=1}^{n} \sum_{j=1}^{m} \underbrace{\textsc{Il} \cdot \ldots \cdot \textsc{Il}}_{m-j \times} \cdot   \underbrace{\textsc{Ir} \cdot \ldots \cdot \textsc{Ir}}_{j-1 \times} \cdot \hspace{2pt} Q_{ij} (O_{i}) \\
%\underbrace{\textsc{Il} \cdot \ldots \cdot \textsc{Il}}_{m-1 \times} \cdot \hspace{1pt} Q_{11}  (O_{1}) + \ldots +   \underbrace{\textsc{Ir} \cdot \ldots \cdot \textsc{Ir}}_{m-1 \times}\cdot \hspace{1pt} Q_{1m} (O_{1}) + \ldots +  \underbrace{\textsc{Il} \cdot \ldots \cdot \textsc{Il}}_{m-1 \times} \cdot\hspace{1pt} Q_{n1} (O_{n}) +  \ldots \\
%& \hspace{10pt}  + \underbrace{\textsc{Ir} \cdot \ldots \cdot \textsc{Ir}}_{m-1 \times}\cdot \hspace{1pt} Q_{nm}  (O_{n})
\end{split}
\end{equation}

As a result, considering \autoref{def:gen_1norm_matrix}, the generalized trace norm of $Q(O)$ corresponds to:
\begin{equation} \label{eq:qo}
  \begin{split}
  \lVert Q(O)  \rVert_{1 \text{ gen}} & = Q_1(O_1) + \ldots + Q_n(O_n)=  \sum_{i=1}^{n} \sum_{j=1}^{m} \lVert Q_{ij} (O_{i}) \rVert_{1}.
  %\lVert Q_{11} (O_{1}) \rVert_{1} + \ldots + \lVert Q_{1m} (O_{1}) \rVert_{1} + \ldots +  \lVert Q_{n1} (O_{n})  \rVert_{1} +  \ldots +  \lVert Q_{nm} (O_{n}) \rVert_{1}. 
  \end{split}
\end {equation}
The generalized trace norm of $Q$ is given by:
\begin{equation}
  \begin{split} \label{eq:q}
  \lVert Q  \rVert_{1 \text{ gen}} & =   \max  \Bigg\{ \max \left\{ \sum_{i=1}^{m} \|Q_{1i} (A_1)\|_{1}   \mid \hspace{1pt} \|A_1\|_{1} = 1 \right\} & \{\text{\autoref{def:gen_1norm}}\} \\
  & \hspace{15pt} ,\hspace{2pt}  \ldots \hspace{2pt}  , \max \left\{ \sum_{i=1}^{m} \|Q_{ni} (A_n)\|_{1}   \mid \|A_n\|_{1} = 1 \right\} \Bigg\} \\
 & = \max \Bigg\{  \sum_{i=1}^{m}  \lVert Q_{1i} (A_{1}) \rVert_{1} \hspace{1pt} , \hspace{2pt} \ldots \hspace{2pt}, \sum_{i=1}^{m} \lVert Q_{ni} (A_{n}) \rVert_{1} \mid \hspace{1pt}   \lVert A_{1} \rVert_{1} = 1  & \{\text{\autoref{lemma:max_max}} \} \\
 & \hspace{15pt} , \ldots,\lVert A_{n} \rVert_{1} = 1 \Bigg\}.
  \end{split}
\end{equation}


To prove that  if  $\lVert O \rVert_{1 \text{ gen}} = 1$
\begin{equation}
  \lVert Q(O) \rVert_{1 \text{ gen}} \leq \lVert Q \rVert_{1 \text{ gen}},
\end{equation}
is equivalent to demonstrating that,
\begin{equation}
 \max \left\{ \lVert Q(O) \rVert_{1 \text{ gen}} \hspace{2pt}  \Biggm| \hspace{2pt}  \sum_{i=1}^{n} \lVert O_i  \rVert_1 = 1 \right\} \leq \lVert Q \rVert_{1 \text{ gen}},
\end{equation}

Thus, 
\begin{align*}
  \hspace{-30pt}&   \max \left\{ \lVert Q(O) \rVert_{1 \text{ gen}} \hspace{2pt}  \Biggm| \hspace{2pt}  \sum_{i=1}^{n} \lVert O_i  \rVert_1 = 1 \right\} \leq \lVert Q \rVert_{1 \text{ gen}} \\
  \hspace{-30pt} \Leftrightarrow  & 
  \max \left\{   \sum_{i=1}^{n} \sum_{j=1}^{m} \lVert Q_{ij} (O_{i}) \rVert_{1} \hspace{2pt}  \Biggm| \hspace{2pt}  \sum_{i=1}^{n} \lVert O_i  \rVert_1 = 1 \right\} \leq \max \Bigg\{  \sum_{i=1}^{m}  \lVert Q_{1i} (A_{1}) \rVert_{1} &   \{\text{\autoref{eq:qo}}, \\
  \hspace{-30pt}& , \hspace{2pt} \ldots \hspace{2pt}, \sum_{i=1}^{m} \lVert Q_{ni} (A_{n}) \rVert_{1} \mid \hspace{1pt}   \lVert A_{1} \rVert_{1} = 1 , \ldots,\lVert A_{n} \rVert_{1} = 1 \Bigg\}   & \text{\autoref{eq:q}} \}  \\
  %\hspace{-30pt} \Leftrightarrow  & \lVert O_1  \rVert_1  + \ldots + \lVert O_n  \rVert_1 = 1  \wedge  \lVert Q_{11} \cdot O_{1} \rVert_{1} + \ldots +  \lVert Q_{1m} \cdot O_{1} \rVert_{1} + \ldots +  \hspace{10pt}  \\
  %\hspace{-30pt}& \leq \max \Bigg\{ \left\lVert Q_{11} \left(\frac{O_{1}} {\lVert O_{1} \rVert_1}\right) \right\rVert_{1} \hspace{1pt} + \ldots +  \left\lVert Q_{1m} \left(\frac{O_{1}} {\lVert O_{1} \rVert_1}\right)  \right\rVert_{1} \hspace{1pt}, \hspace{2pt} \ldots \hspace{2pt},   \\
  %\hspace{-30pt}& \left\lVert Q_{n1}  \left(\frac{O_{n}} {\lVert O_{1} \rVert_1}\right) \right\rVert_{1} + \ldots + \left\lVert Q_{nm}  \left(\frac{O_{n}} {\lVert O_{1} \rVert_1}\right) \right\rVert_{1} \hspace{1pt}  \Bigg\}  \\
  \hspace{-30pt} \Leftrightarrow  &  \sum_{i=1}^{n} \lVert O_i  \rVert_1 = 1   \wedge \max \left\{ \sum_{i=1}^{n} \sum_{j=1}^{m} \lVert Q_{ij} (O_{i}) \rVert_{1} \right\} \leq  \\
  \hspace{-30pt}& \max \Bigg\{  \sum_{i=1}^{m}  \lVert Q_{1i}  \left(O_{1} / \lVert O_{1} \rVert_1\right) \rVert_{1} , \hspace{2pt} \ldots \hspace{2pt}, \sum_{i=1}^{m} \lVert Q_{ni}  \left(O_{n} / \lVert O_{n} \rVert_1\right) \rVert_{1} \Bigg\}  \\
  \hspace{-30pt} \Leftrightarrow  &  \sum_{i=1}^{n} \lVert O_i  \rVert_1 = 1   \wedge \max \left\{ \sum_{i=1}^{n} \sum_{j=1}^{m} \lVert Q_{ij} (O_{i}) \rVert_{1} \right\} \leq   \\
  \hspace{-30pt}& \max \Bigg\{ (1 / \lVert O_{1} \rVert_1) \sum_{i=1}^{m}  \lVert Q_{1i}  \left(O_{1} \right) \rVert_{1}, \hspace{2pt} \ldots \hspace{2pt}, (1 / \lVert O_{n} \rVert_1) \sum_{i=1}^{m} \lVert Q_{ni}  \left(O_{n}\right) \rVert_{1} \Bigg\}    
  \end{align*}

  This is equivalent to demonstrating that for all $a_1, \ldots, a_n, x_1, \ldots, x_n \in \mathbb{R}^{+}_{0}$ with $a_1+ \ldots + a_n=1$,
  \begin{equation} 
  \begin{split}
      x_1 + \ldots + x_n  \leq  \max \left\{   \dfrac{1}{a_1} x_1  , \ldots , \dfrac{1}{a_n} x_n   \right\} \\
  \end{split}
  \end{equation}

  Designating $M = \max \left\{   \dfrac{1}{a_1} x_1  , \ldots , \dfrac{1}{a_n} x_n   \right\}$, from the definition of maximum it follows that, for all $1 \leq i \leq n$, $x_i \leq M \cdot a_i$, and consequently, $x_1 + \ldots + x_n \leq M \cdot (a_1 + \ldots + a_n) = M$. Therefore, it holds that:
  \begin{equation}
    \lVert Q(O) \rVert_{1 \text{ gen}} \leq  \lVert Q \rVert_{1 \text{ gen}}.
  \end{equation} 

  As a result, it follows that for an operator $O \in \mathbb{C}^{o_1 \times o_1} \oplus \ldots \oplus  \mathbb{C}^{o_m \times o_m}$,  $ \left\lVert Q\left(\frac{O}{\lVert O \rVert_{1 \text{ gen}}}\right)  \right\rVert_{1 \text{ gen}}$ is upper bounded by $\lVert Q  \rVert_{1 \text{ gen}}$. Thus, \autoref{eq:qo<q} holds.

\part{Core of the Dissertation}

\chapter{Contribution}

Main result(s) and their scientific evidence

\section{Introduction}

\section{Summary}

%ToDo: Extending the quantum model with conditionals -> Talk about the model (operadores como morfismos, a nossa estrutura algebrica é o espaços vetorias); CPTP model; measurents and the diamond norm é menor ou igual a um; examples (qualtum walk)
%\todo[inline]{The original todo note withouth changed colours.\newline Here's another line.}
%ToDo: Discard operations -> The operator should de unique (but there is for instance trace and the null operator_maps everything to zero) and is unique bacausce only trace is trace preserving; example: mallicious attack discard a qubit ou outside of quantum teleportation : setting a qubit do 0, by discarding it and initiating a new one

\section{Integration of conditionals}

The notion of approximate equivalence for quantum programming explored in [\cite{dahlqvist2022syntactic}] does not encompass classical control flow. As a result, preliminary work based on [\cite{crole1993categories,selinger2013lecture}]   has been undertaken to address the integration of conditionals. 

\subsection{Integration of conditionals}

The term formation rules for conditionals are depicted in
\autoref{fig:typing_rules_cond}. 

\begin{figure} [H]
\begin{equation*}
\begin{split}
\begin{aligned}
& \hspace{55pt}
\begin{minipage}[t]{0.3\textwidth}
$\begin{array}{c}
     \Gamma \triangleright v: \mathbb{A} \\
    \hline
   \Gamma \triangleright \text{inl}(v):  \mathbb{A} \oplus \mathbb{B}
\end{array}
$
\end{minipage}
\hspace{-38pt}
\text{(inl)} 
 \hspace{20pt}
\begin{minipage}[t]{0.3\textwidth}
$\begin{array}{c}
      \Gamma \triangleright v:  \mathbb{B} \\
    \hline
   \Gamma \triangleright \text{inr}(v): \mathbb{A} \oplus \mathbb{B}
\end{array}
$ \end{minipage} 
\hspace{-35pt} \text{(inr)} \\
&\hspace{15pt}
\begin{minipage}[t]{0.3\textwidth}
$\begin{array}{c}
     \Gamma\triangleright v: \mathbb{A} \oplus \mathbb{B} \quad \Delta, x: \mathbb{A} \triangleright w: \mathbb{C} \quad \Delta, y: \mathbb{B}  \triangleright u : \mathbb{C}   \quad E \in \text{Sf}(\Gamma;\Delta)  \\
    \hline
   E \triangleright \text{ cond } v \hspace{2pt} \{\text{inl} (x) \Rightarrow w ; \hspace{1pt} \text{inr} (y) \Rightarrow u\}: \mathbb{C} 
\end{array}
$
\end{minipage}
\hspace{200pt}
\text{(case)} 
\end{aligned}
\end{split}
\end{equation*}
\caption{Term formation rules for conditionals}
\label{fig:typing_rules_cond}
\end{figure}
Considering  $v \in V$, $w \in W$, and $u \in U$ where $V, W, U$ represent vector spaces, $\textsc{Il}_{V}: V \xrightarrow{} V\oplus W$, denotes the left injection operator, defined as $\textsc{Il}_{V}= v \mapsto (v,0) $; $\textsc{Ir}_{V}: V \xrightarrow{} W \oplus V$, denotes the right injection operator, defined as $\textsc{Ir}_{V}= v \mapsto (0,v) $; and $\text{dist}_{V, W,U}: V \otimes  \left(W \oplus U\right) \xrightarrow{} \left(V \otimes W\right) \oplus \left(V \otimes U\right)$, denotes the distributive property of the tensor product over the direct sum, defined as $\text{dist}_{V, W,U} =  v \otimes  \left(w, u\right) \mapsto \left(v \otimes w, v \otimes u\right)$. The subscripts in these operators will be omitted unless ambiguity arises. Moreover, the operation \text{either} corresponds to:
\begin{figure} [H]
\begin{equation}
\begin{split}
\begin{aligned}
\hspace{95pt}&
\begin{minipage}[t]{0.3\textwidth}
$\begin{array}{c}
     V  \xrightarrow{} U  \\
      W \xrightarrow{} U  \\
    \hline
  [T,S]: V \oplus W \xrightarrow{} U
\end{array}
$
\end{minipage} \\
\hspace{95pt}&
\begin{minipage}[t]{0.3\textwidth}
$\begin{array}{c}
  [T,S] = (v,w) \mapsto T(v)+S(w) 
\end{array}
$
\end{minipage}
\end{aligned}
\end{split}
\end{equation}
\label{fig:either}
\end{figure}

The interpretation of conditionals is illustrated in \autoref{fig:denotational_sem cond}.

\begin{figure} [H]
\begin{equation}
\begin{split}
\begin{aligned}
&\hspace{-80pt} 
 \begin{minipage}[t]{0.3\textwidth}
$\begin{array}{c} 
     [\![\Gamma \triangleright v: \mathbb{A}]\!] = m   \\
    \hline
  [\![ \Gamma \triangleright \text{inl} (v):  \mathbb{A} \oplus \mathbb{B}  ]\!] = \textsc{Il}  \cdot m
\end{array}
$ \end{minipage}
\hspace{30pt} 
\begin{minipage}[t]{0.3\textwidth}
$\begin{array}{c}
     [\![\Gamma \triangleright v:\mathbb{B} ]\!]  = m  \\
    \hline
   [\![\Gamma \triangleright \text{inr} (v):  \mathbb{A} \oplus \mathbb{B}]\!]\!] = \textsc{Ir} \cdot m
\end{array}
$
\end{minipage}\\
\hspace{-25pt}
 \begin{minipage}[t]{0.3\textwidth}
$\begin{array}{c} 
    [\![\Gamma\triangleright v: \mathbb{A} \oplus \mathbb{B} ]\!] = b \quad [\![\Delta, x:\mathbb{A} \triangleright w: \mathbb{C} ]\!] = p  \quad  [\![\Delta,x:\mathbb{B} \triangleright w_{2}: \mathbb{C} ]\!] = q    \quad E \in \text{Sf}(\Gamma;\Delta)  \\
    \hline
  [\![E \triangleright \text{ case } v \hspace{2pt}  \{\text{inl} (x) \Rightarrow w ; \hspace{1pt} \text{inr} (y) \Rightarrow u\}: \mathbb{C} ]\!] =   \text{either}(p,q) \cdot \text{dist} \cdot \text{sw} \cdot (b \otimes \text{id}) \cdot \text{sp}_{\Gamma;\Delta} \cdot \text{sh}_{E}
\end{array}
$ \end{minipage}
\end{aligned}
\end{split}
\end{equation}
\caption{Judgment interpretation for conditionals}
\label{fig:denotational_sem cond}
\end{figure}

\paragraph{Proof} In order to validate the judgment interpretation for conditionals, it is necessary to demonstrate its correctness.

For the booleans: 
\begin{equation} \label{eq:proof_bool}
 \begin{aligned} 
    \hspace{120pt}&  [\![\Gamma ]\!]   \xrightarrow{\hspace{5pt}m\hspace{5pt}} [\![\mathbb{A} ]\!] \xrightarrow{\hspace{6pt}\textsc{Il}\hspace{6pt}} [\![\mathbb{A} \oplus \mathbb{B}]\!] \\ 
     &[\![\Gamma ]\!]   \xrightarrow{\hspace{5pt}m\hspace{5pt}} [\![\mathbb{B} ]\!] \xrightarrow{\hspace{6pt}\textsc{Ir}\hspace{6pt}} [\![\mathbb{A} \oplus \mathbb{B}]\!]
\end{aligned}   
\end{equation}
Now, for the conditional statement:
\begin{equation} \label{eq:proof_bool_2}
 \begin{aligned} 
    [\![E]\!] & \xrightarrow{\hspace{2pt}\text{sh}_{E}\hspace{2pt}}   [\![\Gamma,\Delta ]\!]   \xrightarrow{\hspace{1pt}\text{sp}_{\Gamma;\Delta}\hspace{1pt}}  [\![\Gamma ]\!] \otimes [\![\Delta ]\!] \xrightarrow{ b \hspace{1pt} \otimes \hspace{1pt} \text{id}} ([\![\mathbb{A} ]\!] \oplus [\![\mathbb{B} ]\!]) \otimes [\![\Delta ]\!] \xrightarrow{\hspace{2pt}\text{sw}\hspace{2pt}}  [\![\Delta ]\!] \otimes ([\![\mathbb{A} ]\!] \oplus [\![\mathbb{B} ]\!])  \\
    & \xrightarrow{\hspace{3pt}\text{dist}\hspace{3pt}} ([\![\Delta ]\!] \otimes [\![\mathbb{A} ]\!]  ) \oplus (  [\![\Delta ]\!] \otimes [\![\mathbb{B} ]\!] ) \xrightarrow{\hspace{1pt}\text{either}(p,q)\hspace{1pt}} [\![\mathbb{C} ]\!]
\end{aligned}   
\end{equation}


The quantum lambda calculus with conditionals is illustrated with an example —the quantum teleportation protocol— in \autoref{appendice:teleport}.


The metric equations for conditionals are presented in \autoref{fig:metric conditionals}. Note that the first two equations are redundant.
\begin{figure} [H]
\begin{equation*}
\begin{split}
\begin{aligned}
 &
\begin{minipage}[t]{0.3\textwidth}
$\begin{array}{c}
  v =_{q} w \\
    \hline
   \text{inl}(v) =_{q} \text{inl}(w)
\end{array}
$
\end{minipage}
\hspace{-30pt}
\begin{minipage}[t]{0.3\textwidth}
$\begin{array}{c}
   v =_{q} w \\
    \hline
   \text{inr}(v) =_{q} \text{inr}(w)
\end{array}
$ \end{minipage} \\
\hspace{-30pt}
&
\begin{minipage}[t]{0.3\textwidth}
$\begin{array}{c}
   v =_{q} v' \quad w=_{r} w' \quad u=_{s}u'   \\
    \hline
  \text{ case } v \hspace{2pt}  \{\text{inl} (x) \Rightarrow w ; \hspace{1pt} \text{inr} (y) \Rightarrow u\} =_{q+\text{max}(r, s )} \text{ case } v' \hspace{2pt}  \{\text{inl} (x) \Rightarrow w' ; \hspace{1pt} \text{inr} (y) \Rightarrow u'\} 
\end{array}
$ \end{minipage}
\end{aligned}
\end{split}
\end{equation*}
\caption{Metric equational system for condicionals}
\label{fig:metric conditionals}
\end{figure}

\paragraph{Proof} In order to validate the metric equational system for conditionals, it is necessary to demonstrate its correctness.

The diamond norm is a particular instance of the operator norm. The operator norm [\cite{guide2006infinite}] for a super-operator $E$ is defined as:
\begin{equation} \label{eq:op_norm}
  \lVert E \rVert_{\sigma} = \text{sup} \{ \lVert E(v) \rVert \hspace{2pt} | \hspace{2pt} \lVert v \rVert = 1 \}
\end{equation}


For the booleans:




%Pergunta: nós só precisamos de trace norm e diamond norm em quantica. Usmos a operator norm porque é mais gereal, ou seja a trace norm está nela contida? Also no livro do waltrous eles chamam operator norm de shatten norm em dimensão infinita


Now, regarding the metric equation for the conditional statement, before validating its correctness, it is necessary to prove a few intermediate results. 

The first step is to demonstrate that for any super-operators $P$ and $Q$ the following holds:
\begin{lemma}\label{lem1}
  $\lVert [P,Q] \rVert_{\sigma} \leq \max \{ \lVert P \rVert_{\sigma}, \lVert Q \rVert_{\sigma} \}$
\end{lemma}



$\textit{Proof.}$ \quad Employing the definition of the operator norm in \autoref{eq:op_norm}, it follows that:
\begin{equation} \label{eq:cond_opnorm2}
  \begin{split}
  &\text{sup}{\{ \lVert [P,Q] (v) \rVert  \hspace{2pt} |  \hspace{2pt}  \lVert v \rVert=1  \}}  \leq \text{max} \{  \text{sup} \{ \lVert P (w) \rVert  \hspace{2pt} |  \hspace{2pt}  \lVert w \rVert =1 \}, \text{sup} \{\lVert Q (u) \rVert  \hspace{2pt} |  \hspace{2pt}  \lVert u \rVert=1  \} \} \\
  & = \text{sup}{\{ \lVert [P,Q] (w+u) \rVert  \hspace{2pt} |  \hspace{2pt}  \lVert w+u \rVert=1  \}} \leq \text{max} \{  \text{sup} \{ \lVert P (w) \rVert  \hspace{2pt} |  \hspace{2pt}  \lVert w \rVert = 1, \lVert Q (u) \rVert  \hspace{2pt} |  \hspace{2pt}  \lVert u \rVert=1  \} \} \\
  & =  \text{sup}{\{ \lVert P (w) + Q (u) \rVert  \hspace{2pt} |  \hspace{2pt}  \lVert w+u \rVert=1  \}} \leq \text{max} \{  \text{sup} \{ \lVert P (w) \rVert  \hspace{2pt} |  \hspace{2pt}  \lVert w \rVert =1, \lVert Q (u) \rVert  \hspace{2pt} |  \hspace{2pt}  \lVert u \rVert=1  \} \} \\
  &  =  \text{sup}{\{ \lVert P (w) + Q (u) \rVert  \hspace{2pt} |  \hspace{2pt}  \lVert w+u \rVert=1  \}} \leq \text{sup} \{  \text{max} \{ \lVert P (w) \rVert  \hspace{2pt} |  \hspace{2pt}  \lVert w \rVert =1, \lVert Q (u) \rVert  \hspace{2pt} |  \hspace{2pt}  \lVert u \rVert=1  \} \} \\
\end{split}
\end{equation}

Therefore, by the triangle inequality, proving the inequality in \autoref{eq:cond_opnorm3} suffices to establish  \autoref{lem1}.
\begin{equation} \label{eq:cond_opnorm3}
  \begin{split}
  \text{sup}{\{ \lVert P (w)  \rVert + \lVert Q (u)  \rVert  \hspace{2pt} |  \hspace{2pt}  \lVert w+u \rVert_{1}=1  \}} \leq \text{sup} \{  \text{max} \{ \lVert P (w) \rVert  \hspace{2pt} |  \hspace{2pt}  \lVert w  \rVert =1, \lVert Q (u) \rVert  \hspace{2pt} |  \hspace{2pt}  \lVert u \rVert=1  \} \} \\
  \end{split}
\end{equation}


This can be rewritten as:

\begin{equation} 
  \begin{split}
  \lVert w + u   \rVert = 1 \wedge \{ \lVert P (w)  \rVert + \lVert Q (u)  \rVert  \hspace{2pt} |  \hspace{2pt}  \lVert w+u \rVert=1  \}  \leq \text{max}   \left\{ \dfrac{1}{\lVert w \rVert} \lVert P (w) \rVert  \hspace{2pt},  \dfrac{1}{\lVert u \rVert} \lVert Q (u) \rVert   \right\}
\end{split}
\end{equation}

As a result,
\begin{equation} 
  \begin{split}
  \lVert w + u   \rVert = 1 \wedge \text{sup}{\{ \lVert P (w)  \rVert + \lVert Q (u)  \rVert  \hspace{2pt} |  \hspace{2pt}  \lVert w+u \rVert_{1}  \}}  \leq \text{max}   \left\{  \left\lVert P \left( \dfrac{1}{\lVert w \rVert} w \right) \right\rVert  \hspace{2pt},  \left\lVert Q \left( \dfrac{1}{\lVert u \rVert} u \right) \right\rVert   \right\}
\end{split}
\end{equation}

This is equivalent to demonstrating that for $a+b=1$,
\begin{equation} 
\begin{split}
\hspace{110 pt}
    x + y  \leq  \max \left\{   \dfrac{1}{a}x  ,   \dfrac{1}{b} y   \right\} \\
\end{split}
\end{equation}

This is done by arguing by \textit{reductio ad absurdum}, \textit{i.e.}, supposing otherwise leads to a contradiction:
\begin{equation} 
\begin{split} 
    \hspace{90pt}&
     x + y  >  \max \left\{   \dfrac{1}{a}x  ,   \dfrac{1}{b} y   \right\} \\
    & \Rightarrow  x + y > \dfrac{1}{a}x  \wedge x + y > \dfrac{1}{b}y \\
    & \Rightarrow  a (x + y) > x  \wedge b (x + y)> y \\
    & \Rightarrow  a x + a y > x  \wedge b x + by > y \\
    & \Rightarrow  a x + a y > x  \wedge (1-a) x + (1-a)y > y\\
    & \Rightarrow  a x + a y > x  \wedge x-ax + y -ay > y\\
    & \Rightarrow  x < a x + a y   \wedge x > a x + a y  \\
\end{split}
\end{equation}

\vspace{5pt}

Subsequently, it is imperative to prove that:
\begin{lemma}\label{lemiso}
  $ i= [\textsc{Il} \otimes I, \textsc{Ir} \otimes I ]$ \text{is an isomorphism}.
\end{lemma}

\textit{Proof.} \quad The proof is as follows:

For any vector spaces $V$, $W$, and $U$, $i: (V \otimes U) \oplus (W \otimes U) \xrightarrow{} (V  \oplus W) \otimes U $. If $V$ has dimension $m$, $W$ has dimension $n$, and $U$ has dimension $o$, then the space $(V \otimes U) \oplus (W \otimes U) $ has dimension $mo+no=(m+n)\cdot o$. Similarly, the space $(V\oplus W) \otimes U$ has dimension $(m+n)\cdot o$. Hence, the spaces have the same dimension. Given that spaces with the same dimension are isomorphic [\cite{hefferon2006linear}], it follows that $i$ is an isomorphism.

\vspace{5pt}

Next, it is necessary to demonstrate that for any operators $P$ and $Q$, the identity operator $I$, and an isomorphism $i=[\textsc{Il} \otimes I, \textsc{Ir} \otimes I ]$ the following holds:

\begin{lemma}\label{lem2}
  $( [P,Q] \otimes I) \cdot  i  = [P \otimes I, Q \otimes I]$
\end{lemma}

Which is equivalent to showing that for any vector spaces $V$, $W$, $U$, and $Z$  and super-operators $P: V \xrightarrow{} Z$, $Q: W \xrightarrow{} Z$, and $I: U \xrightarrow{} U$, the following diagram holds:

\vspace{10pt}


\begin{tikzpicture}
  \matrix (m) [matrix of math nodes,row sep=4em,column sep=7em,minimum width=2em]
  {
    V \otimes U \oplus W \otimes U & (V  \oplus W) \otimes U \\
     Z \otimes U \\
  };
  \path[-stealth]
    (m-1-1) edge node [left] {$[P \otimes I, Q \otimes I]$} (m-2-1)
    (m-1-1) edge node [above] {$i$} (m-1-2)
    (m-1-2) edge node [right=0.2cm] {$[P,Q] \otimes I$} (m-2-1);
\end{tikzpicture}


\vspace{10pt}

\textit{Proof.} \quad The proof is straightforward:
\begin{equation}
\begin{split}
    & ( [P,Q] \otimes I) \cdot  [\textsc{Il} \otimes I, \textsc{Ir} \otimes I ]  \\
    &=  [([P,Q] \otimes I) \cdot (\textsc{Il} \otimes I),([P,Q] \otimes I) \cdot (\textsc{Ir} \otimes I) ]\\
    &=  [P \otimes I, Q \otimes I]
\end{split}
\end{equation}

\vspace{15pt}

Furhtermore, it is imperative to show that the following relation holds:

\begin{lemma}\label{lemi-1}
  $ [P \otimes I, Q \otimes I] \cdot  i^{-1}  = [P,Q] \otimes I$
\end{lemma}

Demonstrating this is equivalent to establishing that for any vector spaces $V$, $W$, $U$, and $Z$, and super-operators $P: V \xrightarrow{} Z$, $Q: W \xrightarrow{} Z$, and $I: U \xrightarrow{} U$, the following diagram commutes:

\vspace{10pt}

\begin{tikzpicture}
  \matrix (m) [matrix of math nodes,row sep=4em,column sep=7em,minimum width=2em]
  {
    V \otimes U \oplus W \otimes U & (V  \oplus W) \otimes U \\
     Z \otimes U \\
  };
  \path[-stealth]
    (m-1-1) edge node [left] {$[P \otimes I, Q \otimes I]$} (m-2-1)
    (m-1-2) edge node [above] {$i^{-1}$} (m-1-1)
    (m-1-2) edge node [right=0.2cm] {$[P,Q] \otimes I$} (m-2-1);
\end{tikzpicture}


\textit{Proof.} \quad The proof is as follows:
\begin{equation}
\begin{split}
    & ( [P,Q] \otimes I) \cdot  i  = [P \otimes I, Q \otimes I]  \hspace{100pt} & \text{\{\autoref{lem2}\}} \\
    \Leftrightarrow &  \hspace{2pt} ( [P,Q] \otimes I) \cdot  i \cdot i^{-1} = [P \otimes I, Q \otimes I] \cdot  i^{-1}\\
    \Leftrightarrow &  \hspace{2pt} ( [P,Q] \otimes I)  = [P \otimes I, Q \otimes I] \cdot  i^{-1}  &\text{\{\autoref{lemiso}\}} \\
\end{split}
\end{equation}

\vspace{10pt}
With \autoref{lem2} and \autoref{lemi-1}, it has been proved that the diagram below is valid:
\vspace{5pt}

\begin{tikzpicture}
  \matrix (m) [matrix of math nodes,row sep=4em,column sep=7em,minimum width=2em]
  {
    V \otimes U \oplus W \otimes U & (V  \oplus W) \otimes U \\
     Z \otimes U \\
  };
  \path[-stealth]
    (m-1-1) edge node [left] {$[P \otimes I, Q \otimes I]$} (m-2-1)
    edge[bend left=5] node [above] {$i$}  (m-1-2) % Adjusted minimum width
    (m-1-2) edge node [right=0.5cm] {$[P,Q] \otimes I$} (m-2-1)
    (m-1-2) edge[bend right=-5] node [below] {$i^{-1}$} (m-1-1); % Added the label to the arrow
\end{tikzpicture}

\vspace{5pt}

Subsequently, it is imperative to prove the following:

\begin{lemma} \label{lem3}
  $  \lVert [\textsc{Il} \otimes I, \textsc{Ir} \otimes I ]  \rVert_{1} = 1   $
\end{lemma}

\vspace{10pt}

\textit{Proof.} \quad Employing the definition of the trace norm for a super-operator in \autoref{eq:trace_distance},  considering vector spaces $V$, $W$ and $U$ and vectors $v_i \in V$, $w_i \in W$ and $u_i \in U$, it follows that:
\begin{equation}
\begin{split}
    & \lVert [\textsc{Il} \otimes I, \textsc{Ir} \otimes I ]  \rVert  \\
    &= \text{max} \{ \lVert [\textsc{Il} \otimes I, \textsc{Ir} \otimes I ] (A) \rVert \hspace{2pt} \vert \hspace{2pt}  \lVert A\rVert =1   \} \\
    &= \text{max} \left\{ \left\lVert [\textsc{Il} \otimes I, \textsc{Ir} \otimes I ] \left(\sum_{i} v_i \otimes u_i,\sum_{i} w_i \otimes u_i  \right) \right\rVert \hspace{2pt} \Bigg\vert \hspace{2pt}  \left\lVert \left(\sum_{i} v_i \otimes u_i,\sum_{i} w_i \otimes u_i  \right) \right\rVert =1    \right\} \\
    & =\text{max} \left\{ \left\lVert \textsc{Il} \otimes I \left(\sum_{i} v_i \otimes u_i  \right)  +  \textsc{Ir} \otimes I \left(\sum_{i} w_i \otimes u_i  \right) \right\rVert \hspace{2pt} \Bigg\vert \hspace{2pt}  \left\lVert \left(\sum_{i} v_i \otimes u_i,\sum_{i} w_i \otimes u_i  \right) \right\rVert =1    \right\} \\
    & = \text{max} \Bigg\{ \left\lVert \textsc{Il} \left(\sum_{i} v_i  \right) \otimes I \left(\sum_{i} u_i  \right) + \textsc{Ir} \left(\sum_{i} w_i \right)\otimes I \left(\sum_{i} u_i  \right) \right\rVert \hspace{2pt} \\
    & \hspace{50pt}\Bigg\vert \hspace{2pt}  \left\lVert \left(\sum_{i} v_i \otimes u_i,\sum_{i} w_i \otimes u_i  \right) \right\rVert =1    \Bigg\} \\
    &= \text{max} \Bigg\{ \left\lVert \left(\sum_{i} v_i,0  \right) \otimes \sum_{i} u_i +  \left(0,\sum_{i} w_i \right) \otimes \sum_{i} u_i   \right\rVert  \hspace{2pt} \Bigg\vert \hspace{2pt}  \left\lVert \left(\sum_{i} v_i \otimes u_i,\sum_{i} w_i \otimes u_i  \right) \right\rVert =1    \Bigg\} \\
    &= \text{max} \Bigg\{ \left\lVert \left(\sum_{i} v_i \otimes  u_i ,0  \right) + \left(0,\sum_{i} w_i \otimes u_i  \right)   \right\rVert  \hspace{2pt} \Bigg\vert \hspace{2pt}  \left\lVert \left(\sum_{i} v_i \otimes u_i,\sum_{i} w_i \otimes u_i  \right) \right\rVert =1    \Bigg\} \\
    &= \text{max} \Bigg\{ \left\lVert \left(\sum_{i} v_i \otimes  u_i ,\sum_{i} w_i \otimes u_i   \right)    \right\rVert  \hspace{2pt} \Bigg\vert \hspace{2pt}  \left\lVert \left(\sum_{i} v_i \otimes u_i,\sum_{i} w_i \otimes u_i  \right) \right\rVert =1    \Bigg\} \\
    &=1
\end{split}
\end{equation}


Now, it is finally possible to approach the proof of the metric equation for the conditional statement. Considering the the semantics of the "case" rule in \autoref{fig:denotational_sem cond}, proving that the "case" rule in \autoref{fig:metric conditionals} is valid is equivalent to demonstrating that for any super-operators $P$ and $Q$ and their respective erroneous versions $P'$ and $Q'$, the following holds:
\begin{theorem} \label {theorem:1.1}
  $\text{d} ([P,Q],[P',Q']) \leq \text{max} \{\text{d} (P,P'),\text{d} (Q,Q')\}$
\end{theorem}
\vspace{10pt}
\textit {Proof.} 
Here, $\text{d}(A,B)$ denotes the distance between super-operators $A$ and $B$, which in the quantum paradigm corresponds to the diamond norm between the two super-operators. Hence, employing the definition of the diamond norm in \autoref{eq:diamond_distance}, and denoting $ [\textsc{Il} \otimes I, \textsc{Ir} \otimes I ]$ by $i$ it follows that:

%\begin{equation}
%\begin{split}
  %& \text{d} ([P,Q],[P',Q'])  \\
  %&=   \lVert  [P,Q] \otimes I - [P',Q'] \otimes I   \rVert_{1}  \\
  %&=   \lVert [P \otimes I, Q \otimes I]  - [P' \otimes I, Q' \otimes I]  \rVert_{1}  \\
  %&=  \lVert [P - P' \otimes I, Q-Q' \otimes I]  \rVert_{1}   \\
  %&= \lVert [P -P', Q-Q' ] \otimes I \cdot i \rVert_{1}  \\
%\end{split}
%\end{equation}

\begin{equation} \label{eq:proof_theorem1.1_esq}
  \begin{split}
    & \text{d} ([P,Q],[P',Q'])  \\
    &=  \lVert  [P,Q] \otimes I - [P',Q'] \otimes I   \rVert  \\
    &=   \lVert [P \otimes I, Q \otimes I] \cdot i  - [P' \otimes I, Q' \otimes I]  \cdot i  \rVert  \\
    &=  \lVert [P - P' \otimes I, Q-Q' \otimes I] \cdot i  \rVert   \\
    & \leq \lVert [P - P' \otimes I, Q-Q' \otimes I]  \rVert \lVert i  \rVert  \\
    &=  \lVert [P - P' \otimes I, Q-Q' \otimes I]  \rVert \hspace{200 pt} \text{by  \autoref{lem3}} \\
  \end{split}
  \end{equation}



%\begin{equation}
  %\begin{split}
    %& \text{d} ([P,Q],[P',Q'])  \\
    %&=   \lVert  [P,Q] \otimes I - [P',Q'] \otimes I    \rVert_{1} \hspace{2pt} \\
    %&=   \lVert  [P-P',Q-Q'] \otimes I  \rVert_{1}   \\
    %&=    \lVert  [P-P',Q-Q'] \rVert_{1} \lVert I \rVert_{1}\hspace{2pt} \\
    %&=    \lVert  [P-P',Q-Q'] \rVert_{1} \hspace{2pt} \\
  %\end{split}
  %\end{equation}

  % The spectral norm is submultiplicative with respect to compositions and multiplicative with respect to tensor products,

  % Flar sobre definições de normas


%m

and
\begin{equation} \label {eq:proof_theorem1.1_dir}
\begin{split}
   &  \text{max} \{\text{d} (P,P'),\text{d} (Q,Q')\} \\
   &=  \text{max} \{ \text{max} \{ \lVert (P \otimes I - P' \otimes I) (v)  \rVert \vert \lVert v \rVert=1 \}, \text{max} \{ \lVert (Q \otimes I - Q' \otimes I) (w)  \rVert \vert \lVert w \rVert=1 \} \} \\
   &=  \text{max} \{ \text{max} \{ \lVert (P \otimes I - P' \otimes I) (v)  \rVert ,  \lVert (Q \otimes I - Q' \otimes I) (w)  \rVert \vert  \lVert v \rVert=1 \wedge \lVert w \rVert=1  \} \} \\
\end{split}
\end{equation}

Finally, with results in \autoref{eq:proof_theorem1.1_esq} and \autoref{eq:proof_theorem1.1_dir}, by Lemma \autoref{lem1}, it follows that $\text{d} ([P,Q],[P',Q']) \leq \text{max} \{\text{d} (P,P'),\text{d} (Q,Q')\}$, which concludes the proof of theorem \autoref{theorem:1.1}.






%hefferon2006linear


\subsection{Ilustration: Noisy Quantum Teleportation}

\vspace{0pt}

To study decoherence in a quantum channel within the presented metric deductive system, one can consider the application of a dephasing channel in the quantum teleportation protocol with a certain probability $p$. This is exemplified for probabilities $p=0.5$ and $p=0.25$. It is worth noting that similar exercises can be done for scenarios such as a malicious attack involving a bit flip during measurement or the presence of a noisy channel.
\chapter{Grades modalities}



\section{Conditionals}

\section{Quantum and relation with symmetric subspaces and construction of graded modalities}

\section{Discriminating Two Pure Quantum States}
\todo[inline,size=\normalsize]{pôr introdução a quantum state discrimination e a sua importância e de onde vem a melhor estragegia -> livro Barret}



Given a pure $d$-dimensional state \ket{\psi} known to be either \ket{\psi_0} or \ket{\psi_1}, one must guess which state \ket{\psi} is. In quantum state discrimination, we wish to design a measurement to distinguish optimally between \ket{\psi_0} or \ket{\psi_1}. 

Assume with out loss of generality the angle between \ket{\psi_0}  and \ket{\psi_1}, designated $\alpha$, is between $0$ and $\frac{pi}{2}$. Otherwise, replace \ket{\psi_0} is replaced by $- \ket{\psi_0}$.

In this case the best strategy is is to do the projective measurement with \{\ket{v_0}, \ket{v_1}\}, where \ket{v_0}, \ket{v_1}\ are in the span of \ket{\psi_0}  and \ket{\psi_1} such that $\langle v_{0}| v_{0} \rangle = 0$, they are symmetric with
respect to the angle bisector of \ket{\psi_0}  and \ket{\psi_1}, and \ket{v_i} is closer to  \ket{\psi_i} for $i = 0, 1$. On outcome $“i”$, we guess $\psi_i$.

\begin{figure} [H] 
    \centering
    \begin{center}
        \includegraphics[width=0.6\textwidth]{images/qsd_0.png}
    \end{center}
\caption{Optimal minimum error measurement for discriminating between the pure states \ket{\psi_0}  and  \ket{\psi_1}. This is a
projective measurement onto the states \ket{v_0}  and  \ket{v_1}, symmetrically located on either side of the signal states and shown in red here. $\gamma$ is  the angle between the states \ket{\psi_0}  and  \ket{\psi_1}. $\beta$ is the angle between the states \ket{\psi_0}  and  \ket{v_0} \big/ \ket{\psi_1}  and  \ket{v_1} .}
\label{fig:qsd}
\end{figure}

The probability of success using the best strategy is

\begin{equation*}
    P_{succ} = \langle \psi_{0}| v_{0} \rangle = \cos^2(\beta) = \cos^2 \left(\frac{\pi/2 - \gamma}{2}\right) = \frac{1}{2} + \frac{1}{2} \cos(\pi/2 -\gamma) = \frac{1}{2} + \frac{1}{2} \sin(\gamma)
\end{equation*}

\subsection{QSD for two pure states: quantum lambda calculus formulation}

Since the quantum lambda calculus presented allows only for explicit projective measurements in the computational basis, it is necessary to rotate the state \ket{\psi} so that \ket{v_0} and \ket{v_1} coincide with the computational basis. This can be done by applying a unitary $U$ to the state \ket{\psi} such that $U \ket{v_0} = \ket{0}$. 

Observing \autoref{fig:qsd} is possible to conclude that $\beta =  \frac{\frac{\pi}{2}-\gamma}{2} = \frac{\pi}{4} - \frac{\gamma}{2}$. Therefore,

\begin{equation}
    \begin{split}
        \ket{v_0} =R^{\dag}_{\beta} \ket{\psi_0} & = \begin{pmatrix}
        \cos{ \left(\frac{\pi}{4} - \frac{\gamma}{2}\right)}  & \sin{ \left(\frac{\pi}{4} - \frac{\gamma}{2}\right)} \\
       - \sin{ \left(\frac{\pi}{4} - \frac{\gamma}{2}\right)}  &  \cos{ \left(\frac{\pi}{4} - \frac{\gamma}{2}\right)} \end{pmatrix} \begin{pmatrix}
        \cos{ \left(\frac{\theta}{2}\right)}   \\
       e^{i \phi}\sin{ \left(\frac{\theta}{2}\right)}  \end{pmatrix} \\
       & = \begin{pmatrix}
        \cos{ \left(\frac{\pi}{4} - \frac{\gamma}{2}\right)}  \cos{ \left(\frac{\theta}{2}\right)}  + \sin{ \left(\frac{\pi}{4} - \frac{\gamma}{2}\right)} e^{i \phi}\sin{ \left(\frac{\theta}{2}\right)}  \\
       - \sin{ \left(\frac{\pi}{4} - \frac{\gamma}{2}\right)}  \cos{ \left(\frac{\theta}{2}\right)}  +  \cos{ \left(\frac{\pi}{4} - \frac{\gamma}{2}\right)} e^{i \phi}\sin{ \left(\frac{\theta}{2}\right)}   \end{pmatrix} 
    \end{split}
\end{equation}

Consequently, 

\begin{equation}
    \begin{split}
        \ket{v_1}  = \begin{pmatrix}
         \sin{ \left(\frac{\pi}{4} - \frac{\gamma}{2}\right)}  \cos{ \left(\frac{\theta}{2}\right)}  -  \cos{ \left(\frac{\pi}{4} - \frac{\gamma}{2}\right)} e^{i \phi}\sin{ \left(\frac{\theta}{2}\right)}   
        \\
        \cos{ \left(\frac{\pi}{4} - \frac{\gamma}{2}\right)}  \cos{ \left(\frac{\theta}{2}\right)}  + \sin{ \left(\frac{\pi}{4} - \frac{\gamma}{2}\right)} e^{i \phi}\sin{ \left(\frac{\theta}{2}\right)}  \end{pmatrix} 
    \end{split}
\end{equation}

As a result,
\begin{equation}
    \begin{split}
        U =  \begin{pmatrix} \cos{ \left(\frac{\pi}{4} - \frac{\gamma}{2}\right)}  \cos{ \left(\frac{\theta}{2}\right)}  + \sin{ \left(\frac{\pi}{4} - \frac{\gamma}{2}\right)} e^{-i \phi}\sin{ \left(\frac{\theta}{2}\right)}  & - \sin{ \left(\frac{\pi}{4} - \frac{\gamma}{2}\right)}  \cos{ \left(\frac{\theta}{2}\right)}  +  \cos{ \left(\frac{\pi}{4} - \frac{\gamma}{2}\right)} e^{-i \phi}\sin{ \left(\frac{\theta}{2}\right)}   \\
            \sin{ \left(\frac{\pi}{4} - \frac{\gamma}{2}\right)}  \cos{ \left(\frac{\theta}{2}\right)}  -  \cos{ \left(\frac{\pi}{4} - \frac{\gamma}{2}\right)} e^{-i \phi}\sin{ \left(\frac{\theta}{2}\right)}  &  \cos{ \left(\frac{\pi}{4} - \frac{\gamma}{2}\right)}  \cos{ \left(\frac{\theta}{2}\right)}  + \sin{ \left(\frac{\pi}{4} - \frac{\gamma}{2}\right)} e^{-i \phi}\sin{ \left(\frac{\theta}{2}\right)}  \end{pmatrix} 
    \end{split}
\end{equation}

When $\ket{\psi} = \ket{\psi_0}$, one has that,
\begin{equation}
    \begin{split}
        &U \ket{\psi} = \\
        &\scriptsize{
            \begin{pmatrix} 
                \left[\cos{ \left(\frac{\pi}{4} - \frac{\gamma}{2}\right)}  \cos{ \left(\frac{\theta}{2}\right)}  + \sin{ \left(\frac{\pi}{4} - \frac{\gamma}{2}\right)} e^{-i \phi}\sin{ \left(\frac{\theta}{2}\right)}\right] \cos{ \left(\frac{\theta}{2}\right)}  +  
                \left[ - \sin{ \left(\frac{\pi}{4} - \frac{\gamma}{2}\right)}  \cos{ \left(\frac{\theta}{2}\right)}  +  \cos{ \left(\frac{\pi}{4} - \frac{\gamma}{2}\right)} e^{-i \phi}\sin{ \left(\frac{\theta}{2}\right)} \right] e^{i \phi}\sin{ \left(\frac{\theta}{2}\right)} \\
                \left[\sin{ \left(\frac{\pi}{4} - \frac{\gamma}{2}\right)}  \cos{ \left(\frac{\theta}{2}\right)}  -  \cos{ \left(\frac{\pi}{4} - \frac{\gamma}{2}\right)} e^{-i \phi}\sin{ \left(\frac{\theta}{2}\right)}\right]\cos{ \left(\frac{\theta}{2}\right)} + \left[ \cos{ \left(\frac{\pi}{4} - \frac{\gamma}{2}\right)}  \cos{ \left(\frac{\theta}{2}\right)}  + \sin{ \left(\frac{\pi}{4} - \frac{\gamma}{2}\right)} e^{-i \phi}\sin{ \left(\frac{\theta}{2}\right)} \right] e^{i \phi}\sin{ \left(\frac{\theta}{2}\right)}
            \end{pmatrix}  } \\
        & = \begin{pmatrix} \cos{ \left(\frac{\pi}{4} - \frac{\gamma}{2}\right)} \frac{\cos{ \left(\theta\right)}+1} {2}   + e^{-i \phi} \sin{ \left(\frac{\pi}{4} - \frac{\gamma}{2}\right)} \frac{\sin{ \left(\theta\right)}} {2} - e^{i \phi} \sin{ \left(\frac{\pi}{4} - \frac{\gamma}{2}\right)} \frac{\sin{ \left(\theta\right)}} {2} +  \cos{ \left(\frac{\pi}{4} - \frac{\gamma}{2}\right)} \frac{\cos{ 1- \left(\theta\right)}} {2} \\
        \sin{ \left(\frac{\pi}{4} - \frac{\gamma}{2}\right)} \frac{\cos{ \left(\theta\right)}+1} {2}   - e^{-i \phi} \cos{ \left(\frac{\pi}{4} - \frac{\gamma}{2}\right)} \frac{\sin{ \left(\theta\right)}} {2} + e^{i \phi} \cos{ \left(\frac{\pi}{4} - \frac{\gamma}{2}\right)} \frac{\sin{ \left(\theta\right)}} {2} +  \sin{ \left(\frac{\pi}{4} - \frac{\gamma}{2}\right)} \frac{\cos{ 1- \left(\theta\right)}} {2}  
            \end{pmatrix} \\
            & = \begin{pmatrix} \cos{ \left(\frac{\pi}{4} - \frac{\gamma}{2}\right)}   + (e^{-i \phi}-e^{i \phi}) \sin{ \left(\frac{\pi}{4} - \frac{\gamma}{2}\right)} \frac{\sin{ \left(\theta\right)}} {2} \\
            \sin{ \left(\frac{\pi}{4} - \frac{\gamma}{2}\right)} \  (- e^{-i \phi} + e^{i \phi} ) \cos{ \left(\frac{\pi}{4} - \frac{\gamma}{2}\right)} \frac{\sin{ \left(\theta\right)}} {2} \end{pmatrix} \\
            & = \begin{pmatrix} \cos{ \left(\frac{\pi}{4} - \frac{\gamma}{2}\right)}   + 2i \sin{(\phi)} (\cos{ \left(\frac{\pi}{4} - \frac{\gamma}{2}-\theta\right)} - \cos{ \left(\frac{\pi}{4} - \frac{\gamma}{2}+\theta \right)} )\\
            \sin{ \left(\frac{\pi}{4} - \frac{\gamma}{2}\right)} \  - 2i \sin{(\phi)} (\sin{ \left(\frac{\pi}{4} - \frac{\gamma}{2}+\theta\right)} - \sin{ \left(\frac{\pi}{4} - \frac{\gamma}{2}-\theta \right)} ) \end{pmatrix} 
    \end{split}
\end{equation}

%Given the direction of the rotation, the angle $\alpha$ is negative, so the rotation is $R_{-\alpha}$ which also correspons to $R^{\dag}_{\alpha}$.

%As a result, the quantum discrimination for two pure states can be formulated as follows:
%\begin{align*}
  %q: \textit{qbit}\hspace{3 pt} \triangleright \hspace{3 pt} \textit{meas} (U (q)) 
%\end{align*}


\chapter{Future work}\label{ch:future_work}

% Quantais

% Boolean (partial order), ultrametric (ultrametric spaces), 

%whichis (tacitly) used to interpret Nakano’s guarded λ-calculus [BSS10] and also to interpret a higher-order language for functional reactive programming [KB11]. Another interesting quantale is the G¨odel one which is a basis for fuzzy logic [DEW13] and whose V-equations give rise to what we call fuzzy inequations.

% Real-time computation ->  Now, it may be the case that is unnecessary to know the distance between the execution time of two programs– instead it suffices to know whether a program finishes its execution before another one. 

%Equations t =0 s state that the terms t and s are exactly the same and equations t =ϵ s state that t and s differ by at most ϵ seconds in their execution time.

% Boolean quantale, V-equations are labelled by {0,1}. We will see that Γ ▷v =1 w : A can be effectively treated as an inequation Γ ▷v ≤ w : A, whilst Γ▷v =0 w : A corresponds to a trivial V-equation, i.e. a V-equation that always holds.

% fuzzy logic -> imprecise date

% guarded -> guarded recursion (so bad things dont happen)

%functional reative programming -> functional + reative (real time data and events)


% Estender parte quantica para higher order

%The quantum models discussed in SECALGO are not closed. In \cite{dahlqvist2023syntactic}, the authors used general results from category theory to address a similar issue in the category $\catCPTP$. A natural next step would be to extend such a construction for $\catQ$ and $\WstarCPSUop$.



% Graded cenas

%estender modelo qunatico seria util por causa de quantum discrimination e quatum metrology -> ver onde andam os artigos

%\cite{Multiple_copy_two_state_discrimination}

%\cite{Giovannetti_Quantum_Metrology, Zhou_Limits_Noisy_Quantum_Metrology}

%The goal is to estimate an unknown parameter encoded in a quantum channel, To improve the sensitivity and accuracy of measurements, quantum metrology takes advantage of the peculiar properties of quantum systems, such as entanglement and superposition. It is possible, for example, to estimate physical quantities more precisely using entangled states rather than classical states. Numerous applications of quantum metrology — which we will go into more detail shortly — can be found in fields such as navigation, communication, and medicine.




\chapter{Planned Schedule}

\section{Activities}

\newacronym{pdr}{PDR}{Preliminary Dissertation Report}
\newacronym{soa}{SOA}{State of the Art}

\begin{table}[H]
\begin{center}
\begin{tabular}{| c | c | c | c | c | c | c | c | c | c | c |}
\hline
\textbf{Task} & \textbf{Oct} & \textbf{Nov} & \textbf{Dec} & \textbf{Jan} & \textbf{Feb} & \textbf{Mar} & \textbf{Apr} & \textbf{May} & \textbf{Jun} & \textbf{Jul}\\
\hline
Background and \acrshort{soa} & $\bullet$ & $\bullet$ & $\bullet$ & & & & & & & \\
\hline
\acrshort{pdr} preparation & & $\bullet$ & $\bullet$ & $\bullet$ & & & & & & \\
\hline
Contribution & & & &$\bullet$ &$\bullet$ &$\bullet$ &$\bullet$ &$\bullet$ &$\bullet$ & \\
\hline
Writing up & & & & & & & $\bullet$ & $\bullet$ & $\bullet$ & $\bullet$ \\
\hline
\end{tabular}
\end{center}
\caption{Activities Plan}
\end{table}

%\chapter{Graded modalities} \label{chap:graded}

%{selinger2009quantum} -> type !qubit does not exist in the quantum lambda calculus

The linearity
constraint is often deemed too restrictive, prompting research into relaxing it in various computational paradigms. In \cite{dahlqvist2023complete}, the controlled use of a resource multiple times is explored within approximate program equivalence paradigms. Moreover, the grammar introduced allows the specification of how many times a resource can be used—a notion particularly relevant in quantum computation, especially within the NISQ era where resources are scarce.

\todo[inline,size=\normalsize]{Intro} 

\section{Syntax}

Here, the following grammar of types is used.
\begin{equation*} \label{eq:grammar_graded}
  \centering
   \mathbb{A} ::= X  \hspace{3 pt} \vert \hspace{3 pt} \mathbb{I}  \hspace{3 pt}  \vert \hspace{3 pt} \mathbb{A}  \otimes  \mathbb{A} \hspace{3 pt} \vert \hspace{3 pt} \mathbb{A} \oplus \mathbb{A} \hspace{3 pt}  \vert \hspace{3 pt}   \mathbb{A} \multimap  \mathbb{A} \vert \hspace{3 pt} !_{r} \mathbb{A} \hspace{100pt} {X \in G,r \in \mathbb{N}} 
  \end{equation*}


  \begin{figure} [H]
    {\small
    \begin{equation*}
    \begin{split}
    \begin{aligned}
    &
    \begin{minipage}[t]{0.3\textwidth}
    $\begin{array}{c}
         \Gamma_{i} \triangleright v_{i}: !_{r\cdot s_{i}} \mathbb{A}_{i} \quad x_{1}:!_{ s_{1}} \mathbb{A}_{1},\ldots, x_{n}:!_{s_{n}} \mathbb{A}_{n}\triangleright u: \mathbb{A} \quad E \in \text{Sf}(\Gamma_{1}; \ldots; \Gamma_{n})\\
        \hline
       E \triangleright \text{pr}_{(r,[s_1,\ldots,s_n])} v_1, \ldots, v_n \text{ fr } x_1, \ldots, x_n. \hspace{1pt} u : !_{r} \mathbb{A}
    \end{array}
    $
    \end{minipage}
    \hspace{193pt}
    (!_{i})
     \hspace{10pt}
    \begin{minipage}[t]{0.3\textwidth}
    $\begin{array}{c}
      \Gamma  \triangleright v : !_{1} \mathbb{A} \\
        \hline
        \Gamma \triangleright \text{dr } v:\mathbb{A}
    \end{array}
    $ \end{minipage}
    \hspace{-74pt} (!_{e}) \\
    &
    \begin{minipage}[t]{0.3\textwidth}
    $\begin{array}{c}
        \Gamma \triangleright v : !_{0} \mathbb{A} \hspace{7pt} \Delta \triangleright u: \mathbb{B} \hspace{7pt} E \in \text{Sf}(\Gamma, \Delta)\\
        \hline
       E \triangleright v.\hspace{1pt}u: \mathbb{B}
    \end{array}
    $
    \end{minipage}
    \hspace{35pt}
    (!_{0})
    \hspace{5pt}
    \begin{minipage}[t]{0.3\textwidth}
    $\begin{array}{c}
      \Gamma\triangleright v: !_{n+m} \mathbb{A} \hspace{7pt} \Delta,x:!_{n} \mathbb{A}, y:!_{m} \mathbb{B}\triangleright u: \mathbb{B} \hspace{7pt} E \in\text{Sf}(\Gamma, \Delta)\\
        \hline
       E \triangleright \text{cp}_{(n,m)} v \text{ to } x,y.  \hspace{1pt} u: \mathbb{B}
    \end{array}
    $ \end{minipage}
    \hspace{120pt} (!_{n+m}) \\
    \end{aligned}
    \end{split}
    \end{equation*}
    }  
    \caption{Term formation rules of graded lambda calculus.}
    \label{fig:typing_rules_graded}
    \end{figure}


\section{Interpretation}
%falar aqui do subspaço simetrico

\section{Quantum State Discrimination}


\renewcommand{\baselinestretch}{1}
\bibliographystyle{plainnat}
\bibliography{dissertation}
\printindex

\appendix
\renewcommand\chaptername{Appendix}

\part{Appendices}

\input{appendices/SupportWork}
\input{appendices/DetailsOfResults}
\input{appendices/Listings}
\input{appendices/Tooling}

\pagestyle{empty}
\cleartooddpage
\null
\thispagestyle{empty}
\pagecolor{PANTONECoolGray7C}
\afterpage{\nopagecolor}
\newpage

\begin{backcover}
\thispagestyle{empty}{~\vfill
\noindent
This work is financed by National Funds through the FCT - Fundação para a Ciência e a Tecnologia, I.P. (Portuguese Foundation for Science and Technology) within the project \textit{IBEX}, with reference \texttt{PTDC/CCI-COM/4280/2021} (\href{https://doi.org/10.54499/PTDC/CCI-COM/4280/2021}{https://doi.org/10.54499/PTDC/CCI-COM/4280/2021}).
\vfill ~}
\end{backcover}



\end{document}

Prove that the norm of the commutator of two operators is less than or equal to 1.
\begin{equation}
    \begin{split}
        & \lVert [\textsc{Il} \otimes I, \textsc{Ir} \otimes I ]  \rVert  \\
        &= \text{max} \{ \lVert [\textsc{Il} \otimes I, \textsc{Ir} \otimes I ] (A) \rVert \hspace{2pt} \vert \hspace{2pt}  \lVert A\rVert =1   \} \\
        &= \text{max} \left\{ \left\lVert [\textsc{Il} \otimes I, \textsc{Ir} \otimes I ] \left(\sum_{i} v_i \otimes u_i,\sum_{i} w_i \otimes u_i  \right) \right\rVert \hspace{2pt} \Bigg\vert \hspace{2pt}  \left\lVert \left(\sum_{i} v_i \otimes u_i,\sum_{i} w_i \otimes u_i  \right) \right\rVert =1    \right\} \\
        & =\text{max} \left\{ \left\lVert \textsc{Il} \otimes I \left(\sum_{i} v_i \otimes u_i  \right)  +  \textsc{Ir} \otimes I \left(\sum_{i} w_i \otimes u_i  \right) \right\rVert \hspace{2pt} \Bigg\vert \hspace{2pt}  \left\lVert \left(\sum_{i} v_i \otimes u_i,\sum_{i} w_i \otimes u_i  \right) \right\rVert =1    \right\} \\
        & = \text{max} \Bigg\{ \left\lVert \textsc{Il} \left(\sum_{i} v_i  \right) \otimes I \left(\sum_{i} u_i  \right) + \textsc{Ir} \left(\sum_{i} w_i \right)\otimes I \left(\sum_{i} u_i  \right) \right\rVert \hspace{2pt} \\
        & \hspace{50pt}\Bigg\vert \hspace{2pt}  \left\lVert \left(\sum_{i} v_i \otimes u_i,\sum_{i} w_i \otimes u_i  \right) \right\rVert =1    \Bigg\} \\
        &= \text{max} \Bigg\{ \left\lVert \left(\sum_{i} v_i,0  \right) \otimes \sum_{i} u_i +  \left(0,\sum_{i} w_i \right) \otimes \sum_{i} u_i   \right\rVert  \hspace{2pt} \Bigg\vert \hspace{2pt}  \left\lVert \left(\sum_{i} v_i \otimes u_i,\sum_{i} w_i \otimes u_i  \right) \right\rVert =1    \Bigg\} \\
        &= \text{max} \Bigg\{ \left\lVert \left(\sum_{i} v_i \otimes  u_i ,0  \right) + \left(0,\sum_{i} w_i \otimes u_i  \right)   \right\rVert  \hspace{2pt} \Bigg\vert \hspace{2pt}  \left\lVert \left(\sum_{i} v_i \otimes u_i,\sum_{i} w_i \otimes u_i  \right) \right\rVert =1    \Bigg\} \\
        &= \text{max} \Bigg\{ \left\lVert \left(\sum_{i} v_i \otimes  u_i ,\sum_{i} w_i \otimes u_i   \right)    \right\rVert  \hspace{2pt} \Bigg\vert \hspace{2pt}  \left\lVert \left(\sum_{i} v_i \otimes u_i,\sum_{i} w_i \otimes u_i  \right) \right\rVert =1    \Bigg\} \\
        &=1
    \end{split}
    \end{equation}

    \begin{equation}
        \begin{split}
            & \hspace{3pt} \lVert [\textsc{Il} \otimes I, \textsc{Ir} \otimes I ]  \rVert_{}  \\
            &= \text{max} \{ \lVert [\textsc{Il} \otimes I, \textsc{Ir} \otimes I ] (A) \rVert \hspace{2pt} \vert \hspace{2pt}  \lVert A\rVert =1   \} \\
            &= \text{max} \left\{ \left\lVert [\textsc{Il} \otimes I, \textsc{Ir} \otimes I ] \left(\sum_{i} v_i \otimes u_i,\sum_{i} w_i \otimes u_i  \right) \right\rVert \hspace{2pt} \Bigg\vert \hspace{2pt}  \left\lVert \left(\sum_{i} v_i \otimes u_i,\sum_{i} w_i \otimes u_i  \right) \right\rVert =1    \right\} \\
            & =\text{max} \left\{ \left\lVert \textsc{Il} \otimes I \left(\sum_{i} v_i \otimes u_i  \right)  +  \textsc{Ir} \otimes I \left(\sum_{i} w_i \otimes u_i  \right) \right\rVert \hspace{2pt} \Bigg\vert \hspace{2pt}  \left\lVert \left(\sum_{i} v_i \otimes u_i,\sum_{i} w_i \otimes u_i  \right) \right\rVert =1    \right\} \\
            & = \text{max} \Bigg\{ \left\lVert \textsc{Il} \left(\sum_{i} v_i  \right) \otimes I \left(\sum_{i} u_i  \right) + \textsc{Ir} \left(\sum_{i} w_i \right)\otimes I \left(\sum_{i} u_i  \right) \right\rVert \hspace{2pt} \\
            & \hspace{50pt}\Bigg\vert \hspace{2pt}  \left\lVert \left(\sum_{i} v_i \otimes u_i,\sum_{i} w_i \otimes u_i  \right) \right\rVert =1    \Bigg\} \\
            &= \text{max} \Bigg\{ \left\lVert \left(\sum_{i} v_i,0  \right) \otimes \sum_{i} u_i +  \left(0,\sum_{i} w_i \right) \otimes \sum_{i} u_i   \right\rVert  \hspace{2pt} \Bigg\vert \hspace{2pt}  \left\lVert \left(\sum_{i} v_i \otimes u_i,\sum_{i} w_i \otimes u_i  \right) \right\rVert =1    \Bigg\} \\
            &= \text{max} \Bigg\{ \left\lVert \left(\sum_{i} v_i \otimes  u_i ,0  \right) + \left(0,\sum_{i} w_i \otimes u_i  \right)   \right\rVert  \hspace{2pt} \Bigg\vert \hspace{2pt}  \left\lVert \left(\sum_{i} v_i \otimes u_i,\sum_{i} w_i \otimes u_i  \right) \right\rVert =1    \Bigg\} \\
            &= \text{max} \Bigg\{ \left\lVert \left(\sum_{i} v_i \otimes  u_i ,\sum_{i} w_i \otimes u_i   \right)    \right\rVert  \hspace{2pt} \Bigg\vert \hspace{2pt}  \left\lVert \left(\sum_{i} v_i \otimes u_i,\sum_{i} w_i \otimes u_i  \right) \right\rVert =1    \Bigg\} \\
            &=1
        \end{split}
        \end{equation}


        %maliciuos atack a,b,c,d
        
\begin{figure} [H]
    \centering
    \begin{quantikz} [column sep=0.2cm, row sep=0.5cm,wire
      types={n,n,n,n,n}]%
        \lstick{$\ket{\psi}$}  & \qw &\qw & \qw & \qw & \qw &  \ctrl{1} \qw \gategroup[2,steps=2,style={dashed,rounded
        corners,fill=blue!20, inner
        xsep=2pt},background,label style={label
        position=below,anchor=north,yshift=-0.2cm}]{{\sc
        B}} & \gate{H} \qw & \qw & \qw &\qw \gategroup[5,steps=9,style={dashed,rounded
        corners,fill=blue!20, inner
        xsep=2pt},background,label style={label
        position=below,anchor=north,yshift=-0.2cm}]{{\sc
        C}}  & \qw & \qw & \qw & \targ{} \qw  & \qw & \qw & \qw & \qw & \qw & \qw & \meter{} \qw \gategroup[3,steps=6,style={dashed,rounded
        corners,fill=blue!20, inner
        xsep=2pt},background,label style={label
        position=below,anchor=north,yshift=-0.2cm}]{{\sc
        D}}     & \setwiretype{c}  &  &  & & \ctrl[vertical
  wire=c]{2}  \\
        \lstick {$\ket{0}$}  &\gate{H}\gategroup[2,steps=3,style={dashed,rounded
        corners,fill=blue!20, inner
        xsep=2pt},background,label style={label
        position=below,anchor=north,yshift=-0.2cm}]{{\sc
        A}} & \qw  & \ctrl{1} \qw& \qw & \qw  & \targ{} \qw & \qw & \qw & \qw & \qw& \qw & \qw & \qw & \qw &  \qw & \qw &  \targ{} \qw & \qw & \qw & \qw & \meter{} \qw & \setwiretype{c}  & & \ctrl[vertical
  wire=c]{1} \\
        \lstick{$\ket{0}$}  &  \qw & \qw &  \targ{} \qw \qw & \qw &\qw&\qw & \qw & \qw& \qw & \qw & \qw & \qw& \qw & \qw & \qw & \qw &  \qw & \qw & \qw & \qw & \qw & \qw & \qw &  \gate{X} \qw & \qw & \gate{Z} \qw\\
         &   & &  &  & & &  &  & & & & \lstick{$\ket{0}$}  & \qw & \qw & \qw  & \gate{R_X(\frac{\pi}{2})} \qw& \ctrl{-2} \qw & \gate{\text{Disc}} \qw  \\
        &   &  & & & & &  &  & & & & \lstick{$\ket{0}$}   &\gate{R_X(\frac{\pi}{2})} \qw  & \ctrl{-4} \qw & \gate{\text{Disc}} \qw & &  &  &  &  &   & &  &  &  &  &  
      \end{quantikz}
    \caption{Quantum Teleportation Protocol: Bit flip with a 50\% probability before measurement.}
    \label{fig:teleport_bit_flip}
  \end{figure}