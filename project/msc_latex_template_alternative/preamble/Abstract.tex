\chapter*{Abstract}

\begin{comment}
Noisy intermediate-scale quantum (NISQ)  computers are expected to operate with severely limited hardware resources. Precisely controlling qubits in these systems comes at a high cost, is susceptible to errors, and faces scarcity challenges. Therefore, error analysis is indispensable for the design, optimization, and assessment of NISQ computing. Nevertheless, the analysis of errors in quantum programs poses a significant challenge. The overarching goal of the M.Sc. project is to provide a fully-fledged
quantum programming language on which to study metric program equivalence
in various scenarios, such as in quantum algorithmics and quantum information theory.
\end{comment}

In recent decades, there has been an effort in computer science to move beyond rigid binary notions—such as equality and bisimulation—toward more flexible approaches that better reflect the  subtleties of real-world computation. Traditional program equivalence, for example, is purely dichotomous: two programs are either equivalent or not. Yet in many computational paradigms, this binary perspective proves too restrictive. For instance, in contexts involving physical environments and noisy data, a more nuanced notion --- such as approximate program equivalence---becomes imperative. It is within this evolving landscape that our work is situated.

We build on the work of \cite{dahlqvist2023syntactic} by introducing a metric equation for conditionals and proving its soundness and completeness. 
Syntactically, to illustrate the utility of the metric equation introduced, we present a illustrative example: a metric version of the copairing's extensionality. On the semantic side, we present five categories that satisfy the necessary requirements for interpreting this equation, thereby demonstrating the broad applicability of our approach across several domains. Finally, we illustrate the use of the metric equation in more detail within both the probabilistic and quantum computing paradigms. For quantum models, we focus on the first-order fragment of the $\lambda$-calculus, though extensions to higher-order are possible using advanced categorical tools, as in \cite{dahlqvist2023syntactic}. 

\begin{comment}
On the semantic side, we show that the following categories satisfy the necessary requirements for interpreting this equation:
\begin{itemize}
    \item the category of metric spaces;
    \item the cocompletion of a $\catMet$-category $\catC$;
    \item the category $\catBan$ of Banach spaces and short maps;
    \item Selinger’s category \( \catQ \) of quantum operations (i.e., completely positive, trace-nonincreasing superoperators)~\cite{selinger2004towards};
    \item Cho’s category \( \WstarCPSUop \), the opposite of \( \WstarCPSU \), consisting of W$^*$-algebras and normal, completely positive, subunital maps.
\end{itemize}
This demonstrates the broad applicability of our approach across several domains.
\end{comment}




\paragraph{Keywords} quantitative reasoning, $\lambda$-calculus, metric equations

\cleardoublepage

\chapter*{Resumo}

Escrever aqui o resumo (pt)

\paragraph{Palavras-chave} palavras, chave, aqui, separadas, por, vírgulas

\cleardoublepage
