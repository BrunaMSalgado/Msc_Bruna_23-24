\chapter{Background}

\section{Linear Lambda Calculus}


The Lambda-Calculus, developed by Church and Curry in the 1930s, serves as a formal language capturing the key attribute of higher-order functional languages, treating functions as first-class citizens, allowing them to be passed as arguments [\cite{barendregt1984lambda}].  Beyond its foundational aspects, this calculus incorporates extensions for modeling side effects, including probabilistic or non-deterministic behaviors and shared memory. Centered around the expression of higher-order functions, where functions can serve as inputs or outputs, it emerges as a potent computational tool.  Higher-order functions form a pivotal abstraction in practical programming languages such as LISP, Scheme, ML, and Haskell.

% Exemplo de função que só se possa usar uma vez, porque não estou a perceber? 

In quantum information theory, the role of higher-order functions encompasses two fundamental aspects. The first involves the concept of entangled functions and how well-known quantum phenomena find natural descriptions through such functions. The second concerns the interplay between classical objects and quantum objects in a higher-order context. Quantum computation conventionally handles classical and quantum data, while the higher-order context introduces a third data type: functions. These functions fall into two categories - those "quantum-like" (entangled, single-use) and those "classical-like" (duplicable, reusable). Remarkably, this classification transcends input/output types, highlighting the coexistence of quantum-like functions operating on classical data and classical-like functions operating on quantum data. [\cite{selinger2009quantum}].

\subsection{Syntax}
The grammar and term formation rules of the linear lambda calculus, discussed in [\cite{dahlqvist2022syntactic}], are presented in this subsection.

The definition of the grammar for linear lambda calculus is as follows, where $G$ represents a set of ground types.
\vspace{-10pt}
\begin{equation} \label{eq:grammar}
\centering
\hspace{95pt} \mathbb{A} ::= X \in G \hspace{3 pt} \vert \hspace{3 pt} \mathbb{I}  \hspace{3 pt}  \vert \hspace{3 pt} \mathbb{A}  \otimes  \mathbb{A} \hspace{3 pt} \vert \hspace{3 pt} \mathbb{A} \oplus \mathbb{A} \hspace{3 pt}  \vert \hspace{3 pt}   \mathbb{A} \multimap  \mathbb{A}
\end{equation}
\vspace{-25pt}

 Regarding the term formation rules, $\Sigma$ corresponds to a class of sorted operation symbols $f: \mathbb{A}_{1}, \ldots, \mathbb{A}_{n} \xrightarrow{} \mathbb{A}$, where $n \geq 1$.   Typing contexts are represented as lists $x_{1}: \mathbb{A}_{1}, \ldots, x_{n}: \mathbb{A}_{n}$ of typed variables, with each variable $x_i$ (where $1 \leq i \leq n$) occurring at most once in $x_1, \ldots, x_n$. The typing contexts are denoted by greek letters $\Gamma$, $\Delta$, and $E$.
The concept of shuffling is employed to construct a linear typing system that ensures the admissibility of the exchange rule and enables unambiguous reference to judgment's denotations $[\![ \Gamma \triangleright v: \mathbb{A} ]\!]$. Shuffling is defined as a permutation of typed variables in a sequence of contexts, $\Gamma_1, \ldots, \Gamma_n$, preserving the relative order of variables within each $\Gamma_i$. For instance, if $\Gamma_1=x:\mathbb{A}, y:\mathbb{B}$ and $\Gamma_2=z:\mathbb{C}$, then $z:\mathbb{C}, x:\mathbb{A}, y:\mathbb{B}$ is a valid shuffle of $\Gamma_1, \Gamma_2$. On the other hand, $y:\mathbb{B}, x:\mathbb{A}, z:\mathbb{C}$ is not a shuffle because it alters the occurrence order of $x$ and $y$ in $\Gamma_1$. The set of shuffles in $\Gamma_1, \ldots, \Gamma_n$ is denoted as $\text{Sf} (\Gamma_1, \ldots, \Gamma_n)$.
The term formation rules of the linear lambda calculus are shown in
\autoref{fig:typing_rules_linear}.
\vspace{-10pt}
\begin{figure} [H]
\begin{equation*}
\begin{split}
\begin{aligned}
&
\begin{minipage}[t]{0.3\textwidth}
$\begin{array}{c}
     \Gamma_{i} \triangleright v_{i}: \mathbb{A}_{i} \quad f: \mathbb{A}_{1}, \ldots, \mathbb{A}_{n} \xrightarrow{} \mathbb{A} \in \Sigma \quad E \in \text{Sf}(\Gamma_{1}; \ldots; \Gamma_{n})\\
    \hline
   E \triangleright f( v_{1},\ldots,v_{n}): \mathbb{A}
\end{array}
$
\end{minipage}
\hspace{168pt}
\text{(ax)} 
 \hspace{10pt}
\begin{minipage}[t]{0.3\textwidth}
$\begin{array}{c}
      \\
    \hline
   x:\mathbb{A} \triangleright x:\mathbb{A}
\end{array}
$ \end{minipage}
\hspace{-60pt} \text{(hyp)} \\
&
\begin{minipage}[t]{0.3\textwidth}
$\begin{array}{c}
     \\
    \hline
   - \triangleright *: \mathbb{I}
\end{array}
$
\end{minipage}
\hspace{-80pt}
\text{($\mathbb{I}_{i}$)} 
 \hspace{20pt}
\begin{minipage}[t]{0.3\textwidth}
$\begin{array}{c}
     \Gamma \triangleright v: \mathbb{A} \otimes \mathbb{B} \quad  \Delta,x: \mathbb{A}, y: \mathbb{B}  \triangleright w: \mathbb{C}  \quad E \in \text{Sf}(\Gamma;\Delta)\\
    \hline
   E\triangleright \text{pm } v \text{ to } x \otimes y. w :\mathbb{C}
\end{array}
$ \end{minipage}
\hspace{144pt} (\otimes_{e}) \\
&
\begin{minipage}[t]{0.3\textwidth}
$\begin{array}{c}
     \Gamma \triangleright v: \mathbb{A} \quad  \Delta \triangleright w: \mathbb{B}  \quad E \in \text{Sf}(\Gamma;\Delta) \\
    \hline
   E \triangleright v \otimes w: \mathbb{A} \otimes \mathbb{B} 
\end{array}
$
\end{minipage}
\hspace{60pt} (\otimes_{i}) 
 \hspace{20pt}
 \begin{minipage}[t]{0.3\textwidth}
$\begin{array}{c}
     \Gamma \triangleright v: \mathbb{I} \quad  \Delta \triangleright w: \mathbb{A}  \quad E \in \text{Sf}(\Gamma;\Delta)  \\
    \hline
   E \triangleright v \text { to } *.w: \mathbb{A}  
\end{array}
$ \end{minipage}
\hspace{55pt} (\mathbb{I}_{e}) \\
&
\begin{minipage}[t]{0.3\textwidth}
$\begin{array}{c}
     \Gamma,x:\mathbb{A} \triangleright v: \mathbb{B} \\
    \hline
   \Gamma \triangleright \lambda x:\mathbb{A} . v: \mathbb{A} \multimap \mathbb{B} 
\end{array}
$
\end{minipage}
\hspace{-20pt} (\multimap_{i}) 
 \hspace{5pt}
 \begin{minipage}[t]{0.3\textwidth}
$\begin{array}{c}
     \Gamma \triangleright v: \mathbb{A} \multimap \mathbb{B} \quad  \Delta \triangleright w: \mathbb{A}  \quad E \in \text{Sf}(\Gamma;\Delta)  \\
    \hline
   E \triangleright v w: \mathbb{B}  
\end{array}
$ \end{minipage}
\hspace{83pt} (\multimap_{e}) 
\hspace{5pt}
\begin{minipage}[t]{0.3\textwidth}
$\begin{array}{c}
     \Gamma \triangleright v: \mathbb{A}  \\
    \hline
   \Gamma \triangleright \text{dis}(v):  \mathbb{I} 
\end{array}
$
\end{minipage}
\hspace{-63pt} (\text{dis})
\end{aligned}
\end{split}
\end{equation*}
\caption{Term formation rules of affine lambda calculus.}
\label{fig:typing_rules_linear}
\end{figure}
\vspace{-8pt}
The no-cloning theorem states that it is impossible to duplicate a quantum bit [\cite{wootters1982single}]. This principle is upheld by the type system outlined in \autoref{fig:typing_rules_linear}, which does not allow the repeated use of a variable (seen as a quantum resource).
Nevertheless, the linearity
constraint is often deemed too restrictive, prompting research into relaxing it in various computational paradigms. In [\cite{dahlqvist2023complete}], the controlled use of a resource multiple times is explored within approximate program equivalence paradigms. Moreover, the grammar introduced allows the specification of how many times a resource can be used—a notion particularly relevant in quantum computation, especially within the NISQ era where resources are scarce.
\vspace{-5pt}
\subsection{Metric equational system}
Metric equations [\cite{mardare2016quantitative}, \cite{mardare2017axiomatizability}] are a strong candidate for reasoning about approximate program equivalence. These equations take the form of $t=_{\epsilon} s$, where  $\epsilon$ is a non-negative rational representing the ``maximum distance" between the two terms $t$ and $s$. The metric equational system for linear lambda calculus is depicted in \autoref{fig:metric deductive system} [\cite{dahlqvist2022syntactic}].
\vspace{-10pt}
\begin{figure} [H]
\begin{equation*}
\begin{split}
\begin{aligned}
&
\begin{minipage}[t]{0.3\textwidth}
$\begin{array}{c}
     \\
    \hline
   v=_{0}v
\end{array}
$
\end{minipage}
\hspace{-90pt}
\text{(refl)} 
 \hspace{55pt}
\begin{minipage}[t]{0.3\textwidth}
$\begin{array}{c}
    v=_{q}w \quad w=_{r}u  \\
    \hline
   v=_{q + r} u
\end{array}
$ \end{minipage}
\hspace{-40pt} \text{(trans)} 
\hspace{55pt}
\begin{minipage}[t]{0.3\textwidth}
$\begin{array}{c}
    v=_{q}w \quad r\geq q  \\
    \hline
   v=_{r} w
\end{array}
$ \end{minipage}
\hspace{-45pt} \text{(weak)} \\
&
\begin{minipage}[t]{0.3\textwidth}
$\begin{array}{c}
    \forall r > q . \hspace{4pt} v=_{r} w \\
    \hline
   v=_{q}w
\end{array}
$
\end{minipage}
\hspace{-45pt}
\text{(arch)} 
 \hspace{40pt}
\begin{minipage}[t]{0.3\textwidth}
$\begin{array}{c}
    \forall i \leq n. \hspace{4pt} v=_{q_i} w\\
    \hline
   v=_{\vee q_i} w
\end{array}
$ \end{minipage}
\hspace{-40pt} \text{(join)} 
\hspace{40pt}
\begin{minipage}[t]{0.3\textwidth}
$\begin{array}{c}
    v=_{q} w \quad v'=_{r} w' \\
    \hline
   v \otimes v' =_{q + r} w \otimes w'
\end{array}
$ \end{minipage}
\hspace{-45pt}  \\
&
\begin{minipage}[t]{0.3\textwidth}
$\begin{array}{c}
   \forall i \leq n. \hspace{4pt} v_{i}=_{q_i} w_{i}\\
    \hline
   f(v_{1},...,v_{n})=_{\Sigma q_i} f(w_{1},...,,w_{n}) 
\end{array}
$
\end{minipage}
 \hspace{34pt}
\begin{minipage}[t]{0.3\textwidth}
$\begin{array}{c}
   v=_{q}w  \quad v'=_{r}w'\\
    \hline
    v \text { to } *.v' =_{q+r} w \text { to } *.w'
\end{array}
$ \end{minipage}
\hspace{0pt}
\begin{minipage}[t]{0.3\textwidth}
$\begin{array}{c}
    v=_{q} w  \\
    \hline
  \lambda x : \mathbb{A}. \hspace{4pt} v=_{q} \lambda x:\mathbb{A}. \hspace{4pt} w
\end{array}
$ \end{minipage}
\hspace{20pt}  \\
&
\begin{minipage}[t]{0.3\textwidth}
$\begin{array}{c}
    v=_{q} w \quad  v'=_{r} w'  \\
    \hline
   \text{pm} \hspace{4pt} v \hspace{4pt} \text{to} \hspace{4pt} x \otimes y. \hspace{4pt} v'=_{q + r}\text{pm} \hspace{4pt} w \hspace{4pt} \text{to} \hspace{4pt} x \otimes y .  \hspace{4pt} w'
\end{array}
$
\end{minipage}
\hspace{200pt}
\begin{minipage}[t]{0.3\textwidth}
$\begin{array}{c}
    v =_{q} w \quad v'=_{r} w'   \\
    \hline
  v v' =_{q + r} w w'
\end{array}
$ \end{minipage}
 \\
 &
\begin{minipage}[t]{0.3\textwidth}
$\begin{array}{c}
  \Gamma \triangleright v =_{q} w: \mathbb{A} \quad \Delta \in \text{perm}(\Gamma)\\
    \hline
   \Delta \triangleright v =_{q} w: \mathbb{A}
\end{array}
$
\end{minipage}
\hspace{180pt}
\begin{minipage}[t]{0.3\textwidth}
$\begin{array}{c}
    v =_{q} w \quad v'=_{r} w'    \\
    \hline
  v[v'/x]=_{q + r} w[w'/x]
\end{array}
$ \end{minipage}
\hspace{10pt}
\end{aligned}
\end{split}
\end{equation*}
\caption{Metric equational system}
\label{fig:metric deductive system}
\end{figure}
\vspace{-5pt}



In the quantum paradigm, a potential notion of approximate equivalence arises from the so-called diamond norm [\cite{watrous2018theory}], which induces a metric (roughly, a distance function) on the space of quantum programs (seen semantically as completely positive trace-preserving super-operators). This norm relies on another norm known as the trace norm. The $\lVert  \rVert_{1}$ latter is defined by  $\lVert A \rVert_{1} = \text{Tr}\sqrt{A^{\dag}A}$  for matrices $A \in \mathbb{C}^{n\times n}$. The trace norm induces a metric on the set of density matrices which is defined by $d(\rho, \sigma) = \lVert \rho -\sigma\rVert_{1}$. On the other hand, it is well known that the distance $d(vv^{\dag}, uu^{\dag})$ between two quantum states $v$ and $u$ is their Euclidean distance in the Bloch
sphere [\cite{wallman2016noise,nielsen2010quantum}]. The Euclidean norm of a vector $u \in \mathbb{C}^{n} $ is defined as:
\begin{equation} \label{eq:euclidean_distance}
\begin{centering}
\hspace{80pt}
\lVert u \rVert_{2} = \sqrt{\langle u, u\rangle} 
 \end{centering}
\end{equation}
The trace distance between two super-operators $E, E': \mathbb{C}^{n\times n} \xrightarrow{} \mathbb{C}^{m\times m }$,  denoted as $T(E,E')$, is defined as follows:
\begin{equation} \label{eq:trace_distance}
\begin{centering}
\hspace{80pt}
 T(E,E')= \max\{\lVert (E-E') A \rVert_{1} \hspace{2pt}  \vert \hspace{2pt}  \lVert A \rVert_{1}=1\} 
 \end{centering}
\end{equation}
Unfortunately, this norm is not stable under tensoring [\cite{watrous2018theory}], and consequently, the diamond norm, which is based on the trace norm, is used instead. The diamond norm between two super-operators $E, E': \mathbb{C}^{n\times n} \xrightarrow{} \mathbb{C}^{m\times m }$ is defined as:
\begin{equation}  \label{eq:diamond_distance}
\begin{centering}
\hspace{110pt}
 \lVert E-E'\rVert_{\diamondsuit} =  T(E \otimes I_{n},E' \otimes I_{n}) 
 \end{centering}
\end{equation}
where $I_{n} $ is the identity super-operator over the space $\mathbb{C}^{n\times n}$.


Consider an operator  $ \text{r} : (\mathbb{C}^{n} \xrightarrow[]{} \mathbb{C}^{m}) \xrightarrow[]{} (\mathbb{C}^{n\times n}\xrightarrow[]{} \mathbb{C}^{m\times m})$ that sends an operator $T$ to the mapping $A \mapsto TAT^{\dag}$. The exact calculation of distances induced by $\lVert \rVert_{\diamondsuit}$ tends to be quite complicated, but a useful property for calculating the distance between quantum channels in the image of $r$ is provided [\cite{watrous2018theory}]:
Consider two operators $T, S : n \xrightarrow{} m$. There exists a unit vector $v \in \mathbb{C}^{n}$ such that, 
\begin{equation} \label{eq:norm_iso_r}
\begin{centering}
\lVert r(T) (vv^{\dag})-r(S) (vv^{\dag}) \rVert_{1} = \lVert r(T)-r(S) \rVert_{\diamondsuit}
 \end{centering}
\end{equation}


The notion of a diamond norm is used in [\cite{dahlqvist2022syntactic}] which introduces a simple metric theory based on the idea of approximating a quantum operation. The authors argue that their deductive system allows to compute an approximate distance between two quantum programs easily as opposed to computing an exact distance ``semantically" which tends to involve quite complex operators. 
Other works in this spirit include [\cite{hung2019quantitative}] and [\cite{tao2021gleipnir}]. They reason about the issue of noise in a quantum while-language by developing a deductive system to determine how similar a quantum program is from its idealised, noise-free version. The former introduces the ($Q$,$\lambda$)-diamond norm which analyzes the output error given that the input quantum state satisfies some quantum predicate $Q$ to degree $\lambda$. However, it does not specify any practical method for obtaining non-trivial quantum predicates. In fact, the methods used in [\cite{hung2019quantitative}] cannot produce any post conditions other than $(I,0)$ (\textit{i.e.}, the identity matrix $I$ to degree 0, analogous to a ``true” predicate) for large quantum programs. The latter specifically addresses and delves into this aspect. 

\vspace{-7pt}

\subsection{Interpretation}
In order to define the interpretation of judgments $\Gamma \triangleright v: \mathbb{A}$, it is necessary to establish some notation first. Considering $v \in V$, $w \in W$, and $u \in U$  where $V, W, U$ represent vector spaces,  $\textsc{sw}_{V,W} : V\otimes W \xrightarrow{} W \otimes V$, denotes the swap operator, defined as $\textsc{sw}_{V,W}= v\otimes w \mapsto w \otimes v$;    $\rho_{V} : \mathbb{C} \otimes V \xrightarrow{} V $ is the left unitor defined as $\rho_{V}= 1 \otimes v \mapsto v $; $\lambda_{V} : V  \otimes \mathbb{C} \xrightarrow{} V $ is the right unitor defined as $\lambda_{V}= v \otimes 1 \mapsto v$; $\alpha_{V,W,U} : V  \otimes (W \otimes U) \xrightarrow{} (V  \otimes W) \otimes U$ is the left associator, defined as $\alpha_{V,W,U}= v \otimes (w \otimes u) \mapsto (v \otimes w) \otimes u $; and $!_{V}: V \xrightarrow{} \mathbb{C}$ is the trace operation applied to a vector, defined as  $!_{V}= v \xrightarrow{} \text{Tr} \hspace{1pt}v$. Moreover, for all operators $f: V \otimes W \xrightarrow{} U$, the operator $\overline{f} : V \xrightarrow{} (W \multimap U)$ denotes the corresponding curried version, defined as $\overline{f}(v) = w \mapsto  f(v,w)$. The subscripts in these operators will be omitted unless ambiguity arises.

For all ground types $X \in G$  the interpretation of $[\![X]\!]$  is postulated as a vector space $V$. Types are interpreted inductively using the unit $\mathbb{I}$, the tensor $\otimes$, and the linear map $\multimap$. Given a non-empty context $\Gamma=\Gamma',x: \mathbb{A}$, its interpretation is defined by $[\![\Gamma',x: \mathbb{A}]\!] = [\![\Gamma']\!] \otimes [\![\mathbb{A}]\!]$ if $\Gamma'$ is non-empty and $[\![\Gamma',x: \mathbb{A}]\!] = [\![\mathbb{A}]\!]$ otherwise. The empty context $-$ is interpreted as $[\![-]\!] = \mathbb{I}$. Given $X_{1}, . . . ,X_{n} \in V$, the $n$-tensor $(\ldots (X_1 \otimes X_2) \otimes \ldots ) \otimes X_{n}$ is denoted as $X_1 \otimes \ldots \otimes X_{n}$, and similarly for operators. 


``Housekeeping" operators are employed to handle interactions between context interpretation and the vectorial model. Given $\Gamma_{1}, \ldots, \Gamma_{n}$, the operator that splits $[\![\Gamma_{1}, \ldots, \Gamma_{n}]\!]$ into $[\![\Gamma_{1}]\!] \otimes \ldots \otimes [\![\Gamma_{n}]\!]  $ is denoted by $\text{sp}_{\Gamma_1;\ldots;\Gamma_n}: [\![\Gamma_{1}, \ldots, \Gamma_{n}]\!] \xrightarrow{} [\![\Gamma_{1}]\!] \otimes \ldots \otimes [\![\Gamma_{n}]\!] $.
On the other hand, $\text{jn}_{\Gamma_1;\ldots;\Gamma_n}$ denotes the inverse of $\text{sp}_{\Gamma_1;\ldots;\Gamma_n}$. Next, given $\Gamma, x : \mathbb{A}, y : \mathbb{B},\Delta$, the operator permuting $x$ and $y$ is denoted by $\text{exch}_{\Gamma, x : \mathbb{A}, y : \mathbb{B},\Delta}: [\![\Gamma, x : \mathbb{A}, y : \mathbb{B},\Delta]\!] \xrightarrow{} [\![\Gamma, y : \mathbb{B}, x : \mathbb{A}, \Delta]\!] $. The shuffling operator $\text{sh}_{E}: [\![E]\!] \xrightarrow{} [\![\Gamma_1, \ldots, \Gamma_n ]\!]$ is defined as a suitable composition
of exchange operators.

For every operation symbol $f: \mathbb{A}_{1}, \ldots, \mathbb{A}_{n} \xrightarrow{} \mathbb{A}$ we assume the existence of an operator $[\![f]\!]: [\![\mathbb{A}_{1}]\!] \otimes \ldots \otimes [\![\mathbb{A}_{n}]\!] \xrightarrow{}  [\![\mathbb{A}]\!] $.
The interpretation of judgments is defined by induction over derivations according to the rules in \autoref{fig:denotational_sem} [\cite{dahlqvist2022syntactic}].
\vspace{-7pt}
\begin{figure} [H]
\begin{equation*}
\begin{split}
\begin{aligned}
&
\begin{minipage}[t]{0.3\textwidth}
$\begin{array}{c}
      [\![\Gamma_{i} \triangleright v_{i}: \mathbb{A}_{i} ]\!]=m_{i}  \quad f: \mathbb{A}_{1}, \ldots, \mathbb{A}_{n} \in \Sigma \quad E \in \text{Sf}(\Gamma_{1}; \ldots; \Gamma_{n})\\
    \hline
  [\![E \triangleright f( v_{1},\ldots,v_{n}): \mathbb{A} ]\!] = [\![ f]\!] \cdot (m_{1}\otimes \ldots \otimes m_{n}) \cdot \text{sp}_{\Gamma_1;\ldots;\Gamma_n}\cdot \text{sh}_{E}
\end{array}
$
\end{minipage}
\hspace{204pt}
\begin{minipage}[t]{0.3\textwidth}
$\begin{array}{c}
      \\
    \hline
  [\![ x:\mathbb{A} \triangleright x:\mathbb{A}]\!] = \text{id}_{[\![\mathbb{A} ]\!]}
\end{array}
$ \end{minipage} \\
&
\begin{minipage}[t]{0.3\textwidth}
$\begin{array}{c}
     \\  
    \hline
   [\![ - \triangleright *: \mathbb{I}]\!] = \text{id}_{[\![\mathbb{I} ]\!]}
\end{array}
$
\end{minipage}
\hspace{-31pt}
\begin{minipage}[t]{0.3\textwidth}
$\begin{array}{c}
      [\![\Gamma \triangleright v: \mathbb{A} \otimes \mathbb{B} ]\!]=m  \quad [\![\Delta,x: \mathbb{A}, y: \mathbb{B}  \triangleright w: \mathbb{C} ]\!] =n  \quad E \in \text{Sf}(\Gamma;\Delta)\\
    \hline
  [\![ E\triangleright \text{pm } v \text{ to } x \otimes y. w :\mathbb{C}]\!] = n \cdot \text{jn}_{\Delta;\mathbb{A};\mathbb{B}} \cdot \alpha \cdot \text{sw}\cdot (m \otimes \text{id}) \cdot \text{sp}_{\Gamma;\Delta} \cdot \text{sh}_{E}
\end{array}
$ \end{minipage} \\
&
\begin{minipage}[t]{0.3\textwidth}
$\begin{array}{c}  
     [\![ \Gamma \triangleright v: \mathbb{A} ]\!] =m \quad [\![\Delta \triangleright w: \mathbb{B} ]\!]=n \quad E \in \text{Sf}(\Gamma;\Delta) \\
    \hline
  [\![ E \triangleright v \otimes w: \mathbb{A} \otimes \mathbb{B} ]\!] = (m \otimes n) \cdot \text{sp}_{\Gamma;\Delta} \cdot \text{sh}_{E}
\end{array} 
$
\end{minipage}\\
&
 \begin{minipage}[t]{0.3\textwidth}
$\begin{array}{c} 
    [\![\Gamma \triangleright v: \mathbb{I} ]\!]=m  \quad [\![\Delta \triangleright w: \mathbb{A}]\!]=n \quad E \in \text{Sf}(\Gamma;\Delta)  \\
    \hline
  [\![E \triangleright v \text { to } *.w: \mathbb{A} ]\!]=n \cdot \lambda \cdot (m \otimes \text{id}) \cdot \text{sp}_{\Gamma;\Delta} \cdot \text{sh}_{E}
\end{array}
$ \end{minipage} 
\hspace{130 pt}
\begin{minipage}[t]{0.3\textwidth}
$\begin{array}{c} 
     [\![\Gamma,x:\mathbb{A} \triangleright v: \mathbb{B} ]\!] = m \\
    \hline
   [\![ \Gamma \triangleright \lambda x:\mathbb{A} . \hspace{2pt } v: \mathbb{A} \multimap \mathbb{B}]\!] = \overline{m \cdot \text{jn}_{\Gamma;\mathbb{A}}}
\end{array}
$
\end{minipage} \\
&
 \begin{minipage}[t]{0.3\textwidth}
$\begin{array}{c} 
     [\![\Gamma \triangleright v: \mathbb{A} \multimap \mathbb{B} ]\!] = m \quad [\![  \Delta \triangleright w: \mathbb{A} ]\!] =n \quad E \in S\hspace{-3pt}f(\Gamma;\Delta)  \\
    \hline
  [\![ E \triangleright v w: \mathbb{A} ]\!] = \text{app} \cdot (m \otimes n) \cdot \text{sp}_{\Gamma;\Delta} \cdot \text{sh}_{E}
\end{array}
$ \end{minipage}
\hspace{190 pt} %[\![ ]\!]
\begin{minipage}[t]{0.3\textwidth}
$\begin{array}{c}
     [\![\Gamma \triangleright v: \mathbb{A}]\!]  = f \\
    \hline
   [\![ \Gamma \triangleright \text{dis}(v):  \mathbb{I} ]\!] = !_{[\![ \mathbb{A} ]\!]} \cdot f
\end{array}
$
\end{minipage}
\end{aligned}
\end{split}
\end{equation*}
\caption{Judgment interpretation}
\label{fig:denotational_sem}
\end{figure}


% Perguntar ao professor Renato: "Equations corresponding to the axiomatics of autonomous categories" e onde por o background de quantum computing. Also é para fazer glossário com simbolos matemáticos?

Linear $\lambda$-calculus comes equipped with a class of equations, given in \autoref{fig:equations-linear-lambda}, specifically equations-in-context $\Gamma \triangleright v = w : \mathbb {A}$.

\begin{figure}[H]
  \centering
  \begin{tabular}{ |c|c| }
      \hline 
      Monoidal structure & Higher-order structure \\
      \hline
      pm $v \otimes w$ to $x \otimes y.$ $u = u[v/x,w/y]$& \\
      pm $v$ to $x \otimes y.$ $u[x \otimes y/z] = u[v/z]$ & $(\lambda x : A.$ $v) w = v[w/x]$\\
      $* \text { to } *.$ $v = v$ & $\lambda x : A.(v x) = v$ \\
      $v$ to $*$ . $w[* / z] = w[v / z]$ & \\
      \hline
      \multicolumn{2}{|c|}{Commuting conversions} \\
      \hline
      \multicolumn{2}{|c|}{$u[v \text{ to } \ast . w/z] = v \text{ to } \ast . u[w/z]$}  \\
      \multicolumn{2}{|c|}{$u[$pm $v$ to $x \otimes y.$ $w/z] =$ pm $v$ to $x \otimes y.$ $u[w/z]$} \\
      \hline
      \multicolumn{2}{|c|}{Discard} \\
      \hline
      \multicolumn{2}{|c|}{$v = \text{dis}(x_1) \text{ to } \ast.$ $\ldots$ $\text{dis}(x_{n-1}) \text{ to } \ast \text{ dis}(x_{n})$} \\
      \hline
  \end{tabular}
  \caption{Equations-in-context for linear lambda calculus}
  \label{fig:equations-linear-lambda}
\end{figure}
\section{Quantum Computing Preliminaries} \label{sec:Quantum Computing Preliminaries}
This section presents background on quantum information and quantum computation [\cite{nielsen2010quantum}].

The basic unit of information in quantum computation is a quantum bit or qubit [\cite{perdrix2008quantum}]. The state of a single qubit is described by a normalized vector of the 2-dimensional Hilbert space $\mathbb{C}^{2}$.  An $n$-qubit state can be represented by a unit vector in $2^n$-dimensional Hilbert space $\mathbb{C}^{2^{n}}$. An $n$-qubit mixed state can be represented by a density operator $ \mathbb{C}^{2^{n}} \xrightarrow{} \mathbb{C}^{2^{n}}$, whose matrix representation is $\rho = \Sigma_{i} \hspace{2pt} p_{i} \vert \phi_{i} \rangle \langle \phi_{i} \vert$. A density operator encodes uncertainty about the current state of the quantum system at hand. For example, a mixed state with half probability of $\vert 0 \rangle$ and $\vert 1 \rangle$ can be represented by $\frac{\vert 0 \rangle \langle 0 \vert + \vert 1 \rangle \langle 1 \vert}{2}=I/2$, where $I$ is the identity matrix.  One usually denotes density matrices by the greek letters $\rho$, $\sigma$, and so forth. The set of density operators is denoted by $\mathcal{D}_{n} \subseteq \mathbb{C}^{ 2^{n \times n}}$.

Measurements extract classical information from quantum states.  If a measurement ${M_m}$ is performed on a state $\rho$, the outcome $m$ is observed with probability $p_m = \text{Tr}(M_{m} \rho M^{\dag}_{m})$ for each $m$. Moreover, after a measurement yielding outcome $m$, the state collapses to $M_{m}\rho M^{\dag}_{m}/p_{m}$. 


Operations on quantum systems can be described using unitary operators. An operator, $U$, is unitary if its Hermitian conjugate is its own inverse, \textit{i.e.}, $U^{\dag} U= UU^{\dag} = I$. For a pure state $|\psi \rangle$, a unitary operator $U$ describes an evolution from $|\psi \rangle$ to $ U|\psi \rangle$. Similarly, for a density operator $\rho $, the corresponding evolution is $\rho \mapsto U\rho U^{\dag}$. For example, the bit flip gate $X=\big(\begin{smallmatrix}
  0 & 1\\
  1 & 0
\end{smallmatrix}\big)$ maps  $\vert 0 \rangle$  to  $\vert 1 \rangle$ and  $\vert 1 \rangle$  to  $\vert 0 \rangle$. On the other hand, the Hadamard gate $H= \frac{1}{\sqrt{2}}\big(\begin{smallmatrix}
  1 & 1\\
  1 & -1
\end{smallmatrix}\big)$ maps $\vert 0 \rangle$ to  $ \frac{\vert 0 \rangle + \vert 1 \rangle }{\sqrt{2}} $ (denoted as $\vert + \rangle $) and $\vert 1 \rangle$ to $ \frac{\vert 0 \rangle - \vert 1 \rangle }{\sqrt{2}} $ (denoted as $\vert - \rangle $). There are also multi-qubit gates, such as $C\hspace{-2pt}N\hspace{-1pt}O\hspace{-1pt}T$, which leaves the states $\vert 0 0 \rangle$ and  $\vert 0 1 \rangle$ unchanged, and maps $\vert 1 0 \rangle$ and $\vert 1 1 \rangle$ to each other.

More broadly, the evolution of a quantum system can be defined by a super-operator $E$, which is a completely-positive and trace-preserving linear map from $\mathcal{D} (n)$ to $\mathcal{D} (m)$. A super-operator $E$ is called positive if it sends positive matrices to positive matrices, \textit{i.e.} $A \geq 0 \Rightarrow{} E A \geq 0$. A super-operator is said to be completely positive if, for any positive integer $k$ and any $k$-dimensional Hilbert space $\mathbb{C}^{2^{k}}$, the super-operator $E \otimes I_{\mathbb{C}^{2^{k}}}$ is a positive map on $\mathcal{D}(n \times k)$. Finally, a super-operator
$E$ is called trace-preserving if $\text{Tr} \hspace{ 2pt} E A = \text{Tr} A$ [\cite{watrous2018theory}]. Completely-positive, trace-preserving super-operators are traditionally called quantum channels.

For every super-operator $ E: \mathcal{D}(n) \xrightarrow{} \mathcal{D}(m)$, there exists a set of Kraus operators $\{\epsilon_{k}\}_{k}$ such that $ E(\rho)= \Sigma_{k} \hspace{2pt} \epsilon_{k}\hspace{2pt} \rho \hspace{2pt} \epsilon_{k}^{\dag}$  for any input $\rho \in  \mathcal{D}(n) $. Note that the set of Kraus operators is finite if the Hilbert space is finite-dimensional. The Kraus form of $E$ is written as $E = \Sigma_{k} \hspace{2pt} \epsilon_{k} \circ  \epsilon_{k}^{\dag}$.

A matrix $A \in  \mathbb{C}^{n\times n}$ is Hermitian if $A = A^{\dag}$.
A matrix $A \in \mathbb{C}^{n\times n}$ is said to be normal if $AA^{\dag} =  A^{\dag}A$. Clearly every Hermitian matrix is normal. Note also that for every matrix $A \in C^{n\times n}$ the matrix $A^{\dag}A$ is Hermitian. Next, it is well-known that by appealing to the spectral theorem [NC16], every normal matrix $A \in  \mathbb{C}^{n\times n}$ can be expressed as a linear combination $\sum_{i} \lambda_{i}b_{i}b_{i}^{\dag}$ where the set $\{b_{i}, \ldots , b{n}\}$ is an orthonormal basis of $\mathbb{C}^{n}$. Using this last result we can extend any function $f:\mathbb{C} \xrightarrow{} \mathbb{C}$, to normal matrices via,
\begin{equation} \label{eq:apply_f_diag}
  f(A) = \sum_{i} f(\lambda_{i})b_{i}b_{i}^{\dag}
\end{equation}

\todo[inline,size=\normalsize]{add trace, partial trace, reduced density matrix, and respective Bloch Vector, put the last paragragh in the right place and rewrite it} %\label{eq:Bloch\_vector})}

\section{Quantum Lambda Calculus}



\todo[inline,size=\normalsize]{Adicionar os operadores CPTP que vou usar} %\label{eq:Bloch\_vector})}

In the case of quantum lambda calculus, which combines classical and quantum features, it is natural to consider two distinct basic data types: a type $\textit{bit}$ of classical bits and a type $\textit{qbit}$ of quantum bits.  The interpretation of these types is defined as  $[\![\textit{bit}]\!]=\mathbb{C}\oplus\mathbb{C}$ and $[\![\textit{qbit}]\!]=\mathbb{C}^{2\cdot 2}$. The type $\mathbb{I}$ is interpreted as $[\![\mathbb{I}]\!]=\mathbb{C}$.

The following operations are considered: $\textit{new} \hspace{2pt} 0  :\mathbb{I}  \multimap \textit{bit} $, $\textit{new} \hspace{2pt} 1  :\mathbb{I}  \multimap \textit{bit} $, $q : \textit{bit}  \multimap \textit{qbit}$, $\textit{meas}:\textit{qbit} \xrightarrow{} \textit{bit}$, and $\textit{U}:\textit{qbit},\ldots,\textit{qbit} \xrightarrow{} \textit{qbit}^{\otimes n}$. Their correspondent judgment interpretation is shown in \autoref{fig:interpret_ops}. . 



\begin{figure}[H]
\begin{equation*}
\begin{split}
\begin{aligned}
&
\hspace{0pt}
\begin{minipage}[t]{0.45\textwidth}
$\begin{aligned}
  [\![\textit{new} \, 0 ]\!] : \hspace{2pt}& \mathbb{C} \multimap \llbracket \textit{bit} \rrbracket  \\
& 1 \mapsto (1,0)
\end{aligned}$
\end{minipage}
\hspace{-30pt}
\begin{minipage}[t]{0.45\textwidth}
$\begin{aligned}
  [\![\textit{new} \, 1 ]\!] :\hspace{2pt}& \mathbb{C} \multimap \llbracket \textit{bit} \rrbracket  \\
  & 1 \mapsto (0,1)
\end{aligned}$
\end{minipage} 
\hspace{-30pt}
\begin{minipage}[t]{0.45\textwidth}
$\begin{aligned}
  [\![q ]\!] : \hspace{2pt}&\llbracket \textit{bit} \rrbracket \multimap \llbracket \textit{qbit} \rrbracket\\
   &(a,b) \mapsto \big(\begin{smallmatrix}
  a & 0\\
  0 & b
\end{smallmatrix}\big) 
\end{aligned}$
\end{minipage} \\
&
\hspace{0pt}
\begin{minipage}[t]{0.45\textwidth}
$\begin{aligned}
  [\![\textit{meas}]\!]:\hspace{2pt} & \llbracket \textit{qbit} \rrbracket \xrightarrow{} \llbracket \textit{bit} \rrbracket  \\
  &\rho \mapsto ( \text{Tr} (M_{0} \rho M_{0}^{\dag}), \text{Tr} (M_{1} \rho M_{1}^{\dag})) 
\end{aligned}$
\end{minipage} 
\hspace{0pt}
\begin{minipage}[t]{0.45\textwidth}
$\begin{aligned}
  [\![\textit{U} ]\!] : \hspace{2pt} & \llbracket \textit{qbit} \rrbracket^{\otimes n} \xrightarrow{} \llbracket 
  \textit{qbit} \rrbracket^{\otimes n} \\
  & \rho \mapsto U \rho \hspace{2pt}  U^{\dag}
% & \rho,...,\rho_{n} \mapsto U \hspace{-2pt}\left(\bigotimes_{i=1}^{n}\rho_{i}\right ) U^{\dag}
\end{aligned}$
\end{minipage} 
\hspace{-70pt}
\begin{minipage}[t]{0.45\textwidth}
$\begin{aligned}
  [\![\textit{K}_{0}, \ldots,\textit{K}_{n}]\!] : \hspace{2pt} & \llbracket \textit{qbit} \rrbracket^{\otimes n} \xrightarrow{} \llbracket 
  \textit{qbit} \rrbracket^{\otimes n} \\
& \rho \mapsto \sum_{i} K_{i} \rho K_{i}^{\dag} 
\end{aligned}$
\end{minipage} \\
\end{aligned}
\end{split}
\end{equation*}
\caption{Judgment interpretation of the operations in quantum lambda calculus.}
\label{fig:interpret_ops}
\end{figure}

