\chapter{Future work}\label{ch:future_work}

% Quantais

% Boolean (partial order), ultrametric (ultrametric spaces), 

%whichis (tacitly) used to interpret Nakano’s guarded λ-calculus [BSS10] and also to interpret a higher-order language for functional reactive programming [KB11]. Another interesting quantale is the G¨odel one which is a basis for fuzzy logic [DEW13] and whose V-equations give rise to what we call fuzzy inequations.

% Real-time computation ->  Now, it may be the case that is unnecessary to know the distance between the execution time of two programs– instead it suffices to know whether a program finishes its execution before another one. 

%Equations t =0 s state that the terms t and s are exactly the same and equations t =ϵ s state that t and s differ by at most ϵ seconds in their execution time.

% Boolean quantale, V-equations are labelled by {0,1}. We will see that Γ ▷v =1 w : A can be effectively treated as an inequation Γ ▷v ≤ w : A, whilst Γ▷v =0 w : A corresponds to a trivial V-equation, i.e. a V-equation that always holds.

% fuzzy logic -> imprecise date

% guarded -> guarded recursion (so bad things dont happen)

%functional reative programming -> functional + reative (real time data and events)


% Estender parte quantica para higher order

%The quantum models discussed in SECALGO are not closed. In \cite{dahlqvist2023syntactic}, the authors used general results from category theory to address a similar issue in the category $\catCPTP$. A natural next step would be to extend such a construction for $\catQ$ and $\WstarCPSUop$.



% Graded cenas

%estender modelo qunatico seria util por causa de quantum discrimination e quatum metrology -> ver onde andam os artigos

%\cite{Multiple_copy_two_state_discrimination}

%\cite{Giovannetti_Quantum_Metrology, Zhou_Limits_Noisy_Quantum_Metrology}

%The goal is to estimate an unknown parameter encoded in a quantum channel, To improve the sensitivity and accuracy of measurements, quantum metrology takes advantage of the peculiar properties of quantum systems, such as entanglement and superposition. It is possible, for example, to estimate physical quantities more precisely using entangled states rather than classical states. Numerous applications of quantum metrology — which we will go into more detail shortly — can be found in fields such as navigation, communication, and medicine.



