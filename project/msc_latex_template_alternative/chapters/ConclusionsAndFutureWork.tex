\chapter{Future work}\label{ch:future_work}


In this work, we extend~\cite{dahlqvist2023syntactic} by introducing a metric equation for conditionals. We prove its soundness and completeness, and illustrate its syntactic and semantic applicability across several domains.  Nevertheless, much remains to be done.


\textbf{Generalizing to quantales}
This work can be generalized to other quantales, such as the Boolean, ultrametric, and Gödel quantales.
Indeed, we have already taken initial steps in this direction in \cite{bmql25}, 
where a broader range of quantales is considered; this work was accepted at an international workshop.
Readers unfamiliar with quantales may also consult \cite{STUBBE201495} for an accessible introduction to the concept.
This generalization also explains why in the equational system that we introduced, we chose to use the expression \( q + \sup\{r, s\} \) rather than \( \sup\{q + r, q + s\} \), given that these are equal.
The fact is that at the level of arbitrary quantales they need not to be the same. For instance, consider the quantale \(\mathcal{P}(\Sigma^*)\), where \(\Sigma\) is a finite non-empty set of symbols, \ie, the powerset of all finite lists over \(\Sigma\) \cite{STUBBE201495}. In this quantale, the associative operation \(\otimes\) and the infimum/meet are defined as follows:
\[ I \otimes J =  \{ i \mathbin{+\!+} j \,|\, i \in I, j \in J \} \quad \text{and} \quad \inf \{ I, J\} = I \cap  J  \]
where \(I, J \in \mathcal{P}(\Sigma^*)\) and \(\mathbin{+\!+}\) denotes list concatenation.
Consider the sets $X = \{\varepsilon, a\}$, $Y = \{a\}$, and $Z = \{aa\}$, where $\varepsilon$ denotes the empty string over $\Sigma$ (i.e., for any $s \in \Sigma^*$, we have $\varepsilon \mathbin{+\!+} s = s = s \mathbin{+\!+} \varepsilon$). Then:
\begin{align*}
  & X \otimes \inf\{Y, Z\} = X \otimes \emptyset = \emptyset, \\
  &\inf\{X \otimes Y,\, X \otimes Z\} =  \{a, aa\} \cap \{aa, aaa\} = \{aa\}.
\end{align*}
Consequently, $X \otimes \inf\{Y,Z\} \neq  \inf\{X \otimes Y, X \otimes Z \}.$
Moreover, it becomes clear after inspecting the soundness proof that it is the expression \( q + \sup\{r, s\} \) that arises naturally. 

%This work focuses specifically on the metric quantale. A natural direction for future research would be to generalize the metric equations and the associated results of soundness and completeness to other quantales, such as the Boolean, ultrametric, and Gödel quantales.

As an illustrative example of the aforementioned generalisation to quantales, consider, for example, the Boolean quantale, where equations are labelled by elements of $\{0,1\}$. The judgement $\Gamma \vljud v =_{1} w : \typeA$ can be interpreted as an inequation $\Gamma \vljud v \leq w : \typeA$, whereas $\Gamma \vljud v =_{0} w : \typeA$ corresponds to a trivial equation, that is, one that always holds.  In this context, it would be interesting to explore, for instance, inequational $\lambda$-theories in the setting of real-time computation, particularly in scenarios where the exact timing difference between two programs is irrelevant---what matters is simply whether one program finishes before the other~\cite{dahlqvist2023syntactic}. For the ultrametric quantale, one could investigate ultrametric $\lambda$-theories within computational paradigms such as the guarded $\lambda$-calculus~\cite{guarded_ultrametric} and functional reactive programming~\cite{Ultrametric_reactive}. Finally, the Gödel quantale, which underlies fuzzy logic~\cite{deneckeGaloisConnectionsApplications2004}, gives rise to what we refer to as \emph{fuzzy inequations}.



% fuzzy logic -> imprecise date

% guarded -> guarded recursion (so bad things dont happen)

%functional reative programming -> functional + reative (real time data and events)


% Estender parte quantica para higher order

\textbf{Closing quantum (first-order) categories}
Another possible direction stems from the fact that, as previously mentioned, the quantum categories discussed in \Cref{sec:quantum_cats} are not closed. In \cite{dahlqvist2023syntactic}, the authors used general results from category theory to address a similar issue in the category $\catCPTP$. A natural next step would be to extend such a construction for $\catQ$ and $\WstarCPSUop$.

%Speaking of $\WstarCPSUop$, it would be useful to establish a connection between the introduced norm and the completely bounded norm, along with some properties, so that we could, for instance, reason about quantum walks on a line \cite{venegasQuantumWalksComprehensive2012}.

\textbf{The completely bounded norm and the $W^*$ completely bounded norm}
In this work, our focus is limited to showing that the metric induced by the $W^*$ completely bounded norm makes $\WstarCPSUop$ into a first-order model. However, as previously noted, another norm---the \emph{completely bounded} norm---is widely used in the context of $C^*$-algebras and comes equipped with established results that could simplify distance computations between programs.  A natural next step would be to determine whether the completely bounded norm itself constitutes a suitable metric, \ie, if it satisfies $\cbnorm{\id \overline{\otimes} \Phi} \leq \cbnorm{\Phi}$, for any completely bounded normal map $\Phi$ between $W^*$-algebras (recall \autoref{def:cb_norm}). Of course, we aim as well to establish additional results regarding the $W^*$ completely bounded norm that simplify distance computations between morphisms. That would, for instance, allow us to reason about quantum walks on a line \cite{venegasQuantumWalksComprehensive2012}.

\textbf{A metric on Selinger's $\catQ$ by an embedding into $\mathcal{K}(\catCPS)$}
As previously mentioned in \cite{selinger2004towards}, Selinger introduced a first-order functional quantum language, QPL, whose denotational semantics is given by the distributive symmetric monoidal category $\catQ$.  Here, Selinger works with \emph{vectors} (\ie, direct sums) of square matrices, and extends the standard notions of \emph{positivity} and \emph{trace} to matrix tuples.

\begin{comment}
Given a tuple of matrices 
\[
A = (A_1, \ldots, A_s) \in \mathcal{M}^{n_1} \oplus \cdots \oplus \mathcal{M}^{n_s},
\]
we say that $A$ is \emph{positive} if each matrix $A_i$ is positive. The \emph{trace} of such a tuple is then defined as the sum of the traces of its components:
\[
\tr(A) = \sum_i \tr(A_i).
\]

\begin{definition} \label{def:catQ}
The category $\catQ$ is defined as follows:
\begin{itemize}
  \item An object is a signature $\sigma= n_1, \ldots, n_s$. We denote these signatures by the Greek letters $\sigma, \tau$ and $\mu$ and each signature $\sigma$ is associated with the complex vector space
  \[
  \mathcal{M}_\sigma \coloneqq \mathcal{M}_{n_1} \oplus \cdots \oplus \mathcal{M}_{n_s}.
  \]
  
  \item  For signatures $\sigma = (n_1, \ldots, n_s)$ and $\tau = (m_1, \ldots, m_t)$, a morphism $\Phi: \sigma \to \tau$ is a linear map $\Phi: \mathcal{M}_\sigma \to \mathcal{M}_\tau$ satisfying:
    \begin{enumerate}
        \item \emph{Complete positivity:} For every signature $\mu$, the extended map
        \[
        \mathrm{id}_{\mu} \otimes \Phi: \mathcal{M}_{\mu \otimes \sigma} \to \mathcal{M}_{\mu \otimes \tau}
        \]
        preserves positivity (maps positive tuples to positive tuples).
        
        \item It is \emph{trace non-increasing:}
        \(
        \tr(\Phi(A)) \leq \tr(A),
        \)
        for all positive  $A \in \mathcal{M}_\sigma$.
    \end{enumerate}
\end{itemize}
\end{definition}
\end{comment}

Another direction for future work would be to define a functor 
\( F \colon \catQ \to \mathcal{K}(\catCPS) \) 
and use it to induce a norm on the morphisms \(\Phi\) in \(\catQ\) via 
\( \|\Phi\| \coloneqq \diamondnorm{F(\Phi)}, \) 
taking advantage of the fact that \(\mathcal{K}(\catCPS)\) is a model of our calculus.

\begin{comment}
In this work Selinger works with vectors (direct sums) of square matrices, and so he extends the definitions of positivity and trace in this setting. As a result given a tuple of matrices $A=(A_1, \ldots, A_n)\in \mathcal{M}^{n_1} \oplus \ldots \oplus \ldots \mathcal{M}^{n_s}$, $A$ is said to be \emph{positive} if each $A_i$ is positive and we define the trace of a matrix tuple to be the sum of the traces of its components $\tr(A)= \sum_i \tr(A_i)$.


\begin{definition} \label{def:catQ}
  The category $\catQ$ is defined as 
  \begin{itemize}
    \item An object is a signature $\sigma= n_1, \ldots, n_s$. We denote these signatures by the Greek letters $\sigma, \tau$ and $\mu$.
    \item  Every signature $\sigma$ is associated with a complex vector space   
$\mathcal{M}_\sigma \coloneqq \SqMatrix{n_1} \oplus \cdots \oplus \SqMatrix{n_s}$ and so given signatures  $\sigma = n_1, \ldots, n_s $ and $\tau= m_1, \ldots, m_t $, a morphism from $\sigma$ to $\tau$ is quantum operation $\Phi: \mathcal{M}_{\sigma} \to \mathcal{M}_{\tau} $ as defined by Selinger, \ie,  if the map \( \mathrm{id}_{\sigma'} \otimes F \colon \mathcal{M}_{\sigma' \otimes \sigma} \to \mathcal{M}_{\sigma' \otimes \sigma} \) is positive (maps positive tuples of matrices to positive tupled of matrices) for all signatures \( \sigma' \) and and satisfies the following trace condition:
\[
\tr(F(A)) \leq \tr(A), \quad \text{for all positive } A \in V_\sigma.
\]
    
    %More concretely, $ij$-component of $\Phi$ is given by the function $\Phi_{ij} = \pi_{j} \comp \Phi \comp \mathrm{in}_{i} : \mathcal{M}_{n_i} \rightarrow \mathcal{M}_{n_j} $, where $\mathrm{in}_{i}$ is the injection of  $\mathcal{M}_{n_i}$ into the input space of $\Phi$ and  $\pi_{j}$ is the projection onto the $j$-th component.
  \end{itemize}
\end{definition}
\end{comment}






%Trace-non increasing  maps are used here instead of strictly trace-preserving maps to account for possible non-termination in programs, particularly since the language  QPL includes while loops.

\begin{comment}

  \begin{remark}
  $\catQ$ corresponds to the finite biproduct completion of $\catCP$ (which extends $\catCPTP$ to include all completely positive maps),  further restricted to trace-nonincreasing morphisms. 
\end{remark}

Every signature $\sigma$ is associated with a complex vector space   
$\mathcal{M}_\sigma \coloneqq \SqMatrix{n_1} \oplus \cdots \oplus \SqMatrix{n_s}$

This space consists of matrix vectors 
$$A = \begin{pmatrix} A_1 \\ \vdots \\ A_n \end{pmatrix}$$
where the signature $\sigma$ specifies both the number of matrices, $s$,  and their respective dimensions, $n_i \times n_i$. The elements of $\mathcal{M}_\sigma$ are represented uppercase letters such as $A$, $B$, etc.
We also establish that $\mathbb{C}^\sigma \coloneqq \mathbb{C}^{n_1} \oplus \cdots \oplus \mathbb{C}^{n_s}$.

Note that the definition of trace induces a generalized notion of trace applicable to elements of $\mathcal{M}_\sigma$. Specifically, for $A = \begin{pmatrix} A_1 & \ldots & A_n \end{pmatrix}^T$, we have $\tr(A) = \sum_{i=1}^n \tr(A_i).$
 

\begin{definition} \label{def:biproduct} [\emph{Coproduct}]
Concatenation $ \sigma \oplus \sigma'$ of signatures $\sigma$ and $\sigma'$ yields coproducts in $\catQ$. The co-pairing map $[\Phi, \Psi]: \sigma \oplus \sigma' \to \tau$ is defined as  $[\Phi, \Psi](A, B) = \Phi(A) + \Psi(B)$, where addition uses the fact that the codomain is always a direct sum of vector spaces.
\end{definition}

\begin{remark}
  The category $\catQ$ does not have finite products. To see this, observe that the diagonal morphism 
\[
\langle \mathrm{id}, \mathrm{id} \rangle \colon \tau \to \tau \oplus \tau
\] 
is not trace-nonincreasing and hence is not a valid morphism. 
However, $\catQ$ does contain the two projection morphisms 
\[
\pi_1 \colon \sigma \oplus \sigma \to \sigma' \quad \text{and} \quad \pi_2 \colon \sigma \oplus \sigma \to \sigma'.
\] 

\end{remark}

\begin{definition} \label{def:tensor} [\emph{Tensor Product }]
  For signatures $\sigma = n_1, \ldots, n_s $ and $\tau= m_1, \ldots, m_t $, the tensor product of $\sigma$ and $\tau$ is defined as $\sigma \otimes \tau = n_1 m_1, \ldots ,n_1 m_t, \ldots, n_s m_1,...,n_s m_t$. 
  The morphism part of the tensor product follows the definition in the category of vector spaces. If $\Phi: \sigma \rightarrow \tau$ and $\Psi: \sigma' \rightarrow  \tau'$, then their tensor product $\Psi \otimes \Phi: \sigma \otimes \sigma' \rightarrow  \tau \otimes \tau' $ is 
  %defined on a basis element $A \otimes B$ by  
%$$
%(\Phi \otimes \Psi)(A \otimes B) = \Phi(A) \otimes \Psi(B),
%$$
%and extends to arbitrary elements by linearity. More concretly, the tensor product $\Phi \otimes \Psi$ is 
the Kronecker product of their matrices representation, \ie,
\begin{align*}
  \Phi \otimes \Psi = 
  \begin{pmatrix}
  \Phi_{11} \otimes \Psi_{11} & \ldots & \Phi_{11} \otimes \Psi_{s'1} & \ldots   & \Phi_{s1} \otimes \Psi_{11} & \ldots & \Phi_{s1} \otimes \Psi_{s'1} \\
  \vdots & & & & & & \vdots \\
  \Phi_{1t} \otimes \Psi_{1t'} & \ldots & \Phi_{1t} \otimes \Psi_{s't'} & \ldots   & \Phi_{st} \otimes \Psi_{1t'} & \ldots & \Phi_{st} \otimes \Psi_{s't'}
\end{pmatrix}
\end{align*}



 Moreover, $\dist$ is an identity map:
  \[ (\sigma \oplus \sigma') \otimes \tau = (\sigma \otimes \tau ) \oplus (\sigma' \otimes \tau ) \]

  
\end{definition}


The category $\catQ$ is a distributive symmetric monoidal category with binary coproducts. However, this category is not closed \cite{selinger2004b}.

\end{comment}

\textbf{A metric on $\catQ$ by $W^*$-algebras}
In \cite{choSemanticsQuantumProgramming2016}, the author established an equivalence of categories  
$\catQ \simeq (\FdWstarCPSU)^{\mathrm{op}}$.  
Using this equivalence, along the same lines as above, we aim to induce an alternative norm on $\catQ$ by assigning to each superoperator $\Phi \in \catQ$ the norm of its corresponding map $\Phi^* \in (\FdWstarCPSU)^{\mathrm{op}}$. %This approach mirrors the construction we presented in \Cref{subsec:shodinger}.

\begin{comment}
Now we describe how to obtain  $\Phi^*$ from $\Phi$. 
Recall that for a Hilbert space \( \mathcal{H} \), \( \mathcal{B}(\mathcal{H}) \), \ie, the set of bounded operators on \( \mathcal{H} \), is a \( W^* \)-algebra with the predual \( \mathcal{T}(\mathcal{H}) \).
A quantum channel (resp. quantum operatoration) $\Phi_i: \mathcal{T}(\mathcal{H})  \to \mathcal{T}(\mathcal{K})$ defines a linear mapping 
$\Phi_i^*: \mathcal{B}(\mathcal{K}) \to \mathcal{B}(\mathcal{H})$ completely positive and unital (resp. subunital) \cite[Proposition 5.1,]{choSemanticsQuantumProgramming2016}. These maps are related in the following way:
\begin{equation*}
\tr\left[\Phi_i(T) \cdot S\right] = \tr \left[T \cdot \Phi_i^*(S)\right]
\end{equation*}
which holds for all trace-class operators $T \in \mathcal{T}(\mathcal{H})$ and all bounded operators $S \in \mathcal{B}(\mathcal{K})$.
Intuitively, this means that observing the transformed state $\Phi_i(T)$ w.r.t. experiment $S$ is the same as observing the original state $S$ w.r.t. the transformed state $\Phi_i^*(S)$.
The maps 
\( \Phi :  \bigoplus_i M_{n_i} \to \bigoplus_j M_{m_j} \in \catQ \quad \text{and} \quad \Phi^* : \bigoplus_j M_{m_j} \to \bigoplus_i M_{n_i} \in \WstarCPSU \)
are also related via (extended) trace:
\[
\tr \left( \Phi(A) \cdot B \right) = \operatorname{tr} \left( A \cdot \Phi^*(B) \right)
\]
for \(A \in \bigoplus_i M_{n_i}\) and \(B \in \bigoplus_j M_{m_j}\). Here, the trace is defined as the sum of traces of coordinates and the multiplication is coordinatewise:
\[
(A_1, \ldots, A_n) \cdot (A'_1, \ldots,A'_n ) = (A_1 \cdot A'_1, \ldots, A_n \cdot A'_n).
\]

%The expectation value of the observable \( S \) on the output state \( \Phi(T) \) equals the expectation of the evolved observable \( \Phi^*(S) \) on the original state \( T \).

A more explicit way of describing the relationship between $\Phi$ and $\Phi^*$ in the finite-dimensional setting is given by the diagram below:

\[
\hspace{-110pt}
\begin{minipage}{0.45\textwidth}
\centering
\begin{tikzpicture}
  \matrix (m) [matrix of math nodes,row sep=3em,column sep=2em,minimum width=1em]
  { 
    \bigoplus_i M_{n_i} \\
     \bigoplus_j M_{n_j}   \\
  };
  \path[-stealth]
    (m-1-1) edge  node [right] {$\Phi$} (m-2-1);
\end{tikzpicture}
\end{minipage}
\hspace{-60pt}
\begin{minipage}{0.45\textwidth}
\centering
\begin{tikzpicture}
  \matrix (m) [matrix of math nodes,row sep=3em,column sep=23em,minimum width=1em]
  {
  (\bigoplus_i M_{n_i})^*&  \bigoplus_i M_{n_i} \\
   (\bigoplus_j M_{n_j})^* &  \bigoplus_j M_{n_j} \\
  };
  \path[-stealth]
    (m-1-1) edge  node [above] {$\varphi \mapsto \left(\sum_{ij} \varphi_1(\ket{j}\bra{i}) \cdot \ket{i} \bra{j}, \ldots, \sum_{ij} \varphi_n(\ket{j}\bra{i}) \cdot \ket{i} \bra{j}\right)   $} (m-1-2)
    (m-2-2) edge [dotted]  node [right] {$\Phi^*$} (m-1-2)
    (m-2-2) edge  node [below] {$ A \mapsto (B \mapsto \tr(AB)) $} (m-2-1)
    (m-2-1) edge  node [right] {$ \varphi \mapsto \varphi \cdot \Phi $} (m-1-1)
    ;
\end{tikzpicture}
\end{minipage}
\]
where $\varphi \in (\bigoplus_i M_{n_i})^*$ is defined as $\varphi(v_1, \ldots, v_n)= \sum_i^n \varphi_i(v_i).$


\vspace{3pt}
\end{comment}

% Graded cenas

\textbf{Quantum graded $\lambda$-calculus}
  \cite{dahlqvistCompleteVEquationalSystem2023} extends  \cite{dahlqvist2023syntactic} by introducing a sound and complete quantalic equational system ---which includes the metric quantale--- for a$\lambda$-calculus with graded modal types, allowing multiple uses of the same resource. Since the \textit{op. cit.} \cite{dahlqvistCompleteVEquationalSystem2023} does not consider quantum computation, a natural next step would be to explore categorical models suited for this setting. Such an extension would enable us to reason about approximate equivalence in various scenarios, such as discriminating between two known states given $n$ copies of an unknown state \cite{Multiple_copy_two_state_discrimination}, or estimating an unknown parameter across $n$ copies of a quantum state in quantum metrology \cite{Giovannetti_Quantum_Metrology, Zhou_Limits_Noisy_Quantum_Metrology}.

%The subsequent work \cite{dahlqvistCompleteVEquationalSystem2023} extends these ideas by introducing a sound and complete metric equational system for a $\lambda$-calculus with graded modal types, interpreted using what the authors call a \emph{Lipschitz exponential comonad}.

%estender modelo qunatico seria util por causa de quantum discrimination e quatum metrology -> ver onde andam os artigos

%\cite{Multiple_copy_two_state_discrimination}

%\cite{Giovannetti_Quantum_Metrology, Zhou_Limits_Noisy_Quantum_Metrology}





