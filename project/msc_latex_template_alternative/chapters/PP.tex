\chapter{Probabilistic Programming}




The  next section  introduces the fundamentals of measure theory, a key component in defining the semantics of probabilistic programs, drawing inspiration primarily from \cite{dahlqvistSemanticsProbabilisticProgramming2020a} and \cite{athreyaMeasureTheoryProbability2006}. For a more in-depth exploration of measure theory, see reference \cite{aliprantisBanachLattices1999}. 

\todo[inline,size=\normalsize]{Meter esta intro direita} 
%ryanIntroductionTensorProducts2013, rudin91functional,guide2006infinite-> refs banach spaces

\section{Banach spaces}




\subsection{Normed spaces}

The definition of normed spaces given in \autoref{def:normed_space} extends to infinite-dimensional vector spaces.

\begin{definition} \label{def:op_norm}
Let \( V \) and \( W \) be normed vector spaces, and let \( T: V \rightarrow W \) be a linear operator. The \emph{operator norm}, denoted by \gls{op-norm}, is defined as
\[
\norm{T}_{\text{op}} = \sup \{ \norm{T(v)} : v \in V, \, \norm{v} = 1 \}.
\]
If \( \norm{T}_{\text{op}} < \infty \), we say that \( T \) is a \emph{bounded operator}; otherwise, if \( \norm{T}_{\text{op}} = \infty \), we say that \( T \) is \emph{unbounded}. We denote by \( \mathcal{B}(V, W) \) the vector space of all bounded linear operators from \( V \) to \( W \), and we write \( \mathcal{B}(V) \)  for \( \mathcal{B}(V, V) \).
\end{definition}

\begin{lemma} \label{lemma:op_norm_submult} %inf_dim_anal
Let \( T: V \rightarrow W \) be a bounded linear operator between normed spaces. Then the following statements hold:
\begin{enumerate}
  \item For every \( v \in V \), we have $
  \norm{T(v)} \leq \norm{T}_{\text{op}} \cdot \norm{x}.$
  \item The operator \( T \) is continuous if and only if it is bounded.
\end{enumerate}
\end{lemma}


\begin{definition}
  Let \( V \) and \( W \) be normed vector spaces, and let \( T: V \rightarrow W \) be a linear operator. $T$ is called a \emph{short map} if $\opnorm{T} \leq 1$.
\end{definition}



\subsection{Banach spaces}

\begin{definition} (\emph{Cauchy sequence})
  Suppose $ d $ is a metric on a set $ X $. A sequence $ \{x_n\} \subset X $ is called a \emph{Cauchy sequence} if, for every $ \varepsilon > 0 $, there exists an integer $N \in \mathbb{N} $ such that $ d(x_m, x_n) < \varepsilon $ for all $ m, n > N $.  The metric \( d \) is said to be \emph{complete} if every Cauchy sequence in \( X \) converges to a point in \( X \)
\end{definition}

\begin{definition}
  A \emph{Banach space} is a normed vector space that is complete with respect to the metric induced by its norm. In other words, every Cauchy sequence in the space must converge to a limit within the space.
\end{definition}
 

\begin{definition} [Algebraic Tensor Product]
  Let \( V_1, \dots, V_n \) and \( W \) be vector spaces, and let
\[
\varphi : V_1 \times \cdots \times V_n \to W
\]
be a multilinear function, meaning a function for which the mapping
\[
u_k \mapsto \varphi(u_1, \dots, u_n)
\]
is linear for each \( k \in \{1, \dots, n\} \) and every fixed choice of vectors \( u_1, \dots, u_{k-1}, u_{k+1}, \dots, u_n \).

Then there exists a unique linear mapping
\[
A : V_1 \otimes \cdots \otimes V_n \to W
\]
such that
\[
\varphi(u_1, \dots, u_n) = A(u_1 \otimes \cdots \otimes u_n)
\]
for all choices of \( u_1 \in V_1, \dots, u_n \in V_n \).
\end{definition}

\begin{definition}
  Let $V$ and $W$ be Banach spaces. Let $u$ be any element of $V \otimes W$. The \emph{projective norm}, denoted \gls{proj_norm},  is defined by: 
  \[
 \projnorm{u} = \inf\left\{ \sum_{i=1}^n \norm{v_i} \norm{w_i} \Bigg| \, u = \sum_{i=1}^n v_i \otimes w_i\right\}.
\]
\end{definition}

\begin{proposition} %ryan-artigoprofRenato \label{prop:proj_norm_tensor_mult}
   Let $V$ and $W$ be Banach spaces. Then, for every $v \in V$ and $w \in W$, it holds that $ \projnorm{v \otimes w} = \norm{v} \otimes \norm{w}$.
\end{proposition}

\begin{definition}
  Let \( V \) and \( W \) be Banach spaces. The \emph{projective tensor product} of \( V \) and \( W \), denoted  \gls{proj_tensor}, is the completion of the algebraic tensor product \( V \otimes W \) with respect to the projective norm \( \projnorm{\cdot} \).
\end{definition}


\section{Category Ban}

\begin{example}
  The category $\catBan$ is the category of Banach spaces and short maps.
\end{example}

\begin{example}
  In the category \(\catBan\), the coproduct of Banach spaces is given by their direct sum equipped with the \(L_1\)-norm. The injections and copairing map are defined analogously to those in \(\catVect\). 
\end{example}

\section{Measure theory}
Probabilistic computation involves running programs incorporating randomness, leading to output behaviors characterized by probability distributions rather than deterministic outcomes. To effectively understand and analyze these programs, it is essential to have a solid foundation in reasoning about probability distributions. That is where measure theory comes into play.


\todo[inline,size=\normalsize]{Acrescentar medidas negativas tb} 

%In this work, we only consider finite (positive) measures; hence, the term "measure" implicitly refers to a finite measure unless stated otherwise.

\subsection{What is measure theory?}

Throughout history, mathematicians sought to extend the ideas of length, area, and volume. The most effective way to generalize these concepts is through the idea of a measure. Abstractly, a measure is a function defined on sets with additive properties mirroring length, area, and volume.

We begin with a simple example inspired by \cite{athreyaMeasureTheoryProbability2006} to develop an intuition for the concepts of measure and measure space.
Imagine an open field $S$ covered in snow after a storm. Suppose we wish to measure the amount of snow accumulated in as many field regions as possible.
Assume we have accurate tools for measuring snow over standard geometric shapes like triangles, rectangles, and circles. 
We can approximate irregularly shaped regions using combinations of these standard shapes and then apply a limiting process to assign a consistent measure to such regions. 
Let $\mathcal{B}$ denote the collection of subsets of $S$ that are deemed \emph{measurable}, let $\lambda(A)$ represent the amount of snow in each $A \in \mathcal{B}$, and let $A^c$ denote the complement of a set $A$.


For this framework to make sense, it is reasonable to require that $ \mathcal{B}$ and $\lambda (\cdot)$ satisfy the following properties:

\subsubsection*{Properties of $B$:}
\begin{enumerate}
    \item If $A \in B$, then the complement $A^c \in \mathcal{B}$. (\ie, if we can measure the snow on a set $A$, and we know the total amount on $S$, then we can determine the snow on the remaining part $A^c$.)
    \item If $A_1, A_2 \in \mathcal{B}$, then \(A_1 \cup A_2 \in \mathcal{B}\).  
    (\ie, if we can measure the snow on two regions \(A_1\) and \(A_2\), we should also be able to measure it on their union.)
    \item If $\{A_n\}_{n \geq 1} \subset \mathcal{B}$ is an increasing sequence, \ie, $A_n \subset A_{n+1}$ for all $n$, then $\bigcup_{n=1}^{\infty} A_n \in \mathcal{B}$.  
    \\(\ie, if each set in a increasing sequence of regions is measurable, then their limit—the union—should also be measurable.)
    \item The collection $\mathcal{B}$ contains a base class $C$ of simple, well-behaved sets (e.g., triangles, rectangles, circles) for which measurement is initially defined.
\end{enumerate}

\subsubsection*{Properties of $\lambda (\cdot)$:}
\begin{enumerate}
    \item  $\lambda(A) \geq 0$ for all $A \in \mathcal{B}$ (\ie, the amount of snow on any set must be nonnegative). 
    \item If $A_1, A_2 \in \mathcal{B}$ and $A_1 \cap A_2 = \emptyset$, then  
    $\lambda(A_1 \cup A_2) = \lambda(A_1) + \lambda(A_2)$
 (\ie, The total amount of snow over two non-overlapping regions is just the sum of the snow in each region. This characteristic of 
$\lambda$ is known as \emph{finite additivity}.)
    \item If $\{A_n\}_{n \geq 1} \subset \mathcal{B}$ is an increasing sequence, \ie, $A_n \subset A_{n+1}$ then  
    $
    \lambda( \lim_{n \to \infty} A_n) = \lambda\left( \bigcup_{n=1}^\infty A_n \right) = \lim_{n \to \infty} \lambda(A_n).
    $
    (\ie, if a set can be approximated by an increasing sequence of measurable sets $\{A_n\}_{n \geq 1}$, then $\lambda(A) = \lim_{n \to \infty} \lambda(A_n)$. This is known as \emph{monotone continuity from below}.)
\end{enumerate}

\subsection{Measurable spaces and measures}
Remarkably, these  intuitive conditions give rise to a profoundly versatile and far-reaching theoretical framework. The requirements imposed on $\mathcal{B}$ and $\lambda$ can equivalently be stated as follows:

\subsubsection*{Properties of $\mathcal{B}$:}
\begin{enumerate}
    \item $\emptyset \in \mathcal{B}$.
    \item $A \in \mathcal{B}  \implies A^c \in \mathcal{B}$. 
    \item  $A_1, A_2, \dots \in \mathcal{B} \implies \bigcup_i A_i \in \mathcal{B}$ (this is known as \emph{closure
 under countable unions}).
\end{enumerate}

\subsubsection*{Properties of $\lambda$:}
\begin{enumerate}
    \item $\lambda(\cdot) \geq 0$  and $\lambda(\emptyset) = 0$.
    \item  If \(\{A_n\}_{n \geq 1} \subset \mathcal{B}\) is a sequence of pairwise disjoint sets (i.e., \(A_i \cap A_j = \emptyset\) for \(i \neq j\)), then
    $\lambda\left( \bigcup_{n=1}^\infty A_n \right) = \sum_{n=1}^\infty \lambda(A_n)$ (this is known as \emph{countable additivity}). 
\end{enumerate}

A collection $\mathcal{B}$ of subsets of $S$ satisfying the above conditions for $B$ is designated a $\sigma$\emph{-algebra}. Similarly, a set function $\lambda$ defined on a $\sigma$-algebra $\mathcal{B}$ that fulfills the above properties for $\lambda$ qualifies as a \emph{measure}.

\begin{definition} 
  A $\sigma$\emph{-algebra} $\mathcal{B}$ on a set $S$ is a collection of subsets of $S$ that includes the empty set, is closed under complementation with respect to $S$, and is closed under countable unions. 
\end{definition}

\begin{definition} 
  The pair $(S, \mathcal{B})$ where $\mathcal{B}$ is a $\sigma$-algebra constitutes a \emph{measurable space}. The elements of $\mathcal{B}$ are called  the \emph{measurable sets} of the space. 
\end{definition}

\begin{definition}  
   The Borel $\sigma$-algebra of $\mathbb{R}^n$ is the $\sigma$-algebra $\mathcal{B}(\mathbb{R}^n)$ generated by all open sets. The sets in $\mathcal{B}(\mathbb{R}^n)$ are called Borel sets. %For any set $E \subset \mathbb{R}^n$, let B(E) denote the class of all sets of the form $E \cap \mathbb{R}$,where $B \in \mathbb{B}(\mathbb{R}^n)$
\end{definition}


\begin{definition} 
  A  \emph{measure} on the measurable space $(S,\mathcal{B})$ is a function $\mu: \mathcal{B} \to  [0,\infty] $ that is countably additive and satisfies $\mu(\emptyset)=0$. A measure  on $(S,\mathcal{B})$ is called a \emph{probability measure} if it assigns total measure 1 to the entire space, that is, $\mu(S) = 1$.
\end{definition}

 In the context of probability theory, such measures are often referred to as \emph{distributions}. In what follows, we will use the terms measure and distribution interchangeably.

 
% lebesgue measure

 One of the most important measures is the Lebesgue measure on the real line (\ie, the length in $\mathbb{R}$), and its generalizations to $\mathbb{R}^n$.  It is characterized as the unique measure on the Borel sets, whose value on every interval is its length. That is, for any interval $(a, b) \subset \mathbb{R}$,
$
\lambda((a, b)) = b - a.$

The Lebesgue measure is \emph{translation-invariant}, meaning that shifting a set does not change its measure, and it is also \emph{$\sigma$-finite}.

For any $ s \in S $, the \emph{Dirac measure} (also known as the \emph{Dirac delta} or \emph{point mass} at \( s \)) is the probability measure defined by
$$
\delta_s(B) =
\begin{cases}
1, & \text{if } s \in B, \\
0, & \text{if } s \notin B.
\end{cases}
$$
A measure is called \emph{discrete} if represented as a countable weighted sum of Dirac measures. In particular, a \emph{convex combination} of Dirac measures yields a discrete probability measure. These are of the form $\sum_{s \in C} a_s \delta_s$,
where $ a_s \geq 0 $ for all $ s \in C $, and the weights satisfy $ \sum_{s \in C} a_s = 1 $.

On the other hand, a measure $ \mu $ on a measurable space $ (S, \mathcal{B}) $ is called \emph{continuous} if it assigns zero measure to all singleton sets, i.e., $ \mu(\{s\}) = 0 $ for every $ s \in S $. An example of a continuous measure is the \emph{Lebesgue measure} on $ \mathbb{R}^n $ (for $ n \in \mathbb{N} $).



\begin{definition} \label{def:pushforward_meas_1}
  Let $(S, \mathcal{B}_S)$ and $(T, \mathcal{B}_T)$ be measurable spaces. Given a function $f: (S, \mathcal{B}_S) \to (T, \mathcal{B}_T)$ and a measure $\mu$ on $\mathcal{B}_S$, the \emph{pushforward measure} $f_*(\mu)$ on $\mathcal{B}_T$ is defined by:
$$
f_*(\mu)(B) = \mu(f^{-1}(B)), \quad  B \in \mathcal{B}_T.
$$
\end{definition}


% lebesgue integral

\begin{definition}
  Let $(S, \mathcal{B}_S)$ and $(T, \mathcal{B}_T)$ be measurable spaces. A function $f: S \to T$ is said to be \emph{measurable} if for every  measurable subset  $B \in \mathcal{B}_T$, the preimage $f^{-1}(B) \in \mathcal{B}_S$.
\end{definition}


\begin{definition} 
  The \emph{Lebesgue integral} generalizes the familiar Riemann integral. Consider a measurable space $(S, \mathcal{B})$ and a bounded measurable function $f \colon S \to \mathbb{R}_0^+$, with upper and lower bounds $M$ and $m$, respectively. The \emph{Lebesgue integral} of $f$ with respect to a measure $\mu \colon \mathcal{B} \to \mathbb{R}_0^+$, denoted $\int f \, d\mu$, is defined as the limit of finite weighted sums of the form:
$$
\sum_{i=0}^n f(s_i) \mu(B_i),
$$
where $\{B_0, \dots, B_n\}$ forms a measurable partition of $S$, and within each $B_i$, the variation of $f$ does not exceed $(M - m)/n$. Here, $s_i \in B_i$ for each $i$, and the limit is taken over increasingly refined partitions. 
\end{definition}


In the case of a finite discrete space $n = \{1, 2, \dots, n\}$, the Lebesgue integral simplifies to a weighted sum:
$$
\int f \, d\mu = \sum_{i=1}^n f(i) \mu(i).
$$

Given two measurable spaces $(S_1, \mathcal{B}_1)$ and $(S_2, \mathcal{B}_2)$, their product is the measurable space $(S_1 \times S_2, \mathcal{B}_1 \otimes \mathcal{B}_2)$, where $S_1  \times S_2$ is the cartesian product and $\mathcal{B}_1 \otimes \mathcal{B}_2$ is the $\sigma$-algebra generated by all measurable rectangles $B_1 \times B_2$ with $B_1 \in \mathcal{B}_1$ and $B_2 \in \mathcal{B}_2$:
$$
\mathcal{B}_1 \otimes_{\text{meas}} \mathcal{B}_2 \coloneqq \sigma\left(\{ B_1 \times B_2 \mid B_1 \in \mathcal{B}_1,\, B_2 \in \mathcal{B}_2 \}\right).
$$
Measures on this product space are called \emph{joint distributions}, and are uniquely determined by their values on measurable rectangles due to the inductive structure of the product $\sigma$-algebra. \emph{Product measures} are a significant class of joint distributions and are defined from measures

\begin{definition} 
   Let $(S_1, \mathcal{B}_1)$ and $(S_2, \mathcal{B}_2)$ be measurable spaces, and let $\mu_1$ and $\mu_2$ be measures on these spaces, respectively. The \emph{product measure} $\mu_1 \gls{product_meas} \, \mu_2$ is  defined on measurable rectangles by
$$
(\mu_1 \otimes_{\text{meas}} \mu_2)(B_1 \times B_2) = \mu_1(B_1)\mu_2(B_2).
$$
This definition extends uniquely to a joint distribution $\mu_1 \otimes \mu_2: \mathcal{B}_1 \otimes \mathcal{B}_2 \to \mathbb{R}_0^+$, and reflects the notion of probabilistic independence: sampling from $\mu_1 \otimes \mu_2$ is equivalent to independently sampling from $\mu_1$ and $\mu_2$.
\end{definition}


%MarkovKernels
\begin{definition}
Let $(S, \mathcal{B}_S)$ and $(T, \mathcal{B}_T)$ be measurable spaces. A \emph{Markov kernel} is a mapping $P : S \times \mathcal{B}_T \to [0,1]$
satisfying the following two properties:
\begin{enumerate}
    \item For each fixed $s \in S$, the function $P(s, \cdot) : \mathcal{B}_T \to [0,1]$ is a probability measure on $(T, \mathcal{B}_T)$.
    \item For each fixed $A \in \mathcal{B}_T$, the function $P(\cdot, A) : S \to [0,1]$ is a measurable function on $(S, \mathcal{B}_S)$.
\end{enumerate}
\end{definition}


By a slight abuse of terminology, we will also use the term \emph{Markov kernel} to refer to a mapping  $P': S \to (\mathcal{B}_T \to [0,1]) \, P'(s)=\mu$, when $P : S \times \mathcal{B}_T \to [0,1], , P(s,A)=\mu(A)$ is a Markov kernel in the definition above.

By a slight abuse of terminology, we will also refer to a mapping $ P' : S \to (\mathcal{B}_T \to [0,1])\, P'(s)=\mu $, as a Markov kernel, when there exists a Markov kernel $ P : S \times \mathcal{B}_T \to [0,1] \, P(s,A)=\mu (A) $  for all $s \in S $ and $A \in \mathcal{B}_T$.

\begin{definition}
  Given a measure $\mu$ on $S$ and a Markov kernel $P: S \times \mathcal{B}_T \to [0,1]$, we can define the \emph{pushforward} of $\mu$   under the Markov kernel $P$  as:
$$
P_*(\mu)(B) = \int_S P(s, B) \, \mu(ds), \quad \forall B \in \mathcal{B}_T.
$$
\end{definition}

Any measurable function \( f : (S, \mathcal{B}_S) \to (T, \mathcal{B}_T) \) induces a simple Markov kernel $ s \mapsto \delta_{f(s)} $. As a result,  the usual definition of the pushforward measure (Definition \ref{def:pushforward_meas_1}) becomes a special case of the general formulation defined above.

\subsection{Spaces of Measures}

The set of all finite measures on a measurable space $(S, \mathcal{B})$ will be denoted by $\mathcal{M}(S, \mathcal{B})$, 
or simply $\mathcal{M}S$ when the $\sigma$-algebra $\mathcal{B}$ is clear from context. In particular, $\gls{mr}$ denotes  the Banach space of finite Borel measures on $\mathbb{R}$.

 $\mathcal{M}S $ forms a real vector space, where addition and scalar multiplication are defined pointwise. Specifically, for  $ B \in \mathcal{B} $, $ \mu, \nu \in \mathcal{M}S $, and $ \alpha \in \mathbb{R} $, the operations are given by
$$
(\mu + \nu)(B) = \mu(B) + \nu(B), \quad (a\mu)(B) = \alpha \mu(B).
$$


\( \mathcal{M}S \) is also a normed space equipped with the total variation norm. 

\begin{definition}
  For a measure \( \mu \), the total variation norm is defined as
$$
\|\mu\|_{TV} := \sup \left\{ \sum_{i=1}^n |\mu(B_i)| : \{B_1, \dots, B_n\} \text{ is a finite measurable partition of } S \right\}.
$$
\end{definition}

For positive measures, this reduces to \( \mu(S) \), and for probability measures, the norm is 1. The total variation norm turns \( \mathcal{M}_S \) into a Banach space, meaning it is complete under this norm. 

The following alternative definition is useful to compute the total variation norm between measures.

  \begin{definition} \cite{tsybakov2008}
    Let  let $P$ and $Q$ be two probability measures on $\mathcal{M}\mathbb{R}$. Define
    $$
      \nu = P + Q, \quad p = \frac{dP}{d\nu}, \quad q = \frac{dQ}{d\nu},
    $$
    where $\frac{dP}{d\nu}$ denotes the  the Radon–Nikodym derivative of the measure $P$ with respect to the measure $\nu$. It holds that:
    \begin{align*}
      \norm{P-Q} = \sup_{A \in  \mathcal{B}} \left\{ \left\vert \int_{A} (p - q) \, d\nu \right\vert \right\}.
    \end{align*}

  \end{definition}

  
\section{Examples}