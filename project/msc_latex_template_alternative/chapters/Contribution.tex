\chapter{Conditionals}

\todo[inline,size=\normalsize]{Adicionar as coisas das notas dq elas estiverem corrigidas + doc booleanos + intro às coisas} 

\todo[inline,size=\normalsize]{Na parte quantica mencionar o artigo do selinger de 2009, questão da linearidade} 

\todo[inline,size=\normalsize]{Por isto no sitio certo -> modelo quantico aqui mudamos a op de meas e de conversão em bit em qb} 

Within the model of quantum lambda calculus, introduced in \autoref{sec:Quantum Lambda Calculus:Interpretation}, the measurement operation is defined in accordance with the standard approach in physics \cite{watrous2018theory}. However, within this definiton the distinction between classical and quantum states is made through the elements of the spaces and not the spaces themselves, in the sense that both the result of measuring a quantum state, which is a classical bit, and the quantum state itself are elements of the same space, $\mathbb{C}^{2 \times 2}$.





%In order to established that the theory introduced is valid in quantum programming, it is necessary to build a model. The model can be seen as a category where the morphisms are the CPTP super-operators (quantum channels). The algebric structure of this model is given by the vector spaces. 
%Any completely-positive and trace-preserving map has a diamond norm equal to one \cite{watrous2018theory}. Since the measurement operation is completely positive and trace-preserving, its  diamond norm is equal to one. This is a desirable property, as it ensures that the measurement operation does not increase the distance between states, and as a consequence, composition of programs remains valid.





\section{Syntax}

\section{Interpretation}

\section{Metric Equations}

\section{Examples}

\begin{comment}
\begin{figure} [H]
  \centering
  \begin{quantikz} [column sep=0.2cm, row sep=0.5cm] 
      \lstick{$\ket{\psi}$}  & \qw &\qw & \qw & \qw & \qw& \qw & \qw &\qw  & \ctrl{1}\gategroup[2,steps=4,style={dashed,rounded
      corners,fill=blue!20, inner
      xsep=2pt},background,label style={label
      position=below,anchor=north,yshift=-0.2cm}]{{\sc
      BellMeasure}} & \gate{H} & \qw & \meter{} & \setwiretype{c}  &  & \gategroup[3,steps=4,style={dashed,rounded
      corners,fill=blue!20, inner
      xsep=2pt},background,label style={label
      position=below,anchor=north,yshift=-0.2cm}]{{\sc
      Correction}}  &  & & \ctrl[vertical
wire=c]{2}  \\
      \lstick {$\ket{0}$}  &\gate{H}\gategroup[2,steps=3,style={dashed,rounded
      corners,fill=blue!20, inner
      xsep=2pt},background,label style={label
      position=below,anchor=north,yshift=-0.2cm}]{{\sc
      EPR}}  & \qw  & \ctrl{1}& \qw &    & \gate{D_{p}}\gategroup[1,steps=1,style={dashed,rounded
      corners,fill=blue!20, inner
      xsep=2pt},background,label style={label
      position=below,anchor=north,yshift=-0.2cm}]{{\sc
      Dephasing}}  & \qw & \qw & \targ{} & \qw & \qw & \meter{} & \setwiretype{c} & & & \ctrl[vertical
wire=c]{1} \\
      \lstick{$\ket{0}$}  &  \qw & \qw &  \targ{} & \qw \qw & & \qw & \qw &\qw&\qw & \qw & \qw& \qw & \qw & \qw &  \qw & \gate{X} & \qw & \gate{Z} 
 \end{quantikz}
  \caption{Quantum Teleportation Protocol: Dephasing with probability $p$ after EPR pair creation.}
  \label{fig:teleport_dephasing}
\end{figure}


\begin{figure} [H]
  \centering
  \begin{quantikz} [column sep=0.2cm, row sep=0.5cm] 
      \lstick{$\ket{\psi}$}  & \qw &\qw & \qw & \qw & \qw& \ctrl{1}\gategroup[2,steps=4,style={dashed,rounded
      corners,fill=blue!20, inner
      xsep=2pt},background,label style={label
      position=below,anchor=north,yshift=-0.2cm}]{{\sc
      BellMeasure}} & \gate{H} & \qw & \meter{} & \setwiretype{c}  &  & \gategroup[3,steps=4,style={dashed,rounded
      corners,fill=blue!20, inner
      xsep=2pt},background,label style={label
      position=below,anchor=north,yshift=-0.2cm}]{{\sc
      Correction}}  &  & & \ctrl[vertical
wire=c]{2}  \\
      \lstick {$\ket{0}$}  &\gate{H}\gategroup[2,steps=3,style={dashed,rounded
      corners,fill=blue!20, inner
      xsep=2pt},background,label style={label
      position=below,anchor=north,yshift=-0.2cm}]{{\sc
      EPR}} & \qw  & \ctrl{1}& \qw & \qw & \targ{} & \qw & \qw & \meter{} & \setwiretype{c} & & & \ctrl[vertical
wire=c]{1} \\
      \lstick{$\ket{0}$}  &  \qw & \qw &  \targ{} & \qw &\qw&\qw & \qw & \qw& \qw & \qw & \qw &  \qw & \gate{X} & \qw & \gate{Z} & \qw & \qw   & \gate{A_{\gamma}}\gategroup[1,steps=2,style={dashed,rounded
      corners,fill=blue!20, inner
      xsep=2pt},background,label style={label
      position=below,anchor=north,yshift=-0.2cm}]{{\sc
      { \hspace{50 pt} Amplitude Damping}}} & \qw
 \end{quantikz}
  \caption{Quantum Teleportation Protocol: Amplitude Dampling with probability $\gamma$ after Correction.}
  \label{fig:teleport_amplitude_damping}
\end{figure}



\begin{figure} [H]
  \centering
  \begin{quantikz} [column sep=0.2cm, row sep=0.5cm] 
      \lstick{$\ket{\psi}$}  & \qw &\qw & \qw & \qw & \qw& \ctrl{1}\gategroup[2,steps=4,style={dashed,rounded
      corners,fill=blue!20, inner
      xsep=2pt},background,label style={label
      position=below,anchor=north,yshift=-0.2cm}]{{\sc
      BellMeasure}} & \gate{H^{\epsilon}} & \qw & \meter{} & \setwiretype{c}  &  & \gategroup[3,steps=4,style={dashed,rounded
      corners,fill=blue!20, inner
      xsep=2pt},background,label style={label
      position=below,anchor=north,yshift=-0.2cm}]{{\sc
      Correction}}  &  & & \ctrl[vertical
wire=c]{2}  \\
      \lstick {$\ket{0}$}  &\gate{H^{\epsilon}}\gategroup[2,steps=3,style={dashed,rounded
      corners,fill=blue!20, inner
      xsep=2pt},background,label style={label
      position=below,anchor=north,yshift=-0.2cm}]{{\sc
      EPR}} & \qw  & \ctrl{1}& \qw & \qw & \targ{} & \qw & \qw & \meter{} & \setwiretype{c} & & & \ctrl[vertical
wire=c]{1} \\
      \lstick{$\ket{0}$}  &  \qw & \qw &  \targ{} & \qw &\qw&\qw & \qw & \qw& \qw & \qw & \qw &  \qw & \gate{X} & \qw & \gate{Z} 
 \end{quantikz}
  \caption{Quantum Teleportation Protocol: Erroneous implementation of the Hadamard gate. $H^{\epsilon}$ is regarded as the composition $R_{y}(\frac{\pi}{2})\cdot P(\pi + \epsilon)$.}
  \label{fig:teleport_h}
\end{figure}




This operation is depicted in \autoref{fig:Operation_T}.

\begin{figure} [H]
  \centering
  \begin{quantikz} [column sep=0.2cm, row sep=0.5cm,wire
    types={n,n}]%
      \lstick{$\ket{\phi}$}  &\qw \gategroup[2,steps=9,style={dashed,rounded
      corners,fill=blue!20, inner
      xsep=2pt},background,label style={label
      position=below,anchor=north,yshift=-0.2cm}]{{\sc
      T}} & \qw  & \qw   & \qw  & \qw & \qw & \gate{U} \qw &\qw & \qw & \qw \\
      & & & \lstick {$\ket{0}$}  & \qw &\gate{R_X(\frac{\pi}{2})} \qw & \qw & \ctrl{-1} \qw & \qw & \gate{\text{Disc}} \qw 
    \end{quantikz}
  \caption{T operation}
  \label{fig:Operation_T}
\end{figure}


\begin{figure} [H]
  \centering
  \begin{quantikz} [column sep=0.2cm, row sep=0.5cm,wire
    types={n,n,n,n,n}]%
      \lstick{$\ket{\psi}$}  & \qw &\qw & \qw & \qw & \qw &  \ctrl{1} \qw \gategroup[2,steps=2,style={dashed,rounded
      corners,fill=blue!20, inner
      xsep=2pt},background,label style={label
      position=above,anchor=south,yshift=-0.2cm}]{{\sc
      TeleportIntra-gate}} & \gate{H} \qw & \qw & \qw &\qw \gategroup[5,steps=18,style={dashed,rounded
      corners,fill=blue!20, inner
      xsep=2pt},background,label style={label
      position=below,anchor=north,yshift=-0.2cm}]{{\sc
      TMeasureCorrection}}  & \qw & \qw & \qw & \targ{} \qw  & \qw & \qw & \qw & \qw & \qw & \qw  & \qw& \meter{} \qw      & \setwiretype{c}  &  &  & & \ctrl[vertical
wire=c]{2}  \\
      \lstick {$\ket{0}$}  &\gate{H} \qw \gategroup[2,steps=3,style={dashed,rounded
      corners,fill=blue!20, inner
      xsep=2pt},background,label style={label
      position=below,anchor=north,yshift=-0.2cm}]{{\sc
      EPR}} & \qw  & \ctrl{1} \qw & \qw & \qw  & \targ{} \qw & \qw & \qw & \qw & \qw& \qw & \qw & \qw & \qw &  \qw & \qw & \qw & \qw &  \qw &  \targ{} \qw & \qw& \meter{} \qw & \setwiretype{c}  & & \ctrl[vertical
wire=c]{1} \\
      \lstick{$\ket{0}$}  &  \qw & \qw &  \targ{} \qw \qw & \qw &\qw&\qw & \qw & \qw& \qw & \qw & \qw & \qw& \qw & \qw & \qw & \qw &  \qw & \qw & \qw & \qw & \qw & \qw & \qw& \qw& \gate{X} \qw & \qw & \gate{Z} \qw\\
       &   & &  &  & & &  &  & & & & & & &  & & & \lstick{$\ket{0}$}  & \gate{R_X(\frac{\pi}{2})} \qw& \ctrl{-2} \qw & \gate{\text{Disc}} \qw  \\
      &   &  & & & & &  &  & & & & \lstick{$\ket{0}$}   &\gate{R_X(\frac{\pi}{2})} \qw  & \ctrl{-4} \qw & \gate{\text{Disc}} \qw & && &  &  &  &  &   & &  &  &  &  &  &
    \end{quantikz}
  \caption{Quantum Teleportation Protocol: Bit flip with  50\% probability before measurement.}
  \label{fig:teleport_bit_flip}
\end{figure}

\end{comment}
