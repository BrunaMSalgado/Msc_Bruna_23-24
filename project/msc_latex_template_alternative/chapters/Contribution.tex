\chapter{Conditionals} \label{ch:conditionals}

In the calculus introduced in the previous chapter, something appears to be missing: a metric equation for conditionals. In this chapter, we address this gap by introducing such an equation and proving its soundness and completeness.




%For the metric quantale, the way-below relation corresponds to the strictly greater relation with ∞ > ∞, and a basis for the underlying lattice that satisfies the conditions above is the set of extended non-negative rational numbers.

\todo[inline,size=\normalsize]{Adicionar as coisas das notas dq elas estiverem corrigidas} 

\todo[inline,size=\normalsize]{Intro} 




%In order to established that the theory introduced is valid in quantum programming, it is necessary to build a model. The model can be seen as a category where the morphisms are the CPTP super-operators (quantum channels). The algebric structure of this model is given by the vector spaces. 
%Any completely-positive and trace-preserving map has a diamond norm equal to one \cite{watrous2018theory}. Since the measurement operation is completely positive and trace-preserving, its  diamond norm is equal to one. This is a desirable property, as it ensures that the measurement operation does not increase the distance between states, and as a consequence, composition of programs remains valid.





\section{Syntax}

\section{Soundeness and Completeness}

\begin{definition}
  Consider a tuple \( (G, \Sigma) \), where \( G \) is a class of ground types and \( \Sigma \) is a class of sorted operation symbols of the form \( f : A_1, \ldots, A_n \to A \) with \( n \geq 1 \). A \emph{linear metric $\lambda$-theory} is a tuple \( ((G, \Sigma), Ax) \), where \( Ax \) is a class of \emph{metric equations-in-context} over linear $\lambda$-terms constructed from \( (G, \Sigma) \).
\end{definition}

The elements of \( Ax \) are called the \emph{axioms} of the theory. Let \( Th(Ax) \) denote the smallest class that contains \( Ax \) and is closed under the rules presented in \autoref{fig:equations-linear-lambda} (i.e., the classical equational system) and \autoref{fig:metric deductive system}. The elements of \( Th(Ax) \) are called the \emph{theorems} of the theory.


\todo[inline,size=\normalsize]{Way bellow and lattice stuff (complete lattice, distributive latice, R+ é uma distributive lattice)}

\todo[inline,size=\normalsize]{Crole: tem as defs e diz que "Any poset which is a chain is distributive." tendo em conta que "A subset C of a preorder X is called a chain if for every
x, y G C we have x < y or y < x."}

\begin{definition}
	Consider a complete lattice $L$.  For every $x, y \in L$ we say that
	$y$ is \emph{way-below} $x$ (in symbols, $y \ll x$) if for every
	subset $X \subseteq L$ whenever $x \leq \bigvee X$ there exists a
	\emph{finite} subset $A \subseteq X$ such that $y \leq \bigvee A$.
	The lattice $L$ is called \emph{continuous} iff for every $x \in L$,
	\begin{flalign*}
		x = \bigvee \{ y  \mid y \in L\ \text{and} \ y \ll x \}
	\end{flalign*}
\end{definition}

\begin{definition}
	Let $L$ be a complete lattice. A \emph{basis} $B$ of $L$ is a subset
	$B \subseteq L$ such that for every $x \in L$ the set
	$B \cap \{ y \mid y \in L\ \text{and} \ y \ll x \}$ is directed and
	has $x$ as the least upper bound.
\end{definition}

\section{A small example}

\begin{comment}
\begin{figure} [H]
  \centering
  \begin{quantikz} [column sep=0.2cm, row sep=0.5cm] 
      \lstick{$\ket{\psi}$}  & \qw &\qw & \qw & \qw & \qw& \qw & \qw &\qw  & \ctrl{1}\gategroup[2,steps=4,style={dashed,rounded
      corners,fill=blue!20, inner
      xsep=2pt},background,label style={label
      position=below,anchor=north,yshift=-0.2cm}]{{\sc
      BellMeasure}} & \gate{H} & \qw & \meter{} & \setwiretype{c}  &  & \gategroup[3,steps=4,style={dashed,rounded
      corners,fill=blue!20, inner
      xsep=2pt},background,label style={label
      position=below,anchor=north,yshift=-0.2cm}]{{\sc
      Correction}}  &  & & \ctrl[vertical
wire=c]{2}  \\
      \lstick {$\ket{0}$}  &\gate{H}\gategroup[2,steps=3,style={dashed,rounded
      corners,fill=blue!20, inner
      xsep=2pt},background,label style={label
      position=below,anchor=north,yshift=-0.2cm}]{{\sc
      EPR}}  & \qw  & \ctrl{1}& \qw &    & \gate{D_{p}}\gategroup[1,steps=1,style={dashed,rounded
      corners,fill=blue!20, inner
      xsep=2pt},background,label style={label
      position=below,anchor=north,yshift=-0.2cm}]{{\sc
      Dephasing}}  & \qw & \qw & \targ{} & \qw & \qw & \meter{} & \setwiretype{c} & & & \ctrl[vertical
wire=c]{1} \\
      \lstick{$\ket{0}$}  &  \qw & \qw &  \targ{} & \qw \qw & & \qw & \qw &\qw&\qw & \qw & \qw& \qw & \qw & \qw &  \qw & \gate{X} & \qw & \gate{Z} 
 \end{quantikz}
  \caption{Quantum Teleportation Protocol: Dephasing with probability $p$ after EPR pair creation.}
  \label{fig:teleport_dephasing}
\end{figure}


\begin{figure} [H]
  \centering
  \begin{quantikz} [column sep=0.2cm, row sep=0.5cm] 
      \lstick{$\ket{\psi}$}  & \qw &\qw & \qw & \qw & \qw& \ctrl{1}\gategroup[2,steps=4,style={dashed,rounded
      corners,fill=blue!20, inner
      xsep=2pt},background,label style={label
      position=below,anchor=north,yshift=-0.2cm}]{{\sc
      BellMeasure}} & \gate{H} & \qw & \meter{} & \setwiretype{c}  &  & \gategroup[3,steps=4,style={dashed,rounded
      corners,fill=blue!20, inner
      xsep=2pt},background,label style={label
      position=below,anchor=north,yshift=-0.2cm}]{{\sc
      Correction}}  &  & & \ctrl[vertical
wire=c]{2}  \\
      \lstick {$\ket{0}$}  &\gate{H}\gategroup[2,steps=3,style={dashed,rounded
      corners,fill=blue!20, inner
      xsep=2pt},background,label style={label
      position=below,anchor=north,yshift=-0.2cm}]{{\sc
      EPR}} & \qw  & \ctrl{1}& \qw & \qw & \targ{} & \qw & \qw & \meter{} & \setwiretype{c} & & & \ctrl[vertical
wire=c]{1} \\
      \lstick{$\ket{0}$}  &  \qw & \qw &  \targ{} & \qw &\qw&\qw & \qw & \qw& \qw & \qw & \qw &  \qw & \gate{X} & \qw & \gate{Z} & \qw & \qw   & \gate{A_{\gamma}}\gategroup[1,steps=2,style={dashed,rounded
      corners,fill=blue!20, inner
      xsep=2pt},background,label style={label
      position=below,anchor=north,yshift=-0.2cm}]{{\sc
      { \hspace{50 pt} Amplitude Damping}}} & \qw
 \end{quantikz}
  \caption{Quantum Teleportation Protocol: Amplitude Dampling with probability $\gamma$ after Correction.}
  \label{fig:teleport_amplitude_damping}
\end{figure}



\begin{figure} [H]
  \centering
  \begin{quantikz} [column sep=0.2cm, row sep=0.5cm] 
      \lstick{$\ket{\psi}$}  & \qw &\qw & \qw & \qw & \qw& \ctrl{1}\gategroup[2,steps=4,style={dashed,rounded
      corners,fill=blue!20, inner
      xsep=2pt},background,label style={label
      position=below,anchor=north,yshift=-0.2cm}]{{\sc
      BellMeasure}} & \gate{H^{\epsilon}} & \qw & \meter{} & \setwiretype{c}  &  & \gategroup[3,steps=4,style={dashed,rounded
      corners,fill=blue!20, inner
      xsep=2pt},background,label style={label
      position=below,anchor=north,yshift=-0.2cm}]{{\sc
      Correction}}  &  & & \ctrl[vertical
wire=c]{2}  \\
      \lstick {$\ket{0}$}  &\gate{H^{\epsilon}}\gategroup[2,steps=3,style={dashed,rounded
      corners,fill=blue!20, inner
      xsep=2pt},background,label style={label
      position=below,anchor=north,yshift=-0.2cm}]{{\sc
      EPR}} & \qw  & \ctrl{1}& \qw & \qw & \targ{} & \qw & \qw & \meter{} & \setwiretype{c} & & & \ctrl[vertical
wire=c]{1} \\
      \lstick{$\ket{0}$}  &  \qw & \qw &  \targ{} & \qw &\qw&\qw & \qw & \qw& \qw & \qw & \qw &  \qw & \gate{X} & \qw & \gate{Z} 
 \end{quantikz}
  \caption{Quantum Teleportation Protocol: Erroneous implementation of the Hadamard gate. $H^{\epsilon}$ is regarded as the composition $R_{y}(\frac{\pi}{2})\cdot P(\pi + \epsilon)$.}
  \label{fig:teleport_h}
\end{figure}




This operation is depicted in \autoref{fig:Operation_T}.

\begin{figure} [H]
  \centering
  \begin{quantikz} [column sep=0.2cm, row sep=0.5cm,wire
    types={n,n}]%
      \lstick{$\ket{\phi}$}  &\qw \gategroup[2,steps=9,style={dashed,rounded
      corners,fill=blue!20, inner
      xsep=2pt},background,label style={label
      position=below,anchor=north,yshift=-0.2cm}]{{\sc
      T}} & \qw  & \qw   & \qw  & \qw & \qw & \gate{U} \qw &\qw & \qw & \qw \\
      & & & \lstick {$\ket{0}$}  & \qw &\gate{R_X(\frac{\pi}{2})} \qw & \qw & \ctrl{-1} \qw & \qw & \gate{\text{Disc}} \qw 
    \end{quantikz}
  \caption{T operation}
  \label{fig:Operation_T}
\end{figure}


\begin{figure} [H]
  \centering
  \begin{quantikz} [column sep=0.2cm, row sep=0.5cm,wire
    types={n,n,n,n,n}]%
      \lstick{$\ket{\psi}$}  & \qw &\qw & \qw & \qw & \qw &  \ctrl{1} \qw \gategroup[2,steps=2,style={dashed,rounded
      corners,fill=blue!20, inner
      xsep=2pt},background,label style={label
      position=above,anchor=south,yshift=-0.2cm}]{{\sc
      TeleportIntra-gate}} & \gate{H} \qw & \qw & \qw &\qw \gategroup[5,steps=18,style={dashed,rounded
      corners,fill=blue!20, inner
      xsep=2pt},background,label style={label
      position=below,anchor=north,yshift=-0.2cm}]{{\sc
      TMeasureCorrection}}  & \qw & \qw & \qw & \targ{} \qw  & \qw & \qw & \qw & \qw & \qw & \qw  & \qw& \meter{} \qw      & \setwiretype{c}  &  &  & & \ctrl[vertical
wire=c]{2}  \\
      \lstick {$\ket{0}$}  &\gate{H} \qw \gategroup[2,steps=3,style={dashed,rounded
      corners,fill=blue!20, inner
      xsep=2pt},background,label style={label
      position=below,anchor=north,yshift=-0.2cm}]{{\sc
      EPR}} & \qw  & \ctrl{1} \qw & \qw & \qw  & \targ{} \qw & \qw & \qw & \qw & \qw& \qw & \qw & \qw & \qw &  \qw & \qw & \qw & \qw &  \qw &  \targ{} \qw & \qw& \meter{} \qw & \setwiretype{c}  & & \ctrl[vertical
wire=c]{1} \\
      \lstick{$\ket{0}$}  &  \qw & \qw &  \targ{} \qw \qw & \qw &\qw&\qw & \qw & \qw& \qw & \qw & \qw & \qw& \qw & \qw & \qw & \qw &  \qw & \qw & \qw & \qw & \qw & \qw & \qw& \qw& \gate{X} \qw & \qw & \gate{Z} \qw\\
       &   & &  &  & & &  &  & & & & & & &  & & & \lstick{$\ket{0}$}  & \gate{R_X(\frac{\pi}{2})} \qw& \ctrl{-2} \qw & \gate{\text{Disc}} \qw  \\
      &   &  & & & & &  &  & & & & \lstick{$\ket{0}$}   &\gate{R_X(\frac{\pi}{2})} \qw  & \ctrl{-4} \qw & \gate{\text{Disc}} \qw & && &  &  &  &  &   & &  &  &  &  &  &
    \end{quantikz}
  \caption{Quantum Teleportation Protocol: Bit flip with  50\% probability before measurement.}
  \label{fig:teleport_bit_flip}
\end{figure}

\end{comment}
