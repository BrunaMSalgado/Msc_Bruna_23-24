\chapter{Metric $\lambda$-calculus with conditionals} \label{ch:conditionals}

As noted in \autoref{subsec:interlude_bool}, a metric equation for the conditional statement would be extremely helpful for reasoning about approximate equivalence within the $\lambda$-calculus with conditionals. In this chapter, we address this gap by introducing such an equation and proving both its soundness and completeness. 
We present the category of metric spaces as an example of a setting suitable for reasoning about approximate equivalence in this context. Furthermore, we prove if $\catC$ is a $\catMet$-category, then its coproduct cocompletion admits a metric making it a distributive $\catMet$-enriched category.
Finally, we illustrate the usefulness of the introduced equation through two small examples: reasoning about approximate equivalence between boolean terms (i.e., terms of type $\typeI \oplus \typeI$) and the extensionality of copairing. We conclude the chapter with a brief discussion on the suitability of the chosen metric equation.






\section{Syntax}

The metric equation for conditionals is presented in \autoref{fig:metric conditionals}. 

\begin{figure}[H]
  \begin{equation*}
  \begin{aligned}
  &
  &
  %
  \begin{prooftree}
      \hypo{ v =_{q} v' }
      \hypo{w=_{r} w'}
      \hypo{u=_{s}u'}
      \infer3[]{\text{ case } v \,   \{\text{inl} (x) \Rightarrow w ; \, \text{inr} (y) \Rightarrow u\} =_{q+\sup{\{ r, s \}}} \text{ case } v' \,  \{\text{inl} (x) \Rightarrow w' ; \,\text{inr} (y) \Rightarrow u'\} }
  \end{prooftree}
  %
  \\[10pt]
  \end{aligned}
  \end{equation*}
  \caption{Metric equation for condicionals}
  \label{fig:metric conditionals}
\end{figure}


Intuitively, this equation states that the maximum distance between the two conditional statements is determined the:

\begin{itemize}
    \item The larger of the maximum distances between the corresponding branches, \ie, the program pairs $(w, w')$ and $(u, u')$; and
    \item The maximum distance between the terms $v$ and $v'$ on which their execution depends.
\end{itemize}


Recall \autoref{def:metric_lambda_theory}. Now, we extend \( Th(Ax) \) to denote the smallest class that contains \( Ax \) and is closed under the rules presented in \autoref{fig:equations-linear-lambda}, \autoref{fig:metric deductive system} and \autoref{fig:metric conditionals}.


\section{Interpretation}

In this subsection, we extend the concepts introduced in \autoref{subsec:semantics_metric} to include the interpretation of the new metric equation.

\begin{definition}
  A \emph{distributive symmetric monoidal closed $\catMet$-category} $\catC$ is a category that is both distributive symmetric monoidal closed  and a symmetric monoidal closed $\catMet$‑category, such that, for all morphisms $f_1, f_2 \in \catC(A, C)$ and $g_1, g_2 \in \catC(B, C)$, we have:
\[
  d_{A \oplus B, C}([f_1, g_1], [f_2, g_2]) \leq \sup \{d_{A,C}(f_1, f_2), d_{B,C}(g_1, g_2)\}.
\]
\end{definition}


\begin{definition}
  Consider a metric $\lambda$-theory $((G,\Sigma),Ax)$ and a distributive symmetric monoidal closed $\catMet$-category $\catC$. Suppose that for each $X \in G$ we have an interpretation $\llbracket X \rrbracket$ as a $\catC$-object and analogously for the operation symbols. This interpretation structure is a model of the theory if all axioms in $Ax$ are satisfied by the interpretation.
\end{definition}



\section{Lattice Theory Preliminaries}
In this section, we introduce a few concepts from lattice theory that will be useful for the completeness proof and for the broader discussion of our results.
%For the metric quantale, the way-below relation corresponds to the strictly greater relation with ∞ > ∞, and a basis for the underlying lattice that satisfies the conditions above is the set of extended non-negative rational numbers.


\begin{definition}
   A \emph{lattice} is a partial order in which every finite subset has both a meet and a join. A \emph{complete lattice} is a partial order  in which every subset, finite or infinite, has a meet and a join.
\end{definition}

\begin{definition}
  \emph{Quantales} are complete lattices equipped with an associative composition, typically denoted $\otimes$, that preserves suprema in both arguments.
\end{definition}

\begin{example}
  In this work, we consider only the standard metric quantale, whose underlying complete lattice is the interval $[0, \infty]$ ordered by $\geq$, with associative composition given by addition. Note that, since the order is reversed in this quantale, suprema in general lattices correspond to infima here, and vice versa.
\end{example}


\begin{lemma}[Way-below relation for the metric quantale] \cite{dahlqvist2022syntactic}
Let $L$ denote the complete lattice $[0, \infty]$ ordered by $\geq$.  
For all $x, y \in L$ and for every subset $X \subseteq L$, if $y > x$, then whenever $x \geq \inf X$, there exists a \emph{finite} subset $A \subseteq X$ such that $y \geq \inf A$.

This property characterizes the \emph{way-below relation} in the metric quantale. The lattice $L$ is \emph{continuous}, \ie for every $x \in L$,
	\begin{flalign*}
		x = \inf \{ y  \mid y \in L\ \text{and} \ y > x \}.
	\end{flalign*}
\end{lemma}


\begin{definition}
  A subset $D$ of a lattice $L$ is called \emph{directed} if it is nonempty and every finite subset of $D$ has an upper bound in $D$. A partially order is said to be \emph{directed complete} if every directed subset has a supremum. A directed complete poset is commonly referred to as a \emph{dcpo}.
\end{definition}

\begin{definition}
  A semilattice $L$ is called \emph{meet continuous} if it is directed complete, i.e., a dcpo, and satisfies the condition
  \[
    x \wedge \sup D = \sup \{x \wedge d \mid d \in D\} \tag{MC}
  \]
  for all $x \in L$ and all directed subsets $D \subseteq L$.
\end{definition}

\begin{lemma} \cite[Proposition I-1.8]{gierzContinuousLatticesDomains2003} \label{lem:meet_continuos}
  Every continuous lattice is meet continuous.
\end{lemma}

\begin{lemma} \cite[Lemma 2.23]{daveyIntroductionLatticesOrder2002} \label{lem:sup_union}
Let \( P \) be a lattice, let \( S, T \subseteq P \) and assume that \( \bigvee S \),
\( \bigvee T \), \( \bigwedge S \) and \( \bigwedge T \) exist in \( P \). Then
$$ \bigvee(S \cup T) = \left(\bigvee S\right) \vee \left(\bigvee T\right) \quad \text{and} \quad \bigwedge(S \cup T) = \left(\bigwedge S\right) \wedge \left(\bigwedge T\right) . $$
\end{lemma}


\section{Soundeness and Completeness}
The proofs in this section are based on the proofs of 
\autoref{thm:soundness_metric_no_cond} and \autoref{thm:completeness_metric_no_cond} in \cite{dahlqvist2022syntactic}.

\begin{theorem}
  The rules in Figures \ref{fig:equations-linear-lambda}, \ref{fig:metric deductive system} and \ref{fig:metric conditionals} are sound for a  distributive symmetric monoidal closed $\catMet$-category $\catC$. Specifically, if $\Gamma \vljud v =_{q} w : \typeA $ results from the rules in Figures  \ref{fig:equations-linear-lambda}, \ref{fig:metric deductive system} and \ref{fig:metric conditionals} then $q \geq d( \llbracket \Gamma  \vljud v : \typeA \rrbracket, \llbracket\Gamma \vljud w : \typeA \rrbracket)$.
\end{theorem}

\begin{proof}
  We follow the same strategy as \cite{dahlqvist2022syntactic}.
  This proof uses induction over the depth of proof trees that
  arise from the metric deductive system. The general strategy for each
  inference rule is to use the definition of a distributive symmetric monoidal closed $\catMet$-category. 

   More concretly, first, consider the equations on \autoref{fig:equations-linear-lambda} which abreviate equations $\Gamma \triangleright w =_0 v : \typeA$. By \autoref{thm:soundness_classical}, these equations are sound for distributive symmetric monoidal closed categories, \ie if $v=w$, then $\sem{v}=\sem{w}$ in $ \catC$. Then by the definition of a $\catMet$-Category (\autoref{def:monoidal_closed_met_cat}) we obtain $d(\sem{v},\sem{w}) = d(\sem{w},\sem{v})=0.$ 
 
   The rules in \autoref{fig:metric deductive system} follow from the definition of a symmetric monoidal closed $\catMet$-category. Finally, regarding  the rule in  \autoref{fig:metric conditionals}, it follows from the fact that $\catC$ is distributive symmetric monoidal closed $\catMet$-category. We reason as follows:
  \begin{align*}
    & \hspace{-20pt} d( \llbracket  \text{ case } v \, \{\text{inl}_{\typeB}  (x) \Rightarrow w ; \, \text{inr}_{\typeA} (y) \Rightarrow u\}  \rrbracket, \llbracket \text{ case } v' \, \{\text{inl}_{\typeB}  (x) \Rightarrow w' ; \, \text{inr}_{\typeA} (y) \Rightarrow u'\} \rrbracket)  \\
    &  \hspace{-20pt} = d( [\llbracket w\rrbracket ,\llbracket u\rrbracket] \comp (\text{jn}_{\Delta;\typeA}\comp \sw \oplus \text{jn}_{\Delta;\typeB}\comp \sw) \comp \dist \comp (\llbracket v\rrbracket \otimes \text{id}) \comp \text{sp}_{\Gamma;\Delta} \comp \text{sh}_{E}, \\
    & \hspace{7pt} [\llbracket w'\rrbracket ,\llbracket u'\rrbracket] \comp (\text{jn}_{\Delta;\typeA}\comp \sw \oplus \text{jn}_{\Delta;\typeB}\comp \sw) \comp \dist \comp (\llbracket v'\rrbracket \otimes \text{id}) \comp \text{sp}_{\Gamma;\Delta} \comp \text{sh}_{E})\\
    & \hspace{-20pt} \leq  d( [\llbracket w\rrbracket ,\llbracket u\rrbracket] \comp (\text{jn}_{\Delta;\typeA}\comp \sw \oplus \text{jn}_{\Delta;\typeB}\comp \sw) \comp \dist \comp (\llbracket v\rrbracket \otimes \text{id}),  [\llbracket w'\rrbracket ,\llbracket u'\rrbracket] \comp (\text{jn}_{\Delta;\typeA}\comp \sw \, \oplus \\
    & \hspace{7pt}  \text{jn}_{\Delta;\typeB}\comp \sw) \comp \dist \comp (\llbracket v'\rrbracket \otimes \text{id}) )\\
    & \hspace{-20pt} \leq  d(\llbracket v\rrbracket \otimes \text{id}, \llbracket v'\rrbracket \otimes \text{id}) +  d( [\llbracket w\rrbracket ,\llbracket u\rrbracket] \comp (\text{jn}_{\Delta;\typeA}\comp \sw \oplus \text{jn}_{\Delta;\typeB}\comp \sw) \comp \dist,  [\llbracket w'\rrbracket ,\llbracket u'\rrbracket]\comp \\
    & \hspace{7pt} (\text{jn}_{\Delta;\typeA}\comp \sw \oplus \text{jn}_{\Delta;\typeB}\comp \sw) \comp \dist) \\
    & \hspace{-20pt} \leq  q +  d( [\llbracket w\rrbracket ,\llbracket u\rrbracket] \comp (\text{jn}_{\Delta;\typeA}\comp \sw \oplus \text{jn}_{\Delta;\typeB}\comp \sw) \comp \dist,  [\llbracket w'\rrbracket ,\llbracket u'\rrbracket] \comp (\text{jn}_{\Delta;\typeA}\comp \sw \oplus \text{jn}_{\Delta;\typeB}\comp \sw)  \\
    & \hspace{20pt} \comp \dist)\\
    &  \hspace{-20pt} \leq q +  d( [\llbracket w\rrbracket ,\llbracket u\rrbracket],  [\llbracket w'\rrbracket ,\llbracket u'\rrbracket]) \\
    & \hspace{-20pt}  \leq q + \sup(d( \llbracket w\rrbracket , \llbracket w' \rrbracket), d( \llbracket u\rrbracket , \llbracket u' \rrbracket)) \\
    & \hspace{-20pt} \leq  q + \sup(r, s) \\
  \end{align*}
  The second step follows from the fact that $\text{sp}_{\Gamma;\Delta} \comp \text{sh}_{E}$  is a morphism in $\catC$  and that $\catC$ is a $\catMet$-category.  The third and fifth step follow from an analogous reasoning. The fourth step follows from the premises of the rule in question and the fact that $\catC$ is a symmetric monoidal $\catMet$-category. The sixth step follows from the fact that $\catC$ is a distributive symmetric monoidal $\catMet$-enriched. Finally, the last step follows from the premise of the rule in question.

\end{proof}

Next, we extend  \texttt{Values}$((\typeA,\typeB),d)$, by incorporating the new theorems and quotient this category into a  $\catMet$-category (\texttt{Values}$(\typeA,\typeB),d)$/$\sim$.


\begin{theorem}[Completeness]
Consider a varietal metric $\lambda$-theory. A metric equation in context
$\Gamma \vljud v =_{q} w : \typeA$
is a theorem if and only if it holds in all models of the theory.
\end{theorem}

\begin{proof}
  Here, we focus only on the coproduct, as the remaining cases are proven in~\cite{dahlqvist2022syntactic}.
  Completeness arises from constructing the syntactic category $\catSyn(\mathscr{T})$ of the underlying  theory
  $\mathscr{T}$ and then show that provability of $\Gamma \vljud v =_q w : \typeA$
  in $\mathscr{T}$ is equivalent to $a(\sem{v},\sem{w}) \geq q$ in the category
  $\catfont{Syn}(\mathscr{T})$. We use the category (\texttt{Values}$(\typeA,\typeB),d)$/$\sim$ to this end.  Note that the quotienting process identifies all terms $x : \typeA \triangleright v : \typeB$ and $x : \typeA \triangleright w : \typeB$ such that $v =_0 w$ and $w =_0 v$. This relation includes the equations-in-context from \autoref{fig:equations-linear-lambda}.
     First, we remark that all sets of the form $\{ q \mid v =_q w \}$ are directed: they are non-empty, since by equation~(join) we always have at least $v =_{\infty} w$, and again by~(join), every finite subset of $\{ q \mid v =_q w \}$ has a lower bound in the set. This will be useful for applying \autoref{lem:meet_continuos}.
   
   Next, we need to prove that the copairing is well defined in (\texttt{Values}$(\typeA,\typeB),d)$/$\sim$, \ie, for any $a, a', b, b'$ if $a\sim a'$ and $b\sim b'$, then $[a,b] \sim [a',b']$. 

   This is equivalent to demonstrating the following implication:
   \begin{align*}
    \begin{cases}
      \inf \{ q \, \vert \, v=_q w \} \leq 0 \\
      \inf \{ r \, \vert \, v=_r w \} \leq 0 \\
    \end{cases} \implies \inf \left\{q \, \bigg\vert \,  
    \begin{aligned}
       &\text{ case } z \,   \{\text{inl} (x) \Rightarrow v ; \, \text{inr} (y) \Rightarrow w\} =_{q} \\
       &\text{ case } z \,  \{\text{inl} (x) \Rightarrow v' ; \,\text{inr} (y) \Rightarrow w'\}
    \end{aligned}\right\} \leq 0.
   \end{align*}
%which follows from \autoref{lem:meet_continuos}, \autoref{lem:sup_union} and the metric equation for conditionals. 
Let $L$ be a lattice, and let $D, F \subseteq L$ be directed sets. Observe the following:
\begin{equation} \label{eq:completeness}
  \begin{split}
     & \sup \{ \inf D, \inf F \} \\\
    & = \inf \{ \sup \{\inf D, f\} \, \vert \, f \in F \} & (\text{\autoref{lem:meet_continuos}}) \\
    & = \inf \{ \inf\{ \sup \{\inf d, f\} \,\vert \, d \in D \} \vert \, f \in F \} & (\text{\autoref{lem:meet_continuos}}) \\
    & =  \inf \{  \sup \{d,f\} \, \vert \, d \in D, f \in F \} & (\text{\autoref{lem:sup_union}})
  \end{split}
  \end{equation}
With the equality above, we have
\begin{align*}
    &\begin{cases}
      \inf \{ q \, \vert \, v=_q w \} \leq 0 \\
      \inf \{ r \, \vert \, v=_r w \} \leq 0 \\
    \end{cases} \\
    &\implies  \sup \{ \inf \{ q \, \vert \, v=_q w \} , \inf \{ r \, \vert \, v=_r w \} \} \leq 0 \\
    & \implies \inf \{ \sup \{q,r\} \vert \,  v=_q w,  v=_r w  \} \leq 0 & (\text{\autoref{eq:completeness}})  \\
    & \implies \left\{q \, \bigg\vert \,  
    \begin{aligned}
       &\text{ case } z \,   \{\text{inl} (x) \Rightarrow v ; \, \text{inr} (y) \Rightarrow w\} =_{q}\\
        &\text{ case } z \,  \{\text{inl} (x) \Rightarrow v' ; \,\text{inr} (y) \Rightarrow w'\}
    \end{aligned}\right\} \leq 0
  \end{align*}



  Thus, we obtain a category $\catSyn(\mathscr{T})$ whose objects are the types of the language and whose hom-sets are the underlying sets of the  $\catMet$-category (\textbf{Values}$(\typeA,\typeB),d)$/$\sim$.


  The next step is to show that $\catSyn(\mathscr{T})$ is a distributive symmetric monoidal $\catMet$-category. To this effect, with repect to the case statement, we reason as follows:

  \begin{align*}
    & \sup{\{d([v],[w]),d([v'],[w']) \}}  \\
    & = \sup{\{d(v,w),d(v',w') \}} \\
    & = \sup {\{ \inf{\{q \, \vert \, v=_q v'\}},\inf{\{r \, \vert \, w=_r w'\}}  \}} \\
    & = \inf{\{ \sup \{ q, r \} \vert \, v=_q v', w=_r w' \}} &  \\
    & \geq  \inf{ \{ q  \,\vert \, \text{ case } z \,   \{\text{inl} (x) \Rightarrow v ; \, \text{inr} (y) \Rightarrow w\} =_{q} \text{ case } z \,  \{\text{inl} (x) \Rightarrow v' ; \,\text{inr} (y) \Rightarrow w'\} \} } &  \\ 
    & = d(\text{case } z \,   \{\text{inl} (x) \Rightarrow v ; \, \text{inr} (y) \Rightarrow w\}, \text{case } z \,  \{\text{inl} (x) \Rightarrow v' ; \,\text{inr} (y) \Rightarrow w'\}) \\
    & = d([\text{case } z \,   \{\text{inl} (x) \Rightarrow v ; \, \text{inr} (y) \Rightarrow w\}], [\text{case } z \,  \{\text{inl} (x) \Rightarrow v' ; \,\text{inr} (y) \Rightarrow w'\}]) \\
    & = d([[v],[v']],[[w],[w']])  
  \end{align*}
   The third step follows from \autoref{eq:completeness}, and the fourth step follows from the fact that for any sets $A$ and $B$, if $A \subseteq B$, then $\inf\{A\} \geq \inf\{B\}$.

  The final step is to show that if an equation $\Gamma \vljud v =_q v' : \typeA$ with $q \in [0, \infty]$ is satisfied by Syn$(\mathscr{T})$ then it is a theorem of the linear metric $\lambda$-theory. By assumption, $d([v],[v']) = d(v,v') =  \inf{ \{r \, \vert \, v =_r v'\}} \leq q$. It follows from the definition of the way-below relation that for all
 $x \in [0, \infty]$ with $x>q$ there exists a finite set $A \subseteq \{r \, \vert \, v =_r v'\}$ such that $x \geq \inf{A}$. Then by an
 application of rule (join) we obtain $v =_{\inf{A}} v'$, and consequently, rule (weak) provides $v =_x v'$ for all $x > q$. Finally, by an application of rule (arch)  we deduce that $v =_q v'$ is part of the theory.
\end{proof}

\section{A small example: $\catMet$}

\begin{proposition}
        \label{prop:vcat}
        The category $\catMet$ is a distributive symmetric monoidal closed $\catMet$-category 
\end{proposition}

\begin{proof}
  By~\cite[Example 3.8]{dahlqvist2022syntactic}, the category $\catMet$ is a symmetric monoidal closed $\catMet$-category. As a result, it suffices to show that for all morphisms \( f_1, f_2 \in \catMet(A, C) \) and \( g_1, g_2 \in \catMet(B, C) \), the following inequality holds:
  
\[
  d_{A \oplus B, C}([f_1, g_1], [f_2, g_2]) \leq \sup \{d_{A,C}(f_1, f_2), d_{B,C}(g_1, g_2)\}.
\]

In this category, the coproduct is defined as in $\catSet$. The distance function $d$ on the coproduct $A \oplus B$ is given by:
   \[
    \begin{cases}
    d_{A \oplus B}(\inl(a_1),\inl(x_2)) = d_A(a_1,a_2) \\
    d_{A \oplus B}(\inr(b_1),\inr(b_2)) = d_B(b_1,b_2) \\
    d_{A \oplus B}(\inl(a), \inr(b)) = 
    d_{A \oplus B}(\inr(b), \inl(a))  = \infty
    \end{cases}
    \]

    The co-pairing is defined as in $\catfont{Set}$. The inequality we aim to prove follows directly from the fact that, given two morphisms  $f,g\in \catMet(A, B)$ the distance between them is defined as 
  $$ d_{B}(\inl(x_1),\inl(x_2)) = \sup \{ d_A (f a, g a) \,\vert \, a \in A \} ,$$  
    together with \autoref{lem:sup_union}. For  \( f_1, f_2 \in \catMet(A, C) \) and \( g_1, g_2 \in \catMet(B, C) \), we calculate:
    \begin{align*}
      & d_{A \oplus B, C}([f_1, g_1], [f_2, g_2]) \\
      &  =  \sup \{ d_{A\oplus B} ([f_1, g_1] (a,b), [f_2, g_2] (a,b)) \,\vert \, (a,b) \in A \oplus B \} \\
      & =  \sup \{ \{ d_{A\oplus B} ([f_1, g_1] (\inl(a)), [f_2, g_2] (\inl(a))) \,\vert \, a \in A \}    \\
      & \hspace{15pt} \cup \{d_{A\oplus B} ([f_1, g_1] (\inr(b)), [f_2, g_2] (\inr(b))) \,\vert \, b \in B\}\} \\
      & = \sup \{ \{ d_{A}(f_1 (a), f_2 (a)) \,\vert \, a \in A \} \cup  \{d_{B}(g_1 (b), g_2 (b)) \,\vert \, b \in B \}\} \\
      & = \sup   \{ \sup \{ d_{A}(f_1 (a), f_2 (a)) \,\vert \, a \in A \}, \sup\{d_{B}(g_1 (b), g_2 (b)) \,\vert \, b \in B \}\} \\
      &=  \sup \{d_{A,C}(f_1, f_2), d_{B,C}(g_1, g_2)\}. 
    \end{align*}
\end{proof}





\section{Coproduct cocompletion}
\todo[inline,size=\normalsize]{Ir à parte das cats e acrescentar que não temos só produtos e coproductos binários -> p64 prod, Proposition 3.12. Coproducts are unique up to isomorphism. ->awodey}

We note that more powerful machinery, based on advanced categorical structures such as presheaves, can be employed to address the fact that the coproduct cocompletion of a category $\catC$ forms a $\catMet$-category. 


Up until how we have only discussed binary products/coproducts. However ....

The idea behind the coproduct completion of a category $\catC$ is to create a new category where all small coproducts exist by formally adding them to $\catC$  in the simplest way possible.

\begin{definition}
  A \emph{(free) coproduct completion} of a category \(\mathcal{C}\) is the category whose objects are families \((X_i)_{i \in I}\) of objects of \(\mathcal{C}\), where \(I\) is a set; an arrow \((X_i)_{i \in I} \to (Y_j)_{j \in J}\) consists of a pair \((f, (\phi_i)_{i \in I})\), where \(f : I \to J\) is a function between the index sets, and \((\phi_i)_{i \in I}\) is a family of morphisms \(\phi_i : X_i \to Y_{f(i)}\) in \(\mathcal{C}\). Given morphisms $(f,(\phi)_i): (X_i)_{i \in I} \to (Y_j)_{j \in J}$ and $(g,(\psi)_i): (Y_j)_{j \in J} \to (Z_k)_{k \in K}$, their composition is defined as the morphism, given by the pair \((g \comp f, (\theta_i)_{i \in I})\), where $ \theta_i := \psi_{f(i)} \circ \varphi_i : X_i \to Z_{g(f(i))}$. From now on, unless ambiguity arises, we will omit the indexing function and use uppercase letters to refer to families of morphisms.
\end{definition}

If $\catC$ is a $\catMet$-category, one may define a metric on the morphims of its coproduct cocompletion as follows:

\begin{equation*}
  d(\Phi, \Psi) = \sup \left\{ d'(\phi_i, \psi_i) \;\middle\vert\; i \in I \right\},\text{ where } 
  d'(\phi_i, \psi_i) 
  = 
  \begin{cases}
    \infty, & \text{if } f(i) \neq g(i), \\
    d(\phi_i, \psi_i), & \text{otherwise}.
  \end{cases}
\end{equation*}





\todo[inline,size=\normalsize]{Mostrar q tb funciona para a soma?}
Alternatively we may take the sum as the metric:


\section{Interlude:Booleans - Part 2}

Having proven that the metric equation for conditionals is both sound and complete, we can now use it to explore the booleans introduced in the previous chapter in \autoref{subsec:interlude_bool}. For instance, given axioms
$$ v' =_{\epsilon_1} a \quad \text{and}  \quad w' =_{\epsilon_2} b $$
we can postulate 
$$ \textbf{Conjunction} [v'/v, w'/w]  =_{\epsilon_1 + \epsilon_2}  \textbf{Conjunction} [a/v, b/w]. $$

First note that using the equation $\eta_{\text{dis}}$, we have that $\Gamma \vljud \text{dis}(w) =_{0} \text{dis}(b)$.
Then,using our metric deductive system we calculate

\begin{align*}
  &\text{case } v'  
    \left\{ \begin{aligned}
    \inl_{\typeI}(x) \Rightarrow  x \text{ to} *. \, w ;  \inl_{\typeI}(y) \Rightarrow y \text{ to} *. \,  \text{dis}(w) \text{ to} *. \, \inr_{\typeI}(*)
  \end{aligned}  \right\}   \\
  =_{\epsilon_1 + \epsilon_2} \, & \text{case } a  
    \left\{ \begin{aligned}
     \inl_{\typeI}(x) \Rightarrow  x \text{ to} *. \, b ;  \inl_{\typeI}(y) \Rightarrow y \text{ to} *. \,  \text{dis}(b) \text{ to} *. \, \inr_{\typeI}(*)
  \end{aligned}  \right\}.
\end{align*}

Applying similar reasoning, is straightforward to see that we have a similar equation for the \textbf{Disjunction} program and that given an axiom $ v' =_{\epsilon} a $, we can postulate $\textbf{Negation} [v'/v] =_{\epsilon} \textbf{Negation} [a/v] $.

Now, consider the axiom
$$ v' =_{\epsilon} \inl_{\typeI}(*)$$

Considering Lemma \ref{lemma:inl_neutral}, we have
\begin{align*}
   \textbf{Conjunction} [v'/v, w'/w]  &=_{\epsilon}  \textbf{Conjunction} [ \inl_{\typeI}(*)'/v, w'/w] = w'
\end{align*}


\section{Extensionality of the copairing}

A $\lambda$-abstraction that receives inputs of a disjunctive type is determined by what it does to inputs ``from the left and from the right'', \ie,
\[
\left\{
\begin{aligned}
&(\lambda x.t)\, \text{inl}(y) = (\lambda x.s)\, \text{inl}(y) \\
&(\lambda x.t)\, \text{inr}(z) =(\lambda x.s)\, \text{inr}(z)
\end{aligned}
\right.
\implies \lambda x.t = \lambda x.s
\]

The proof is as follows:

Using the $\beta$ equation we have
\begin{equation} \label{eq:beta_red}
  \left\{
\begin{aligned}
&(\lambda x.t)\, \text{inl}(y) = (\lambda x.s)\, \text{inl}(y) \\
&(\lambda x.t)\, \text{inr}(z) =(\lambda x.s)\, \text{inr}(z)
\end{aligned}
\right.
= \left\{
\begin{aligned}
&t\,[\text{inl}(y)/x]  = s\, [\text{inl}(y)/x] \\
&t\,[\text{inr}(z)/x]  = s\, [\text{inr}(z]/x] 
\end{aligned}
\right.
\end{equation}
Next, considering the equations above, $\eta_{case}$, and the $\alpha$-equivalence we reason as follows:
\begin{equation*}
\begin{split}
  t = & t[x'/x] = \text{case } x'  
    \left\{ \begin{aligned}
    \inl_{\typeI}(y) \Rightarrow t\,[\text{inl}(y)/x] ;  
    \inl_{\typeI}(z) \Rightarrow t\,[\text{inr}(z)/x]
  \end{aligned}  \right\} & (\alpha,\eta_{case})\\
  & = \text{case } x'  
    \left\{ \begin{aligned}
    \inl_{\typeI}(y) \Rightarrow s\,[\text{inl}(y)/x] ;  
    \inl_{\typeI}(z) \Rightarrow s\,[\text{inr}(z)/x]
  \end{aligned}  \right\} = s & (\text{\autoref{eq:beta_red}},\eta_{case},\alpha)
\end{split}
\end{equation*}
Finally, it is straightforward that $ t=s \implies \lambda x.t = \lambda x.s$.

Now assume that 
\[
\left\{
\begin{aligned}
&(\lambda x.t)\, \text{inl}(y) =_{\epsilon_1} (\lambda x.s)\, \text{inl}(y) \\
&(\lambda x.t)\, \text{inr}(z) =_{\epsilon_2}(\lambda x.s)\, \text{inr}(z)
\end{aligned}
\right.
\implies \lambda x.t = \lambda x.s
\]
Following the same reasoning as before and applying the metric equation for condicionals we obtain:
\begin{equation*}
\begin{split}
  &t =  \text{case } x'  
    \left\{ \begin{aligned}
    \inl_{\typeI}(y) \Rightarrow t\,[\text{inl}(y)/x] ;  
    \inl_{\typeI}(z) \Rightarrow t\,[\text{inr}(z)/x]
  \end{aligned}  \right\} & (\alpha,\eta_{case})\\
  &=_{\max \{\epsilon_1,\epsilon_2\}} \text{case } x'  
    \left\{ \begin{aligned}
    \inl_{\typeI}(y) \Rightarrow s\,[\text{inl}(y)/x] ;  
    \inl_{\typeI}(z) \Rightarrow s\,[\text{inr}(z)/x]
  \end{aligned}  \right\} = s & (\text{\autoref{fig:metric conditionals}},\eta_{case},\alpha)
\end{split}
\end{equation*}
As a result, given our metric deductive system we obtain $\lambda x.t =_{\max \{\epsilon_1,\epsilon_2\}} \lambda x.s$.



\section{Some remarks on the metric equation chosen}

\todo[inline,size=\normalsize]{Falta exemplo de quantale em que não yemos distributividade}

One might wonder why, in the definition of the metric, we chose to use the expression \( q + \sup\{r, s\} \) rather than \( \sup\{q + r, q + s\} \), given that these two expressions are, in fact, equal in the metric quantale. More generally, this equality corresponds to the identity
\[
q \otimes \inf\{r, s\} = \inf\{q \otimes r, q \otimes s\},
\]
for \( r, s \in L \), where \( L \) denotes the underlying lattice of the quantale.

However, this identity does \emph{not} hold in arbitrary quantales~\textbf{[CONTRAEXAMPLE NEEDED]}. Moreover, as becomes clear in the soundness proof, it is the expression \( q \otimes \sup\{r, s\} \) that arises naturally. 