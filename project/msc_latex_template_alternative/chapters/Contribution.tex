\chapter{Metric $\lambda$-calculus with conditionals} \label{ch:conditionals}

As noted in \autoref{subsec:interlude_bool}, a metric equation for the conditional statement would be extremely helpful for reasoning about approximate equivalence within the $\lambda$-calculus with conditionals. In this chapter, we address this gap by introducing such an equation and proving both its soundness and completeness. We present the category of metric spaces as an example of a setting suitable for reasoning about approximate equivalence in this context and illustrate the usefulness of the introduced equation through two small examples: reasoning about approximate equivalence between boolean terms (i.e., terms of type $\typeI \oplus \typeI$) and the extensionality of copairing. We conclude the chapter with a brief discussion on the suitability of the chosen metric equation.


\todo[inline,size=\normalsize]{ (Tentar) Meter tudo direito de soundness and completeness }

\todo[inline,size=\normalsize]{ Exemplo MET : referir Example 3.13. que diz de MET é symmetric monoidal met-enriched}

\todo[inline,size=\normalsize]{ Exemplo booleano }


\todo[inline,size=\normalsize]{ Exemplo extensionalidade do copairing}

\todo[inline,size=\normalsize]{ Discussao sobre max(q+r, q+s) != max(q+r, q+s) }



\section{Syntax}

The metric equation for conditionals is presented in \autoref{fig:metric conditionals}. 

\begin{figure}[H]
  \begin{equation*}
  \begin{aligned}
  &
  &
  %
  \begin{prooftree}
      \hypo{ v =_{q} v' }
      \hypo{w=_{r} w'}
      \hypo{u=_{s}u'}
      \infer3[]{\text{ case } v \,   \{\text{inl} (x) \Rightarrow w ; \, \text{inr} (y) \Rightarrow u\} =_{q+\sup{\{ r, s \}}} \text{ case } v' \,  \{\text{inl} (x) \Rightarrow w' ; \,\text{inr} (y) \Rightarrow u'\} }
  \end{prooftree}
  %
  \\[10pt]
  \end{aligned}
  \end{equation*}
  \caption{Metric equation for condicionals}
  \label{fig:metric conditionals}
\end{figure}


Intuitively, this equation states that the maximum distance between the two conditional statements is determined the:

\begin{itemize}
    \item The larger of the maximum distances between the corresponding branches, \ie, the program pairs $(w, w')$ and $(u, u')$; and
    \item The maximum distance between the terms $v$ and $v'$ on which their execution depends.
\end{itemize}


Recall \autoref{def:metric_lambda_theory}. Now, we extend \( Th(Ax) \) to denote the smallest class that contains \( Ax \) and is closed under the rules presented in \autoref{fig:equations-linear-lambda}, \autoref{fig:metric deductive system} and \autoref{fig:metric conditionals}.



\section{Soundeness and Completeness}





%For the metric quantale, the way-below relation corresponds to the strictly greater relation with ∞ > ∞, and a basis for the underlying lattice that satisfies the conditions above is the set of extended non-negative rational numbers.

\todo[inline,size=\normalsize]{Way bellow and lattice stuff (complete lattice, distributive latice, R+ é uma distributive lattice)}

\todo[inline,size=\normalsize]{Crole: tem as defs e diz que "Any poset which is a chain is distributive." tendo em conta que "A subset C of a preorder X is called a chain if for every
x, y G C we have x < y or y < x."}

\begin{definition}
	Consider a complete lattice $L$.  For every $x, y \in L$ we say that
	$y$ is \emph{way-below} $x$ (in symbols, $y \ll x$) if for every
	subset $X \subseteq L$ whenever $x \leq \bigvee X$ there exists a
	\emph{finite} subset $A \subseteq X$ such that $y \leq \bigvee A$.
	The lattice $L$ is called \emph{continuous} iff for every $x \in L$,
	\begin{flalign*}
		x = \bigvee \{ y  \mid y \in L\ \text{and} \ y \ll x \}
	\end{flalign*}
\end{definition}

\begin{definition}
	Let $L$ be a complete lattice. A \emph{basis} $B$ of $L$ is a subset
	$B \subseteq L$ such that for every $x \in L$ the set
	$B \cap \{ y \mid y \in L\ \text{and} \ y \ll x \}$ is directed and
	has $x$ as the least upper bound.
\end{definition}

\section{A small example}



