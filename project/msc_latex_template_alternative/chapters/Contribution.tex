\chapter{Contribution}

Main result(s) and their scientific evidence

\section{Introduction}

\section{Summary}

%ToDo: Extending the quantum model with conditionals -> Talk about the model (operadores como morfismos, a nossa estrutura algebrica é o espaços vetorias); CPTP model; measurents and the diamond norm é menor ou igual a um; examples (qualtum walk)
%\todo[inline]{The original todo note withouth changed colours.\newline Here's another line.}
%ToDo: Discard operations -> The operator should de unique (but there is for instance trace and the null operator_maps everything to zero) and is unique bacausce only trace is trace preserving; example: mallicious attack discard a qubit ou outside of quantum teleportation : setting a qubit do 0, by discarding it and initiating a new one

\section{Integration of conditionals}

The notion of approximate equivalence for quantum programming explored in [\cite{dahlqvist2022syntactic}] does not encompass classical control flow. As a result, preliminary work based on [\cite{crole1993categories,selinger2013lecture}]   has been undertaken to address the integration of conditionals. 

\subsection{Integration of conditionals}

The term formation rules for conditionals are depicted in
\autoref{fig:typing_rules_cond}. 

\begin{figure} [H]
\begin{equation*}
\begin{split}
\begin{aligned}
& \hspace{55pt}
\begin{minipage}[t]{0.3\textwidth}
$\begin{array}{c}
     \Gamma \triangleright v: \mathbb{A} \\
    \hline
   \Gamma \triangleright \text{inl}(v):  \mathbb{A} \oplus \mathbb{B}
\end{array}
$
\end{minipage}
\hspace{-38pt}
\text{(inl)} 
 \hspace{20pt}
\begin{minipage}[t]{0.3\textwidth}
$\begin{array}{c}
      \Gamma \triangleright v:  \mathbb{B} \\
    \hline
   \Gamma \triangleright \text{inr}(v): \mathbb{A} \oplus \mathbb{B}
\end{array}
$ \end{minipage} 
\hspace{-35pt} \text{(inr)} \\
&\hspace{15pt}
\begin{minipage}[t]{0.3\textwidth}
$\begin{array}{c}
     \Gamma\triangleright v: \mathbb{A} \oplus \mathbb{B} \quad \Delta, x: \mathbb{A} \triangleright w: \mathbb{C} \quad \Delta, y: \mathbb{B}  \triangleright u : \mathbb{C}   \quad E \in \text{Sf}(\Gamma;\Delta)  \\
    \hline
   E \triangleright \text{ cond } v \hspace{2pt} \{\text{inl} (x) \Rightarrow w ; \hspace{1pt} \text{inr} (y) \Rightarrow u\}: \mathbb{C} 
\end{array}
$
\end{minipage}
\hspace{200pt}
\text{(case)} 
\end{aligned}
\end{split}
\end{equation*}
\caption{Term formation rules for conditionals}
\label{fig:typing_rules_cond}
\end{figure}
Considering  $v \in V$, $w \in W$, and $u \in U$ where $V, W, U$ represent vector spaces, $\textsc{Il}_{V}: V \xrightarrow{} V\oplus W$, denotes the left injection operator, defined as $\textsc{Il}_{V}= v \mapsto (v,0) $; $\textsc{Ir}_{V}: V \xrightarrow{} W \oplus V$, denotes the right injection operator, defined as $\textsc{Ir}_{V}= v \mapsto (0,v) $; and $\text{dist}_{V, W,U}: V \otimes  \left(W \oplus U\right) \xrightarrow{} \left(V \otimes W\right) \oplus \left(V \otimes U\right)$, denotes the distributive property of the tensor product over the direct sum, defined as $\text{dist}_{V, W,U} =  v \otimes  \left(w, u\right) \mapsto \left(v \otimes w, v \otimes u\right)$. The subscripts in these operators will be omitted unless ambiguity arises. Moreover, the operation \text{either} corresponds to:
\begin{figure} [H]
\begin{equation}
\begin{split}
\begin{aligned}
\hspace{95pt}&
\begin{minipage}[t]{0.3\textwidth}
$\begin{array}{c}
     V  \xrightarrow{} U  \\
      W \xrightarrow{} U  \\
    \hline
  [T,S]: V \oplus W \xrightarrow{} U
\end{array}
$
\end{minipage} \\
\hspace{95pt}&
\begin{minipage}[t]{0.3\textwidth}
$\begin{array}{c}
  [T,S] = (v,w) \mapsto T(v)+S(w) 
\end{array}
$
\end{minipage}
\end{aligned}
\end{split}
\end{equation}
\label{fig:either}
\end{figure}

The interpretation of conditionals is illustrated in \autoref{fig:denotational_sem cond}.

\begin{figure} [H]
\begin{equation}
\begin{split}
\begin{aligned}
&\hspace{-80pt} 
 \begin{minipage}[t]{0.3\textwidth}
$\begin{array}{c} 
     [\![\Gamma \triangleright v: \mathbb{A}]\!] = m   \\
    \hline
  [\![ \Gamma \triangleright \text{inl} (v):  \mathbb{A} \oplus \mathbb{B}  ]\!] = \textsc{Il}  \cdot m
\end{array}
$ \end{minipage}
\hspace{30pt} 
\begin{minipage}[t]{0.3\textwidth}
$\begin{array}{c}
     [\![\Gamma \triangleright v:\mathbb{B} ]\!]  = m  \\
    \hline
   [\![\Gamma \triangleright \text{inr} (v):  \mathbb{A} \oplus \mathbb{B}]\!]\!] = \textsc{Ir} \cdot m
\end{array}
$
\end{minipage}\\
\hspace{-25pt}
 \begin{minipage}[t]{0.3\textwidth}
$\begin{array}{c} 
    [\![\Gamma\triangleright v: \mathbb{A} \oplus \mathbb{B} ]\!] = b \quad [\![\Delta, x:\mathbb{A} \triangleright w: \mathbb{C} ]\!] = p  \quad  [\![\Delta,x:\mathbb{B} \triangleright w_{2}: \mathbb{C} ]\!] = q    \quad E \in \text{Sf}(\Gamma;\Delta)  \\
    \hline
  [\![E \triangleright \text{ case } v \hspace{2pt}  \{\text{inl} (x) \Rightarrow w ; \hspace{1pt} \text{inr} (y) \Rightarrow u\}: \mathbb{C} ]\!] =   \text{either}(p,q) \cdot \text{dist} \cdot \text{sw} \cdot (b \otimes \text{id}) \cdot \text{sp}_{\Gamma;\Delta} \cdot \text{sh}_{E}
\end{array}
$ \end{minipage}
\end{aligned}
\end{split}
\end{equation}
\caption{Judgment interpretation for conditionals}
\label{fig:denotational_sem cond}
\end{figure}

\paragraph{Proof} In order to validate the judgment interpretation for conditionals, it is necessary to demonstrate its correctness.

For the booleans: 
\begin{equation} \label{eq:proof_bool}
 \begin{aligned} 
    \hspace{120pt}&  [\![\Gamma ]\!]   \xrightarrow{\hspace{5pt}m\hspace{5pt}} [\![\mathbb{A} ]\!] \xrightarrow{\hspace{6pt}\textsc{Il}\hspace{6pt}} [\![\mathbb{A} \oplus \mathbb{B}]\!] \\ 
     &[\![\Gamma ]\!]   \xrightarrow{\hspace{5pt}m\hspace{5pt}} [\![\mathbb{B} ]\!] \xrightarrow{\hspace{6pt}\textsc{Ir}\hspace{6pt}} [\![\mathbb{A} \oplus \mathbb{B}]\!]
\end{aligned}   
\end{equation}
Now, for the conditional statement:
\begin{equation} \label{eq:proof_bool_2}
 \begin{aligned} 
    [\![E]\!] & \xrightarrow{\hspace{2pt}\text{sh}_{E}\hspace{2pt}}   [\![\Gamma,\Delta ]\!]   \xrightarrow{\hspace{1pt}\text{sp}_{\Gamma;\Delta}\hspace{1pt}}  [\![\Gamma ]\!] \otimes [\![\Delta ]\!] \xrightarrow{ b \hspace{1pt} \otimes \hspace{1pt} \text{id}} ([\![\mathbb{A} ]\!] \oplus [\![\mathbb{B} ]\!]) \otimes [\![\Delta ]\!] \xrightarrow{\hspace{2pt}\text{sw}\hspace{2pt}}  [\![\Delta ]\!] \otimes ([\![\mathbb{A} ]\!] \oplus [\![\mathbb{B} ]\!])  \\
    & \xrightarrow{\hspace{3pt}\text{dist}\hspace{3pt}} ([\![\Delta ]\!] \otimes [\![\mathbb{A} ]\!]  ) \oplus (  [\![\Delta ]\!] \otimes [\![\mathbb{B} ]\!] ) \xrightarrow{\hspace{1pt}\text{either}(p,q)\hspace{1pt}} [\![\mathbb{C} ]\!]
\end{aligned}   
\end{equation}


The quantum lambda calculus with conditionals is illustrated with an example —the quantum teleportation protocol— in \autoref{sec:teleport}.


The metric equations for conditionals are presented in \autoref{fig:metric conditionals}. Note that the first two equations are redundant.
\begin{figure} [H]
\begin{equation*}
\begin{split}
\begin{aligned}
 &
\begin{minipage}[t]{0.3\textwidth}
$\begin{array}{c}
  v =_{q} w \\
    \hline
   \text{inl}(v) =_{q} \text{inl}(w)
\end{array}
$
\end{minipage}
\hspace{-30pt}
\begin{minipage}[t]{0.3\textwidth}
$\begin{array}{c}
   v =_{q} w \\
    \hline
   \text{inr}(v) =_{q} \text{inr}(w)
\end{array}
$ \end{minipage} \\
\hspace{-30pt}
&
\begin{minipage}[t]{0.3\textwidth}
$\begin{array}{c}
   v =_{q} v' \quad w=_{r} w' \quad u=_{s}u'   \\
    \hline
  \text{ case } v \hspace{2pt}  \{\text{inl} (x) \Rightarrow w ; \hspace{1pt} \text{inr} (y) \Rightarrow u\} =_{q+\text{max}(r, s )} \text{ case } v' \hspace{2pt}  \{\text{inl} (x) \Rightarrow w' ; \hspace{1pt} \text{inr} (y) \Rightarrow u'\} 
\end{array}
$ \end{minipage}
\end{aligned}
\end{split}
\end{equation*}
\caption{Metric equational system for condicionals}
\label{fig:metric conditionals}
\end{figure}

\paragraph{Proof} In order to validate the metric equational system for conditionals, it is necessary to demonstrate its correctness.

The diamond norm is a particular instance of the operator norm. The operator norm [\cite{guide2006infinite}] for a super-operator $E$ is defined as:
\begin{equation} \label{eq:op_norm}
  \lVert E \rVert_{\sigma} = \text{sup} \{ \lVert E(v) \rVert \hspace{2pt} | \hspace{2pt} \lVert v \rVert = 1 \}
\end{equation}

\vspace{15pt}

For the \textbf{injections}:

Firstly, it is necessary to prove that the identity operator $I$ has a norm equal to 1.
\begin{lemma} \label{lemid}
  $ \lVert I \rVert_{\sigma} = 1   $
\end{lemma}

\textit{Proof.} \quad Using the definition of operator norm in \autoref{eq:op_norm}, it follows that:
\begin{equation} 
\begin{split}
  \lVert I \rVert_{\sigma} = \text{sup} \{\lVert I (v) \rVert \hspace{2pt} \vert \hspace{2pt}  \lVert v\rVert =1 \} = \text{sup} \{\lVert v \rVert \hspace{2pt} \vert \hspace{2pt}  \lVert v\rVert =1 \} = 1
\end{split}
\end{equation}

\vspace{10pt}

Thereafter, it is imperative to show that the injection operators $\textsc{Il}$ and $\textsc{Ir}$ are have a norm equal to 1.

\begin{lemma} \label{lemil}
  $ \lVert \textsc{Il} \rVert_{\sigma} = 1   $
\end{lemma}

\begin{lemma} \label{lemir}
  $ \lVert \textsc{Ir} \rVert_{\sigma} = 1   $
\end{lemma} 

\textit{Proof.} \quad Employing the definition of operator norm as defined in \autoref{eq:op_norm}, it ensues that:
\begin{equation} 
\begin{split}
  \lVert \textsc{Il} \rVert_{\sigma} &= \text{sup} \{\lVert \textsc{Il} (v) \rVert \hspace{2pt} \vert \hspace{2pt}  \lVert v\rVert =1 \} = \text{sup} \{\lVert (v,0) \rVert \hspace{2pt} \vert \hspace{2pt}  \lVert v\rVert =1 \} = \text{sup} \{\lVert v \rVert + \lVert 0 \rVert  \hspace{2pt} \vert \hspace{2pt}  \lVert v\rVert =1 \} \\
  & = \text{sup} \{\lVert v \rVert \hspace{2pt} + 0    \hspace{2pt}  \vert \lVert v\rVert =1 \} \hspace{160 pt} \text{ \{Positive definiteness\}} \\
  & = \text{sup} \{\lVert v \rVert \hspace{2pt} \vert \hspace{2pt}  \lVert v\rVert =1 \} = 1
\end{split}
\end{equation}

The proof for \autoref{lemir} is analogous to the proof for \autoref{lemil}.
\begin{equation} 
  \begin{split}
    \lVert \textsc{Ir} \rVert_{\sigma} &= \text{sup} \{\lVert \textsc{Ir} (v) \rVert \hspace{2pt} \vert \hspace{2pt}  \lVert v\rVert =1 \} = \text{sup} \{\lVert (0,v) \rVert \hspace{2pt} \vert \hspace{2pt}  \lVert v\rVert =1 \} = \text{sup} \{ \lVert 0 \rVert +\lVert v \rVert   \hspace{2pt} \vert \hspace{2pt}  \lVert v\rVert =1 \} \\
    & = \text{sup} \{0+\lVert v \rVert \hspace{2pt}     \hspace{2pt}  \vert \lVert v\rVert =1 \} \hspace{160 pt} \text{ \{Positive definiteness\}} \\
    & = \text{sup} \{\lVert v \rVert \hspace{2pt} \vert \hspace{2pt}  \lVert v\rVert =1 \} = 1
  \end{split}
  \end{equation}

Futhermore, given the submultiplicative property of the operator norm, for any super-operators $P$ and $Q$,where $\lVert P \rVert_{\sigma} =1  $ the following holds:
\begin{lemma}\label{lemleq}
  $\lVert PQ \rVert_{\sigma} \leq  \lVert Q \rVert_{\sigma}, \quad \lVert P \rVert_{\sigma}  =1 $ 
\end{lemma}

Using these properties it is possible to prove the validity of the metric equations for the injections. Demonstrating the correctness of the metric equations for the injections is equivalent to proving that for any  non‑negative rational $q$ and super-operators $v$ and $w$ such that $d(v,w) \leq q$, where  $d(v,w)$ represents the distance between $v$ and $w$ the following holds:

\begin{theorem} \label{theoremil}
  $d(\textsc{Il}(v),\textsc{Il} (w)) \leq q$
\end{theorem}
\begin{theorem} \label{theoremir}
  $d(\textsc{Ir}(v),\textsc{Ir} (w)) \leq q$
\end{theorem}
\vspace{10pt}
\textit{Proof.} \quad In the quantum paradigm, the distance between two super-operators $E$ and $E'$ corresponds to the diamond norm between $E$ and $E'$. Therefore,
\begin{equation}
\begin{split}
  d(v,w) \leq q \Leftrightarrow \lVert v \otimes I - w \otimes I \rVert_{\sigma} \leq q
\end{split}
\end{equation}

As a result, to prove that $d(\textsc{Il}(v),\textsc{Il} (w)) \leq q$, it suffices to show that:
\begin{align}
  \lVert \textsc{Il}\otimes I (v \otimes I)-\textsc{Il} \otimes I (w \otimes I)\rVert_{\sigma} \leq \lVert v \otimes I - w \otimes I \rVert_{\sigma} \\
  \lVert \textsc{Ir}\otimes I (v \otimes I)-\textsc{Ir} \otimes I (w \otimes I)\rVert_{\sigma} \leq \lVert v \otimes I - w \otimes I \rVert_{\sigma} 
\end{align}
Given that $\textsc{Il}$ and $\textsc{Ir}$ possess a norm equal to 1, as established by Lemmas \ref{lemil} and \ref{lemir} respectively, and considering the multiplicative property of the operator norm with respect to tensor products alongside the fact that the identity operator also exhibits a norm equal to 1, as demonstrated in  \autoref{lemid}, it follows that both $\lVert \textsc{Il} \otimes I \rVert_{\sigma}$ and $\lVert \textsc{Ir} \otimes I \rVert_{\sigma}$ are equal to one 1. Hence, by \autoref{lemleq},
\begin{align}
   \lVert \textsc{Il}\otimes I (v \otimes I)-\textsc{Il} \otimes I (w \otimes I)\rVert_{\sigma}=\lVert \textsc{Il}\otimes I (v \otimes I-w \otimes I)\rVert_{\sigma} \leq \lVert v \otimes I - w \otimes I \rVert_{\sigma} \\
   \lVert \textsc{Ir}\otimes I (v \otimes I)-\textsc{Ir} \otimes I (w \otimes I)\rVert_{\sigma}=\lVert \textsc{Ir}\otimes I (v \otimes I-w \otimes I)\rVert_{\sigma} \leq \lVert v \otimes I - w \otimes I \rVert_{\sigma}
\end{align}

\vspace{10pt}

Now, regarding the metric equation for the \textbf{conditional statement}, before validating its correctness, it is necessary to prove a few intermediate results. 

The first step is to demonstrate that for any super-operators $P$ and $Q$ the following holds:
\begin{lemma}\label{lem1}
  $\lVert [P,Q] \rVert_{\sigma} \leq \max \{ \lVert P \rVert_{\sigma}, \lVert Q \rVert_{\sigma} \}$
\end{lemma}



$\textit{Proof.}$ \quad Employing the definition of the operator norm in \autoref{eq:op_norm}, it follows that:
\begin{equation} \label{eq:cond_opnorm2}
  \begin{split}
  &\text{sup}{\{ \lVert [P,Q] (v) \rVert  \hspace{2pt} |  \hspace{2pt}  \lVert v \rVert=1  \}}  \leq \text{max} \{  \text{sup} \{ \lVert P (w) \rVert  \hspace{2pt} |  \hspace{2pt}  \lVert w \rVert =1 \}, \text{sup} \{\lVert Q (u) \rVert  \hspace{2pt} |  \hspace{2pt}  \lVert u \rVert=1  \} \} \\
  & = \text{sup}{\{ \lVert [P,Q] (w+u) \rVert  \hspace{2pt} |  \hspace{2pt}  \lVert w+u \rVert=1  \}} \leq \text{max} \{  \text{sup} \{ \lVert P (w) \rVert  \hspace{2pt} |  \hspace{2pt}  \lVert w \rVert = 1, \lVert Q (u) \rVert  \hspace{2pt} |  \hspace{2pt}  \lVert u \rVert=1  \} \} \\
  & =  \text{sup}{\{ \lVert P (w) + Q (u) \rVert  \hspace{2pt} |  \hspace{2pt}  \lVert w+u \rVert=1  \}} \leq \text{max} \{  \text{sup} \{ \lVert P (w) \rVert  \hspace{2pt} |  \hspace{2pt}  \lVert w \rVert =1, \lVert Q (u) \rVert  \hspace{2pt} |  \hspace{2pt}  \lVert u \rVert=1  \} \} \\
  &  =  \text{sup}{\{ \lVert P (w) + Q (u) \rVert  \hspace{2pt} |  \hspace{2pt}  \lVert w+u \rVert=1  \}} \leq \text{sup} \{  \text{max} \{ \lVert P (w) \rVert  \hspace{2pt} |  \hspace{2pt}  \lVert w \rVert =1, \lVert Q (u) \rVert  \hspace{2pt} |  \hspace{2pt}  \lVert u \rVert=1  \} \} \\
\end{split}
\end{equation}

Therefore, by the triangle inequality, proving the inequality in \autoref{eq:cond_opnorm3} suffices to establish  \autoref{lem1}.
\begin{equation} \label{eq:cond_opnorm3}
  \begin{split}
  \text{sup}{\{ \lVert P (w)  \rVert + \lVert Q (u)  \rVert  \hspace{2pt} |  \hspace{2pt}  \lVert w+u \rVert_{1}=1  \}} \leq \text{sup} \{  \text{max} \{ \lVert P (w) \rVert  \hspace{2pt} |  \hspace{2pt}  \lVert w  \rVert =1, \lVert Q (u) \rVert  \hspace{2pt} |  \hspace{2pt}  \lVert u \rVert=1  \} \} \\
  \end{split}
\end{equation}


This can be rewritten as:

\begin{equation} 
  \begin{split}
  \lVert w + u   \rVert = 1 \wedge \{ \lVert P (w)  \rVert + \lVert Q (u)  \rVert  \hspace{2pt} |  \hspace{2pt}  \lVert w+u \rVert=1  \}  \leq \text{max}   \left\{ \dfrac{1}{\lVert w \rVert} \lVert P (w) \rVert  \hspace{2pt},  \dfrac{1}{\lVert u \rVert} \lVert Q (u) \rVert   \right\}
\end{split}
\end{equation}

As a result,
\begin{equation} 
  \begin{split}
  \lVert w + u   \rVert = 1 \wedge \text{sup}{\{ \lVert P (w)  \rVert + \lVert Q (u)  \rVert  \hspace{2pt} |  \hspace{2pt}  \lVert w+u \rVert_{1}  \}}  \leq \text{max}   \left\{  \left\lVert P \left( \dfrac{1}{\lVert w \rVert} w \right) \right\rVert  \hspace{2pt},  \left\lVert Q \left( \dfrac{1}{\lVert u \rVert} u \right) \right\rVert   \right\}
\end{split}
\end{equation}

This is equivalent to demonstrating that for $a+b=1$,
\begin{equation} 
\begin{split}
\hspace{110 pt}
    x + y  \leq  \max \left\{   \dfrac{1}{a}x  ,   \dfrac{1}{b} y   \right\} \\
\end{split}
\end{equation}

This is done by arguing by \textit{reductio ad absurdum}, \textit{i.e.}, supposing otherwise leads to a contradiction:
\begin{equation} 
\begin{split} 
    \hspace{90pt}&
     x + y  >  \max \left\{   \dfrac{1}{a}x  ,   \dfrac{1}{b} y   \right\} \\
    & \Rightarrow  x + y > \dfrac{1}{a}x  \wedge x + y > \dfrac{1}{b}y \\
    & \Rightarrow  a (x + y) > x  \wedge b (x + y)> y \\
    & \Rightarrow  a x + a y > x  \wedge b x + by > y \\
    & \Rightarrow  a x + a y > x  \wedge (1-a) x + (1-a)y > y\\
    & \Rightarrow  a x + a y > x  \wedge x-ax + y -ay > y\\
    & \Rightarrow  x < a x + a y   \wedge x > a x + a y  \\
\end{split}
\end{equation}

\vspace{10pt}

Subsequently, it is imperative to prove that:
\begin{lemma}\label{lemiso}
  $ i= [\textsc{Il} \otimes I, \textsc{Ir} \otimes I ]$ \text{is an isomorphism}.
\end{lemma}

\textit{Proof.} \quad The proof is as follows:

For any vector spaces $V$, $W$, and $U$, $i: (V \otimes U) \oplus (W \otimes U) \xrightarrow{} (V  \oplus W) \otimes U $. If $V$ has dimension $m$, $W$ has dimension $n$, and $U$ has dimension $o$, then the space $(V \otimes U) \oplus (W \otimes U) $ has dimension $mo+no=(m+n)\cdot o$. Similarly, the space $(V\oplus W) \otimes U$ has dimension $(m+n)\cdot o$. Hence, the spaces have the same dimension. Given that spaces with the same dimension are isomorphic [\cite{hefferon2006linear}], it follows that $i$ is an isomorphism.

\vspace{10pt}

Next, it is necessary to demonstrate that for any operators $P$ and $Q$, the identity operator $I$, and an isomorphism $i=[\textsc{Il} \otimes I, \textsc{Ir} \otimes I ]$ the following holds:

\begin{lemma}\label{lem2}
  $( [P,Q] \otimes I) \cdot  i  = [P \otimes I, Q \otimes I]$
\end{lemma}

Which is equivalent to showing that for any vector spaces $V$, $W$, $U$, and $Z$  and super-operators $P: V \xrightarrow{} Z$, $Q: W \xrightarrow{} Z$, and $I: U \xrightarrow{} U$, the following diagram holds:

\vspace{10pt}


\begin{tikzpicture}
  \matrix (m) [matrix of math nodes,row sep=4em,column sep=7em,minimum width=2em]
  {
    V \otimes U \oplus W \otimes U & (V  \oplus W) \otimes U \\
     Z \otimes U \\
  };
  \path[-stealth]
    (m-1-1) edge node [left] {$[P \otimes I, Q \otimes I]$} (m-2-1)
    (m-1-1) edge node [above] {$i$} (m-1-2)
    (m-1-2) edge node [right=0.2cm] {$[P,Q] \otimes I$} (m-2-1);
\end{tikzpicture}


\vspace{10pt}

\textit{Proof.} \quad The proof is straightforward:
\begin{equation}
\begin{split}
    & ( [P,Q] \otimes I) \cdot  [\textsc{Il} \otimes I, \textsc{Ir} \otimes I ]  \\
    &=  [([P,Q] \otimes I) \cdot (\textsc{Il} \otimes I),([P,Q] \otimes I) \cdot (\textsc{Ir} \otimes I) ]\\
    &=  [P \otimes I, Q \otimes I]
\end{split}
\end{equation}

\vspace{15pt}

Furhtermore, it is imperative to show that the following relation holds:

\begin{lemma}\label{lemi-1}
  $ [P \otimes I, Q \otimes I] \cdot  i^{-1}  = [P,Q] \otimes I$
\end{lemma}

Demonstrating this is equivalent to establishing that for any vector spaces $V$, $W$, $U$, and $Z$, and super-operators $P: V \xrightarrow{} Z$, $Q: W \xrightarrow{} Z$, and $I: U \xrightarrow{} U$, the following diagram commutes:

\vspace{10pt}

\begin{tikzpicture}
  \matrix (m) [matrix of math nodes,row sep=4em,column sep=7em,minimum width=2em]
  {
    V \otimes U \oplus W \otimes U & (V  \oplus W) \otimes U \\
     Z \otimes U \\
  };
  \path[-stealth]
    (m-1-1) edge node [left] {$[P \otimes I, Q \otimes I]$} (m-2-1)
    (m-1-2) edge node [above] {$i^{-1}$} (m-1-1)
    (m-1-2) edge node [right=0.2cm] {$[P,Q] \otimes I$} (m-2-1);
\end{tikzpicture}


\textit{Proof.} \quad The proof is as follows:
\begin{equation}
\begin{split}
    & ( [P,Q] \otimes I) \cdot  i  = [P \otimes I, Q \otimes I]  \hspace{100pt} & \text{\{\autoref{lem2}\}} \\
    \Leftrightarrow &  \hspace{2pt} ( [P,Q] \otimes I) \cdot  i \cdot i^{-1} = [P \otimes I, Q \otimes I] \cdot  i^{-1}\\
    \Leftrightarrow &  \hspace{2pt} ( [P,Q] \otimes I)  = [P \otimes I, Q \otimes I] \cdot  i^{-1}  &\text{\{\autoref{lemiso}\}} \\
\end{split}
\end{equation}

\vspace{10pt}
With \autoref{lem2} and \autoref{lemi-1}, it has been proved that the diagram below is valid:
\vspace{5pt}

\begin{tikzpicture}
  \matrix (m) [matrix of math nodes,row sep=4em,column sep=7em,minimum width=2em]
  {
    V \otimes U \oplus W \otimes U & (V  \oplus W) \otimes U \\
     Z \otimes U \\
  };
  \path[-stealth]
    (m-1-1) edge node [left] {$[P \otimes I, Q \otimes I]$} (m-2-1)
    edge[bend left=5] node [above] {$i$}  (m-1-2) % Adjusted minimum width
    (m-1-2) edge node [right=0.5cm] {$[P,Q] \otimes I$} (m-2-1)
    (m-1-2) edge[bend right=-5] node [below] {$i^{-1}$} (m-1-1); % Added the label to the arrow
\end{tikzpicture}

\vspace{10pt}




%Next, it is necessary to demonstrate that the coproduct of two super-operators $P$ and $Q$ has a norm equal to 1.
%\begin{lemma} \label{lemeither}
  %$  \lVert [P, Q]  \rVert_{\sigma} = 1   $
%\end{lemma}

%\textit{Proof.} \quad Utilizing the definition of the operator norm as defined in Equation \ref{eq:op_norm}, it follows that:
%\begin{equation} 
  %\begin{split}
    %\lVert [P, Q]  \rVert_{\sigma}  \\
  %\end{split}
  %\end{equation}
%\vspace{10pt}

Now, it is possivel to prove that $i$ has a norm equal to 1.

\begin{lemma} \label{lem3}
  $  \lVert i\rVert_{\sigma} = 1 $
\end{lemma}

\vspace{10pt}

\textit{Proof.} \quad Employing the definition of the operator norm as defined in \autoref{eq:op_norm}, it follows that:
\begin{equation}
  \begin{split}
      & \hspace{3pt} \lVert [\textsc{Il} \otimes I, \textsc{Ir} \otimes I ]  \rVert_{}  \\
      &= \text{max} \{ \lVert [\textsc{Il} \otimes I, \textsc{Ir} \otimes I ] (A) \rVert \hspace{2pt} \vert \hspace{2pt}  \lVert A\rVert =1   \} \\
      &= \text{max} \left\{ \left\lVert [\textsc{Il} \otimes I, \textsc{Ir} \otimes I ] \left(\sum_{i} v_i \otimes u_i,\sum_{i} w_i \otimes u_i  \right) \right\rVert \hspace{2pt} \Bigg\vert \hspace{2pt}  \left\lVert \left(\sum_{i} v_i \otimes u_i,\sum_{i} w_i \otimes u_i  \right) \right\rVert =1    \right\} \\
      & =\text{max} \left\{ \left\lVert \textsc{Il} \otimes I \left(\sum_{i} v_i \otimes u_i  \right)  +  \textsc{Ir} \otimes I \left(\sum_{i} w_i \otimes u_i  \right) \right\rVert \hspace{2pt} \Bigg\vert \hspace{2pt}  \left\lVert \left(\sum_{i} v_i \otimes u_i,\sum_{i} w_i \otimes u_i  \right) \right\rVert =1    \right\} \\
      & = \text{max} \Bigg\{ \left\lVert \textsc{Il} \left(\sum_{i} v_i  \right) \otimes I \left(\sum_{i} u_i  \right) + \textsc{Ir} \left(\sum_{i} w_i \right)\otimes I \left(\sum_{i} u_i  \right) \right\rVert \hspace{2pt} \\
      & \hspace{50pt}\Bigg\vert \hspace{2pt}  \left\lVert \left(\sum_{i} v_i \otimes u_i,\sum_{i} w_i \otimes u_i  \right) \right\rVert =1    \Bigg\} \\
      &= \text{max} \Bigg\{ \left\lVert \left(\sum_{i} v_i,0  \right) \otimes \sum_{i} u_i +  \left(0,\sum_{i} w_i \right) \otimes \sum_{i} u_i   \right\rVert  \hspace{2pt} \Bigg\vert \hspace{2pt}  \left\lVert \left(\sum_{i} v_i \otimes u_i,\sum_{i} w_i \otimes u_i  \right) \right\rVert =1    \Bigg\} \\
      &= \text{max} \Bigg\{ \left\lVert \left(\sum_{i} v_i \otimes  u_i ,0  \right) + \left(0,\sum_{i} w_i \otimes u_i  \right)   \right\rVert  \hspace{2pt} \Bigg\vert \hspace{2pt}  \left\lVert \left(\sum_{i} v_i \otimes u_i,\sum_{i} w_i \otimes u_i  \right) \right\rVert =1    \Bigg\} \\
      &= \text{max} \Bigg\{ \left\lVert \left(\sum_{i} v_i \otimes  u_i ,\sum_{i} w_i \otimes u_i   \right)    \right\rVert  \hspace{2pt} \Bigg\vert \hspace{2pt}  \left\lVert \left(\sum_{i} v_i \otimes u_i,\sum_{i} w_i \otimes u_i  \right) \right\rVert =1    \Bigg\} \\
      &=1
  \end{split}
  \end{equation}

%Hence, by \autoref{lem1}, it follows that:
%\begin{equation} 
  %\begin{split}
    %\lVert [\textsc{Il} \otimes I, \textsc{Ir} \otimes I]  \rVert_{\sigma} &\leq \text{max} \{ \lVert\textsc{Il} \otimes I \rVert_{\sigma}, \lVert\textsc{Ir} \otimes I \rVert_{\sigma} \} \\
   %& \leq \text{max} \{ 1, 1 \}\\
   %& \leq 1
  %\end{split}   
  %\end{equation}

%On the other hand, employing the definition of the operator norm as defined in \autoref{eq:op_norm}, it follows that:
%\begin{equation} 
  %\begin{split}
    %\lVert i  \rVert_{\sigma}&=\lVert [\textsc{Il} \otimes I, \textsc{Ir} \otimes I]  \rVert_{\sigma}  \\
    %& = \text{sup} \{ \lVert [\textsc{Il} \otimes I , \textsc{Ir} \otimes I ] (v) \rVert \hspace{2pt} | \hspace{2pt} \lVert v \rVert = 1 \} \\
    %&= \text{sup} \{ \lVert [\textsc{Il} \otimes I , \textsc{Ir} \otimes I ] (u+w) \rVert \hspace{2pt} | \hspace{2pt} \lVert w+u \rVert = 1 \}\\
    %&= \text{sup} \{ \lVert \textsc{Il} \otimes I  (w) + \textsc{Ir} \otimes I  (u) \rVert \hspace{2pt} | \hspace{2pt} \lVert w+u \rVert = 1 \} \\
    %&=
  %\end{split}   
  %\end{equation}

  
Finally, it is possible to demontrate that $i^{-1}$ has a norm less or equal to 1

\begin{lemma} \label{lem4}
  $  \lVert i^{-1}  \rVert_{\sigma} \leq 1 $
\end{lemma}

\textit{Proof.} \quad Given that $i$ is an isomophism, it follows that 
\begin{equation} 
  \begin{split}
    &\lVert i \cdot i^{-1}  \rVert_{\sigma} = 1  \\
    \leq \hspace{2pt}& \lVert i  \rVert_{\sigma} \cdot \lVert i^{-1}  \rVert_{\sigma} = 1 \hspace{50pt} & \text{\{Norm submultiplicative with respect to compositions\}} \\
    \Leftrightarrow & 1 \cdot \lVert i^{-1}  \rVert_{\sigma} = 1 & \text{\{\autoref{lem4}\}}  \\
    \Leftrightarrow &  \lVert i^{-1}  \rVert_{\sigma} = 1  \\
  \end{split}   
  \end{equation}




Now, it is finally possible to adress the proof of the metric equation for the conditional statement. Considering the the semantics of the "case" rule in \autoref{fig:denotational_sem cond}, proving that the "case" rule in \autoref{fig:metric conditionals} is valid is equivalent to demonstrating that for any super-operators $P$ and $Q$ and their respective erroneous versions $P'$ and $Q'$, denoting the distance between super-operators $A$ and $B$ as $d(A,B)$,  the following holds:
\begin{theorem} \label {theorem:1.1}
  $\text{d} ([P,Q],[P',Q']) \leq \text{max} \{\text{d} (P,P'),\text{d} (Q,Q')\}$
\end{theorem}
\vspace{10pt}
\textit {Proof.} 
In the quantum paradigm, the distance between two super-operators  corresponds to the diamond norm between the two super-operators. Hence, denoting $ [\textsc{Il} \otimes I, \textsc{Ir} \otimes I ]$ by $i$ it follows that:

%\begin{equation}
%\begin{split}
  %& \text{d} ([P,Q],[P',Q'])  \\
  %&=   \lVert  [P,Q] \otimes I - [P',Q'] \otimes I   \rVert_{1}  \\
  %&=   \lVert [P \otimes I, Q \otimes I]  - [P' \otimes I, Q' \otimes I]  \rVert_{1}  \\
  %&=  \lVert [P - P' \otimes I, Q-Q' \otimes I]  \rVert_{1}   \\
  %&= \lVert [P -P', Q-Q' ] \otimes I \cdot i \rVert_{1}  \\
%\end{split}
%\end{equation}

\begin{equation} \label{eq:proof_theorem1.1_esq}
  \begin{split}
    & \text{d} ([P,Q],[P',Q'])  \\
    &=  \lVert  [P,Q] \otimes I - [P',Q'] \otimes I   \rVert_{\sigma}  \\
    &=   \lVert [P \otimes I, Q \otimes I] \cdot i^{-1}  - [P' \otimes I, Q' \otimes I]  \cdot i^{-1}  \rVert_{\sigma}   \hspace{165pt}  \text{\{\autoref{lemi-1}\}} \\
    &=  \lVert [P - P' \otimes I, Q-Q' \otimes I] \cdot i^{-1}  \rVert_{\sigma}   \\
    & \leq \lVert [P - P' \otimes I, Q-Q' \otimes I]  \rVert \lVert i^{-1}  \rVert \rVert_{\sigma} \hspace{20pt} \text{\{Norm submultiplicative with respect to compositions\}}  \\  
    & \leq \lVert [(P - P') \otimes I, (Q-Q') \otimes I]  \rVert_{\sigma} \hspace{235pt} \text{ \{\autoref{lem4}\}} \\
  \end{split}
  \end{equation}
and
\begin{equation} \label {eq:proof_theorem1.1_dir}
\begin{split}
   &  \text{max} \{\text{d} (P,P'),\text{d} (Q,Q')\} \\
   = &  \text{max}\{ \lVert P \otimes I - P' \otimes I \rVert_{\sigma}, \lVert Q \otimes I - Q'\otimes I \rVert_{\sigma} \}\\
   = &  \text{max}\{ \lVert (P - P') \otimes I \rVert_{\sigma}, \lVert (Q - Q') \otimes I \rVert_{\sigma} \}\\
\end{split}
\end{equation}

Finally, by  \autoref{lem1}, it can be deduced that $\text{d} ([P,Q],[P',Q']) \leq \text{max} \{\text{d} (P,P'),\text{d} (Q,Q')\}$, which concludes the proof of theorem \autoref{theorem:1.1}.

\vspace{10pt}

An alternative method to establish \autoref{theorem:1.1} is now presented.
\vspace{5pt}
\textit {Proof.} The proof is as follows:
\begin{equation}
  \begin{split}
    & \text{d} ([P,Q],[P',Q'])  \\
    &=   \lVert  [P,Q] \otimes I - [P',Q'] \otimes I    \rVert_{\sigma} \hspace{2pt} \\
    &=   \lVert  ([P,Q]  - [P',Q']) \otimes I    \rVert_{\sigma} \hspace{2pt} \\
    &=   \lVert  [P-P',Q-Q'] \otimes I  \rVert_{\sigma}   \\
    &=    \lVert  [P-P',Q-Q'] \rVert_{\sigma} \lVert I \rVert_{\sigma}\hspace{2pt} & \hspace {20pt} \text{\{Norm multiplicative with respect to tensor products\}} \\ 
    &=    \lVert  [P-P',Q-Q'] \rVert_{\sigma} & \text{\{\autoref{lemid}\}}  \\
  \end{split}
  \end{equation}
Moreover,
\begin{equation}
  \begin{split}
     &  \text{max} \{\text{d} (P,P'),\text{d} (Q,Q')\} \\
     = &  \text{max}\{ \lVert P \otimes I - P' \otimes I \rVert_{\sigma}, \lVert Q \otimes I - Q'\otimes I \rVert_{\sigma} \}\\
     = &  \text{max}\{ \lVert (P - P') \otimes I \rVert_{\sigma}, \lVert (Q - Q') \otimes I \rVert_{\sigma} \}\\
     = &\text{max}\{ \lVert (P - P') \rVert_{\sigma} \lVert  I \rVert_{\sigma}, \lVert (Q - Q') \rVert_{\sigma} \lVert I \rVert_{\sigma} \} & \hspace{60pt} \text{\{Norm multiplicative with}\\
     && \text{respect to tensor products\}} \\
     = & \text{max}\{ \lVert (P - P') \rVert_{\sigma}, \lVert (Q - Q') \rVert_{\sigma}  \}  & \text{\{\autoref{lemid}\}}  \\
    \end{split}
  \end{equation}

Therefore, by \autoref{lem1}, it can be deduced that $\text{d} ([P,Q],[P',Q']) \leq \text{max} \{\text{d} (P,P'),\text{d} (Q,Q')\}$, which concludes the proof of theorem \autoref{theorem:1.1}.

\vspace{10pt}

Next, one has to prove that for any super-operators $P$ and $Q$ and their respective erroneous versions $P'$ and $Q'$, the following holds:
\begin{lemma} \label {lemmasum}
  $  \lVert P\cdot Q - P'\cdot Q'  \rVert_{\sigma} \leq  \lVert P - P'  \rVert_{\sigma} + \lVert Q - Q' \rVert_{\sigma}   $
\end{lemma} 



  % The spectral norm is submultiplicative with respect to compositions and multiplicative with respect to tensor products,

  % Flar sobre definições de normas


%m


%hefferon2006linear


\subsection{Quantum Teleportation} \label{sec:teleport}


\begin{figure} [H]
  \centering
  \begin{quantikz} [column sep=0.2cm, row sep=0.5cm] 
      \lstick{$\ket{\psi}$}  & \qw &\qw & \qw & \qw & \qw& \ctrl{1}\gategroup[2,steps=4,style={dashed,rounded
      corners,fill=blue!20, inner
      xsep=2pt},background,label style={label
      position=below,anchor=north,yshift=-0.2cm}]{{\sc
      BellMeasure}} & \gate{H} & \qw & \meter{} & \setwiretype{c}  &  & \gategroup[3,steps=4,style={dashed,rounded
      corners,fill=blue!20, inner
      xsep=2pt},background,label style={label
      position=below,anchor=north,yshift=-0.2cm}]{{\sc
      Correction}}  &  & & \ctrl[vertical
wire=c]{2}  \\
      \lstick {$\ket{0}$}  &\gate{H}\gategroup[2,steps=3,style={dashed,rounded
      corners,fill=blue!20, inner
      xsep=2pt},background,label style={label
      position=below,anchor=north,yshift=-0.2cm}]{{\sc
      EPR}} & \qw  & \ctrl{1}& \qw & \qw & \targ{} & \qw & \qw & \meter{} & \setwiretype{c} & & & \ctrl[vertical
wire=c]{1} \\
      \lstick{$\ket{0}$}  &  \qw & \qw &  \targ{} & \qw &\qw&\qw & \qw & \qw& \qw & \qw & \qw &  \qw & \gate{X} & \qw & \gate{Z} 
 \end{quantikz}
  \caption{Quantum Teleportation Protocol}
  \label{fig:teleport}
\end{figure}


%\begin{figure} [H]
    %\centering
    %\begin{quantikz} [column sep=0.2cm, row sep=0.5cm] 
        %\lstick{$\ket{0}$} & \qw & \qw & \qw & \qw & \qw& \ctrl{1}& \gate{H}& \qw &  \meter{} & \qw & \qw & \qw & \ctrl{2} & \qw \\
        %\lstick{$\ket{0}$} & \qw &\gate{H} & \qw  & \ctrl{1}& \qw & \targ{} & \qw &  \qw & \meter{} & \qw & \ctrl{1} & \qw & \qw & \qw \\
        %\lstick{$\ket{0}$} & \qw &  \qw & \qw &  \targ{} & \qw &\qw & \qw& \qw & \qw & \qw & \gate{X} & \qw & \gate{Z} & \qw 
   %\end{quantikz}
    %\caption{Quantum Teleportation Protocol}
    %\label{fig:teleport}
%\end{figure}



When formalizing the quantum teleportation protocol within the lambda calculus framework, each part of the protocol is instantiated as a distinct function. This entails the definition of three specific functions:
\begin{align*}
   \hspace{100pt} & \textbf{EPR}: \hspace{5pt} \mathbb{I} \multimap (\textit{qbit} \otimes \textit{qbit}) \\ 
    &\textbf{BellMeasure}: \hspace{5pt} \textit{qbit} \otimes \textit{qbit}  \multimap \textit{bit} \otimes \textit{bit} \\
    &\textbf{Correction}: \hspace{5pt} \textit{qbit} \otimes \textit{bit} \otimes \textit{bit}  \multimap \textit{qbit} \\
\end{align*}

Considering the unitary operations $H: \textit{qubit} \xrightarrow{}  \textit{qubit}$, $X: \textit{qubit} \xrightarrow{}  \textit{qubit}$, $Z: \textit{qubit} \xrightarrow{}  \textit{qubit}$, $I: \textit{qubit} \xrightarrow{}  \textit{qubit}$, and $\textit{CNOT}: \textit{qubit}, \textit{qubit} \xrightarrow{}  \textit{qubit} \otimes \textit{qubit}$ , these functions are defined as follows:

\begin{align*}
  \hspace{-28pt}
      &\textbf{EPR} =  - \triangleright  \textit{CNOT} \hspace{2pt} (\textit{H}\hspace{2pt} (q  \hspace{2pt}    ( \textit{new}   \hspace{2pt}  0 \hspace{1pt}(*))),(q  \hspace{2pt}   ( \textit{new}   \hspace{2pt}  0 \hspace{1pt}(*))))  \\ 
      \hspace{-28pt}
      &\textbf{BellMeasure} =  q: \text{qubit}, q_{2}: \text{qubit}  \triangleright  (\text{pm}  \hspace{5pt} \textit{CNOT} (q_{1},q_{2})  \hspace{2pt}  \text{to} \hspace{2pt} x \otimes y.  \hspace{2pt}  \textit{meas} (\textit{H} (x)) \otimes \textit{meas} (y) ) \\
      \hspace{-28pt}
      &\textbf{Correction}= q: \text{qubit}, x: \text{bit},  y: \text{bit} \triangleright  \text{cond}\hspace{2pt} x  \hspace{2pt}  \{\text{inl} (x_{0}) \Rightarrow  (\text{cond}\hspace{2pt} y  \hspace{2pt}  \{\text{inl} (y_{0})  \Rightarrow  \textit{Z} (\textit{X}(q));  \\
      \hspace{-28pt}
      &\hspace{335pt} \text{inr} (y_{1}) \Rightarrow{} \textit{Z}(q))\}\\
      \hspace{-28pt}
      & \hspace{240pt}\text{inr} (x_{1})  \Rightarrow  (\text{cond}\hspace{2pt} y  \hspace{2pt}  \{\text{inl} (y_{0})  \Rightarrow  \hspace{2pt}   \textit{X} (q); \\
      \hspace{-28pt}
      & \hspace{335pt} \hspace{5pt} \text{inr} (y_{1}) \Rightarrow{}  \textit{I}(q)\})\}
 \end{align*}



\subsection{Ilustration: Noisy Quantum Teleportation}

\vspace{0pt}

%To study decoherence in a quantum channel within the presented metric deductive system, one can consider the application of a dephasing channel in the quantum teleportation protocol with a certain probability $p$. This is exemplified for probabilities $p=0.5$ and $p=0.25$. It is worth noting that similar exercises can be done for scenarios such as a malicious attack involving a bit flip during measurement or the presence of a noisy channel.



\begin{figure} [H]
  \centering
  \begin{quantikz} [column sep=0.2cm, row sep=0.5cm] 
      \lstick{$\ket{\psi}$} & \qw & \qw & \qw & \qw & \qw& \ctrl{1}& \gate{H}& \qw & \qw & \targ{} &\qw &\qw &\qw &\qw& \qw&  \meter{} &  \setwiretype{c} &  &  &  \ctrl[vertical
wire=c]{2}  \\
      \lstick{$\ket{0}$} & \qw &\gate{H} & \qw  & \ctrl{1}& \qw & \qw & \targ{} & \qw &  \qw & \qw & \qw & \qw &\targ{} &\qw &\qw  & \meter{} & \setwiretype{c} & \ctrl[vertical
wire=c]{1} \\
      \lstick{$\ket{0}$} & \qw &  \qw & \qw &  \targ{} & \qw &\qw & \qw & \qw & \qw& \qw & \qw & \qw & \qw &\qw &\qw &\qw & \qw & \gate{X} & \qw & \gate{Z} \\
      \lstick{$\ket{0}$} & \qw &  \qw & \qw &  \qw & \qw &\qw & \qw & \qw & \qw& \qw & \qw & \gate{R_X(\frac{\pi}{2})} & \ctrl{-2} & \gate{\text{Disc}} \\
      \lstick{$\ket{0}$} & \qw &  \qw & \qw &  \qw & \qw &\qw & \qw & \qw & \gate{R_X(\frac{\pi}{2})} & \ctrl{-4}  & \gate{\text{Disc}}   \\
 \end{quantikz}
  \caption{Quantum Teleportation Protocol with bit flip with a 50\% probability. }
  \label{fig:dephasing}
\end{figure}

