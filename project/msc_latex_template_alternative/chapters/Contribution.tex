\chapter{A Metric Equational System for Conditionals} \label{ch:conditionals}

A metric equational system for the conditionals would be extremely helpful for reasoning about approximate equivalence in the setting of programming and beyond.
In this chapter, we address this gap by introducing such a system and corresponding models, and we then establish a soundness and completeness result in the same style as before. Additionally, we illustrate our system at work via a small example: a metric version of copairing's extensionality. 
%Additionally, we illustrate our system at work via two small examples: reasoning about approximate equivalence between boolean terms (i.e., terms of type $\typeI \oplus \typeI$) and a metric version of copairing's extensionality. 
Next, we introduce different models of our equational system. These include the category of metric spaces and all categories arising from a $\catMet$-enriched version of coproduct cocompletion. We conclude the chapter with a brief discussion about our system.






\section{System}

Our system for conditionals is presented in \autoref{fig:metric conditionals}. 

\begin{figure}[H]
  \begin{equation*}
    \hspace{90pt}
            \begin{prooftree}
                    \hypo{ v =_q w}
                    \infer1[]{\inl_\typeB(v) =_q \inl_\typeB(w)}
            \end{prooftree}
            \hspace{40pt}
            \begin{prooftree}
                    \hypo{ v =_q w}
                    \infer1[]{\inr_\typeA(v) =_q \inl_\typeA(w)}
            \end{prooftree}
    \end{equation*}
    \begin{equation*}
    \hspace{20pt}
  \begin{aligned}
  \hspace{-20pt}
  \begin{prooftree}
      \hypo{ v =_{q} v' }
      \hypo{w=_{r} w'}
      \hypo{u=_{s}u'}
      \infer3[]{\text{ case } v \,   \{\text{inl} (x) \Rightarrow w ; \, \text{inr} (y) \Rightarrow u\} =_{q+\sup\{ r, s \}} \text{ case } v' \,  \{\text{inl} (x) \Rightarrow w' ; \,\text{inr} (y) \Rightarrow u'\} }
  \end{prooftree}
  %
  \\[10pt]
  \end{aligned}
  \end{equation*}
  \caption{Equational system for condicionals}
  \label{fig:metric conditionals}
\end{figure}


\todo[inline,size=\normalsize]{Porque é que queremos as eqs metricas das injeções se são redundantes?}

Firstly, we observe that our equational system encompasses both \autoref{fig:metric conditionals} and \autoref{fig:metric deductive system}. Consequently, the $\inl$ and $\inr$ equations are redundant, as they can be derived from the substitution rule in \autoref{fig:metric deductive system}. Moreover, note that, strictly speaking, we can always use substitution to reason about case statements; however, the introduced equation provides a tighter bound.

 Intuitively, the equation for the case statement provides a bound for the worst-case scenario: the only branch executed (either $w$ or $u$) is the one that is at the greatest distance from its counterpart ($w'$ or $u'$, respectively). Therefore, the maximum distance between the two branches is taken as the bound.   

Additionally, observe that while the original metric system gives the operation ``$+$'' a predominant role, the extended version assigns similar importance to the $\sup$ operator. 


\vspace{5pt}
Recall \autoref{def:metric_lambda_theory} where we presented the notion of metria $\lambda$-theory. Now, we extend the set of theorems \( Th(Ax) \) to denote the smallest set that contains \( Ax \) and is closed under the rules presented in \autoref{fig:equations-linear-lambda}, \autoref{fig:metric deductive system}, and \autoref{fig:metric conditionals}.


%Recall bla, note that in this case using the case metric equation and the substitution rule we obtain the same final equation - this is because in this case we have only one program $w$ which is erroenous, the case equation is trully benefitial in cases where we have diferent programs with diferent errors in both branches- here with substitution the error would be added instead of summed.


\subsection{Extensionality of the copairing}

We establish  a metric version of copairing extensionality. Just as the classical extensionality principle for copairings served as the foundation for demonstrating that terms of type $\typeI \oplus \typeI$ satisfy certain axioms of a Boolean algebra, our metric copairing extensionality will play an analogous role in metric-based reasoning.

Assume that 
\[
\left\{
\begin{aligned}
&(\lambda x.v)\, \text{inl}(y) =_{\varepsilon_1} (\lambda x.v)\, \text{inl}(y) \\
&(\lambda x.w)\, \text{inr}(z) =_{\varepsilon_2}(\lambda x.w)\, \text{inr}(z)
\end{aligned}
\right.
\implies \lambda x.v = \lambda x.w
\]
Following the same reasoning as before in \Cref{subsec:interlude_bool} and applying the metric equation for condicionals we obtain:
\begin{equation*}
\begin{split}
  &v =  \text{case } x  
    \left\{ \begin{aligned}
    \inl(y) \Rightarrow v\,[\text{inl}(y)/x] ;  
    \inl(z) \Rightarrow v\,[\text{inr}(z)/x]
  \end{aligned}  \right\} & (\eta_{case})\\
  &=_{\max \{\varepsilon_1,\varepsilon_2\}} \text{case } x  
    \left\{ \begin{aligned}
    \inl(y) \Rightarrow w\,[\text{inl}(y)/x] ;  
    \inl(z) \Rightarrow w\,[\text{inr}(z)/x]
  \end{aligned}  \right\} = w & (\text{\autoref{fig:metric conditionals}},\eta_{case})
\end{split}
\end{equation*}
As a result, we have $v =_{\max \{\varepsilon_1,\varepsilon_2\}} w$ and derive $\lambda x.v =_{\max \{\varepsilon_1,\varepsilon_2\}} \lambda x.w$ using the metric equational system-

\section{Interpretation}

In this subsection, we extend the concepts introduced in \Cref{subsec:semantics_metric} to include the interpretation of the extended equational system.

\begin{definition} \label{def:met_cop_cat}
  A \emph{symmetric monoidal closed $\catMet$-category with binary coproducts} $\catC$ is a  $\catMet$-category that is symmetric monoidal closed and has binary coproducts   such that, for all morphisms $f_1, f_2 \in \catC(A, C)$ and $g_1, g_2 \in \catC(B, C)$, we have:
\[
  d_{A \oplus B, C}([f_1, g_1], [f_2, g_2]) \leq \sup \{d_{A,C}(f_1, f_2), d_{B,C}(g_1, g_2)\}.
\]
\end{definition}


We present the category of metric spaces as an example of \autoref{def:met_cop_cat}.

\begin{proposition}
        \label{prop:vcat}
        The category $\catMet$ is a symmetric monoidal closed with binary coproducts $\catMet$-category 
\end{proposition}

\begin{proof}
  By~\cite[Example 3.8]{dahlqvist2023syntactic}, the category $\catMet$ is a symmetric monoidal closed $\catMet$-category. As a result, it suffices to show that for all morphisms \( f_1, f_2 \in \catMet(A, C) \) and \( g_1, g_2 \in \catMet(B, C) \), the following inequality holds:
\[
  d_{A \oplus B, C}([f_1, g_1], [f_2, g_2]) \leq \sup \{d_{A,C}(f_1, f_2), d_{B,C}(g_1, g_2)\}.
\]

In this category, the coproduct is defined as in $\catSet$. The distance function $d$ on the coproduct $A \oplus B$ is given by:
   \[
    \begin{cases}
    d_{A \oplus B}(\inl(a_1),\inl(x_2)) = d_A(a_1,a_2) \\
    d_{A \oplus B}(\inr(b_1),\inr(b_2)) = d_B(b_1,b_2) \\
    d_{A \oplus B}(\inl(a), \inr(b)) = 
    d_{A \oplus B}(\inr(b), \inl(a))  = \infty
    \end{cases}
    \]

    The co-pairing is defined as in $\catfont{Set}$. The inequality we aim to prove follows directly from the fact that, given two morphisms  $f,g\in \catMet(A, B)$ the distance between them is defined as 
  \[ \sup \{ d_A (f a, g a) \,\vert \, a \in A \} ,\] 
    together with \autoref{lem:sup_union}. For  \( f_1, f_2 \in \catMet(A, C) \) and \( g_1, g_2 \in \catMet(B, C) \), we calculate:
    \todo[inline,size=\normalsize]{Não estou a perceber o problema de $ (a,b) \in A \oplus B$? Nós usamos $A \otimes B$ acima tb, não podemos usar $(a,b)$?}
    \begin{align*}
      & d_{A \oplus B, C}([f_1, g_1], [f_2, g_2]) \\
      &  =  \sup \{ d_{A\oplus B} ([f_1, g_1] (a,b), [f_2, g_2] (a,b)) \,\vert \, (a,b) \in A \oplus B \} \\
      & =  \sup \{ \{ d_{A\oplus B} ([f_1, g_1] (\inl(a)), [f_2, g_2] (\inl(a))) \,\vert \, a \in A \}    \\
      & \hspace{15pt} \cup \{d_{A\oplus B} ([f_1, g_1] (\inr(b)), [f_2, g_2] (\inr(b))) \,\vert \, b \in B\}\} \\
      & = \sup \{ \{ d_{A}(f_1 (a), f_2 (a)) \,\vert \, a \in A \} \cup  \{d_{B}(g_1 (b), g_2 (b)) \,\vert \, b \in B \}\} \\
      & = \sup   \{ \sup \{ d_{A}(f_1 (a), f_2 (a)) \,\vert \, a \in A \}, \sup\{d_{B}(g_1 (b), g_2 (b)) \,\vert \, b \in B \}\} \\
      &=  \sup \{d_{A,C}(f_1, f_2), d_{B,C}(g_1, g_2)\}. 
    \end{align*}
\end{proof}


\begin{definition}
  Consider a metric $\lambda$-theory $((G,\Sigma),Ax)$ and a symmetric monoidal closed $\catMet$-category with binary coproducts $\catC$, in which both the tensor product and the currying are non-expansive. Suppose that for each $X \in G$ we have an interpretation $\llbracket X \rrbracket$ as a $\catC$-object and analogously for the operation symbols. This interpretation structure is a model of the theory if all axioms in $Ax$ are satisfied by the interpretation.
\end{definition}

\section{Soundeness and Completeness}
In this section we establish a soundness and completeness result in the same style as before. 


\subsubsection{Lattice Theory Preliminaries}
First, we introduce a few concepts from lattice theory that will be useful for the completeness proof and for a broader discussion of our results.
%For the metric quantale, the way-below relation corresponds to the strictly greater relation with ∞ > ∞, and a basis for the underlying lattice that satisfies the conditions above is the set of extended non-negative rational numbers.


\begin{definition}
   A \emph{lattice} is a partial order in which every finite subset has both a meet and a join. A \emph{complete lattice} is a partial order  in which every subset, finite or infinite, has a meet and a join.
\end{definition}

\begin{comment}
\begin{definition}
  \emph{Quantales} are complete lattices equipped with an associative composition, typically denoted $\otimes$, that preserves suprema in both arguments.
\end{definition}


\begin{example}
  In this work, we consider only the standard metric quantale, whose underlying complete lattice is the interval $[0, \infty]$ ordered by $\geq$, with associative composition given by addition. Note that, since the order is reversed in this quantale, suprema in general lattices correspond to infima here, and vice versa.
\end{example}


\begin{lemma}[Way-below relation for the metric quantale] \cite{dahlqvist2023syntactic}
Let $L$ denote the complete lattice $[0, \infty]$ ordered by $\geq$.  
For all $x, y \in L$ and for every subset $X \subseteq L$, if $y > x$, then whenever $x \geq \inf X$, there exists a \emph{finite} subset $A \subseteq X$ such that $y \geq \inf A$.


This property characterizes the \emph{way-below relation} in the metric quantale. The lattice $L$ is \emph{continuous}, \ie for every $x \in L$,
	\begin{flalign*}
		x = \inf \{ y  \mid y \in L\ \text{and} \ y > x \}.
	\end{flalign*}
\end{lemma}
\end{comment}


\begin{definition}
  A subset $D$ of a lattice $L$ is called \emph{directed} if it is nonempty and every finite subset of $D$ has an upper bound in $D$. A partially ordered set is said to be \emph{directed complete} if every directed subset has a supremum. A directed complete poset is commonly referred to as a \emph{dcpo}.
\end{definition}

\begin{definition}
  A lattice $L$ is called \emph{meet continuous} if it is directed complete, \ie a dcpo, and satisfies the condition
  \[
    \inf\{ x, \sup D \} = \sup \{ \inf \{x, d\} \mid d \in D\} 
  \]
  for all $x \in L$ and all directed subsets $D \subseteq L$.
\end{definition}

\begin{lemma} \label{lem:meet_continuos}
  The $[0, \infty]$ lattice is meet continuous.
\end{lemma}

\begin{proof}
  Follows from \cite{dahlqvist2023syntactic}  and  \cite[Proposition I-1.8]{gierzContinuousLatticesDomains2003}
\end{proof}

Note that, since the order is reversed in this quantale, suprema in general lattices correspond to infima here, and vice versa.

\begin{lemma} \cite[Lemma 2.23]{daveyIntroductionLatticesOrder2002} \label{lem:sup_union}
Let \( L \) be a lattice, let \( A, B \subseteq L \) and assume that \( \sup A \),
\( \sup B \), \( \inf A \) and \( \inf B \) exist in \( L \). Then
$$ \sup\{A \cup B\} = \sup\{ \sup A, \sup B \}  \quad \text{and} \quad \inf\{A \cup B\} = \inf\{ \inf A, \inf B \}
 . $$
\end{lemma}




\subsubsection{Soundeness and Completeness}
The proofs in this section are based on the proofs of 
\autoref{thm:soundness_metric_no_cond} and \autoref{thm:completeness_metric_no_cond} in \cite{dahlqvist2023syntactic}.

\begin{theorem}
  The rules in Figures \ref{fig:equations-linear-lambda}, \ref{fig:metric deductive system} and \ref{fig:metric conditionals} are sound for a  symmetric monoidal closed $\catMet$-category with coproducts $\catC$, in which both the tensor product and the internal hom-functor (currying) are non-expansive. Specifically, if $\Gamma \vljud v =_{q} w : \typeA $ results from the rules in Figures  \ref{fig:equations-linear-lambda}, \ref{fig:metric deductive system} and \ref{fig:metric conditionals} then $q \geq d( \llbracket \Gamma  \vljud v : \typeA \rrbracket, \llbracket\Gamma \vljud w : \typeA \rrbracket)$.
\end{theorem}

\begin{proof}
 We follow the same strategy as in \cite{dahlqvist2023syntactic}.
 This proof uses induction over the depth of proof trees that
 arise from the metric deductive system. The general strategy for each
 inference rule is to use the definition of a symmetric monoidal closed $\catMet$-category with coproducts. 

 More concretely, first, consider the equations on \autoref{fig:equations-linear-lambda} which abbreviate equations $\Gamma \triangleright w =_0 v : \typeA$. By \autoref{thm:soundness_classical}, these equations are sound for symmetric monoidal closed categories with binary coproducts, \ie if $v=w$, then $\sem{v}=\sem{w}$ in $ \catC$. Then by the definition of metric space we obtain $d(\sem{v},\sem{w}) = d(\sem{w},\sem{v})=0.$ 
 The rules in \autoref{fig:metric deductive system} follow from the definition of  $\catMet$-category, and the fact that the tensor product and the internal hom-functor (currying) are non-expansive. 
 Finally, rules in  \autoref{fig:metric conditionals} follow from the non-expansive law imposed on binary coproducts \autoref{def:met_cop_cat}. Specifically,
  \begin{align*}
    & \hspace{-20pt} d( \llbracket  \text{ case } v \, \{\text{inl}_{\typeB}  (x) \Rightarrow w ; \, \text{inr}_{\typeA} (y) \Rightarrow u\}  \rrbracket, \llbracket \text{ case } v' \, \{\text{inl}_{\typeB}  (x) \Rightarrow w' ; \, \text{inr}_{\typeA} (y) \Rightarrow u'\} \rrbracket)  \\
    &  \hspace{-20pt} = d( [\llbracket w\rrbracket ,\llbracket u\rrbracket] \comp (\text{jn}_{\Delta;\typeA}\comp \sw \oplus \text{jn}_{\Delta;\typeB}\comp \sw) \comp \dist \comp (\llbracket v\rrbracket \otimes \text{id}) \comp \text{sp}_{\Gamma;\Delta} \comp \text{sh}_{E}, \\
    & \hspace{7pt} [\llbracket w'\rrbracket ,\llbracket u'\rrbracket] \comp (\text{jn}_{\Delta;\typeA}\comp \sw \oplus \text{jn}_{\Delta;\typeB}\comp \sw) \comp \dist \comp (\llbracket v'\rrbracket \otimes \text{id}) \comp \text{sp}_{\Gamma;\Delta} \comp \text{sh}_{E})\\
    & \hspace{-20pt} \leq  d( [\llbracket w\rrbracket ,\llbracket u\rrbracket] \comp (\text{jn}_{\Delta;\typeA}\comp \sw \oplus \text{jn}_{\Delta;\typeB}\comp \sw) \comp \dist \comp (\llbracket v\rrbracket \otimes \text{id}),  [\llbracket w'\rrbracket ,\llbracket u'\rrbracket] \comp (\text{jn}_{\Delta;\typeA}\comp \sw \, \oplus \\
    & \hspace{7pt}  \text{jn}_{\Delta;\typeB}\comp \sw) \comp \dist \comp (\llbracket v'\rrbracket \otimes \text{id}) )\\
    & \hspace{-20pt} \leq  d(\llbracket v\rrbracket \otimes \text{id}, \llbracket v'\rrbracket \otimes \text{id}) +  d( [\llbracket w\rrbracket ,\llbracket u\rrbracket] \comp (\text{jn}_{\Delta;\typeA}\comp \sw \oplus \text{jn}_{\Delta;\typeB}\comp \sw) \comp \dist,  [\llbracket w'\rrbracket ,\llbracket u'\rrbracket]\comp \\
    & \hspace{7pt} (\text{jn}_{\Delta;\typeA}\comp \sw \oplus \text{jn}_{\Delta;\typeB}\comp \sw) \comp \dist) \\
    & \hspace{-20pt} \leq  q +  d( [\llbracket w\rrbracket ,\llbracket u\rrbracket] \comp (\text{jn}_{\Delta;\typeA}\comp \sw \oplus \text{jn}_{\Delta;\typeB}\comp \sw) \comp \dist,  [\llbracket w'\rrbracket ,\llbracket u'\rrbracket] \comp (\text{jn}_{\Delta;\typeA}\comp \sw \oplus \text{jn}_{\Delta;\typeB}\comp \sw)  \\
    & \hspace{20pt} \comp \dist)\\
    &  \hspace{-20pt} \leq q +  d( [\llbracket w\rrbracket ,\llbracket u\rrbracket],  [\llbracket w'\rrbracket ,\llbracket u'\rrbracket]) \\
    & \hspace{-20pt}  \leq q + \sup(d( \llbracket w\rrbracket , \llbracket w' \rrbracket), d( \llbracket u\rrbracket , \llbracket u' \rrbracket)) \\
    & \hspace{-20pt} \leq  q + \sup \{r, s\} \\
  \end{align*}
  The second step follows from the fact that $\text{sp}_{\Gamma;\Delta} \comp \text{sh}_{E}$  is a morphism in $\catC$  and that $\catC$ is a $\catMet$-category.  The third and fifth step follow from an analogous reasoning. The fourth step follows from the premises of the rule in question and the fact that $\catC$ is a symmetric monoidal $\catMet$-category. The sixth step follows from the fact that $\catC$ is a symmetric monoidal $\catMet$-category with binary coproducts. Finally, the last step follows from the premise of the rule in question.

\end{proof}

We will now focus on completeness. We extend  \texttt{Values}$((\typeA,\typeB),d)$, by incorporating the new theorems (those concerning the metric equations for conditionals) and quotient this pseudometric space into a metric space (\texttt{Values}$(\typeA,\typeB),d)$/$\sim$ (in the same spirit as before, see \Cref{subsec:semantics_metric}).


\begin{theorem}[Completeness]
Consider a  metric $\lambda$-theory. A metric equation in context
$\Gamma \vljud v =_{q} w : \typeA$
is a theorem if and only if it holds in all models of the theory.
\end{theorem}

\begin{proof}
  We will focus only on conditionals, as the remaining cases are proven in~\cite{dahlqvist2023syntactic}.
  Completeness arises from constructing the syntactic category $\catSyn(T)$ of the underlying  theory
  $T$ and then showing that provability of $\Gamma \vljud v =_q w : \typeA$
  in $T$ is equivalent to $d(\sem{v},\sem{w}) \geq q$ in the category
  $\catfont{Syn}(T)$. We use the category (\texttt{Values}$(\typeA,\typeB),d)$/$\sim$ to this end.  Note that the quotienting process identifies all terms $x : \typeA \triangleright v : \typeB$ and $x : \typeA \triangleright w : \typeB$ such that $v =_0 w$ and $w =_0 v$. This relation includes the equations-in-context from \autoref{fig:equations-linear-lambda}.
     Then, we remark that all sets of the form $\{ q \mid v =_q w \}$ are directed: they are non-empty, since by rule~(join) we always have at least $v =_{\infty} w$, and again by~(join), every finite subset of $\{ q \mid v =_q w \}$ has a lower bound in the set. This will be useful for applying \autoref{lem:meet_continuos}.
   
   Next, we need to prove that the copairing is well defined in ($\catSyn(T))$, \ie, for any $v, v', w, w'$ if $v\sim v'$ and $w\sim w'$, then $[v,w] \sim [v',w']$. 

   This is equivalent to demonstrating the following implication:
   \begin{align*}
    \begin{cases}
      \inf \{ q \, \vert \, v=_q w \} \leq 0 \\
      \inf \{ r \, \vert \, v'=_r w' \} \leq 0 \\
    \end{cases} \implies \inf \left\{q \, \bigg\vert \,  
    \begin{aligned}
       &\text{ case } z \,   \{\text{inl} (x) \Rightarrow v ; \, \text{inr} (y) \Rightarrow w\} =_{q} \\
       &\text{ case } z \,  \{\text{inl} (x) \Rightarrow v' ; \,\text{inr} (y) \Rightarrow w'\}
    \end{aligned}\right\} \leq 0.
   \end{align*}
%which follows from \autoref{lem:meet_continuos}, \autoref{lem:sup_union} and the metric equation for conditionals. 
Let $L$ be a lattice, and let $D, F \subseteq L$ be directed sets. Observe the following:
\begin{equation} \label{eq:completeness}
  \begin{split}
     & \sup \{ \inf D, \inf F \} \\\
    & = \inf \{ \sup \{\inf D, f\} \, \vert \, f \in F \} & (\text{\autoref{lem:meet_continuos}}) \\
    & = \inf \{ \inf\{ \sup \{\inf d, f\} \,\vert \, d \in D \} \vert \, f \in F \} & (\text{\autoref{lem:meet_continuos}}) \\
    & =  \inf \{  \sup \{d,f\} \, \vert \, d \in D, f \in F \} & (\text{\autoref{lem:sup_union}})
  \end{split}
  \end{equation}
With the equality above, we have
\begin{align*}
    &\begin{cases}
      \inf \{ q \, \vert \, v=_q w \} \leq 0 \\
      \inf \{ r \, \vert \, v=_r w \} \leq 0 \\
    \end{cases} \\
    &\implies  \sup \{ \inf \{ q \, \vert \, v=_q w \} , \inf \{ r \, \vert \, v=_r w \} \} \leq 0 \\
    & \implies \inf \{ \sup \{q,r\} \vert \,  v=_q w,  v=_r w  \} \leq 0 & (\text{\autoref{eq:completeness}})  \\
    & \implies \left\{q \, \bigg\vert \,  
    \begin{aligned}
       &\text{ case } z \,   \{\text{inl} (x) \Rightarrow v ; \, \text{inr} (y) \Rightarrow w\} =_{q}\\
        &\text{ case } z \,  \{\text{inl} (x) \Rightarrow v' ; \,\text{inr} (y) \Rightarrow w'\}
    \end{aligned}\right\} \leq 0
  \end{align*}

  Thus, we obtain a category $\catSyn(T)$ with binary coproducts.
 The next step is to prove the required non-expansivity law concerning copairing. To this effect, we reason as follows:

  \begin{align*}
    & \sup{\{d([v],[w]),d([v'],[w']) \}}  \\
    & = \sup{\{d(v,w),d(v',w') \}} \\
    & = \sup {\{ \inf{\{q \, \vert \, v=_q v'\}},\inf{\{r \, \vert \, w=_r w'\}}  \}} \\
    & = \inf{\{ \sup \{ q, r \} \vert \, v=_q v', w=_r w' \}} &  \\
    & \geq  \inf{ \{ q  \,\vert \, \text{ case } z \,   \{\text{inl} (x) \Rightarrow v ; \, \text{inr} (y) \Rightarrow w\} =_{q} \text{ case } z \,  \{\text{inl} (x) \Rightarrow v' ; \,\text{inr} (y) \Rightarrow w'\} \} } &  \\ 
    & = d(\text{case } z \,   \{\text{inl} (x) \Rightarrow v ; \, \text{inr} (y) \Rightarrow w\}, \text{case } z \,  \{\text{inl} (x) \Rightarrow v' ; \,\text{inr} (y) \Rightarrow w'\}) \\
    & = d([\text{case } z \,   \{\text{inl} (x) \Rightarrow v ; \, \text{inr} (y) \Rightarrow w\}], [\text{case } z \,  \{\text{inl} (x) \Rightarrow v' ; \,\text{inr} (y) \Rightarrow w'\}]) \\
    & = d([[v],[v']],[[w],[w']])  
  \end{align*}
   The third step follows from \autoref{eq:completeness}, and the fourth step follows from the fact that for any sets $A$ and $B$, if $A \subseteq B$, then $\inf\{A\} \geq \inf\{B\}$.

 The final step is to show that if an equation $\Gamma \vljud v =_q v' : \typeA$ with $q \in [0, \infty]$ is satisfied by Syn$(T)$, then it is a theorem of the linear metric $\lambda$-theory. By assumption, $d([v],[v']) = d(v,v') =  \inf{ \{r \, \vert \, v =_r v'\}} \leq q$. It follows from the definition of the strictly greater relation that for all
 $x \in [0, \infty]$ with $x>q$ there exists a finite set $A \subseteq \{r \, \vert \, v =_r v'\}$ such that $x \geq \inf{A}$. Then by an
 application of rule (join) we obtain $v =_{\inf{A}} v'$, and consequently, rule (weak) provides $v =_x v'$ for all $x > q$. Finally, by applying rule (arch),  we deduce that $v =_q v'$ is part of the theory.
\end{proof}


\section{Coproduct cocompletion} \label{sec:Coproduct_cocompletion}


The idea behind the coproduct completion of a category $\catC$ is to create a new category where all small coproducts exist by formally adding them to $\catC$  in the simplest way possible. We will show that this construction is compatible with our metric equational system.

\begin{definition}
  A \emph{(free) coproduct completion} of a category \(\catC\), denoted $\catCcop$, is the category whose objects are families \((A_i)_{i \in I}\) of objects of \(\catC\), where \(I\) is a set; an arrow \((A_i)_{i \in I} \to (B_j)_{j \in J}\) consists of a pair \((f, (\phi_i)_{i \in I})\), where \(f : I \to J\) is a function between the index sets, and \((\phi_i)_{i \in I}\) is a family of morphisms \(\phi_i : X_i \to Y_{f(i)}\) in \(\catC\). Given morphisms $(f,\phi_i): (A_i)_{i \in I} \to (B_j)_{j \in J}$ and $(g,\psi_i): (Y_j)_{j \in J} \to (Z_k)_{k \in K}$, their composition is defined as the morphism, given by the pair \((g \comp f, (\theta_i)_{i \in I})\), where $ \theta_i := \psi_{f(i)} \circ \varphi_i : X_i \to Z_{g(f(i))}$. From now on, unless ambiguities arise, we will omit the indexing function and use letters $\Phi, \Psi, \xi$ to refer to families of morphisms.
\end{definition}


If $\catC$ is a $\catMet$-category, one may define a metric on the morphims of its coproduct cocompletion $\catCcop$ as follows:

\begin{equation*}
  d(\Phi, \Psi) = \sup \left\{ d'(\phi_i, \psi_i) \;\middle\vert\; i \in I \right\},\text{ where } 
  d'(\phi_i, \psi_i) 
  = 
  \begin{cases}
    \infty, & \text{if } f(i) \neq g(i), \\
    d(\phi_i, \psi_i), & \text{otherwise}.
  \end{cases}
\end{equation*}

\begin{proposition} \label{prop:cop_completion_met}
  The coproduct completion of a $\catMet$-category $\catC$ is a $\catMet$-category.
\end{proposition}

\begin{proof}
  This follows from the fact that $\catC$ is a $\catMet$-category. Considering the definition of a $\catMet$-category, we need to show that for all objects $A, B,C$ in $\catCcop$, and for any morphisms $\Phi,\Phi':\catCcop(B, C),$  $\Psi, \Psi' \in \catCcop(A, B)$, it holds that:
  \[d(\Phi \cdot \Psi, \Phi' \cdot \Psi') \leq d(\Phi, \Phi') + d(\Psi, \Psi').\]

  Given our choice of metric, we have:
  \begin{align*}
    d(\Phi \cdot \Psi, \Phi \cdot \Psi') = \sup \{d'(\phi_{f(i)} \cdot \psi_i, \phi_{g(i)} \cdot \psi')  \;\vert\; i \in I \}
  \end{align*}
  First, suppose $f(i)=g(i)$ for all $i \in I$,
  \begin{align*}
   & \sup \{d'(\phi_{f(i)} \cdot \psi_i, \phi_{g(i)} \cdot \psi')  \;\vert\; i \in I \} \\
   &= \sup \{d(\phi_{f(i)} \cdot \psi_i,\phi_{f(i)} \cdot \psi_i' ) \;\vert\; i \in I\}\\
      &  \leq \sup \{d( \psi_i, \psi_i' ) \;\vert\; i \in I\} & (\catC \text{ is a $\catMet$-category}) \\
      & = d(\Psi, \Psi')
  \end{align*}
  Next, assume $f(i)\neq g(i)$ for any $i \in I$, it is direct that 
  \begin{align*}
   \sup \{d'(\phi_{f(i)} \cdot \psi_i, \phi_{g(i)} \cdot \psi') \} \leq \infty = \sup \{d'(\psi_i, \psi_i') \} = d(\Psi, \Psi')
  \end{align*}
  The proof for precomposition follows a similar reasoning.
\end{proof}


  \begin{proposition}
  The coproduct completion of a $\catMet$-category $\catC$ is a  $\catMet$-category with binary coproducts.
\end{proposition}

\begin{proof}
  This follows from \autoref{prop:cop_completion_met} and the definition of the metric in $\catCcop$. For any objects $A, B, C$ in $\catCcop$ and morphisms $\Phi, \Phi' \in \catCcop(A, C)$ and $\Psi, \Psi' \in \catCcop(B, C)$, we may regard the copairings
\[
[(f,(\phi_i)_{i \in I'}),\ (g,(\psi_i)_{i \in I''})] \quad \text{and} \quad [(f',(\phi_i')_{i \in I'}),\ (g',(\psi_i')_{i \in I''})]
\]
as morphisms $(h,(\xi_i)_{i \in I})$ and $(h',(\xi_i')_{i \in I})$ from $A \oplus B$ to $C$, where $I = I' \cup I''$. For instance, the indexing function $h$ underlying $(\xi_i)_{i \in I}$ is defined as
\[
h(i) =
\begin{cases}
f(i) & \text{if } i \in I', \\
g(i) & \text{if } i \in I''.
\end{cases}
\]
Then, we have:

  \begin{align*}
    &d([\Phi, \Psi],[\Phi', \Psi'])  \\
    &= d ( \xi, \xi') = \sup\{d'(\xi_i, \xi'_i) \;\vert\; i \in I' \cup I'' \} \\
    & = \sup \{ \{d'(\phi_i, \phi'_i)\;\vert\; i \in I'\}  \cup \{d'(\psi_i, \psi'_i) \;\vert\; i \in I'' \} \} \\
    & = \sup \{ \sup  \{d'(\phi_i, \phi'_i)\;\vert\; i \in I'\}, \sup \{d'(\psi_i, \psi'_i) \;\vert\; i \in I'' \} \} & (\text{\autoref{lem:sup_union}}) \\
    & = \sup \{ d(\Phi,\Phi'), d(\Psi,\Psi') \}
  \end{align*}
\end{proof}

We note that more powerful machinery, based on advanced categorical structures such as presheaves, could be employed to prove the fact that the coproduct cocompletion of a category $\catC$ forms a $\catMet$-category with binary products. However, since categories are used here as a tool rather than being the main focus of the thesis, we have opted for this more down-to-earth description, which assumes from the reader less advanced knowledge of category theory.




\begin{comment}
  Alternatively, we may consider the following metric on the morphisms of the category obtained by the coproduct cocompletion of a $\catMet$-category $\catC$.
\begin{equation*}
  d(\Phi, \Psi) = \sum_{i \in I}  d'(\phi_i, \psi_i) ,\text{ where } 
  d'(\phi_i, \psi_i) 
  = 
  \begin{cases}
    \infty, & \text{if } f(i) \neq g(i), \\
    d(\phi_i, \psi_i), & \text{otherwise}.
  \end{cases}
\end{equation*}


\begin{proposition}
  Considering the metric above, the coproduct completion of a $\catMet$-category $\catC$ is a $\catMet$-category.
\end{proposition}

\begin{proof}
  
  The proof follows similarly to the previous case. For all objects $A, B$ in $\catCcop$, and for any morphisms $\Phi, \Psi, \Psi' \in \catCcop(A, B)$, we have:
\[
d(\Phi \cdot \Psi, \Phi \cdot \Psi') = \sum_{i \in I} d(\phi_{f(i)} \cdot \psi_i, \phi_{f(i)} \cdot \psi_i') \leq \sum_{i \in I} d(\psi_i, \psi_i') = d(\Psi, \Psi'),
\]
if $f(i) = g(i)$ for all $i \in I$. Otherwise, if there exists $i \in I$ such that $f(i) \neq g(i)$, then
\[
d(\Phi \cdot \Psi, \Phi \cdot \Psi') = \infty = d(\Psi, \Psi').
\]
And the precomposition follows analogous reasoning.
\end{proof}

Note that this metric does not always make the category enriched with binary coproducts. It is only the case if d(\Phi,\Phi') or  d(\Psi,\Psi') equals zero. 
 For any objects $A, B, C$ in $\catCcop$ and morphisms $\Phi, \Phi' \in \catCcop(A, C)$ and $\Psi, \Psi' \in \catCcop(B, C)$, and $\xi$ and $\xi'$ defined as before, we have:
  \begin{align*}
    &d([\Phi, \Psi],[\Phi', \Psi']) = \sum_{i \in I' \cup I''} d'(\xi_i, \xi'_i)  \\
    &= \sum_{i \in I'} d'(\phi_i, \phi'_i) + \sum_{i \in I'} d'(\phi_i, \phi'_i) = d(\Phi,\Phi')+ d(\Psi,\Psi')  \geq  \sup \{ d(\Phi,\Phi'), d(\Psi,\Psi') \}
  \end{align*}
  
\end{comment}





\section{Discussion: generalizing to quantales}

\todo[inline,size=\normalsize]{Que referencia é que dou para o artigo se ele (ainda) não está em lado nenhum?}


This work can be generalized to other quantales. Indeed, initial steps in this direction have already been established in \textbf{[BMQL25]}.   Readers unfamiliar with quantales may also consult \cite{STUBBE201495} for an accessible introduction to the concept.
This generalization also explains why in the equational system that we introduced, we chose to use the expression \( q + \sup\{r, s\} \) rather than \( \sup\{q + r, q + s\} \), given that these are equal.

The fact is that at the level of arbitrary quantales they need not to be the same. For instance, consider the quantale \(\mathcal{P}(\Sigma^*)\), where \(\Sigma\) is a finite non-empty set of symbols (\ie, the powerset of all finite lists over \(\Sigma\)) \cite{STUBBE201495}. In this quantale, the associative operation \(\otimes\) and the infimum/meet are defined as follows:
\[ I \otimes J =  \{ i \mathbin{+\!+} j \,|\, i \in I, j \in J \} \quad \text{and} \quad \inf \{ I, J\} = I \cap  J  \]
where \(I, J \in \mathcal{P}(\Sigma^*)\) and \(\mathbin{+\!+}\) denotes list concatenation.
Consider the sets $X = \{\varepsilon, a\}$, $Y = \{a\}$, and $Z = \{aa\}$, where $\varepsilon$ denotes the empty string over $\Sigma$ (i.e., for any $s \in \Sigma^*$, we have $\varepsilon \mathbin{+\!+} s = s = s \mathbin{+\!+} \varepsilon$). Then:
\begin{align*}
  & X \otimes \inf\{Y, Z\} = X \otimes \emptyset = \emptyset, \\
  &\inf\{X \otimes Y,\, X \otimes Z\} =  \{a, aa\} \cap \{aa, aaa\} = \{aa\}.
\end{align*}
Consequently, $X \otimes \inf\{Y,Z\} \neq  \inf\{X \otimes Y, X \otimes Z \}.$

Moreover, it becomes clear after inspecting the soundness proof that it is the expression \( q + \sup\{r, s\} \) that arises naturally. 

%STUBBE201495