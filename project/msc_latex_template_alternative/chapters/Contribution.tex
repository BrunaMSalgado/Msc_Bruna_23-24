\chapter{Contribution}

Main result(s) and their scientific evidence

\section{Introduction}

\section{Summary}


\section{Integration of conditionals}

The notion of approximate equivalence for quantum programming explored in [\cite{dahlqvist2022syntactic}] does not encompass classical control flow. As a result, preliminary work based on [\cite{crole1993categories,selinger2013lecture}]   has been undertaken to address the integration of conditionals. 

\subsection{Integration of conditionals}

The term formation rules for conditionals are depicted in
\autoref{fig:typing_rules_cond}. 

\begin{figure} [H]
\begin{equation*}
\begin{split}
\begin{aligned}
& \hspace{55pt}
\begin{minipage}[t]{0.3\textwidth}
$\begin{array}{c}
     \Gamma \triangleright v: \mathbb{A} \\
    \hline
   \Gamma \triangleright \text{inl}(v):  \mathbb{A} \oplus \mathbb{B}
\end{array}
$
\end{minipage}
\hspace{-38pt}
\text{(inl)} 
 \hspace{20pt}
\begin{minipage}[t]{0.3\textwidth}
$\begin{array}{c}
      \Gamma \triangleright v:  \mathbb{B} \\
    \hline
   \Gamma \triangleright \text{inr}(v): \mathbb{A} \oplus \mathbb{B}
\end{array}
$ \end{minipage} 
\hspace{-35pt} \text{(inr)} \\
&\hspace{15pt}
\begin{minipage}[t]{0.3\textwidth}
$\begin{array}{c}
     \Gamma\triangleright v: \mathbb{A} \oplus \mathbb{B} \quad \Delta, x: \mathbb{A} \triangleright w: \mathbb{C} \quad \Delta, y: \mathbb{B}  \triangleright u : \mathbb{C}   \quad E \in \text{Sf}(\Gamma;\Delta)  \\
    \hline
   E \triangleright \text{ cond } v \hspace{2pt} \{\text{inl} (x) \Rightarrow w ; \hspace{1pt} \text{inr} (y) \Rightarrow u\}: \mathbb{C} 
\end{array}
$
\end{minipage}
\hspace{200pt}
\text{(case)} 
\end{aligned}
\end{split}
\end{equation*}
\caption{Term formation rules for conditionals}
\label{fig:typing_rules_cond}
\end{figure}
Considering  $v \in V$, $w \in W$, and $u \in U$ where $V, W, U$ represent vector spaces, $\textsc{Il}_{V}: V \xrightarrow{} V\oplus W$, denotes the left injection operator, defined as $\textsc{Il}_{V}= v \mapsto (v,0) $; $\textsc{Ir}_{V}: V \xrightarrow{} W \oplus V$, denotes the right injection operator, defined as $\textsc{Ir}_{V}= v \mapsto (0,v) $; and $\text{dist}_{V, W,U}: V \otimes  \left(W \oplus U\right) \xrightarrow{} \left(V \otimes W\right) \oplus \left(V \otimes U\right)$, denotes the distributive property of the tensor product over the direct sum, defined as $\text{dist}_{V, W,U} =  v \otimes  \left(w, u\right) \mapsto \left(v \otimes w, v \otimes u\right)$. The subscripts in these operators will be omitted unless ambiguity arises. Moreover, the operation \text{either} corresponds to:
\begin{figure} [H]
\begin{equation}
\begin{split}
\begin{aligned}
\hspace{95pt}&
\begin{minipage}[t]{0.3\textwidth}
$\begin{array}{c}
     V  \xrightarrow{} U  \\
      W \xrightarrow{} U  \\
    \hline
  [T,S]: V \oplus W \xrightarrow{} U
\end{array}
$
\end{minipage} \\
\hspace{95pt}&
\begin{minipage}[t]{0.3\textwidth}
$\begin{array}{c}
  [T,S] = (v,w) \mapsto T(v)+S(w) 
\end{array}
$
\end{minipage}
\end{aligned}
\end{split}
\end{equation}
\label{fig:either}
\end{figure}

The interpretation of conditionals is illustrated in \autoref{fig:denotational_sem cond}.

\begin{figure} [H]
\begin{equation}
\begin{split}
\begin{aligned}
&\hspace{-80pt} 
 \begin{minipage}[t]{0.3\textwidth}
$\begin{array}{c} 
     [\![\Gamma \triangleright v: \mathbb{A}]\!] = m   \\
    \hline
  [\![ \Gamma \triangleright \text{inl} (v):  \mathbb{A} \oplus \mathbb{B}  ]\!] = \textsc{Il}  \cdot m
\end{array}
$ \end{minipage}
\hspace{30pt} 
\begin{minipage}[t]{0.3\textwidth}
$\begin{array}{c}
     [\![\Gamma \triangleright v:\mathbb{B} ]\!]  = m  \\
    \hline
   [\![\Gamma \triangleright \text{inr} (v):  \mathbb{A} \oplus \mathbb{B}]\!]\!] = \textsc{Ir} \cdot m
\end{array}
$
\end{minipage}\\
\hspace{-25pt}
 \begin{minipage}[t]{0.3\textwidth}
$\begin{array}{c} 
    [\![\Gamma\triangleright v: \mathbb{A} \oplus \mathbb{B} ]\!] = b \quad [\![\Delta, x:\mathbb{A} \triangleright w: \mathbb{C} ]\!] = p  \quad  [\![\Delta,x:\mathbb{B} \triangleright w_{2}: \mathbb{C} ]\!] = q    \quad E \in \text{Sf}(\Gamma;\Delta)  \\
    \hline
  [\![E \triangleright \text{ cond } v \hspace{2pt}  \{\text{inl} (x) \Rightarrow w ; \hspace{1pt} \text{inr} (y) \Rightarrow u\}: \mathbb{C} ]\!] =   \text{either}(p,q) \cdot \text{dist} \cdot \text{sw} \cdot (b \otimes \text{id}) \cdot \text{sp}_{\Gamma;\Delta} \cdot \text{sh}_{E}
\end{array}
$ \end{minipage}
\end{aligned}
\end{split}
\end{equation}
\caption{Judgment interpretation for conditionals}
\label{fig:denotational_sem cond}
\end{figure}

\subparagraph{Proof} In order to validate the judgment interpretation for conditionals, it is necessary to demonstrate its correctness.

For the booleans: 
\begin{equation} \label{eq:proof_bool}
 \begin{aligned} 
    \hspace{120pt}&  [\![\Gamma ]\!]   \xrightarrow{\hspace{5pt}m\hspace{5pt}} [\![\mathbb{A} ]\!] \xrightarrow{\hspace{6pt}\textsc{Il}\hspace{6pt}} [\![\mathbb{A} \oplus \mathbb{B}]\!] \\ 
     &[\![\Gamma ]\!]   \xrightarrow{\hspace{5pt}m\hspace{5pt}} [\![\mathbb{B} ]\!] \xrightarrow{\hspace{6pt}\textsc{Ir}\hspace{6pt}} [\![\mathbb{A} \oplus \mathbb{B}]\!]
\end{aligned}   
\end{equation}
Now, for the conditional statement:
\begin{equation} \label{eq:proof_bool}
 \begin{aligned} 
    [\![E]\!] & \xrightarrow{\hspace{2pt}\text{sh}_{E}\hspace{2pt}}   [\![\Gamma,\Delta ]\!]   \xrightarrow{\hspace{1pt}\text{sp}_{\Gamma;\Delta}\hspace{1pt}}  [\![\Gamma ]\!] \otimes [\![\Delta ]\!] \xrightarrow{ b \hspace{1pt} \otimes \hspace{1pt} \text{id}} ([\![\mathbb{A} ]\!] \oplus [\![\mathbb{B} ]\!]) \otimes [\![\Delta ]\!] \xrightarrow{\hspace{2pt}\text{sw}\hspace{2pt}}  [\![\Delta ]\!] \otimes ([\![\mathbb{A} ]\!] \oplus [\![\mathbb{B} ]\!])  \\
    & \xrightarrow{\hspace{3pt}\text{dist}\hspace{3pt}} ([\![\Delta ]\!] \otimes [\![\mathbb{A} ]\!]  ) \oplus (  [\![\Delta ]\!] \otimes [\![\mathbb{B} ]\!] ) \xrightarrow{\hspace{1pt}\text{either}(p,q)\hspace{1pt}} \![\mathbb{C} ]\!]
\end{aligned}   
\end{equation}


The quantum lambda calculus with conditionals is illustrated with an example —the quantum teleportation protocol— in \autoref{appendice:teleport}.


While the validation of its correctness is ongoing, the metric equations for conditionals are presented in \autoref{fig:metric conditionals}. Note that the first two equations are redundant.
\begin{figure} [H]
\begin{equation*}
\begin{split}
\begin{aligned}
 &
\begin{minipage}[t]{0.3\textwidth}
$\begin{array}{c}
  v =_{q} w \\
    \hline
   \text{irl}(v) =_{q} \text{irl}(w)
\end{array}
$
\end{minipage}
\hspace{-30pt}
\begin{minipage}[t]{0.3\textwidth}
$\begin{array}{c}
   v =_{q} w \\
    \hline
   \text{inr}(v) =_{q} \text{inr}(w)
\end{array}
$ \end{minipage} \\
\hspace{-30pt}
&
\begin{minipage}[t]{0.3\textwidth}
$\begin{array}{c}
   v =_{q} v' \quad w=_{r} w' \quad u=_{s}u'   \\
    \hline
  \text{ cond } v \hspace{2pt}  \{\text{irl} (x) \Rightarrow w ; \hspace{1pt} \text{inr} (y) \Rightarrow u\} =_{\text{max}(q + r;q + s )} \text{ cond } v' \hspace{2pt}  \{\text{irl} (x) \Rightarrow w' ; \hspace{1pt} \text{inr} (y) \Rightarrow u'\} 
\end{array}
$ \end{minipage}
\end{aligned}
\end{split}
\end{equation*}
\caption{Metric equational system for condicionals}
\label{fig:metric conditionals}
\end{figure}

\vspace{-20pt}

\subsection{Ilustration: Noisy Quantum Teleportation}

\vspace{0pt}

To study decoherence in a quantum channel within the presented metric deductive system, one can consider the application of a dephasing channel in the quantum teleportation protocol with a certain probability $p$. This is exemplified for probabilities $p=0.5$ and $p=0.25$. It is worth noting that similar exercises can be done for scenarios such as a malicious attack involving a bit flip during measurement or the presence of a noisy channel.