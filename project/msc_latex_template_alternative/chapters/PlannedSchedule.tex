\chapter{Planned Schedule}

\section{Activities}

\newacronym{pdr}{PDR}{Preliminary Dissertation Report}
\newacronym{soa}{SOA}{State of the Art}

\begin{table}[H]
\begin{center}
\begin{tabular}{| c | c | c | c | c | c | c | c | c | c | c |}
\hline
\textbf{Task} & \textbf{Oct} & \textbf{Nov} & \textbf{Dec} & \textbf{Jan} & \textbf{Feb} & \textbf{Mar} & \textbf{Apr} & \textbf{May} & \textbf{Jun} & \textbf{Jul}\\
\hline
Background and \acrshort{soa} & $\bullet$ & $\bullet$ & $\bullet$ & & & & & & & \\
\hline
\acrshort{pdr} preparation & & $\bullet$ & $\bullet$ & $\bullet$ & & & & & & \\
\hline
Contribution & & & &$\bullet$ &$\bullet$ &$\bullet$ &$\bullet$ &$\bullet$ &$\bullet$ & \\
\hline
Writing up & & & & & & & $\bullet$ & $\bullet$ & $\bullet$ & $\bullet$ \\
\hline
\end{tabular}
\end{center}
\caption{Activities Plan}
\end{table}

%\chapter{Graded modalities} \label{chap:graded}

%{selinger2009quantum} -> type !qubit does not exist in the quantum lambda calculus

The linearity
constraint is often deemed too restrictive, prompting research into relaxing it in various computational paradigms. In \cite{dahlqvist2023complete}, the controlled use of a resource multiple times is explored within approximate program equivalence paradigms. Moreover, the grammar introduced allows the specification of how many times a resource can be used—a notion particularly relevant in quantum computation, especially within the NISQ era where resources are scarce.

\todo[inline,size=\normalsize]{Intro} 

\section{Syntax}

Here, the following grammar of types is used.
\begin{equation*} \label{eq:grammar_graded}
  \centering
   \mathbb{A} ::= X  \hspace{3 pt} \vert \hspace{3 pt} \mathbb{I}  \hspace{3 pt}  \vert \hspace{3 pt} \mathbb{A}  \otimes  \mathbb{A} \hspace{3 pt} \vert \hspace{3 pt} \mathbb{A} \oplus \mathbb{A} \hspace{3 pt}  \vert \hspace{3 pt}   \mathbb{A} \multimap  \mathbb{A} \vert \hspace{3 pt} !_{r} \mathbb{A} \hspace{100pt} {X \in G,r \in \mathbb{N}} 
  \end{equation*}


  \begin{figure} [H]
    {\small
    \begin{equation*}
    \begin{split}
    \begin{aligned}
    &
    \begin{minipage}[t]{0.3\textwidth}
    $\begin{array}{c}
         \Gamma_{i} \triangleright v_{i}: !_{r\cdot s_{i}} \mathbb{A}_{i} \quad x_{1}:!_{ s_{1}} \mathbb{A}_{1},\ldots, x_{n}:!_{s_{n}} \mathbb{A}_{n}\triangleright u: \mathbb{A} \quad E \in \text{Sf}(\Gamma_{1}; \ldots; \Gamma_{n})\\
        \hline
       E \triangleright \text{pr}_{(r,[s_1,\ldots,s_n])} v_1, \ldots, v_n \text{ fr } x_1, \ldots, x_n. \hspace{1pt} u : !_{r} \mathbb{A}
    \end{array}
    $
    \end{minipage}
    \hspace{193pt}
    (!_{i})
     \hspace{10pt}
    \begin{minipage}[t]{0.3\textwidth}
    $\begin{array}{c}
      \Gamma  \triangleright v : !_{1} \mathbb{A} \\
        \hline
        \Gamma \triangleright \text{dr } v:\mathbb{A}
    \end{array}
    $ \end{minipage}
    \hspace{-74pt} (!_{e}) \\
    &
    \begin{minipage}[t]{0.3\textwidth}
    $\begin{array}{c}
        \Gamma \triangleright v : !_{0} \mathbb{A} \hspace{7pt} \Delta \triangleright u: \mathbb{B} \hspace{7pt} E \in \text{Sf}(\Gamma, \Delta)\\
        \hline
       E \triangleright v.\hspace{1pt}u: \mathbb{B}
    \end{array}
    $
    \end{minipage}
    \hspace{35pt}
    (!_{0})
    \hspace{5pt}
    \begin{minipage}[t]{0.3\textwidth}
    $\begin{array}{c}
      \Gamma\triangleright v: !_{n+m} \mathbb{A} \hspace{7pt} \Delta,x:!_{n} \mathbb{A}, y:!_{m} \mathbb{B}\triangleright u: \mathbb{B} \hspace{7pt} E \in\text{Sf}(\Gamma, \Delta)\\
        \hline
       E \triangleright \text{cp}_{(n,m)} v \text{ to } x,y.  \hspace{1pt} u: \mathbb{B}
    \end{array}
    $ \end{minipage}
    \hspace{120pt} (!_{n+m}) \\
    \end{aligned}
    \end{split}
    \end{equation*}
    }  
    \caption{Term formation rules of graded lambda calculus.}
    \label{fig:typing_rules_graded}
    \end{figure}


\section{Interpretation}
%falar aqui do subspaço simetrico

\section{Quantum State Discrimination}
