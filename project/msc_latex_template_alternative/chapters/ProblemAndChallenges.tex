\chapter{The problem and its challenges}

The problem and its challenges.

\section{Images}
Example of inserting an image as displayed text,
\begin{center}
	\includegraphics[width=0.1\textwidth]{images/UM.jpg}
\end{center}

\begin{wrapfigure}{r}{0.15\textwidth}	
	\includegraphics[width=0.1\textwidth]{images/UM.jpg}
\end{wrapfigure}
\noindent --- wrapped into the text,
bla-bla bla-bla bla-bla bla-bla bla-bla bla-bla bla-bla bla-bla bla-bla bla-bla
bla-bla bla-bla bla-bla bla-bla bla-bla bla-bla bla-bla bla-bla bla-bla bla-bla
bla-bla bla-bla bla-bla bla-bla bla-bla bla-bla bla-bla bla-bla bla-bla bla-bla
bla-bla bla-bla bla-bla bla-bla bla-bla bla-bla bla-bla bla-bla bla-bla bla-bla
bla-bla bla-bla bla-bla bla-bla bla-bla bla-bla bla-bla bla-bla bla-bla bla-bla bla-bla bla-bla bla-bla bla-bla
bla-bla bla-bla bla-bla bla-bla bla-bla bla-bla bla-bla bla-bla bla-bla bla-bla bla-bla bla-bla bla-bla bla-bla

\noindent --- or as a floating body.
\begin{figure}
\begin{center}
	\includegraphics[width=0.25\textwidth]{images/UM.jpg}
\end{center}
\caption{Caption}
\end{figure}

\section{Acronyms and Glossary}
\newacronym{gcd}{GCD}{Greatest Common Divisor}
\newacronym{lcm}{LCM}{Least Common Multiple}
\newglossaryentry{maths}
{
    name=mathematics,
    description={Mathematics is what mathematicians do}
}
\newglossaryentry{latex}
{
    name=latex,
    description={Is a markup language specially suited for 
scientific documents}
}
\newglossaryentry{formula}
{
    name=formula,
    description={A mathematical expression}
}

Given a set of numbers, there are elementary methods to compute 
its \acrlong{gcd}, which is abbreviated \acrshort{gcd}. This process 
is similar to that used for the \acrfull{lcm}.

The \Gls{latex} typesetting markup language is specially suitable 
for documents that include \gls{maths}. \Glspl{formula} are rendered 
properly an easily once one gets used to the commands.