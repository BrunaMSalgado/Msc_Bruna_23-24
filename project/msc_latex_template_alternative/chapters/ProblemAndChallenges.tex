\chapter{The problem and its challenges}


\paragraph{Proof} In order to validate the metric equational system for conditionals, it is necessary to demonstrate its correctness.

The diamond norm is a particular instance of the operator norm. 
\vspace{15pt}

  The norm of a tuple is defined as the sum of the norms of its components, \textit{i.e.}, for any operators $v$ and $w$:
\begin{equation} \label{eq:norm_tuple}
  \lVert (v,w) \rVert = \lVert v \rVert + \lVert w \rVert
\end{equation}

For the \textbf{injections}:

Firstly, it is necessary to prove that the identity operator $I$ has a norm equal to 1.
\begin{lemma} \label{lemid}
  $ \lVert I \rVert_{\sigma} = 1   $
\end{lemma}

\textit{Proof.} \quad Using the definition of operator norm in \autoref{eq:op_norm}, it follows that:
\begin{equation} 
\begin{split}
  \lVert I \rVert_{\sigma} = \text{sup} \{\lVert I (v) \rVert \hspace{2pt} \vert \hspace{2pt}  \lVert v\rVert =1 \} = \text{sup} \{\lVert v \rVert \hspace{2pt} \vert \hspace{2pt}  \lVert v\rVert =1 \} = 1
\end{split}
\end{equation}

\vspace{10pt}

Thereafter, it is imperative to show that the injection operators $\textsc{Il}$ and $\textsc{Ir}$ are have a norm equal to 1.

\begin{lemma} \label{lemil}
  $ \lVert \textsc{Il} \rVert_{\sigma} = 1   $
\end{lemma}

\begin{lemma} \label{lemir}
  $ \lVert \textsc{Ir} \rVert_{\sigma} = 1   $
\end{lemma} 

\textit{Proof.} \quad Employing the definition of operator norm as defined in \autoref{eq:op_norm}, it ensues that:
\begin{equation} 
\begin{split}
  \lVert \textsc{Il} \rVert_{\sigma} &= \text{sup} \{\lVert \textsc{Il} (v) \rVert \hspace{2pt} \vert \hspace{2pt}  \lVert v\rVert =1 \} = \text{sup} \{\lVert (v,0) \rVert \hspace{2pt} \vert \hspace{2pt}  \lVert v\rVert =1 \} = \text{sup} \{\lVert v \rVert + \lVert 0 \rVert  \hspace{2pt} \vert \hspace{2pt}  \lVert v\rVert =1 \} \\
  & = \text{sup} \{\lVert v \rVert \hspace{2pt} + 0    \hspace{2pt}  \vert \lVert v\rVert =1 \} \hspace{160 pt} \text{ \{Positive definiteness\}} \\
  & = \text{sup} \{\lVert v \rVert \hspace{2pt} \vert \hspace{2pt}  \lVert v\rVert =1 \} = 1
\end{split}
\end{equation}

The proof for \autoref{lemir} is analogous to the proof for \autoref{lemil}.
\begin{equation} 
  \begin{split}
    \lVert \textsc{Ir} \rVert_{\sigma} &= \text{sup} \{\lVert \textsc{Ir} (v) \rVert \hspace{2pt} \vert \hspace{2pt}  \lVert v\rVert =1 \} = \text{sup} \{\lVert (0,v) \rVert \hspace{2pt} \vert \hspace{2pt}  \lVert v\rVert =1 \} = \text{sup} \{ \lVert 0 \rVert +\lVert v \rVert   \hspace{2pt} \vert \hspace{2pt}  \lVert v\rVert =1 \} \\
    & = \text{sup} \{0+\lVert v \rVert \hspace{2pt}     \hspace{2pt}  \vert \lVert v\rVert =1 \} \hspace{160 pt} \text{ \{Positive definiteness\}} \\
    & = \text{sup} \{\lVert v \rVert \hspace{2pt} \vert \hspace{2pt}  \lVert v\rVert =1 \} = 1
  \end{split}
  \end{equation}

Futhermore, given the submultiplicative property of the operator norm, for any super-operators $P$ and $Q$,where $\lVert P \rVert_{\sigma} =1  $ the following holds:
\begin{lemma}\label{lemleq}
  $\lVert PQ \rVert_{\sigma} \leq  \lVert Q \rVert_{\sigma}, \quad \lVert P \rVert_{\sigma}  =1 $ 
\end{lemma}

Using these properties it is possible to prove the validity of the metric equations for the injections. Demonstrating the correctness of the metric equations for the injections is equivalent to proving that for any  non‑negative rational $q$ and super-operators $v$ and $w$ such that $d(v,w) \leq q$, where  $d(v,w)$ represents the distance between $v$ and $w$ the following holds:

\begin{theorem} \label{theoremil}
  $d(\textsc{Il}(v),\textsc{Il} (w)) \leq q$
\end{theorem}
\begin{theorem} \label{theoremir}
  $d(\textsc{Ir}(v),\textsc{Ir} (w)) \leq q$
\end{theorem}
\vspace{10pt}
\textit{Proof.} \quad In the quantum paradigm, the distance between two super-operators $E$ and $E'$ corresponds to the diamond norm between $E$ and $E'$. Therefore,
\begin{equation}
\begin{split}
  d(v,w) \leq q \Leftrightarrow \lVert v \otimes I - w \otimes I \rVert_{\sigma} \leq q
\end{split}
\end{equation}

As a result, to prove that $d(\textsc{Il}(v),\textsc{Il} (w)) \leq q$, it suffices to show that:
\begin{align}
  \lVert \textsc{Il}\otimes I (v \otimes I)-\textsc{Il} \otimes I (w \otimes I)\rVert_{\sigma} \leq \lVert v \otimes I - w \otimes I \rVert_{\sigma} \\
  \lVert \textsc{Ir}\otimes I (v \otimes I)-\textsc{Ir} \otimes I (w \otimes I)\rVert_{\sigma} \leq \lVert v \otimes I - w \otimes I \rVert_{\sigma} 
\end{align}
Given that $\textsc{Il}$ and $\textsc{Ir}$ possess a norm equal to 1, as established by Lemmas \ref{lemil} and \ref{lemir} respectively, and considering the multiplicative property of the operator norm with respect to tensor products alongside the fact that the identity operator also exhibits a norm equal to 1, as demonstrated in  \autoref{lemid}, it follows that both $\lVert \textsc{Il} \otimes I \rVert_{\sigma}$ and $\lVert \textsc{Ir} \otimes I \rVert_{\sigma}$ are equal to one 1. Hence, by \autoref{lemleq},
\begin{align}
   \lVert \textsc{Il}\otimes I (v \otimes I)-\textsc{Il} \otimes I (w \otimes I)\rVert_{\sigma}=\lVert \textsc{Il}\otimes I (v \otimes I-w \otimes I)\rVert_{\sigma} \leq \lVert v \otimes I - w \otimes I \rVert_{\sigma} \\
   \lVert \textsc{Ir}\otimes I (v \otimes I)-\textsc{Ir} \otimes I (w \otimes I)\rVert_{\sigma}=\lVert \textsc{Ir}\otimes I (v \otimes I-w \otimes I)\rVert_{\sigma} \leq \lVert v \otimes I - w \otimes I \rVert_{\sigma}
\end{align}

\vspace{10pt}

Now, regarding the metric equation for the \textbf{conditional statement}, before validating its correctness, it is necessary to prove a few intermediate results. 

The first step is to demonstrate that for any super-operators $P$ and $Q$ the following holds:
\begin{lemma}\label{lem1}
  $\lVert [P,Q] \rVert_{\sigma} \leq \max \{ \lVert P \rVert_{\sigma}, \lVert Q \rVert_{\sigma} \}$
\end{lemma}



$\textit{Proof.}$ \quad Employing the definition of the operator norm in \autoref{eq:op_norm}, it follows that:
\begin{equation} \label{eq:cond_opnorm2}
  \begin{split}
  &\text{sup}{\{ \lVert [P,Q] (v) \rVert  \hspace{2pt} |  \hspace{2pt}  \lVert v \rVert=1  \}}  \leq \text{max} \{  \text{sup} \{ \lVert P (w) \rVert  \hspace{2pt} |  \hspace{2pt}  \lVert w \rVert =1 \}, \text{sup} \{\lVert Q (u) \rVert  \hspace{2pt} |  \hspace{2pt}  \lVert u \rVert=1  \} \} \\
  & = \text{sup}{\{ \lVert [P,Q] (w,u) \rVert  \hspace{2pt} |  \hspace{2pt}  \lVert w \rVert+ \lVert u \rVert=1  \}} \leq \text{max} \{  \text{sup} \{ \lVert P (w) \rVert  \hspace{2pt} |  \hspace{2pt}  \lVert w \rVert = 1, \lVert Q (u) \rVert  \hspace{2pt} |  \hspace{2pt}  \lVert u \rVert=1  \} \} \\
  & =  \text{sup}{\{ \lVert P (w) + Q (u) \rVert  \hspace{2pt} |  \hspace{2pt}  \lVert w \rVert+ \lVert u \rVert=1 \rVert=1  \}} \leq \text{max} \{  \text{sup} \{ \lVert P (w) \rVert  \hspace{2pt} |  \hspace{2pt}  \lVert w \rVert =1, \lVert Q (u) \rVert  \hspace{2pt} |  \hspace{2pt}  \lVert u \rVert=1  \} \} \\
  &  =  \text{sup}{\{ \lVert P (w) + Q (u) \rVert  \hspace{2pt} |  \hspace{2pt}  \lVert w \rVert+ \lVert u \rVert=1 \}} \leq \text{sup} \{  \text{max} \{ \lVert P (w) \rVert  \hspace{2pt} |  \hspace{2pt}  \lVert w \rVert =1, \lVert Q (u) \rVert  \hspace{2pt} |  \hspace{2pt}  \lVert u \rVert=1  \} \} \\
\end{split}
\end{equation}

Therefore, by the triangle inequality, proving the inequality in \autoref{eq:cond_opnorm3} suffices to establish  \autoref{lem1}.
\begin{equation} \label{eq:cond_opnorm3}
  \begin{split}
  \text{sup}{\{ \lVert P (w)  \rVert + \lVert Q (u)  \rVert  \hspace{2pt} |  \hspace{2pt}  \lVert w \rVert+ \lVert u \rVert=1  \}} \leq \text{sup} \{  \text{max} \{ \lVert P (w) \rVert  \hspace{2pt} |  \hspace{2pt}  \lVert w  \rVert =1, \lVert Q (u) \rVert  \hspace{2pt} |  \hspace{2pt}  \lVert u \rVert=1  \} \} \\
  \end{split}
\end{equation}


This can be rewritten as:

\begin{equation} 
  \begin{split}
    \lVert w \rVert+ \lVert u \rVert=1 \wedge \text{sup} \{ \lVert P (w)  \rVert + \lVert Q (u)  \rVert  \hspace{2pt}   \}  \leq \text{max}   \left\{ \dfrac{1}{\lVert w \rVert} \lVert P (w) \rVert  \hspace{2pt},  \dfrac{1}{\lVert u \rVert} \lVert Q (u) \rVert   \right\}
\end{split}
\end{equation}

As a result,
\begin{equation} 
  \begin{split}
    \lVert w \rVert+ \lVert u \rVert=1 \wedge \text{sup}{\{ \lVert P (w)  \rVert + \lVert Q (u)  \rVert    \}}  \leq \text{max}   \left\{  \left\lVert P \left( \dfrac{1}{\lVert w \rVert} w \right) \right\rVert  \hspace{2pt},  \left\lVert Q \left( \dfrac{1}{\lVert u \rVert} u \right) \right\rVert   \right\}
\end{split}
\end{equation}

This is equivalent to demonstrating that for $a+b=1$,
\begin{equation} 
\begin{split}
\hspace{110 pt}
    x + y  \leq  \max \left\{   \dfrac{1}{a}x  ,   \dfrac{1}{b} y   \right\} \\
\end{split}
\end{equation}

This is done by arguing by \textit{reductio ad absurdum}, \textit{i.e.}, supposing otherwise leads to a contradiction:
\begin{equation} 
\begin{split} 
    \hspace{90pt}&
     x + y  >  \max \left\{   \dfrac{1}{a}x  ,   \dfrac{1}{b} y   \right\} \\
    & \Rightarrow  x + y > \dfrac{1}{a}x  \wedge x + y > \dfrac{1}{b}y \\
    & \Rightarrow  a (x + y) > x  \wedge b (x + y)> y \\
    & \Rightarrow  a x + a y > x  \wedge b x + by > y \\
    & \Rightarrow  a x + a y > x  \wedge (1-a) x + (1-a)y > y\\
    & \Rightarrow  a x + a y > x  \wedge x-ax + y -ay > y\\
    & \Rightarrow  x < a x + a y   \wedge x > a x + a y  \\
\end{split}
\end{equation}

\vspace{10pt}

Subsequently, it is imperative to prove that:
\begin{lemma}\label{lemiso}
  $ i= [\textsc{Il} \otimes I, \textsc{Ir} \otimes I ]$ \text{is an isomorphism}.
\end{lemma}

\textit{Proof.} \quad The proof is as follows:

For any vector spaces $V$, $W$, and $U$, $i: (V \otimes U) \oplus (W \otimes U) \xrightarrow{} (V  \oplus W) \otimes U $. If $V$ has dimension $m$, $W$ has dimension $n$, and $U$ has dimension $o$, then the space $(V \otimes U) \oplus (W \otimes U) $ has dimension $mo+no=(m+n)\cdot o$. Similarly, the space $(V\oplus W) \otimes U$ has dimension $(m+n)\cdot o$. Hence, the spaces have the same dimension. Given that spaces with the same dimension are isomorphic \cite{hefferon2006linear}, it follows that $i$ is an isomorphism.

\vspace{10pt}

Next, it is necessary to demonstrate that for any operators $P$ and $Q$, the identity operator $I$, and an isomorphism $i=[\textsc{Il} \otimes I, \textsc{Ir} \otimes I ]$ the following holds:

\begin{lemma}\label{lem2}
  $( [P,Q] \otimes I) \cdot  i  = [P \otimes I, Q \otimes I]$
\end{lemma}

Which is equivalent to showing that for any vector spaces $V$, $W$, $U$, and $Z$  and super-operators $P: V \xrightarrow{} Z$, $Q: W \xrightarrow{} Z$, and $I: U \xrightarrow{} U$, the following diagram holds:

\vspace{10pt}


\begin{tikzpicture}
  \matrix (m) [matrix of math nodes,row sep=4em,column sep=7em,minimum width=2em]
  {
    V \otimes U \oplus W \otimes U & (V  \oplus W) \otimes U \\
     Z \otimes U \\
  };
  \path[-stealth]
    (m-1-1) edge node [left] {$[P \otimes I, Q \otimes I]$} (m-2-1)
    (m-1-1) edge node [above] {$i$} (m-1-2)
    (m-1-2) edge node [right=0.2cm] {$[P,Q] \otimes I$} (m-2-1);
\end{tikzpicture}


\vspace{10pt}

\textit{Proof.} \quad The proof is straightforward:
\begin{equation}
\begin{split}
    & ( [P,Q] \otimes I) \cdot  [\textsc{Il} \otimes I, \textsc{Ir} \otimes I ]  \\
    &=  [([P,Q] \otimes I) \cdot (\textsc{Il} \otimes I),([P,Q] \otimes I) \cdot (\textsc{Ir} \otimes I) ]\\
    &=  [P \otimes I, Q \otimes I]
\end{split}
\end{equation}

\vspace{15pt}

Furhtermore, it is imperative to show that the following relation holds:

\begin{lemma}\label{lemi-1}
  $ [P \otimes I, Q \otimes I] \cdot  i^{-1}  = [P,Q] \otimes I$
\end{lemma}

Demonstrating this is equivalent to establishing that for any vector spaces $V$, $W$, $U$, and $Z$, and super-operators $P: V \xrightarrow{} Z$, $Q: W \xrightarrow{} Z$, and $I: U \xrightarrow{} U$, the following diagram commutes:

\vspace{10pt}

\begin{tikzpicture}
  \matrix (m) [matrix of math nodes,row sep=4em,column sep=7em,minimum width=2em]
  {
    V \otimes U \oplus W \otimes U & (V  \oplus W) \otimes U \\
     Z \otimes U \\
  };
  \path[-stealth]
    (m-1-1) edge node [left] {$[P \otimes I, Q \otimes I]$} (m-2-1)
    (m-1-2) edge node [above] {$i^{-1}$} (m-1-1)
    (m-1-2) edge node [right=0.2cm] {$[P,Q] \otimes I$} (m-2-1);
\end{tikzpicture}


\textit{Proof.} \quad The proof is as follows:
\begin{equation}
\begin{split}
    & ( [P,Q] \otimes I) \cdot  i  = [P \otimes I, Q \otimes I]  \hspace{100pt} & \text{\{\autoref{lem2}\}} \\
    \Leftrightarrow &  \hspace{2pt} ( [P,Q] \otimes I) \cdot  i \cdot i^{-1} = [P \otimes I, Q \otimes I] \cdot  i^{-1}\\
    \Leftrightarrow &  \hspace{2pt} ( [P,Q] \otimes I)  = [P \otimes I, Q \otimes I] \cdot  i^{-1}  &\text{\{\autoref{lemiso}\}} \\
\end{split}
\end{equation}

\vspace{10pt}
With \autoref{lem2} and \autoref{lemi-1}, it has been proved that the diagram below is valid:
\vspace{5pt}

\begin{tikzpicture}
  \matrix (m) [matrix of math nodes,row sep=4em,column sep=7em,minimum width=2em]
  {
    V \otimes U \oplus W \otimes U & (V  \oplus W) \otimes U \\
     Z \otimes U \\
  };
  \path[-stealth]
    (m-1-1) edge node [left] {$[P \otimes I, Q \otimes I]$} (m-2-1)
    edge[bend left=5] node [above] {$i$}  (m-1-2) % Adjusted minimum width
    (m-1-2) edge node [right=0.5cm] {$[P,Q] \otimes I$} (m-2-1)
    (m-1-2) edge[bend right=-5] node [below] {$i^{-1}$} (m-1-1); % Added the label to the arrow
\end{tikzpicture}

\vspace{10pt}




%Next, it is necessary to demonstrate that the coproduct of two super-operators $P$ and $Q$ has a norm equal to 1.
%\begin{lemma} \label{lemeither}
  %$  \lVert [P, Q]  \rVert_{\sigma} = 1   $
%\end{lemma}

%\textit{Proof.} \quad Utilizing the definition of the operator norm as defined in Equation \ref{eq:op_norm}, it follows that:
%\begin{equation} 
  %\begin{split}
    %\lVert [P, Q]  \rVert_{\sigma}  \\
  %\end{split}
  %\end{equation}
%\vspace{10pt}

Now, it is possivel to prove that $i$ has a norm equal to 1.

\begin{lemma} \label{lem3}
  $  \lVert i\rVert_{\sigma} \geq 1 $
\end{lemma}

\vspace{10pt}

\textit{Proof.} \quad Considering the vector $(v \otimes u, 0)$ with $\lVert(v \otimes u, 0)\rVert = 1$, and  attending the multiplicative property of the operator norm with respect to tensor products, along with the definition of the norm of a tuple as in \autoref{eq:norm_tuple}, it holds that $\lVert v \rVert = 1$ and $\lVert u \rVert =1$. Therefore, using this same property and definition, it is possible to demonstrate that the following holds:
  \begin{equation}
    \begin{split}
      \lVert [\textsc{Il} \otimes I, \textsc{Ir} \otimes I ] (v \otimes u, 0) \rVert = \lVert(v, 0) \otimes u \rVert = (\lVert v \rVert + \lVert 0 \rVert ) \lVert u \rVert = \lVert v \rVert \lVert u \rVert =1
    \end{split}
  \end{equation}
 
Given the definition of the operator norm as presented in \autoref{eq:op_norm}, it follows that:
\begin{equation}
  \begin{split}
      & \hspace{3pt} \lVert [\textsc{Il} \otimes I, \textsc{Ir} \otimes I ]  \rVert_{\sigma}  = \text{sup} \{ \lVert [\textsc{Il} \otimes I , \textsc{Ir} \otimes I ] (a) \rVert \hspace{2pt} | \hspace{2pt} \lVert a \rVert = 1 \} \\
  \end{split}
  \end{equation}
  From this, it can be deduced that $\lVert i \rVert_{\sigma} \geq 1$.

Subsequently, it is possible to demontrate that $i^{-1}$ has a norm greater than or equal to 1,

\begin{lemma} \label{lem4}
  $  \lVert i^{-1}  \rVert_{\sigma} \leq 1 $
\end{lemma}

\textit{Proof.} \quad Given that $i$ is an isomophism, it follows that 
\begin{equation} 
  \begin{split}
    &\lVert i \cdot i^{-1}  \rVert_{\sigma} = 1  \\
    \leq \hspace{2pt}& \lVert i  \rVert_{\sigma} \cdot \lVert i^{-1}  \rVert_{\sigma} = 1 \hspace{50pt} & \text{\{Norm submultiplicative with respect to compositions\}} \\
    \leq & 1 \cdot \lVert i^{-1}  \rVert_{\sigma} = 1 & \text{\{\autoref{lem4}\}}  \\
    \Leftrightarrow &  \lVert i^{-1}  \rVert_{\sigma} = 1  \\
  \end{split}   
  \end{equation}




Next, one has to prove that for any super-operators $P$ and $Q$ and their respective erroneous versions $P'$ and $Q'$, the following holds:
  \begin{lemma} \label {lemmasum}
    $  \lVert P\cdot Q \otimes I - P'\cdot Q'  \otimes I \rVert_{\sigma} \leq  \lVert (P - P') \otimes I  \rVert_{\sigma} + \lVert (Q - Q') \otimes I \rVert_{\sigma}   $
  \end{lemma} 
  
  \textit{Proof.} \quad Applying the triangle inequality, he submultiplicative property of the operator norm with respect to compositions, and given that a positive and trace-preserving operator map, $E$, has norm $\lVert E \otimes I  \rVert_{\sigma} =1$ (\cite{watrous2018theory}), it follows that:
  
  \begin{equation}
    \begin{split}
      & \lVert P\cdot Q \otimes I - P'\cdot Q' \otimes I  \rVert_{\sigma}  \\
      &= \lVert  P\cdot Q \otimes I- P\cdot Q' \otimes I + P\cdot Q' \otimes I - P'\cdot Q' \otimes I  \rVert_{\sigma}  \\
      &\leq \lVert P\cdot Q \otimes I - P\cdot Q' \otimes I  \rVert_{\sigma} + \lVert P\cdot Q' \otimes I - P'\cdot Q' \otimes I  \rVert_{\sigma}  \\
      &\leq \lVert P \rVert_{\sigma} \lVert Q \otimes I - Q' \otimes I  \rVert_{\sigma} + \lVert P \otimes I - P' \otimes I  \rVert_{\sigma} \lVert Q'  \rVert_{\sigma}  \\
      &= \lVert P \rVert_{\sigma} \lVert (Q  - Q') \otimes I  \rVert_{\sigma} + \lVert (P  - P') \otimes I  \rVert_{\sigma} \lVert Q'  \rVert_{\sigma}  \\
      &= \lVert (P - P') \otimes I  \rVert_{\sigma} + \lVert (Q - Q') \otimes I  \rVert_{\sigma}  \\
    \end{split}
    \end{equation}

\vspace{5pt}

Finally, considering the the semantics  the conditional statement  in \autoref{fig:denotational_sem cond}, demonstrating the conditional statement rule in \autoref{fig:metric conditionals} includes proving that for any super-operators $P$, $Q$, $P'$ and $Q'$,  denoting the distance between super-operators $A$ and $B$ as $d(A,B)$,  the following holds:
\begin{lemma} \label {lemma_max_otimes}
  $\text{d} ([P,Q],[P',Q']) \leq \text{max} \{\text{d} (P,P'),\text{d} (Q,Q')\}$
\end{lemma}
\vspace{10pt}
\textit {Proof.} 
In the quantum paradigm, the distance between two super-operators  corresponds to the diamond norm between the two super-operators. Hence, denoting $ [\textsc{Il} \otimes I, \textsc{Ir} \otimes I ]$ by $i$ it follows that:

%\begin{equation}
%\begin{split}
  %& \text{d} ([P,Q],[P',Q'])  \\
  %&=   \lVert  [P,Q] \otimes I - [P',Q'] \otimes I   \rVert_{1}  \\
  %&=   \lVert [P \otimes I, Q \otimes I]  - [P' \otimes I, Q' \otimes I]  \rVert_{1}  \\
  %&=  \lVert [P - P' \otimes I, Q-Q' \otimes I]  \rVert_{1}   \\
  %&= \lVert [P -P', Q-Q' ] \otimes I \cdot i \rVert_{1}  \\
%\end{split}
%\end{equation}

\begin{equation} \label{eq:proof_theorem1.1_esq}
  \begin{split}
    & \text{d} ([P,Q],[P',Q'])  \\
    &=  \lVert  [P,Q] \otimes I - [P',Q'] \otimes I   \rVert_{\sigma}  \\
    &=   \lVert [P \otimes I, Q \otimes I] \cdot i^{-1}  - [P' \otimes I, Q' \otimes I]  \cdot i^{-1}  \rVert_{\sigma}   \hspace{165pt}  \text{\{\autoref{lemi-1}\}} \\
    &=  \lVert [P - P' \otimes I, Q-Q' \otimes I] \cdot i^{-1}  \rVert_{\sigma}   \\
    & \leq \lVert [P - P' \otimes I, Q-Q' \otimes I]  \rVert \lVert i^{-1}  \rVert \rVert_{\sigma} \hspace{20pt} \text{\{Norm submultiplicative with respect to compositions\}}  \\  
    & \leq \lVert [(P - P') \otimes I, (Q-Q') \otimes I]  \rVert_{\sigma} \hspace{235pt} \text{ \{\autoref{lem4}\}} \\
  \end{split}
  \end{equation}
and
\begin{equation} \label {eq:proof_theorem1.1_dir}
\begin{split}
   &  \text{max} \{\text{d} (P,P'),\text{d} (Q,Q')\} \\
   = &  \text{max}\{ \lVert P \otimes I - P' \otimes I \rVert_{\sigma}, \lVert Q \otimes I - Q'\otimes I \rVert_{\sigma} \}\\
   = &  \text{max}\{ \lVert (P - P') \otimes I \rVert_{\sigma}, \lVert (Q - Q') \otimes I \rVert_{\sigma} \}\\
\end{split}
\end{equation}

Finally, by  \autoref{lem1}, it can be deduced that $\text{d} ([P,Q],[P',Q']) \leq \text{max} \{\text{d} (P,P'),\text{d} (Q,Q')\}$, which concludes the proof of theorem \autoref{lemma_max_otimes}.
\vspace{10pt}


An alternative method to establish \autoref{theorem:1.1} is now presented.
\vspace{5pt}


\textit {Proof.} The proof is as follows:
\begin{equation}
  \begin{split}
    & \text{d} ([P,Q],[P',Q'])  \\
    &=   \lVert  [P,Q] \otimes I - [P',Q'] \otimes I    \rVert_{\sigma} \hspace{2pt} \\
    &=   \lVert  ([P,Q]  - [P',Q']) \otimes I    \rVert_{\sigma} \hspace{2pt} \\
    &=   \lVert  [P-P',Q-Q'] \otimes I  \rVert_{\sigma}   \\
    &=    \lVert  [P-P',Q-Q'] \rVert_{\sigma} \lVert I \rVert_{\sigma}\hspace{2pt} & \hspace {20pt} \text{\{Norm multiplicative with respect to tensor products\}} \\ 
    &=    \lVert  [P-P',Q-Q'] \rVert_{\sigma} & \text{\{\autoref{lemid}\}}  \\
  \end{split}
  \end{equation}
Moreover,
\begin{equation}
  \begin{split}
     &  \text{max} \{\text{d} (P,P'),\text{d} (Q,Q')\} \\
     = &  \text{max}\{ \lVert P \otimes I - P' \otimes I \rVert_{\sigma}, \lVert Q \otimes I - Q'\otimes I \rVert_{\sigma} \}\\
     = &  \text{max}\{ \lVert (P - P') \otimes I \rVert_{\sigma}, \lVert (Q - Q') \otimes I \rVert_{\sigma} \}\\
     = &\text{max}\{ \lVert (P - P') \rVert_{\sigma} \lVert  I \rVert_{\sigma}, \lVert (Q - Q') \rVert_{\sigma} \lVert I \rVert_{\sigma} \} & \hspace{60pt} \text{\{Norm multiplicative with}\\
     && \text{respect to tensor products\}} \\
     = & \text{max}\{ \lVert (P - P') \rVert_{\sigma}, \lVert (Q - Q') \rVert_{\sigma}  \}  & \text{\{\autoref{lemid}\}}  \\
    \end{split}
  \end{equation}

Therefore, by \autoref{lem1}, it can be deduced that $\text{d} ([P,Q],[P',Q']) \leq \text{max} \{\text{d} (P,P'),\text{d} (Q,Q')\}$, which concludes the proof of theorem \autoref{lemma_max_otimes}.

\vspace{10pt}
  


Now, it is finally possible to adress the proof of the metric equation for the conditional statement as a whole. Considering the the semantics of the conditional statement in \autoref{fig:denotational_sem cond}, the rule for the conditional statement in \autoref{fig:metric conditionals} is valid is equivalent to demonstrating that the distance between the evalution of a boolen $B$ followed by the execution of a program $P$ or a program $Q$ and the evalution of a boolean $B'$ followed by the execution of a program $P'$ or a program $Q'$ is less or equal to the  distance between the evaluation of the boolean $B$ and the evaluation of the boolean $B'$ plus the maximum distance between the execution of the programs $P$ and $P'$ and the execution of the programs $Q$ and $Q'$, \textit{ergo}, that for any booleand $B$ and $B'$ super-operators $P$, $Q$, $P'$ and $Q'$, the following holds:

\begin{theorem} \label {theorem:1.1}
  $ \text{d} (B \cdot [P,Q], B' \cdot [P',Q']) \leq \text{d} (B,B') + \text{max} \{\text{d} (P,P'),\text{d} (Q,Q')\}$
\end{theorem}
\vspace{10pt}

\textit {Proof.} Considering that in the quantum paradigm, the distance between two super-operators  corresponds to the diamond norm between the two super-operators, it follows that:
\begin{equation}
\begin{split}
  & \text{d} (B \cdot [P,Q], B' \cdot [P',Q'])  \\
  &=   \lVert  B \cdot [P,Q] \otimes I - B' \cdot [P',Q'] \otimes I   \rVert_{\sigma}  \\
  & \leq \lVert  (B - B')  \otimes I   \rVert_{\sigma} + \lVert  ([P,Q] - [P',Q']) \otimes I   \rVert_{\sigma} & \hspace{100 pt}  \text{\{\autoref{lemmasum}\}} \\
  &= d(B,B') + \lVert  [P,Q]\otimes I - [P',Q'] \otimes I   \rVert_{\sigma} & \hspace{100 pt} \\
  &=  \text{d} (B,B') + \text{d} ([P,Q],[P',Q'])    \\
  &=d(B,B') + \text{max} \{\text{d} (P,P'),\text{d} (Q,Q')\} & \text{\{\autoref{lemma_max_otimes}\}} \\ 
\end{split}
\end{equation}

%
%
%
%
%
%
%
%
%
% Cenas necessárias v1

%
%
%
%
%
%
%
%
%
% lemmas v1

\begin{theorem} \label{theorem:tensor_stability} 
  For all super-operators $Q: \mathbb{C}^{o_1 \times o_1} \oplus \ldots \oplus \mathbb{C}^{o_n \times o_n}  \rightarrow \mathbb{C}^{p_1 \times p_1} \oplus \ldots \oplus  \mathbb{C}^{p_m \times p_m}$ and complex spaces $\mathbb{C}^{q_1 \times q_1} \oplus \ldots \oplus \mathbb{C}^{q_t \times q_t}$  it holds that:
      \begin{equation}
        \lVert Q \otimes I_{\mathbb{C}^{q_1 \times q_1} \oplus \ldots \oplus \mathbb{C}^{q_t \times q_t}} \rVert_{1 \text{ gen}} \leq  \lVert Q \rVert_{\diamondsuit \text{ gen}}
      \end{equation}
      with equality holding under the assumption that $\text{dim}(\mathbb{C}^{q_1 \times q_1} \oplus \ldots \oplus \mathbb{C}^{q_t \times q_t}) \geq  \text{dim}(\mathbb{C}^{o_1 \times o_1} \oplus \ldots \oplus \mathbb{C}^{o_n \times o_n})$.
      \end{theorem}
  
  \begin{proof}
    \begin{align*}
      & \hspace{-30pt} \lVert Q \otimes I_{\mathbb{C}^{q_1 \times q_1} \oplus \ldots \oplus \mathbb{C}^{q_t \times q_t}} \rVert_{1 \text{ gen}} \\
      & \hspace{-30pt}  = \max \{ \lVert Q_{11}  \otimes I_{\mathbb{C}^{q_1 \times q_1}}  \rVert_{1}  + \ldots +  \lVert Q_{1m}  \otimes I_{\mathbb{C}^{q_1 \times q_1}} \rVert_{1},  \ldots ,  \lVert Q_{11} \otimes I_{\mathbb{C}^{q_t \times q_t}} \rVert_{1}   \quad \{\text{\autoref{eq:gen_tensor_identity}, } \\
      & \hspace{-18pt}  + \ldots + \lVert Q_{1m} \otimes I_{\mathbb{C}^{q_t \times q_t}} \rVert_{1}, \ldots,  \lVert Q_{n1}  \otimes I_{\mathbb{C}^{q_1 \times q_1}}  \rVert_{1} + \ldots +  \hspace{80 pt} \text{\autoref{def:gen_1norm}} \} \\
      & \hspace{-16pt}\lVert Q_{nm}  \otimes I_{\mathbb{C}^{q_1 \times q_1}} \rVert_{1}, \ldots,  \lVert Q_{n1} \otimes I_{\mathbb{C}^{q_t \times q_t}} \rVert_{1} + \ldots + \lVert Q_{nm} \otimes I_{\mathbb{C}^{q_t \times q_t}} \rVert_{1} \} \\
      &  \hspace{-30pt}  \leq   \max \{ \|Q_{11}\|_{\diamondsuit} + \ldots + \|Q_{1m}\|_{\diamondsuit}, \hspace{2pt} \ldots \hspace{2pt},  \|Q_{11}\|_{\diamondsuit} + \ldots + \|Q_{1m}\|_{\diamondsuit}, , \hspace{2pt} \ldots \hspace{2pt},   \hspace{12 pt}  \{ \text{\autoref{thm:tensor_stability}}\}\\
      & \|Q_{n1}\|_{\diamondsuit} + \ldots + \|Q_{nm}\|_{\diamondsuit}, \hspace{2pt} \ldots \hspace{2pt}, \|Q_{n1}\|_{\diamondsuit} + \ldots + \|Q_{nm}\|_{\diamondsuit} \} \\
      &  \hspace{-30pt} = \max \{\|Q_{11}\|_{\diamondsuit} + \ldots + \|Q_{1m}\|_{\diamondsuit}, \hspace{2pt} \ldots \hspace{2pt}, \|Q_{n1}\|_{\diamondsuit} + \ldots + \|Q_{nm}\|_{\diamondsuit} \} \\
      & \hspace{-30pt}  = \lVert Q \rVert_{\diamondsuit \text{ gen}}  \hspace{308 pt}  \{ \text{\autoref{def:gen_diamond_norm}}\}
    \end{align*} 
  
  Note that if  $\text{dim}(\mathbb{C}^{q_1 \times q_1} \oplus \ldots \oplus \mathbb{C}^{q_t \times q_t}) \geq  \text{dim}(\mathbb{C}^{o_1 \times o_1} \oplus \ldots \oplus \mathbb{C}^{o_n \times o_n})$, at least one of the vector spaces ${\mathbb{C}^{q_k \times q_k}}$, in the direct sum $\mathbb{C}^{q_1 \times q_1} \oplus \ldots \oplus \mathbb{C}^{q_t \times q_t}$ has higher dimension that any of the vector spaces ${\mathbb{C}^{q_l \times q_l}}$, in the direct sum $\mathbb{C}^{o_1 \times o_1} \oplus \ldots \oplus \mathbb{C}^{o_n \times o_n}$. Consequentlty, in such a case, there is a vector space $\mathbb{C}^{q_k \times q_k}$ such that, by \autoref{thm:tensor_stability},  $ \lVert Q_{ij} \otimes I_{\mathbb{C}^{q_k \times q_k}} \rVert_1 = {\|Q_{ij}\|_{\diamondsuit}}$ for all $1\leq i \leq n$, $1 \leq j \leq m$. As a result, when $\text{dim}(\mathbb{C}^{q_1 \times q_1} \oplus \ldots \oplus \mathbb{C}^{q_t \times q_t}) \geq  \text{dim}(\mathbb{C}^{o_1 \times o_1} \oplus \ldots \oplus \mathbb{C}^{o_n \times o_n})$,
  
    \begin{align*}
      & \hspace{-30pt} \lVert Q \otimes I_{\mathbb{C}^{q_1 \times q_1} \oplus \ldots \oplus \mathbb{C}^{q_t \times q_t}} \rVert_{1 \text{ gen}} \\
      & \hspace{-30pt}  = \max \{ \lVert Q_{11}  \otimes I_{\mathbb{C}^{q_1 \times q_1}}  \rVert_{1}  + \ldots +  \lVert Q_{1m}  \otimes I_{\mathbb{C}^{q_1 \times q_1}} \rVert_{1},  \ldots ,  \lVert Q_{11} \otimes I_{\mathbb{C}^{q_t \times q_t}} \rVert_{1}   \quad \{\text{\autoref{eq:gen_tensor_identity}, } \\
      & \hspace{-18pt}  + \ldots + \lVert Q_{1m} \otimes I_{\mathbb{C}^{q_t \times q_t}} \rVert_{1}, \ldots,  \lVert Q_{n1}  \otimes I_{\mathbb{C}^{q_1 \times q_1}}  \rVert_{1} + \ldots +  \hspace{80 pt} \text{\autoref{def:gen_1norm}} \} \\
      & \hspace{-16pt}\lVert Q_{nm}  \otimes I_{\mathbb{C}^{q_1 \times q_1}} \rVert_{1}, \ldots,  \lVert Q_{n1} \otimes I_{\mathbb{C}^{q_t \times q_t}} \rVert_{1} + \ldots + \lVert Q_{nm} \otimes I_{\mathbb{C}^{q_t \times q_t}} \rVert_{1} \} \\
      &  \hspace{-30pt}  = \max \{ \|Q_{11} \otimes \mathbb{C}^{q_k \times q_k}\|_{1} + \ldots + \|Q_{1m} \otimes \mathbb{C}^{q_k \times q_k}\|_{1}, \hspace{2pt} \ldots \hspace{2pt} , \hspace {80 pt}  \{ \text{\autoref{thm:tensor_stability}}\}\\
      & \hspace{-16pt} \|Q_{n1} \otimes \mathbb{C}^{q_k \times q_k}\|_{1} + \ldots + \|Q_{nm} \otimes \mathbb{C}^{q_k \times q_k}\|_{1} \} \\
      &  \hspace{-30pt} = \max \{\|Q_{11}\|_{\diamondsuit} + \ldots + \|Q_{1m}\|_{\diamondsuit}, \hspace{2pt} \ldots \hspace{2pt}, \|Q_{n1}\|_{\diamondsuit} + \ldots + \|Q_{nm}\|_{\diamondsuit} \} \hspace{40pt}\{ \text{\autoref{thm:tensor_stability}}\} \\
      & \hspace{-30pt}  = \lVert Q \rVert_{\diamondsuit \text{ gen}}  \hspace{308 pt}  \{ \text{\autoref{def:gen_diamond_norm}}\}
    \end{align*} 
  \end{proof}
  
  \begin{corollary} \label{cor:tensor_stability}
    For all super-operators  $Q: \mathbb{C}^{o_1 \times o_1} \oplus \ldots \oplus \mathbb{C}^{o_n \times o_n}  \rightarrow \mathbb{C}^{p_1 \times p_1} \oplus \ldots \oplus  \mathbb{C}^{p_m \times p_m}$ and complex spaces $\mathbb{C}^{q_1 \times q_1} \oplus \ldots \oplus \mathbb{C}^{q_t \times q_t}$   it holds that:
  \begin{equation}
     \lVert Q \otimes I_{\mathbb{C}^{q_1 \times q_1} \oplus \ldots \oplus \mathbb{C}^{q_t \times q_t}} \rVert_{\diamondsuit \text{ gen}} = \lVert Q \rVert_{\diamondsuit \text{ gen}} 
  \end{equation}
  \end{corollary}
  
  \begin{lemma}\label{lem:q(o)}
    Let  $Q: \mathbb{C}^{o_1 \times o_1} \oplus \ldots \oplus \mathbb{C}^{o_n \times o_n}  \rightarrow \mathbb{C}^{p_1 \times p_1} \oplus \ldots \oplus  \mathbb{C}^{p_m \times p_m}$ be a superoperator, then for $O \in \mathbb{C}^{o_1 \times o_1} \oplus \ldots \oplus  \mathbb{C}^{o_m \times o_m}$ it holds that:
    \begin{equation} \label{eq:qo<q}
      \lVert Q(O) \rVert_{1 \text{ gen}} \leq \lVert Q  \rVert_{1 \text{ gen}} \cdot \lVert O  \rVert_{1 \text{ gen}}.
    \end{equation}
  \end{lemma}
  
  \begin{proof}
    $O$ can be written as  $O = \underbrace{\textsc{Il} \cdot \ldots \cdot \textsc{Il}}_{n-1 \times} \cdot \hspace{2pt} O_{1} + \ldots +  \underbrace{\textsc{Ir} \cdot \ldots \cdot \textsc{Ir}}_{n-1 \times} \cdot \hspace{2pt} O_{n} 
  $, where for each $1 \leq i \leq n$, $O_{i} \in \mathbb{C}^{p_i \times p_i}$ and $O_i =  \underbrace{\textsc{Pl} \cdot \ldots \cdot \textsc{Pl}}_{n-i \times} \cdot \underbrace{\textsc{Pr} \cdot \ldots \cdot \textsc{Pr}}_{i-1 \times} \cdot O $. Considering \autoref{def:gen_norm}, it follows that: 
  \begin{equation}
    \lVert O  \rVert_{1 \text{ gen}} = \lVert O_1 \rVert_{1} + \ldots + \lVert O_n \rVert_{1}.
  \end{equation}
  
  Applying $O$ to $Q$ results in:
  \begin{equation}
  \begin{split}
  Q(O) & = \underbrace{\textsc{Il} \cdot \ldots \cdot \textsc{Il}}_{m-1 \times} \cdot \hspace{1pt} Q_{11}  (O_{1}) + \ldots +   \underbrace{\textsc{Ir} \cdot \ldots \cdot \textsc{Ir}}_{m-1 \times}\cdot \hspace{1pt} Q_{1m} (O_{1}) + \ldots +  \underbrace{\textsc{Il} \cdot \ldots \cdot \textsc{Il}}_{m-1 \times} \cdot\hspace{1pt} Q_{n1} (O_{n}) +  \ldots \\
  & \hspace{10pt}  + \underbrace{\textsc{Ir} \cdot \ldots \cdot \textsc{Ir}}_{m-1 \times}\cdot \hspace{1pt} Q_{nm}  (O_{n})
  \end{split}
  \end{equation}
  
  As a result, the generalized trace norm of $Q(O)$ corresponds to:
  \begin{equation} \label{eq:qo}
    \begin{split}
    \lVert Q(O)  \rVert_{1 \text{ gen}} & = \lVert Q_{11} (O_{1}) \rVert_{1} + \ldots + \lVert Q_{1m} (O_{1}) \rVert_{1} + \ldots +  \lVert Q_{n1} (O_{m})  \rVert_{1} +  \ldots +  \lVert Q_{nm} (O_{n}) \rVert_{1}. 
    \end{split}
  \end {equation}
  The generalized trace norm of $Q$ is given by:
  \begin{equation}
    \begin{split} \label{eq:q}
    &\lVert Q  \rVert_{1 \text{ gen}} = \max \{ \lVert Q_{11} \rVert_{1} + \ldots + \lVert Q_{1m} \rVert_{1}, \hspace{2pt} \ldots \hspace{2pt}, \lVert Q_{n1} \rVert_{1} + \ldots + \lVert Q_{nm} \rVert_{1} \} \\
   & = \max \{ \max \{ \lVert Q_{11} (A_{1}) \rVert_{1} \hspace{1pt}  \vert \hspace{1pt}  \lVert A_{1} \rVert_{1} = 1 \} + \ldots + &  \{\text{\autoref{def:trace_norm_superoperator}}\} \\
   & \hspace{15pt}  \max \{  \lVert Q_{1m} (A_{1}) \rVert_{1} \hspace{1pt}  \vert   \lVert A_{1} \rVert_{1} = 1 \} , \hspace{1pt} \ldots \hspace{1pt}, \max \{ \lVert Q_{n1} (A_{n}) \rVert_{1} \hspace{1pt}  \vert \hspace{1pt}  \lVert A_{n} \rVert_{1} = 1 \}   \\
   &\hspace{15pt} + \ldots +  \max \{ \lVert Q_{nn} (A_{n}) \rVert_{1} \hspace{1pt}  \vert \lVert A_{n} \rVert_{1} = 1 \}\}\\
   & = \max \{ \lVert Q_{11} (A_{1}) \rVert_{1} \hspace{1pt}  + \ldots +  \lVert Q_{1m} (A_{1}) \rVert_{1} \hspace{1pt} , \hspace{2pt} \ldots \hspace{2pt}, \lVert Q_{n1} (A_{n}) \rVert_{1}  + \ldots +  \\
   & \hspace{15pt} \lVert Q_{nm} (A_{n}) \rVert_{1} \hspace{1pt}  \vert \hspace{1pt}   \lVert A_{1} \rVert_{1} = 1, \ldots, \lVert A_{n} \rVert_{1} = 1 \}
    \end{split}
  \end{equation}
  
  
  Thus, if  $\lVert O \rVert_{1 \text{ gen}} = 1$,
  \begin{align*}
    \hspace{-30pt}&  \lVert O_1  \rVert_1  + \ldots + \lVert O_n  \rVert_1 = 1  \wedge \lVert Q(O) \rVert_{1 \text{ gen}} \leq  \lVert Q \rVert_{1 \text{ gen}} \\
    \hspace{-30pt} \Leftrightarrow  & \lVert O_1  \rVert_1  + \ldots + \lVert O_n  \rVert_1 = 1  \wedge  \lVert Q_{11} (O_{1}) \rVert_{1} + \ldots +  \lVert Q_{1m} (O_{1}) \rVert_{1} + \ldots +  \hspace{10pt}&   \{\text{\autoref{eq:qo}}, \\
    \hspace{-30pt} & \lVert Q_{n1} (O_{m}) \rVert_{1} +  \ldots  + \lVert Q_{nm} (O_{n}) \rVert_{1}   & \text{\autoref{eq:q}} \} \\
    \hspace{-30pt}& \leq \max \{ \lVert Q_{11} (A_{1}) \rVert_{1} \hspace{1pt} + \ldots +  \lVert Q_{1m} (A_{1}) \rVert_{1}, \hspace{2pt} \ldots \hspace{2pt},  \lVert Q_{n1} (A_{n}) \rVert_{1} + \ldots +   \\
    \hspace{-30pt}&\lVert Q_{nm} (A_{n}) \rVert_{1} \hspace{1pt}  \vert  \lVert A_{1} \rVert_{1} = 1, \ldots, \lVert A_{n} \rVert_{1} = 1 \} \\
    %\hspace{-30pt} \Leftrightarrow  & \lVert O_1  \rVert_1  + \ldots + \lVert O_n  \rVert_1 = 1  \wedge  \lVert Q_{11} \cdot O_{1} \rVert_{1} + \ldots +  \lVert Q_{1m} \cdot O_{1} \rVert_{1} + \ldots +  \hspace{10pt}  \\
    %\hspace{-30pt}& \leq \max \Bigg\{ \left\lVert Q_{11} \left(\frac{O_{1}} {\lVert O_{1} \rVert_1}\right) \right\rVert_{1} \hspace{1pt} + \ldots +  \left\lVert Q_{1m} \left(\frac{O_{1}} {\lVert O_{1} \rVert_1}\right)  \right\rVert_{1} \hspace{1pt}, \hspace{2pt} \ldots \hspace{2pt},   \\
    %\hspace{-30pt}& \left\lVert Q_{n1}  \left(\frac{O_{n}} {\lVert O_{1} \rVert_1}\right) \right\rVert_{1} + \ldots + \left\lVert Q_{nm}  \left(\frac{O_{n}} {\lVert O_{1} \rVert_1}\right) \right\rVert_{1} \hspace{1pt}  \Bigg\}  \\
    \hspace{-30pt} \Leftrightarrow  & \lVert O_1  \rVert_1  + \ldots + \lVert O_n  \rVert_1 = 1  \wedge  \lVert Q_{11} (O_{1}) \rVert_{1} + \ldots +  \lVert Q_{1m} (O_{1}) \rVert_{1} + \ldots +  \hspace{10pt}\\
    \hspace{-30pt} & \lVert Q_{n1} (O_{m}) \rVert_{1} +  \ldots  + \lVert Q_{nm} (O_{n}) \rVert_{1}   \\
    \hspace{-30pt}& \leq \max \{ \left\lVert Q_{11} \left(O_{1} / \lVert O_{1} \rVert_1\right) \right\rVert_{1} \hspace{1pt} + \ldots +  \left\lVert Q_{1m}  \left(O_{1} / \lVert O_{1} \rVert_1\right) \right\rVert_{1}, \hspace{2pt} \ldots \hspace{2pt},   \\
    \hspace{-30pt}& \left\lVert Q_{n1}  \left(O_{n} / \lVert O_{n} \rVert_1\right) \right\rVert_{1} + \ldots + \left\lVert Q_{nm}   \left(O_{n} / \lVert O_{n} \rVert_1\right) \right\rVert_{1} \hspace{1pt} \}  \\
    \hspace{-30pt} \Leftrightarrow  & \lVert O_1  \rVert_1  + \ldots + \lVert O_n  \rVert_1 = 1  \wedge  \lVert Q_{11} (O_{1}) \rVert_{1} + \ldots +  \lVert Q_{1m} (O_{1}) \rVert_{1} + \ldots +  \hspace{10pt}\\
    \hspace{-30pt} & \lVert Q_{n1} (O_{m}) \rVert_{1} +  \ldots  + \lVert Q_{nm} (O_{n}) \rVert_{1}   \\
    \hspace{-30pt}& \leq \max \{ (1 / \lVert O_{1} \rVert_1)   \left(\lVert Q_{11} (O_{1}) \rVert_{1} \hspace{1pt} + \ldots +  \lVert Q_{1m} (O_{1})\rVert_{1} \right), \hspace{2pt} \ldots \hspace{2pt},   \\
    \hspace{-30pt}& (1 / \lVert O_{n} \rVert_1)   \left(\lVert Q_{n1} (O_{n}) \rVert_{1} \hspace{1pt} + \ldots +  \lVert Q_{nm} (O_{n})\rVert_{1} \right) \}  
    \end{align*}
  
    This is equivalent to demonstrating that for $a_1, \ldots, a_n, x_1, \ldots, x_n \in \mathbb{R}^{+}_{0}$ with $a_1+ \ldots + a_n=1$,
    \begin{equation} 
    \begin{split}
        x_1 + \ldots + x_n  \leq  \max \left\{   \dfrac{1}{a_1} x_1  , \ldots , \dfrac{1}{a_n} x_n   \right\} \\
    \end{split}
    \end{equation}
  
    Designating $M = \max \left\{   \dfrac{1}{a_1} x_1  , \ldots , \dfrac{1}{a_n} x_n   \right\}$, from the definition of maximum it follows that, for all $1 \leq i \leq n$, $x_i \leq M \cdot a_i$, and consequently, $x_1 + \ldots + x_n \leq M \cdot (a_1 + \ldots + a_n) = M$. Therefore, it holds that:
    \begin{equation}
      \lVert Q(O) \rVert_{1 \text{ gen}} \leq  \lVert Q \rVert_{1 \text{ gen}}.
    \end{equation} 
  
    As a result, it follows that for an operator $O \in \mathbb{C}^{o_1 \times o_1} \oplus \ldots \oplus  \mathbb{C}^{o_m \times o_m}$,  $ \left\lVert Q\left(\frac{O}{\lVert O \rVert_{1 \text{ gen}}}\right)  \right\rVert_{1 \text{ gen}}$ is upper bounded by $\lVert Q  \rVert_{1 \text{ gen}}$. Thus, 
  \begin{equation}
    \lVert Q(O) \rVert_{1 \text{ gen}} \leq \lVert Q  \rVert_{1 \text{ gen}} \cdot \lVert O  \rVert_{1 \text{ gen}}.
  \end{equation}
  \end{proof}
  
  
  \begin{lemma}\label{lem:gen_trace_submultiplicative}
    The generalized trace norm is submultiplicative with respect to composition of super‑operators, \textit{i.e.}, for all super-operators $Q: \mathbb{C}^{o_1 \times o_1} \oplus \ldots \oplus \mathbb{C}^{o_n \times o_n}  \rightarrow \mathbb{C}^{p_1 \times p_1} \oplus \ldots \oplus  \mathbb{C}^{p_m \times p_m}$ and $S: \mathbb{C}^{p_1 \times p_1} \oplus \ldots \oplus \mathbb{C}^{p_m \times p_m}  \rightarrow \mathbb{C}^{q_1 \times q_1} \oplus \ldots \oplus \mathbb{C}^{q_t \times q_t}$, it holds that:
    \begin{equation} \label{eq:gen_trace_submultiplicative}
      \lVert S  Q \rVert_{1 \text{ gen}} \leq \lVert S \rVert_{1 \text{ gen}} \lVert Q \rVert_{1 \text{ gen}}
    \end{equation}
  \end{lemma}
  
  \begin{proof}
  The composition of two superoperators $Q: \mathbb{C}^{o_1 \times o_1} \oplus \ldots \oplus \mathbb{C}^{o_n \times o_n}  \rightarrow \mathbb{C}^{p_1 \times p_1} \oplus \ldots \oplus  \mathbb{C}^{p_m \times p_m}$ and $S: \mathbb{C}^{p_1 \times p_1} \oplus \ldots \oplus \mathbb{C}^{p_m \times p_m}  \rightarrow \mathbb{C}^{q_1 \times q_1} \oplus \ldots \oplus \mathbb{C}^{q_t \times q_t}$ corresponds to 
  \begin{equation}
    \begin{split}
      & S \cdot  Q =[\underbrace{\textsc{Il} \cdot \ldots \cdot \textsc{Il}}_{t-1 \times} \cdot \hspace{1pt} S_{11} + \ldots +   \underbrace{\textsc{Ir} \cdot \ldots \cdot \textsc{Ir}}_{t-1 \times}\cdot \hspace{1pt} S_{1t},  \hspace{2pt} \ldots  \hspace{2pt},  \underbrace{\textsc{Il} \cdot \ldots \cdot \textsc{Il}}_{t-1 \times} \cdot\hspace{1pt} S_{m1} + \ldots + \underbrace{\textsc{Ir} \cdot \ldots \cdot \textsc{Ir}}_{t-1 \times}\cdot \hspace{1pt} S_{mt}] \\
       & \hspace{21pt}\cdot  [\underbrace{\textsc{Il} \cdot \ldots \cdot \textsc{Il}}_{m-1 \times} \cdot \hspace{1pt} Q_{11} + \ldots +   \underbrace{\textsc{Ir} \cdot \ldots \cdot \textsc{Ir}}_{m-1 \times}\cdot \hspace{1pt} Q_{1m},  \hspace{2pt} \ldots  \hspace{2pt},  \underbrace{\textsc{Il} \cdot \ldots \cdot \textsc{Il}}_{m-1 \times} \cdot\hspace{1pt} Q_{n1} + \ldots + \underbrace{\textsc{Ir} \cdot \ldots \cdot \textsc{Ir}}_{m-1 \times}\cdot \hspace{1pt} Q_{nm}]  \\
       & \hspace{21pt} = [ \underbrace{\textsc{Il} \cdot \ldots \cdot \textsc{Il}}_{t-1 \times} \cdot \hspace{1pt} S_{11} \cdot Q_{11} + \ldots +   \underbrace{\textsc{Ir} \cdot \ldots \cdot \textsc{Ir}}_{t-1 \times}\cdot \hspace{1pt} S_{1t} \cdot Q_{11} + \ldots +  \underbrace{\textsc{Il} \cdot \ldots \cdot \textsc{Il}}_{t-1 \times} \cdot\hspace{1pt} S_{m1} \cdot Q_{1m} +  \ldots \\
       & \hspace{22pt}  + \underbrace{\textsc{Ir} \cdot \ldots \cdot \textsc{Ir}}_{t-1 \times}\cdot \hspace{1pt} S_{mt} \cdot Q_{1m}, \hspace{2pt} \ldots  \hspace{2pt},  \underbrace{\textsc{Il} \cdot \ldots \cdot \textsc{Il}}_{t-1 \times} \cdot \hspace{1pt} S_{11} \cdot Q_{n1} + \ldots +   \underbrace{\textsc{Ir} \cdot \ldots \cdot \textsc{Ir}}_{t-1 \times}\cdot \hspace{1pt} S_{1t} \cdot Q_{n1} + \ldots   \\
       & \hspace{22pt} +  \underbrace{\textsc{Il} \cdot \ldots \cdot \textsc{Il}}_{t-1 \times} \cdot\hspace{1pt} S_{m1} \cdot Q_{nm} + \ldots + \underbrace{\textsc{Ir} \cdot \ldots \cdot \textsc{Ir}}_{t-1 \times}\cdot \hspace{1pt} S_{mt} \cdot Q_{nm}]
    \end{split}
  \end{equation}
  
  
  
  %Attending to the definition of the generalized trace norm (\autoref{def:gen_1norm}), it follows that:
  %\begin{equation}
    %\begin{split}
   %& \lVert S \cdot  Q \rVert_{1 \text{ gen}} \\
   %& =  \max \{ \lVert S_{11} \cdot Q_{11} \rVert_{1} + \ldots + \lVert S_{1t} \cdot Q_{11} \rVert_{1} + \ldots +  \lVert S_{m1} \cdot Q_{1m} \rVert_{1} +  \ldots +  \lVert S_{mt} \cdot Q_{1m} \rVert_{1}, \\
    %&  \hspace{15pt} \hspace{2pt} \ldots \hspace{2pt}, \lVert S_{11} \cdot Q_{n1}\rVert_{1}  + \ldots + \lVert S_{1t} \cdot Q_{n1} \rVert_{1} + \ldots +  \lVert S_{m1} \cdot Q_{nm} \rVert_{1} +  \ldots + \lVert  S_{mt} \cdot Q_{nm} \rVert_{1} \}
    %\end{split}
  %\end{equation}
  
  
  Note that if $Q$ is decomposed as  $Q=[Q_1, \ldots, Q_n]$, where $Q_1, \ldots, Q_n$ are defined as in \autoref{def:gen_norm_either}, then for $ 1 \leq i \leq n$  $Q_i = \underbrace{\textsc{Il} \cdot \ldots \cdot \textsc{Il}}_{m-1 \times} \cdot \hspace{2pt} Q_{i1} + \ldots +  \underbrace{\textsc{Ir} \cdot \ldots \cdot \textsc{Ir}}_{m-1 \times} \cdot \hspace{2pt} Q_{im}$.
  Consequently, $S \cdot Q$ can also be defined as follows:
  \begin{equation}
    S \cdot Q = [S \cdot Q_1, \ldots, S \cdot Q_n].
  \end {equation}
  where 
  \begin{equation}
    \begin{split}
    S \cdot Q_i & = \underbrace{\textsc{Il} \cdot \ldots \cdot \textsc{Il}}_{t-1 \times} \cdot \hspace{1pt} S_{11} \cdot Q_{i1} + \ldots +   \underbrace{\textsc{Ir} \cdot \ldots \cdot \textsc{Ir}}_{t-1 \times}\cdot \hspace{1pt} S_{1t} \cdot Q_{i1} + \ldots +  \underbrace{\textsc{Il} \cdot \ldots \cdot \textsc{Il}}_{t-1 \times} \cdot\hspace{1pt} S_{m1} \cdot Q_{im} +  \ldots \\
    & \hspace{10pt}  + \underbrace{\textsc{Ir} \cdot \ldots \cdot \textsc{Ir}}_{t-1 \times}\cdot \hspace{1pt} S_{mt} \cdot Q_{im},
    \end{split}
  \end{equation}
  
  
  Attending to \autoref{def:gen_norm_either}, it follows that:
  \begin{equation} \label{eq:sq_decomposed_norm}
  \lVert S \cdot  Q \rVert_{1 \text{ gen}} = \max \{ \lVert S \cdot Q_1 \rVert_{1 \text{ gen}}, \ldots, \lVert S \cdot Q_n \rVert_{1 \text{ gen}} \},
  \end{equation}
  where 
  \begin{equation} \label{eq:sqi_norm}
  \lVert S \cdot Q_i \rVert_{1 \text{ gen}} =  \lVert S_{11} \cdot Q_{i1} \rVert_{1} + \ldots + \lVert S_{1t} \cdot Q_{i1} \rVert_{1} + \ldots +  \lVert S_{m1} \cdot Q_{im} \rVert_{1} +  \ldots +  \lVert S_{mt} \cdot Q_{im} \rVert_{1}
  \end{equation} 
  for all $1 \leq i \leq n$.
  
  %Let $P \in \mathbb{C}^{p_1 \times p_1} \oplus \ldots \oplus  \mathbb{C}^{p_m \times p_m}$, with the decomposition $P = \underbrace{\textsc{Il} \cdot \ldots \cdot \textsc{Il}}_{m-1 \times} \cdot \hspace{2pt} P_{1} + \ldots +  \underbrace{\textsc{Ir} \cdot \ldots \cdot \textsc{Ir}}_{m-1 \times} \cdot \hspace{2pt} P_{m} 
  %$, where for each $1 \leq i \leq m$, $P_{i} \in \mathbb{C}^{p_i \times p_i}$ and $P_i =  \underbrace{\textsc{Pl} \cdot \ldots \cdot \textsc{Pl}}_{m-i \times} \cdot \underbrace{\textsc{Pr} \cdot \ldots \cdot \textsc{Pr}}_{i-1 \times} (P) $, considering \autoref{def:gen_norm}, it follows that: 
  %\begin{equation}
    %\lVert P  \rVert_{1 \text{ gen}} = \lVert P_1 \rVert_{1} + \ldots + \lVert P_m \rVert_{1}.
  %\end{equation}
  %The application of $S$ to $P$ corresponds to 
  %\begin{equation}
  %\begin{split}
  %S(P) & = \underbrace{\textsc{Il} \cdot \ldots \cdot \textsc{Il}}_{t-1 \times} \cdot \hspace{1pt} S_{11} \cdot P_{1} + \ldots +   \underbrace{\textsc{Ir} \cdot \ldots \cdot \textsc{Ir}}_{t-1 \times}\cdot \hspace{1pt} S_{1t} \cdot P_{1} + \ldots +  \underbrace{\textsc{Il} \cdot \ldots \cdot \textsc{Il}}_{t-1 \times} \cdot\hspace{1pt} S_{m1} \cdot P_{m} +  \ldots \\
  %& \hspace{15pt}  + \underbrace{\textsc{Ir} \cdot \ldots \cdot \textsc{Ir}}_{t-1 \times}\cdot \hspace{1pt} S_{mt} \cdot P_{m}
  %\end{split}
  %\end{equation}
  
  %As a result, the generalized trace norm of $S(P)$ corresponds to:
  %\begin{equation}
    %\begin{split}
    %\lVert S(P)  \rVert_{1 \text{ gen}} & = \lVert S_{11} \cdot P_{1} \rVert_{1} + \ldots + \lVert S_{1t} \cdot P_{1} \rVert_{1} + \ldots +  \lVert S_{m1} \cdot P_{m} \rVert_{1} +  \ldots +  \lVert S_{mt} \cdot P_{m} \rVert_{1}. 
    %\end{split}
  %\end {equation}
  %The generalized trace norm of $S$ is given by:
  %\begin{equation}
    %\begin{split}
    %&\lVert S  \rVert_{1 \text{ gen}} = \max \{ \lVert S_{11} \rVert_{1} + \ldots + \lVert S_{1t} \rVert_{1}, \hspace{2pt} \ldots \hspace{2pt}, \lVert S_{m1} \rVert_{1} + \ldots + \lVert S_{mt} \rVert_{1} \} \\
   %& = \max \{ \max \{ \lVert S_{11} (A_{11}) \rVert_{1} \hspace{1pt}  \vert \hspace{1pt}  \lVert A_{11} \rVert_{1} = 1 \} + \ldots + &  \{\text{\autoref{def:trace_norm_superoperator}}\} \\
   %& \hspace{15pt}  \max \{  \lVert S_{1t} (A_{1t}) \rVert_{1} \hspace{1pt}  \vert   \lVert A_{1t} \rVert_{1} = 1 \} , \hspace{1pt} \ldots \hspace{1pt}, \max \{ \lVert S_{11} (A_{m1}) \rVert_{1} \hspace{1pt}  \vert \hspace{1pt}  \lVert A_{m1} \rVert_{1} = 1 \}   \\
   %&\hspace{15pt} + \ldots +  \max \{ \lVert S_{11} (A_{m1}) \rVert_{1} \hspace{1pt}  \vert \lVert A_{m1} \rVert_{1} = 1 \}\}
    %\end{split}
  %\end{equation}
  Let $P \in \mathbb{C}^{p_1 \times p_1} \oplus \ldots \oplus  \mathbb{C}^{p_m \times p_m}$, by \autoref{lem:q(o)} it follows that:
  \begin{equation}
    \lVert S(P) \rVert_{1 \text{ gen}} \leq \lVert S  \rVert_{1 \text{ gen}} \cdot \lVert P  \rVert_{1 \text{ gen}}.
  \end{equation}
  As a result, 
  \begin{equation} \label{ineq:gen_trace_submultiplicative_O}
    \lVert S (Q_i (O_i)) \rVert_{1 \text{ gen}} \leq \lVert S  \rVert_{1 \text{ gen}} \cdot \lVert Q_i (O_i)  \rVert_{1 \text{ gen}},
  \end{equation}
  for all $O_i \in \mathbb{C}^{o_i \times o_i}$ and $1 \leq i \leq n$. 
  
  Given that
  \begin{equation}
  \begin{split} 
    S (Q_i (O_i)) & = \underbrace{\textsc{Il} \cdot \ldots \cdot \textsc{Il}}_{t-1 \times} \cdot \hspace{1pt} S_{11} \cdot Q_{i1} \cdot O_{i} + \ldots + \underbrace{\textsc{Ir} \cdot \ldots \cdot \textsc{Ir}}_{t-1 \times} \cdot \hspace{1pt} S_{1t} \cdot Q_{i1} \cdot O_{i} + \ldots +      \\
    & \hspace{15pt}  \underbrace{\textsc{Il} \cdot \ldots \cdot \textsc{Il}}_{t-1 \times}\cdot \hspace{1pt} S_{m1} \cdot Q_{im} \cdot O_{i} + \ldots + \underbrace{\textsc{Ir} \cdot \ldots \cdot \textsc{Ir}}_{t-1 \times} \cdot \hspace{1pt} S_{mt} \cdot Q_{im} \cdot O_{i},
  \end{split}
  \end{equation}
  and that
  \begin{equation}
    \begin{split}
       Q_i (O_i) = \underbrace{\textsc{Il} \cdot \ldots \cdot \textsc{Il}}_{m-1 \times} \cdot \hspace{2pt} Q_{i1} \cdot O_{i} + \ldots +  \underbrace{\textsc{Ir } \cdot \ldots \cdot \textsc{Ir}}_{m-1 \times} \cdot \hspace{2pt} Q_{im} \cdot O_{i},
    \end{split}
  \end{equation}
  considering \autoref{def:gen_norm}, \autoref{ineq:gen_trace_submultiplicative_O} can be rewritten as:
  \begin{equation}
    \begin{split}
    &\lVert S_{11} \cdot Q_{i1} \cdot O_{i} \rVert_{1} + \ldots + \lVert S_{1t} \cdot Q_{i1} \cdot O_{i} \rVert_{1} + \ldots + \lVert S_{m1} \cdot Q_{im} \cdot O_{i} \rVert_{1} + \ldots +  \\
    &  \lVert S_{mt} \cdot Q_{im} \cdot O_{i} \rVert_{1} \leq  \lVert S  \rVert_{1 \text{ gen}} \cdot \lVert Q_{i1} \cdot O_{i} \rVert_{1} + \ldots + \lVert Q_{im} \cdot O_{i} \rVert_{1}
    \end{split}
  \end{equation}
  
  Taking the maximum over all $O \in \mathbb{C}^{o_1 \times o_1} \oplus \ldots \oplus \mathbb{C}^{o_n \times o_n}$ such that $\lVert O_i \rVert_{1} = 1$ yields the following inequality 
  \begin{equation}
    \begin{split}
    & \max \{ \lVert S_{11} \cdot Q_{i1} \cdot O_{i} \rVert_{1} + \ldots + \lVert S_{1t} \cdot Q_{i1} \cdot O_{i} \rVert_{1} + \ldots + \lVert S_{m1} \cdot Q_{im} \cdot O_{i} \rVert_{1} + \ldots +  \\
    &  \lVert S_{mt} \cdot Q_{im}  \cdot O_{i} \rVert_{1} \hspace{1pt} |\lVert O_i \rVert_{1} = 1 \} \leq  \lVert S  \rVert_{1 \text{ gen}} \cdot \max \{ \lVert Q_{i1} \cdot O_{i} \rVert_{1} + \ldots + \lVert Q_{im} \cdot O_{i} \rVert_{1} \hspace{1pt} |\lVert O_i \rVert_{1} = 1 \}.
    \end{split}
  \end{equation}
  This is equivalent to
  \begin{equation}
    \begin{split}
    & \max \{ \lVert S_{11} \cdot Q_{i1} \cdot O_{i} \rVert_{1} \hspace{1pt}|\lVert O_i \rVert_{1} = 1 \} + \ldots + \max \{\lVert S_{1t} \cdot Q_{i1} \cdot O_{i} \rVert_{1}  \hspace{1pt}|\lVert O_i \rVert_{1} = 1 \}  + \ldots +   \\
    & \max \{\lVert S_{m1} \cdot Q_{im} \cdot O_{i} \rVert_{1} \hspace{1pt}|\lVert O_i \rVert_{1} = 1 \} + \ldots + \max \{ \lVert S_{mt} \cdot Q_{im}  \cdot O_{i} \rVert_{1} \hspace{1pt} |\lVert O_i \rVert_{1} = 1 \}  \\
    & \leq  \lVert S  \rVert_{1 \text{ gen}} \cdot \max \{ \lVert Q_{i1} \cdot O_{i} \rVert_{1}  \hspace{1pt} |\lVert O_i \rVert_{1} = 1 \}  + \ldots + \max \{\lVert Q_{im} \cdot O_{i} \rVert_{1} \hspace{1pt} |\lVert O_i \rVert_{1} = 1 \} 
    \end{split}
  \end{equation}
  As a result, attending to \autoref{def:trace_norm_superoperator}, it follows that:
  \begin{equation}
    \begin{split}
    & \lVert S_{11} \cdot Q_{i1} \rVert_{1} + \ldots + \lVert S_{1t} \cdot Q_{i1} \rVert_{1} + \ldots + \lVert S_{m1} \cdot Q_{im} \rVert_{1} + \ldots +  \lVert S_{mt} \cdot Q_{im} \rVert_{1} \\
    &  \leq  \lVert S  \rVert_{1 \text{ gen}} \cdot (\lVert Q_{i1} \rVert_{1} + \ldots + \lVert Q_{im} \rVert_{1})
    \end{split}
  \end{equation}
  Considering \autoref{eq:sqi_norm}, and \autoref{def:gen_norm_gen_inj}, the inequality above can be rewritten as: 
  \begin{equation}
     \lVert S \cdot  Q_i \rVert_{1 \text{ gen}} \leq \lVert S  \rVert_{1 \text{ gen}} \cdot \lVert Q_i  \rVert_{1 \text{ gen}}
  \end{equation}
  for all $1 \leq i \leq n$. Given \autoref{def:gen_norm_either}, and considering the fact that for $a,b,c \in \mathbb{R}$ if $a \leq b$ and $b \leq c$, then $a \leq c$, it follows that:
  \begin{equation}
    \lVert S  Q_i \rVert_{1 \text{ gen}} \leq \lVert S \rVert_{1 \text{ gen}} \lVert Q \rVert_{1 \text{ gen}}
  \end{equation}
  Attending to \autoref{eq:sq_decomposed_norm}, and the fact that if all elements of a set verify a certain property, then the maximum of the set also verifies the property, given it is an element of the set, it follows that:
  \begin{equation}
    \lVert S  Q \rVert_{1 \text{ gen}} \leq \lVert S \rVert_{1 \text{ gen}} \lVert Q \rVert_{1 \text{ gen}}
  \end{equation}
  
  Therefore, the inequality in \autoref{eq:gen_trace_submultiplicative} holds.
  
  %no take the maximum a seguir por as eqs das normas (com trace norm explita com cena da norma 1) e depois é que digo que corresponde a ineq in \autoref{eq:gen_trace_submultiplicative}
  
  % \underbrace{\textsc{Il} \cdot \ldots \cdot \textsc{Il}}_{m-1 \times} \cdot \hspace{2pt} P_{1} + \ldots +  \underbrace{\textsc{Ir} \cdot \ldots \cdot \textsc{Ir}}_{m-1 \times} \cdot \hspace{2pt} P_{m} 
  
  \end{proof}
  
  
  \begin{lemma}\label{lem:gen_diamond_submultiplicative}
    The generalized diamond norm is submultiplicative with respect to composition of super‑operators, \textit{i.e.}, for all super-operators $Q: \mathbb{C}^{o_1 \times o_1} \oplus \ldots \oplus \mathbb{C}^{o_n \times o_n}  \rightarrow \mathbb{C}^{p_1 \times p_1} \oplus \ldots \oplus  \mathbb{C}^{p_m \times p_m}$ and $S: \mathbb{C}^{p_1 \times p_1} \oplus \ldots \oplus \mathbb{C}^{p_m \times p_m}  \rightarrow \mathbb{C}^{q_1 \times q_1} \oplus \ldots \oplus \mathbb{C}^{q_t \times q_t}$, it holds that:
    \begin{equation} \label{eq:gen_trace_submultiplicative}
      \lVert S  Q \rVert_{\diamondsuit \text{ gen}} \leq \lVert S \rVert_{\diamondsuit  \text{ gen}} \lVert Q \rVert_{\diamondsuit  \text{ gen}}
    \end{equation}
  \end{lemma}
  
  \begin{proof}
   
    By \autoref{lem:gen_trace_submultiplicative}, it is possible to state that:
  \begin{equation}
    \begin{split}
     \lVert S Q \otimes I_{\mathbb{C}^{o_1 \times o_1} \oplus \ldots \oplus \mathbb{C}^{o_n \times o_n}} \rVert_{1\text{ gen}} \leq \lVert S \otimes I_{\mathbb{C}^{o_1 \times o_1} \oplus \ldots \oplus \mathbb{C}^{o_n \times o_n}} \rVert_{1\text{ gen}} \cdot \lVert Q \otimes I_{\mathbb{C}^{o_1 \times o_1} \oplus \ldots \oplus \mathbb{C}^{o_n \times o_n}} \rVert_{1\text{ gen}}  
      \end{split}
  \end{equation}
  Attending to \autoref{def:gen_diamond_norm}, it follows that:
  \begin{equation}
     \lVert S Q \rVert_{\diamondsuit \text{ gen}} \leq  \lVert S \otimes I_{\mathbb{C}^{o_1 \times o_1} \oplus \ldots \oplus \mathbb{C}^{o_n \times o_n}} \rVert_{1\text{ gen}} \cdot \lVert Q \rVert_{\diamondsuit \text{ gen}}
  \end{equation}
  Given \autoref{theorem:tensor_stability}, it holds that
  \begin{equation}
    \lVert S \otimes I_{\mathbb{C}^{o_1 \times o_1} \oplus \ldots \oplus \mathbb{C}^{o_n \times o_n}} \rVert_{1\text{ gen}} \leq \lVert S \rVert_{\diamondsuit \text{ gen}}
  \end{equation}
  As a result, the inequality in \autoref{eq:gen_trace_submultiplicative} holds.
  \end{proof}
   
  \begin{lemma} \label{lem:gen_trace_ptp_norm1}
    Let  $Q: \mathbb{C}^{o_1 \times o_1} \oplus \ldots \oplus \mathbb{C}^{o_n \times o_n}  \rightarrow \mathbb{C}^{p_1 \times p_1} \oplus \ldots \oplus  \mathbb{C}^{p_m \times p_m}$ be a positive trace-preserving super-operator. Then, it holds that $\lVert Q \rVert_{1 \text{ gen}} = 1$.
  \end{lemma}
  
  \begin{proof}
    Consider the decomposition $Q = [Q_1, \ldots, Q_n]$, where $Q_1, \ldots, Q_n$ are defined as in \autoref{def:gen_norm_either}. Then for $ 1 \leq i \leq n$, one has that $Q_i: \mathbb{C}^{o_i \otimes o_i} \rightarrow  \mathbb{C}^{p_1 \times p_1} \oplus \ldots \oplus  \mathbb{C}^{p_m \times p_m}$ is defined as  $Q_i = \underbrace{\textsc{Il} \cdot \ldots \cdot \textsc{Il}}_{m-1 \times} \cdot \hspace{2pt} Q_{i1} + \ldots +  \underbrace{\textsc{Ir} \cdot \ldots \cdot \textsc{Ir}}_{m-1 \times} \cdot \hspace{2pt} Q_{im}$.
  
    In this case attending to \autoref{def:gen_norm_either} and \autoref{def:gen_norm_either}, 
    \begin{equation}
        \lVert Q \rVert_{1 \text{ gen}} = \max \{ \lVert Q_1 \rVert_{1  \text{ gen}} + \ldots + \lVert Q_n \rVert_{1  \text{ gen}} \},
    \end{equation}
      where
      \begin{equation} \label{eq:qi_norm}
          \lVert Q_i \rVert_{1  \text{ gen}} = \lVert Q_{i1} \rVert_{1} + \ldots + \lVert Q_{im} \rVert_{1}
      \end{equation}
  for $ 1 \leq i \leq n$.
  
  Note that for all $1 \leq i \leq m$, if $Q$ is positive trace-preserving, then $Q_i$ is also positive trace-preserving given that the composition of positive trace-preserving super-operators is also positive trace-preserving.
  
  Considering the definition of $Q_{ij}$ in \autoref{def:gen_norm_ops} for a fixed $1 \leq i \leq n$, $Q_{i1}, \ldots, Q_{im}$ are always given the same argument.
  Consequentlty, attending to \autoref{thm:Russo–Dye}, \autoref{eq:qi_norm} can be rewritten as:
  \begin{align} 
    \hspace{-30pt} \lVert Q_i \rVert_{1} & = \max \{\text{Tr}\left(Q_{i1}(uu^{*}) \right) \vert \hspace{2pt}  \lVert u \rVert_{1}=1 \} + \ldots +  \max \{\text{Tr}\left(Q_{in}(u u^{*}) \right) \vert \hspace{2pt}  \lVert u \rVert_{1}=1 \}\\ \label{eq:qi_ptp_norm_1}
    \hspace{-30pt} & =  \max \{ \text{Tr}\left(Q_{i1}(uu^{*}) \right) + \ldots + \text{Tr}\left(Q_{in}(u u^{*}) \right) \vert \hspace{2pt}  \lVert u \rVert_{1}=1 \} \\
    \hspace{-30pt} & =  \max \{ \text{Tr}\left(Q_{i1}(uu^{*}) + \ldots + Q_{in}(u u^{*}) \right) \vert \hspace{2pt}  \lVert u \rVert_{1}=1 \}  \label{eq:qi_ptp_norm_3}
  \end{align}
  %\begin{align} 
    %\hspace{-30pt} \lVert Q_i \rVert_{1} & = \max \{\text{Tr}\left(Q_{i1}(u_1u_1^{*}) \right) \vert \hspace{2pt}  \lVert u_1 \rVert_{1}=1 \} + \ldots +  \max \{\text{Tr}\left(Q_{in}(u_m u_m^{*}) \right) \vert \hspace{2pt}  \lVert u_m \rVert_{1}=1 \}\\ \label{eq:qi_ptp_norm_1}
    %\hspace{-30pt} & =  \max \{ \text{Tr}\left(Q_{i1}(u_1u_1^{*}) \right) + \ldots + \text{Tr}\left(Q_{in}(u_m u_m^{*}) \right) \vert \hspace{2pt}  \lVert u_1 \rVert_{1}=1, \ldots, \lVert u_m \rVert_{1}=1 \} \\
    %\hspace{-30pt} & =  \max \{ \text{Tr}\left(Q_{i1}(u_1u_1^{*}) + \ldots + Q_{in}(u_m u_m^{*}) \right) \vert \hspace{2pt}  \lVert u_1 \rVert_{1}=1, \ldots, \lVert u_m \rVert_{1}=1 \}  \label{eq:qi_ptp_norm_3}
  %\end{align}
  \todo[inline,size=\normalsize]{ no "and ..." fazer referencia à definição de trace preserving para espaços com somas diretas qd eu a definir} 
  where $u \in \mathbb{C}^{o_i}$.
  If $Q_i$ is trace-preserving, then considering \autoref{def:gen_norm_gen_inj} and ...
  \begin{equation}
    \begin{split}
    &\text{Tr} (Q_i(u u^{*})) =  \text{Tr} \left( \underbrace{\textsc{Il} \cdot \ldots \cdot \textsc{Il}}_{m-1 \times } \cdot \hspace{2pt} Q_{i1} (u u^{*}) + \ldots + \underbrace{\textsc{Ir} \cdot \ldots \cdot \textsc{Ir}}_{m-1 \times } \cdot  \hspace{2pt} Q_{im}(u u^{*}) \right) \\
    =&   \text{Tr} \left( Q_{i1} (u u^{*}) + \ldots + Q_{im}(u u^{*}) \right) = \text{Tr} (u u^{*}) = \text{Tr} (u u^{*})
  \end{split}
  \end{equation}
  As a result, considering the definition of trace-norm for square matrices, \autoref{def:trace-norm-matriz},  it follows that $\lVert Q_i \rVert_{1} = \text{Tr} (u u^{*}) = 1$ for all $1 \leq i \leq n$. Given that if all elements of a set verify a certain property, then the maximum of the set also verifies the property it follows that $\lVert Q \rVert_{1 \text{ gen}} = 1$.
  
  \end{proof}
  
  \begin{lemma} \label{lem:gen_diamond_cptp_norm}
    Let  $Q: \mathbb{C}^{o_1 \times o_1} \oplus \ldots \oplus \mathbb{C}^{o_n \times o_n}  \rightarrow \mathbb{C}^{p_1 \times p_1} \oplus \ldots \oplus  \mathbb{C}^{p_m \times p_m}$ be a \acrshort{cptp} superoperator. Then, it holds that $\lVert Q \rVert_{\diamondsuit \text{ gen}} = 1$.
  \end{lemma}
  
  \begin{proof}
    Given that $Q$ is a \acrshort{cptp} superoperator, if follows that $ Q \otimes I_{\mathbb{C}^{o_1 \times o_1} \oplus \ldots \oplus \mathbb{C}^{o_n \times o_n}}$ is a positive trace-preserving superoperator. As a result, attending to \autoref{lem:gen_trace_ptp_norm1}, it holds that $\lVert Q \otimes I_{\mathbb{C}^{o_1 \times o_1} \oplus \ldots \oplus \mathbb{C}^{o_n \times o_n}} \rVert_{1 \text{ gen}} = 1$. As a result, considering  \autoref{def:gen_diamond_norm}, it follows that $\lVert Q \rVert_{\diamondsuit \text{ gen}} = 1$.
  \end{proof}
  
  \begin{theorem}
    Let  $Q: \mathbb{C}^{o_1 \times o_1} \oplus \ldots \oplus \mathbb{C}^{o_n \times o_n}  \rightarrow \mathbb{C}^{p_1 \times p_1} \oplus \ldots \oplus  \mathbb{C}^{p_m \times p_m}$ and $S: \mathbb{C}^{p_1 \times p_1} \oplus \ldots \oplus \mathbb{C}^{p_m \times p_m}  \rightarrow \mathbb{C}^{q_1 \times q_1} \oplus \ldots \oplus \mathbb{C}^{q_t \times q_t}$ be super-operators. If $Q$ is a \acrshort{cptp} superoperator, then $\lVert S  Q \rVert_{\diamondsuit \text{ gen}} \leq \lVert S \rVert_{\diamondsuit \text{ gen}}$, and if $S$ is a quantum channel, then $\lVert S  Q \rVert_{\diamondsuit \text{ gen}} \leq \lVert Q \rVert_{\diamondsuit \text{ gen}}$
  \end{theorem}
  
  \begin{proof}
    This result is an immediate consequence of \autoref{lem:gen_diamond_submultiplicative} and \autoref{lem:gen_diamond_cptp_norm}. 
  \end{proof}
  
















  %
  %
  %
  %
  %
  %
  %
  %
  %
  %
  %
%
%
%
%
%
%
%
%
%
%
%
%
%
% ||Q(O)||<= ||Q|| ||O||


 %
  %
  %
  %
  %
  %
  %
  %
  %
  %
  %
%
%
%
%
%
%
%
%
%
%
%
%
%




$O$ can be written as  $O = \sum_{i=1}^{m} \underbrace{\textsc{Il} \cdot \ldots \cdot \textsc{Il}}_{n-i \times} \cdot   \underbrace{\textsc{Ir} \cdot \ldots \cdot \textsc{Ir}}_{i-1 \times} \cdot \hspace{2pt} O_{i} 
$, where  $O_{i} \in \mathbb{C}^{p_i \times p_i}$ and $O_i =  \underbrace{\textsc{Pl} \cdot \ldots \cdot \textsc{Pl}}_{n-i \times} \cdot \underbrace{\textsc{Pr} \cdot \ldots \cdot \textsc{Pr}}_{i-1 \times} \cdot O $. Considering \autoref{def:gen_norm}, it follows that: 
\begin{equation}
  \lVert O  \rVert_{1 \text{ gen}} =   \lVert O_1 \rVert_{1} + \ldots + \lVert O_n \rVert_{1}.
\end{equation}

Applying $O$ to $Q$ results in:
\begin{equation}
\begin{split}
Q(O) & = \sum_{i=1}^{n} \sum_{j=1}^{m} \underbrace{\textsc{Il} \cdot \ldots \cdot \textsc{Il}}_{m-j \times} \cdot   \underbrace{\textsc{Ir} \cdot \ldots \cdot \textsc{Ir}}_{j-1 \times} \cdot \hspace{2pt} Q_{ij} (O_{i}) \\
%\underbrace{\textsc{Il} \cdot \ldots \cdot \textsc{Il}}_{m-1 \times} \cdot \hspace{1pt} Q_{11}  (O_{1}) + \ldots +   \underbrace{\textsc{Ir} \cdot \ldots \cdot \textsc{Ir}}_{m-1 \times}\cdot \hspace{1pt} Q_{1m} (O_{1}) + \ldots +  \underbrace{\textsc{Il} \cdot \ldots \cdot \textsc{Il}}_{m-1 \times} \cdot\hspace{1pt} Q_{n1} (O_{n}) +  \ldots \\
%& \hspace{10pt}  + \underbrace{\textsc{Ir} \cdot \ldots \cdot \textsc{Ir}}_{m-1 \times}\cdot \hspace{1pt} Q_{nm}  (O_{n})
\end{split}
\end{equation}

As a result, considering \autoref{def:gen_1norm_matrix}, the generalized trace norm of $Q(O)$ corresponds to:
\begin{equation} \label{eq:qo}
  \begin{split}
  \lVert Q(O)  \rVert_{1 \text{ gen}} & = Q_1(O_1) + \ldots + Q_n(O_n)=  \sum_{i=1}^{n} \sum_{j=1}^{m} \lVert Q_{ij} (O_{i}) \rVert_{1}.
  %\lVert Q_{11} (O_{1}) \rVert_{1} + \ldots + \lVert Q_{1m} (O_{1}) \rVert_{1} + \ldots +  \lVert Q_{n1} (O_{n})  \rVert_{1} +  \ldots +  \lVert Q_{nm} (O_{n}) \rVert_{1}. 
  \end{split}
\end {equation}
The generalized trace norm of $Q$ is given by:
\begin{equation}
  \begin{split} \label{eq:q}
  \lVert Q  \rVert_{1 \text{ gen}} & =   \max  \Bigg\{ \max \left\{ \sum_{i=1}^{m} \|Q_{1i} (A_1)\|_{1}   \mid \hspace{1pt} \|A_1\|_{1} = 1 \right\} & \{\text{\autoref{def:gen_1norm}}\} \\
  & \hspace{15pt} ,\hspace{2pt}  \ldots \hspace{2pt}  , \max \left\{ \sum_{i=1}^{m} \|Q_{ni} (A_n)\|_{1}   \mid \|A_n\|_{1} = 1 \right\} \Bigg\} \\
 & = \max \Bigg\{  \sum_{i=1}^{m}  \lVert Q_{1i} (A_{1}) \rVert_{1} \hspace{1pt} , \hspace{2pt} \ldots \hspace{2pt}, \sum_{i=1}^{m} \lVert Q_{ni} (A_{n}) \rVert_{1} \mid \hspace{1pt}   \lVert A_{1} \rVert_{1} = 1  & \{\text{\autoref{lemma:max_max}} \} \\
 & \hspace{15pt} , \ldots,\lVert A_{n} \rVert_{1} = 1 \Bigg\}.
  \end{split}
\end{equation}


To prove that  if  $\lVert O \rVert_{1 \text{ gen}} = 1$
\begin{equation}
  \lVert Q(O) \rVert_{1 \text{ gen}} \leq \lVert Q \rVert_{1 \text{ gen}},
\end{equation}
is equivalent to demonstrating that,
\begin{equation}
 \max \left\{ \lVert Q(O) \rVert_{1 \text{ gen}} \hspace{2pt}  \Biggm| \hspace{2pt}  \sum_{i=1}^{n} \lVert O_i  \rVert_1 = 1 \right\} \leq \lVert Q \rVert_{1 \text{ gen}},
\end{equation}

Thus, 
\begin{align*}
  \hspace{-30pt}&   \max \left\{ \lVert Q(O) \rVert_{1 \text{ gen}} \hspace{2pt}  \Biggm| \hspace{2pt}  \sum_{i=1}^{n} \lVert O_i  \rVert_1 = 1 \right\} \leq \lVert Q \rVert_{1 \text{ gen}} \\
  \hspace{-30pt} \Leftrightarrow  & 
  \max \left\{   \sum_{i=1}^{n} \sum_{j=1}^{m} \lVert Q_{ij} (O_{i}) \rVert_{1} \hspace{2pt}  \Biggm| \hspace{2pt}  \sum_{i=1}^{n} \lVert O_i  \rVert_1 = 1 \right\} \leq \max \Bigg\{  \sum_{i=1}^{m}  \lVert Q_{1i} (A_{1}) \rVert_{1} &   \{\text{\autoref{eq:qo}}, \\
  \hspace{-30pt}& , \hspace{2pt} \ldots \hspace{2pt}, \sum_{i=1}^{m} \lVert Q_{ni} (A_{n}) \rVert_{1} \mid \hspace{1pt}   \lVert A_{1} \rVert_{1} = 1 , \ldots,\lVert A_{n} \rVert_{1} = 1 \Bigg\}   & \text{\autoref{eq:q}} \}  \\
  %\hspace{-30pt} \Leftrightarrow  & \lVert O_1  \rVert_1  + \ldots + \lVert O_n  \rVert_1 = 1  \wedge  \lVert Q_{11} \cdot O_{1} \rVert_{1} + \ldots +  \lVert Q_{1m} \cdot O_{1} \rVert_{1} + \ldots +  \hspace{10pt}  \\
  %\hspace{-30pt}& \leq \max \Bigg\{ \left\lVert Q_{11} \left(\frac{O_{1}} {\lVert O_{1} \rVert_1}\right) \right\rVert_{1} \hspace{1pt} + \ldots +  \left\lVert Q_{1m} \left(\frac{O_{1}} {\lVert O_{1} \rVert_1}\right)  \right\rVert_{1} \hspace{1pt}, \hspace{2pt} \ldots \hspace{2pt},   \\
  %\hspace{-30pt}& \left\lVert Q_{n1}  \left(\frac{O_{n}} {\lVert O_{1} \rVert_1}\right) \right\rVert_{1} + \ldots + \left\lVert Q_{nm}  \left(\frac{O_{n}} {\lVert O_{1} \rVert_1}\right) \right\rVert_{1} \hspace{1pt}  \Bigg\}  \\
  \hspace{-30pt} \Leftrightarrow  &  \sum_{i=1}^{n} \lVert O_i  \rVert_1 = 1   \wedge \max \left\{ \sum_{i=1}^{n} \sum_{j=1}^{m} \lVert Q_{ij} (O_{i}) \rVert_{1} \right\} \leq  \\
  \hspace{-30pt}& \max \Bigg\{  \sum_{i=1}^{m}  \lVert Q_{1i}  \left(O_{1} / \lVert O_{1} \rVert_1\right) \rVert_{1} , \hspace{2pt} \ldots \hspace{2pt}, \sum_{i=1}^{m} \lVert Q_{ni}  \left(O_{n} / \lVert O_{n} \rVert_1\right) \rVert_{1} \Bigg\}  \\
  \hspace{-30pt} \Leftrightarrow  &  \sum_{i=1}^{n} \lVert O_i  \rVert_1 = 1   \wedge \max \left\{ \sum_{i=1}^{n} \sum_{j=1}^{m} \lVert Q_{ij} (O_{i}) \rVert_{1} \right\} \leq   \\
  \hspace{-30pt}& \max \Bigg\{ (1 / \lVert O_{1} \rVert_1) \sum_{i=1}^{m}  \lVert Q_{1i}  \left(O_{1} \right) \rVert_{1}, \hspace{2pt} \ldots \hspace{2pt}, (1 / \lVert O_{n} \rVert_1) \sum_{i=1}^{m} \lVert Q_{ni}  \left(O_{n}\right) \rVert_{1} \Bigg\}    
  \end{align*}

  This is equivalent to demonstrating that for all $a_1, \ldots, a_n, x_1, \ldots, x_n \in \mathbb{R}^{+}_{0}$ with $a_1+ \ldots + a_n=1$,
  \begin{equation} 
  \begin{split}
      x_1 + \ldots + x_n  \leq  \max \left\{   \dfrac{1}{a_1} x_1  , \ldots , \dfrac{1}{a_n} x_n   \right\} \\
  \end{split}
  \end{equation}

  Designating $M = \max \left\{   \dfrac{1}{a_1} x_1  , \ldots , \dfrac{1}{a_n} x_n   \right\}$, from the definition of maximum it follows that, for all $1 \leq i \leq n$, $x_i \leq M \cdot a_i$, and consequently, $x_1 + \ldots + x_n \leq M \cdot (a_1 + \ldots + a_n) = M$. Therefore, it holds that:
  \begin{equation}
    \lVert Q(O) \rVert_{1 \text{ gen}} \leq  \lVert Q \rVert_{1 \text{ gen}}.
  \end{equation} 

  As a result, it follows that for an operator $O \in \mathbb{C}^{o_1 \times o_1} \oplus \ldots \oplus  \mathbb{C}^{o_m \times o_m}$,  $ \left\lVert Q\left(\frac{O}{\lVert O \rVert_{1 \text{ gen}}}\right)  \right\rVert_{1 \text{ gen}}$ is upper bounded by $\lVert Q  \rVert_{1 \text{ gen}}$. Thus, \autoref{eq:qo<q} holds.