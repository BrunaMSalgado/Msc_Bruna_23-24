\chapter{Introduction}

\section{Motivation and Context}


Quantum computing dates back to 1982 when Nobel laureate Richard Feynman proposed the idea that constructing computers founded on the principles of quantum mechanics could efficiently simulate quantum systems of interest to physicists, whereas this seemed to be very difficult with classical computers [\cite{feynman2018simulating}].

This paradigm holds immense promise, as evidenced by several compelling results in computational complexity theory [\cite{shor1994algorithms,grover1996fast}]. While hardware advancements have brought the scientific community closer to realizing this potential, the ultimate goal the ultimate goal is yet to be accomplished. A NISQ quantum computer equipped with 50-100 qubits may surpass the capabilities of current classical computers, yet the impact of quantum noise, such as decoherence in entangled states, imposes limitations on the size of quantum circuits that can be executed reliably [\cite{preskill2018quantum}]. Unfortunately, general-purpose error correction techniques [\cite{calderbank1996good, gottesman1997stabilizer, steane1996error}] consume a substantial number of qubits, making it difficult for NISQ devices to make use of them in the near term. For instance, the implementation of a single logical qubit may require between $10^3$ and $10^4$ physical qubits [\cite{fowler2012surface}]. 

To reconcile quantum computation with NISQ computers, quantum compilers perform transformations for error mitigation [\cite{wallman2016noise}] and noise-adaptive optimization [\cite{murali2019noise}]. Additionally, current quantum computers only support a restricted, albeit universal, set of quantum operations. As a result, nonnative operations must be decomposed into sequences of native operations before execution [\cite{harrow2002efficient}, \cite{burgholzer2020advanced}]. In general, perfect computational universality is not sought, but only the ability to approximate any quantum algorithm, with a preference for minimizing the use of additional gates beyond the original requirements. The assessment of these compiler transformations necessitates a comparison of the error bounds between the source and compiled quantum programs. Furthermore, in quantum information theory, the concept of an $\epsilon-\text{approximation}$ channel is fundamental when studying quantum teleportation via noisy channels [\cite{watrous2018theory}]. This suggests the development of appropriate notions of approximate program equivalence, \textit {in lieu} of the classical program equivalence and underlying theories that typically hinge on the idea that equivalence is binary, \textit{i.e.} two programs are either equivalent or they are not [\cite{winskel1993formal}].

As previously noted, Shor's and Grover's algorithms have played a pivotal role in sparking heightened interest within the scientific community toward quantum computing research. On these bases, various endeavors to establish quantum programming languages have surfaced over the past 20 years.  These include imperative languages such as Qiskit [\cite{Qiskit}] and Silq [\cite{bichsel2020silq}], as well as functional languages such as Quipper [\cite{green2013quipper}] and Q\# [\cite{svore2018q}]. On one hand, the design of quantum programming languages is strongly oriented towards implementing quantum algorithms. On the other hand, the  definition of functional paradigmatic languages or functional calculi serves as a valuable tool for delving into theoretical aspects of quantum computing, particularly exploring the foundational basis of quantum computation [\cite{zorzi2016quantum}]. Given the nature of this work, the focus will be on quantum languages designed with this latter aspect in mind. 


QPL, a quantum language within the functional programming paradigm, marks a significant milestone in this context [\cite{selinger2004towards}]. This is a first-order functional language featuring a static type system based on the idea of classical control and quantum data.

Most of the current research on algorithms and programming languages assumes that addressing the challenge of noise during program execution will be resolved either by the hardware or through the implementation of fault-tolerant protocols designed independently of any specific application [\cite{chong2017programming}]. As previously stated, this assumption is not realistic in the NISQ era. Nonetheless, there have been efforts to address the challenge of approximate program equivalence in the quantum setting. [\cite{hung2019quantitative}] and [\cite{tao2021gleipnir}] reason about the issue of noise in a quantum while-language by developing a deductive system to determine how similar a quantum program is from its idealised, noise-free version. An alternative approach was explored in [\cite{dahlqvist2022syntactic}], using linear $\lambda$-calculus as basis – \textit{i.e} programs are written as linear $\lambda$-terms – which has deep connections to both logic and category theory [\cite{girard1995advances}, \cite{benton1994mixed}]. Some positive results were achieved in this setting, but much remains to be done.

\section{Goals}
The notion of approximate equivalence for quantum programming explored in [\cite{dahlqvist2022syntactic}] does not take important operations into account. Specifically, the corresponding mathematical model does not include measurements, classical control flow, or discard operations. Also, the corresponding typing system is often times too strict and cannot properly handle multiple uses of the same resource, such as sampling exactly $n$-times from a distribution. The overarching goal of this M.Sc. project is to tackle the aforementioned limitations. A successful completion of this goal will provide a fully-fledged quantum programming language on which to study metric program equivalence in various scenarios. This includes not only quantum algorithmics – where, for example, the number of iterations in Grover’s algorithm involves approximations – but also quantum information theory, where, for instance, quantum teleportation and the problem
of the discrimination of quantum states have important roles [\cite{nielsen2010quantum}].